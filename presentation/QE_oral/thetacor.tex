\documentclass[mathserif,handout]{beamer}
\usepackage{cite}
\usepackage{calrsfs}

\usepackage[all,cmtip]{xy} 


\usepackage{amsfonts}

\usetheme{Singapore}

\usepackage{hyperref}

\usepackage{amsmath,amssymb}
%\usepackage{euruf}
\usepackage{graphics}
\usepackage{multimedia}

\usepackage{color}
\usepackage{braket}
%\usepackage{mathpple}


\author[JJ, Ma]{Ma Jia Jun}
\title{Cohomological induction and Theta correspondence}
\subtitle[Unitarity]{and the unitarizability of the constructions}
\institute[Dept. of Math.]{Departement of Mathematics, \\
  National University of Singapore\\[1ex]
  \texttt{g0701232@nus.edu.sg}
}

\def\bZ{\mathbb{Z}}
\def\bR{\mathbb{R}}

\newtheorem{thm}{Theorem}

\begin{document}

\begin{frame}[plain]
  \titlepage
\end{frame}

\def\fk{{\mathfrak{k}}}
\def\fh{{\mathfrak{h}}}
\def\fg{{\mathfrak{g}}}
\def\ft{{\mathfrak{t}}}
\def\fu{{\mathfrak{u}}}
\def\fm{{\mathfrak{m}}}
\def\fl{{\mathfrak{l}}}
\def\fq{{\mathfrak{q}}}
\def\Ad{{\mathrm{Ad}}}
\def\Hom{{\mathrm{Hom}}}
\def\H{{\mathrm{H}}}
\def\bC{{\mathbb{C}}}


\section{Cohomological induction}
\subsection{Notations}
\frame{
 \frametitle{\subsecname}
 We follow Wallach's approach~\cite{Wallach1984}~\cite{Wallach1988}
 \begin{description}
 \item[$G$] a real reductive Lie group of inner type 
 \item[$\theta$] Cartan involution of $G$, with  compact subalgebra $\fk$. 
 \item[$\fh $] fundamental Cartan subalgebra of $\fg$ 
 \item[$\ft$] $\fk\cap \fh$ is maximal abelian in $\fk$
 \item[$H$] a element in $i\ft$.
 \item[$\fl$] $=\Set{X\in \fg|[H,X]=0}$
 \item[$\fu$] $=\Set{X\in \fg|[H,X]=\lambda X, \lambda >0}$
 \item[$\fu_k$] $= \fu\cap \fk_\bC$
 \item[$\fq$] $=\fl_\bC+ \fu$, $\theta$-stable parabolic subalgebra.
 %%\item[$L$] $\Set{g\in G|\Ad(g)H=H}$
 \item[$\fm$] $=\fk\cap \fl$.
% \item[$\fq_k$] $\triangleq \fq\cap \fk_\bC = \fm_\bC + \fu_k$, 
%   $\theta$-satable parabolic of $\fk$.
 \item[$M$] $=\Set{g\in K|\Ad(g)H=H}=K\cap L$.
 \end{description}
}

\subsection{Cohomology induction}
%\frame[<+->]{
\frame{
  \frametitle{\subsecname}
  Fix a $(\fl,M)$-module $W$.
  \begin{enumerate}[(I)]
  \item Regard $W$ as a $(\fq, M)$-module on which $\fu$ act trivially. 
  \item Define $M(\fq,W) = U(\fg_\bC)\otimes_{U(\fq)}W$, 
    which is a $(\fg,M)$-module.
  \item Apply the Zuckerman functor, $\Gamma^j(M(\fq,W))$.
  \end{enumerate}
  $\Gamma^j(M(\fq,W))$ is
    trivial for all $j<n$. ($n=\dim \fu_k$).
}

\def\cC{\mathcal{C}}
\subsection{Zuckerman functor}
\frame{
  \frametitle{\subsecname}
  The Zuckerman functor is defined by 
  ($H(K)$:left $K$-finite smooth
  functions on $K$)
  \[
  \Gamma^j(V) = \H^j(\fk,M;V\otimes H(K)).
  \]
  $\H^j(V)$ is the $j$-th cohomology group of the cochian complex 
  \[
  C^j(\fk,M;V)= \Hom_M(\bigwedge^j(\fk/\fm), V)
  \]

  $\Gamma^j$ is a functor from $\cC(\fg,M)$ to $\cC(\fg,K)$, where $K$
  action given by right transform of $H(K)$ and $\fg$ action given by
  the commutative relation 
  \[
  \xymatrix{
    \Gamma^j(U(\fg)\otimes V) \ar[d]_{\Gamma^j(m)} 
    \ar[r]^{T_{U(\fg)}(V)}
    &  U(\fg)\otimes \Gamma^j(V) \ar[d]^{m}  \\ 
    \Gamma^j(V) \ar[r]^{\mathrm{Identity}} & \Gamma^j(V)
  }
  \]
}

\def\inn#1#2{\left<#1,#2\right>}

\subsection{Hermitian Form}
\frame{
  \frametitle{\subsecname}
  Now assume that $W$ is an irreducible $(\fl,M)$-module 
  admitting a positive definite 
  Hermitian form $\inn{\cdot}{\cdot}$. 
  We construct a hermitian form on $\Gamma^n(M(\fq,W))$.
  \begin{enumerate}[(I)]
  \item Shapovalov Form on $M(\fq,W)=U(\fg_\bC)\otimes _{U(\fq)} W$.
    \[(x\otimes w, y\otimes v) = \inn{p(y^*x)w}{v},\] 
    $p$ --- projection from $U(\fg_\bC)$ to $U(\fl_\bC)$ by
    decomposition 
    \[
    U(\fg_\bC)=U(\fl_\bC)\oplus(\overline{\fu}U(\fg_\bC)+U(\fg_\bC)\fu).
    \]
  \item Hermitian form on $\Gamma^n(M(\fq,W))$ --- by
    the nature pairing between complexes
    $C^j(M(\fq,W))\subset \bigwedge^j(\ft_\bC/\fm_\bC)^*\otimes M(\fq,W)$ and
    $C^{(2n-j)}(M(\fq,W))$:
    \[
    \inn{\alpha\otimes w}{\beta\otimes v} = (\alpha,\beta)(w,v)
    \]
  \end{enumerate}
  If $M(\fq,W)$ is also irreducible, $(\cdot,\cdot)$ non-degenerate
  and only $\Gamma^n(M(\fq,W))$ nontrivial.
}

\def\hK{\widehat{K}}
\def\hM{\widehat{M}}
\def\ch{\mathrm{ch}}
\def\sgn{\mathrm{sgn}}
\subsection{Signature character}
\frame{
  \frametitle{\subsecname}
  For a $(\fg,K)$ module $V$ define the {\em character} of it:
  \[
  \ch_M(V) = \sum_{\gamma\in \widehat{M}} \dim\,\Hom_M(F_\gamma, V)\gamma.
  \]
  $V$ admits Hermitian form $\inn{\cdot}{\cdot}$, then define
  {\em signature character}:
  \[
  \ch_s(V,\inn{\cdot}{\cdot}) = \sum_{\gamma \in \widehat{M}} \frac{1}{d(\gamma)}
  (p_\gamma-q_\gamma)\gamma
  \]
  $(p_\gamma$,$q_\gamma)$ is the signature of the Hermitian form 
  $\inn{\cdot}{\cdot}$ restricted on isotropic component $V(\gamma)$.

  Note that $\inn{\cdot}{\cdot}$ is definite if and only if 
  \[
  \ch_s(V,\inn{\cdot}{\cdot}) = \pm \ch_M(V)
  \]
}

\subsection{Unitarizablity I}
\frame{
  \frametitle{\subsecname}
  By vanishing theorems and the construction of the Hermition form on 
  $\Gamma^j(V)$ we have:
  \[
  \begin{split}
    \ch_s(\Gamma^n(V)) 
    &= \sum_{\gamma\in \hK} \sgn(
    H^n(\ft,M;V\otimes F_\gamma^*)\otimes F_\gamma) \\
    &= \sum_{\gamma\in \hK} \sgn(\bigoplus_j C^j(\ft,M;V\otimes F_\gamma^*))
    \ch_M \gamma
  \end{split}
  \]
  Hence reduce computing $\ch_s(\Gamma^n(M(\fq,W)))$ to computing
  \[
  \ch_s(\bigwedge^j (\ft_\bC/\fm_\bC)^*\otimes M(\fq,W) \otimes F^*_\gamma)
  \]
  The key is to compute 
  \[
  \ch_s(M(\fq,W)).
  \]
}

\subsection{Unitarizablity II}
\frame{
  \frametitle{\subsecname}
  Now we use a ``continous argument''.
  
  Assume there exists $\mu\in i\fl^*$ such that 
  $(\mu,\alpha)>0$ for $\alpha\in \Delta(\fu_k,\ft_\bC)$ and some technical condition. 
  $M(\fq,W\otimes C_{-t\mu})$ is irreducible for $t\geq 0$. 
  %Note that:
  \begin{enumerate}[(I)]
  \item the decomposition of $M(\fq,W\otimes C_{-t\mu})$ 
    into eigenspaces  is independent of $t$;
  \item signature takes integer value, then is constant for all $t\geq 0$;
  \item Key formula:
    \[
    \begin{split}
      &\inn{X^IY^J\otimes w}{X^{I'}Y^{J'}\otimes w'}_t\\
      =& t^{|I|+|J|}\delta_{I,I'}\delta_{J,J'}(-1)^{|J|} \prod_a(\mu,\alpha_a)^{i_a}
      \prod_b(\mu,\beta_b)^{j_b}\inn{w}{w'} \\
      &+ \text{lower degree terms}.
    \end{split}
    \]
  \end{enumerate}
  This gives an expression for $\ch_s(M(\fq,W))$ by $\ch_M(W)$.
  %Finish the proof by compare the form of siganiture charactor and charactor.
}

\def\Sp{{\mathrm{Sp}}}
\def\GL{{\mathrm{GL}}}
\def\tSp{{\widetilde{\mathrm{Sp}}}}
\def\bR{\mathbb{R}}
\def\cR{{\mathcal{R}}}
\def\bZ{\mathbb{Z}}
\def\tG{{\widetilde{G}}}
\def\htG{{\widehat{\widetilde{G}}}}
\def\cY{\mathcal{Y}}
\def\cN{\mathcal{N}}
\section{Theta correspondence}
\subsection{Representations of double cover of Symplectic group}
\frame{
   \frametitle{\subsecname}
   For simplectic group $\Sp$ (except complex case),
   for each central character $\chi$ there is 
   an {\em oscillator representation} $\omega=\omega_\chi$
   of the double cover $\tSp$ of $\Sp$.
   
   For any subgroup $G$ of $\Sp$, let $\tG$ be the perimage of projecetion 
   $\varphi\colon \tSp \to \Sp$. Suppose $\omega$ is realized on $\cY$, with 
   smooth vector $\cY^\infty$.
   Let
   \[
     \cR(\tG,\omega) = \left.\Set{\begin{array}{c}
        \text{countinous }\\\text{irreducible } \\
         \text{ admissible }\end{array} \rho  \text{ of }\tG|
       \rho\simeq \cY^{\infty}/\cN_\rho } 
     \right/\substack{infinitesimal\\ equivalence}
   \]
}

\def\tr{\mathrm{tr}}
\subsection{Reductive dual pair in Symplectic group}
\frame{
 \frametitle{\subsecname}
 $(G,G')$ called a reductive dual pair if $G$, $G'$ are reductive 
 subgroups of some 
 symplectic group $\Sp$ and are mutual centralizers.
 
% Roughly, we want to know the repsentation of $\tG$ by the represntation of 
% $\tG'$. 
 
 We can only consider irreducible dual pairs. Let $\Sp = \Sp(W)$ 
 \begin{enumerate}[Type I]
 \item $W= V\otimes_D V'$, $D$ with involution $\tau$.
   $G$ and $G'$  
   are group of isometries of sesqui-linear 
   hermitian form $(,)$ and skew-hermitian $(,)'$.
   \[
   \inn{v_1\otimes v_1'}{v_2\otimes v_2'} = \tr((v_1,v_2)(v'_2,v'_1)')
   \]
   E.g. $(O_{p,q}, \Sp_{2n}(\bR)))\subset Sp_{2(p+q)n}(\bR)$.
 \item $W = U\oplus U^*$, $U = V\otimes_D V'$, $U^*=V^*\otimes_D V'^*$
   $G=\mathrm{GL}_D(V)$, $G'=\mathrm{GL}_D(V')$. \\
   E.g. $(\GL_m(\bR),\GL_n(\bR))\subset \Sp_{2nm}(\bR)$
 \end{enumerate}
}


\subsection{Theta correspondence}
\frame{
 \frametitle{\subsecname}
 We follow~\cite{Howe1989}.
 There is a one-one correspondence between $\cR(\tG,\omega)$
 and $\cR(\tG',\omega)$.
  $\cR(\tG \tG',\omega)$ is the graph of this bijection.
 Steps to give this correspondence:
 \begin{enumerate}[(1)]
 \item Given $\rho' \in \cR(\tG',\omega)$, 
 \item Define
   \[
   \cY_{\rho'}=\bigcap_{\rho \cong \cY^{\infty}/\cY_1} \cY_1.
   \]
 \item Consider $\cY^\infty/\cY_{\rho'}$ be a $\tG\tG'$ module, then
   \[
   \cY^\infty/\cY_{\rho'} = \rho_1\otimes \rho'
   \]
   $\rho_1$ is a $\tG$ module
 \item $\rho_1$ has a unique irreducible quotient $\rho$.
 \item $\rho'\leftrightarrow \rho$ gives the correspondence. 
 \end{enumerate}

 The proof is purely algebraic by classical invariant theory.
}



\subsection{Unitary version}
\frame{
 \frametitle{\subsecname}
 Want to know when this correspondence preserves unitarity. Li~\cite{Li1989}
 gives results in {\em stable range} for Type I reductive dual pair.
 $\widehat{\tG}(\varepsilon)$ --- unitary dual of $\tG$ with nontrivial action
 on $\bZ_2={\pm 1}$.
 %A different approch to the correspondence:
 \begin{enumerate}[(1)]
 \item Fix realizaitions $\cY^\infty$ and $H_\sigma^\infty$ 
   of $\omega$ and $\sigma\in \widehat{\tG'}(\varepsilon)$
 \item Consider $\cY^\infty\otimes H_\sigma^\infty$.
 \item 
   \[
   (\Phi,\Phi')_\sigma = \int_{G'}(\Phi,(\omega\otimes \sigma)(g)\Phi')dg
   \]
 \item Make sense in {\em stable range}: Witt index of $V$
   $\geq \dim V'$.
 \item 
   \[
   R = \Set{\Phi \in \cY^\infty\otimes H_\sigma^\infty|(\Phi,\Phi')_\sigma =0 
   \text{ for all} \Phi'}
   \]
 \item $H(\sigma) \triangleq (\cY^\infty\otimes H_\sigma^\infty)/R$ 
   gives irreducible 
   unitary representation denoted by
   $\pi(\sigma)$.
 \end{enumerate}

}

\frame{
  \frametitle{}
   Howe's duality correspondence gives an injection (in the stable range)
   \[
   \widehat{\widetilde{G'}}(\varepsilon)\hookrightarrow \htG(\varepsilon).
   \]
   
   This correspondence is the 
   same as Howe's by $\sigma\to \pi(\sigma^*)$.

   The proof is by analysis.
}


\subsection{Unitary dual}

\frame{
 \frametitle{\subsecname}
 A central problem in represnetaion theory is to understand the unitary dual of 
 group $G$. 
 
 For real reductive group, 
 we can construct unitary representations by parabolic induction. 
 This gives {\rm principle series}.
 
 But they are not all. It may also have {\rm discrete series}
 (can be constructed by cohomological induction or other methods)
 and other types.
 
 Theta-correspondence gives another way to construct. 
 
 There seems to be some relation between cohomological induction 
 and theta-correspondence \cite{Zhu2004}. 
 We are trying to find some kind of such relations.
}


%\frame{
%  \frametitle{References}
%  \bibstyle{apa}
%  %\bibliographystyle{plain}
%  \bibliography{bib/reppapers}	
%}

\section{References}
\begin{frame}[allowframebreaks]{Bibliography}
\bibliographystyle{alpha}
\bibliography{bib/reppapers}
\end{frame}

\end{document}
