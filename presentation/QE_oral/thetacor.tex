\documentclass[11pt]{beamer}
\usepackage{cite}


\usepackage[all,cmtip]{xy} 


\usepackage{amsfonts}

\usetheme{umbc4}

\usepackage{hyperref}

\usepackage{amsmath,amssymb}
%\usepackage{euruf}
\usepackage{graphics}
\usepackage{multimedia}

\usepackage{color}
\usepackage{braket}
%\usepackage{mathpple}


\author[JJ, Ma]{Ma Jia Jun}
\title{Theta correspondence and cohomological induction}
\subtitle[Unitarity]{with the unitarity the constructions}
\institute[Dept. of Math.]{Departement of Mathematics, \\
  National University of Singapore\\[1ex]
  \texttt{g0701232@nus.edu.sg}
}

\def\bZ{\mathbb{Z}}
\def\bR{\mathbb{R}}

\newtheorem{thm}{Theorem}

\begin{document}

\begin{frame}[plain]
  \titlepage
\end{frame}

\def\fk{{\mathfrak{k}}}
\def\fh{{\mathfrak{h}}}
\def\fg{{\mathfrak{g}}}
\def\ft{{\mathfrak{t}}}
\def\fu{{\mathfrak{u}}}
\def\fm{{\mathfrak{m}}}
\def\fl{{\mathfrak{l}}}
\def\fq{{\mathfrak{q}}}
\def\Ad{{\mathrm{Ad}}}
\def\Hom{{\mathrm{Hom}}}
\def\H{{\mathrm{H}}}
\def\bC{{\mathbb{C}}}


\section{Cohomological induction}
\subsection{Notations}
\frame{
 \frametitle{\subsecname}
 We following Wallach's approach~\cite{Wallach1984}~\cite{Wallach1988}
 \begin{description}
 \item[$G$] a real reductive Lie group of inner type 
 \item[$\theta$] Cartan involution of $G$, with $\fk$ compact subalgebra. 
 \item[$\fh $] fundamental Cartan subalgebra of $\fg$ 
 \item[$\ft$] $\fk\cap \fh$ is maximal abelian in $\fk$
 \item[$H$] a element in $i\ft$.
 \item[$\fl$] $\Set{X\in \fg|[H,X]=0}$
 \item[$\fu$] $\Set{X\in \fg|[H,X]=\lambda X, \lambda >0}$
 \item[$\fq$] $=\fl_\bC+ \fu$, $\theta$-stable parabolic subalgebra.
 %%\item[$L$] $\Set{g\in G|\Ad(g)H=H}$
 \item[$\fm$] $=\fk\cap \fl$.
% \item[$\fq_k$] $\triangleq \fq\cap \fk_\bC = \fm_\bC + \fu_k$, 
%   $\theta$-satable parabolic of $\fk$.
 \item[$M$] $=\Set{g\in K|\Ad(g)H=H}=K\cap L$.
 \end{description}
}

\subsection{Cohomology induction}
%\frame[<+->]{
\frame{
  \frametitle{\subsecname}
  Fix a $(\fl,M)$-module $W$.
  \begin{enumerate}[(I)]
  \item Regard $W$ as a $\fq$-module with $\fu$ act trivially on it. 
  \item Define $M(\fq,W) = U(\fg)\otimes_{U(\fl)}W$, 
    which is a $(\fk,M)$-module.
  \item Applying the Zuckerman functor, $\Gamma^j(M(\fl,W))$.
  \item $\Gamma^j(M(\fl,W))$ is
    nontrivial for all $j<n$. ($n=\dim\fu_k$)
  \end{enumerate}
}

\subsection{Zuckerman functor}
\frame{
  \frametitle{\subsecname}
  The Zuckerman functor is defined by 
  ($H(K)$:left $K$-finite smooth
  functions on $K$)
  \[
  \Gamma^j(V) = \H^j(\fk,M;V\otimes H(K)).
  \]
  $\H^j(V)$ is the cohomology of complex 
  \[
  C^j(\fk,M;V)= \Hom_M(\bigwedge^j(\fk/\fm), V)
  \]

  $\Gamma^j$ is a functor from $C(\fg,M)$ to $C(\fg,K)$, where $K$
  action given by right transform of $H(K)$ and $\fg$ action given by
  the commutative relation 
  \[
  \xymatrix{
    \Gamma^j(U(\fg)\otimes V) \ar[d]_{\Gamma^j(m)} 
    \ar[r]^{T_{U(\fg)}(V)}
    &  U(\fg)\otimes \Gamma^j(V) \ar[d]^{m}  \\ 
    \Gamma^j(V) \ar[r]^{\mathrm{Identity}} & \Gamma^j(V)
  }
  \]
}

\def\inn#1#2{\left<#1,#2\right>}

\subsection{Hermitian Form}
\frame{
  \frametitle{\subsecname}
  Now assume $W$ be an irreducible $(\fl,M)$-module admits a positive definite 
  Hermitian form $\inn{\cdot}{\cdot}$. 
  We can construct a hermitian form on $\Gamma^n(M(\fq,W))$.
  \begin{enumerate}[(I)]
  \item Shapovalov Form,
    $(x\otimes w, y\otimes v) = \inn{p(y^*x)w}{v}$, \\
    $p$ is the projection from $U(\fg$ to $U(\fl_\bC)$ by
    decompostion 
    \[
    U(\fg_\bC)=U(\fl_\bC)\oplus(\overline{\fu}U(\fg_\bC)+U(\fg_\bC)\fu).
    \]
  \item The Hermitian form on $\Gamma^n(M(\fq,W))$ comes from 
    the nature pairing between complex 
    $C^j(M(\fq,W))\subset \bigwedge^j(\ft/fm)_\bC^*\otimes M(\fq,W)$ and
    $C^{(2n-j)}(M(\fq,W))$:
    \[
    \inn{\alpha\otimes w}{\beta\otimes v} = (\alpha,\beta)(w,v)
    \]
  \end{enumerate}
  If $M(\fq,W)$ is also irreducible, $(\cdot,\cdot)$ non-degenerate,
  and only $\Gamma^n(M(\fq,W)$ nontrivial.
}

\def\hK{\widehat{K}}
\def\hM{\widehat{M}}
\def\ch{\mathrm{ch}}
\def\sgn{\mathrm{sgn}}
\subsection{Signature charactor}
\frame{
  \frametitle{\subsecname}
  For a $(\fg,K)$ module $V$ define the {\em charactor} of it:
  \[
  \ch_M(V) = \sum_{\gamma\in \widehat{M}} \dim\,\Hom_M(F_\gamma, V)\gamma.
  \]
  $V$ admits Hermitian form $\inn{\cdot}{\cdot}$, then define
  {\em signature charactor}:
  \[
  \ch_s(V,\inn{\cdot}{\cdot}) = \sum_{\gamma \in \widehat{M}} \frac{1}{d(\gamma)}
  (p_\gamma-q_\gamma)\gamma
  \]
  $(p_\gamma$,$q_\gamma)$ is the signature of the Hermitian form 
  $\inn{\cdot}{\cdot}$ restricted on isotropic component $V(\gamma)$.

  Note that $\inn{\cdot}{\cdot}$ definite if and only if 
  \[
  \ch_s(V,\inn{\cdot}{\cdot}) = \pm \ch_M(V)
  \]
}

\subsection{Unitarizablity}
\frame{
  \frametitle{\subsecname}
  By vanishing theorems and the construction of the Hermition form on 
  $\Gamma^j(V)$ we have:
  \[
  \begin{split}
    \ch_s(\Gamma^n(V)) 
    &= \sum_{\gamma} \sgn(\sum_{\gamma\in \hK} 
    H^n(\ft,M;V\otimes F_\gamma^*)\otimes F_\gamma) \\
    &= \sum_{\gamma\in \hK} \sgn(\bigoplus_j C^j(\ft,M;V\otimes F_\gamma^*))
    \ch_M \gamma
  \end{split}
  \]
  Hence deduce compute $\ch_s(\Gamma^n(M(\fq,W)))$ to compute
  \[
  \ch_s(\bigwedge^j (\ft/fm)^*\otimes M(\fq,W) \otimes V^*_\gamma)
  \]
  The key is to compute 
  \[
  \ch_s(M(\fq,W)).
  \]
}

\subsection{Unitarizablity}
\frame{
  \frametitle{\subsecname}
  Now we use a ``continous argument''.
  
  Assume exists $\mu\in i\fl^*$ such that 
  $(\mu,\alpha)>0$ for $\alpha\in \Sigma$. 
  Then $M(\fq,W\otimes C_{-t\mu}$ is irreducible. 
  Note that:
  \begin{enumerate}[(I)]
  \item the decomposition of $M(\fq,W\otimes C_{-t\mu})$ 
    in to isotypic components
    is independent of $t$;
  \item signature take integer value, the constant for all $t>0$;
  \item Key formula:
    \[
    \begin{split}
      &\inn{X^IY^J\otimes w}{X^{I'}Y^{J'}\otimes w'}_t\\
      =& t^{|I|+|J|}\delta_{I,I'}\delta_{J,J'}(-1)^{|J|} \prod_k(\mu,\alpha_k)^{i_k}
      \prod_k(\mu,\beta_k)^{j_k}\inn{w}{w'} \\
      &+ \text{lower order terms}
    \end{split}
    \]
  \end{enumerate}
  Then give an expression of $\ch_s(M(\fq,W))$ by $\ch_M(W)$.
  Finish the proof by compare the form of siganiture charactor and charactor.
}

\def\Sp{{\mathrm{Sp}}}
\def\tSp{{\widetilde{\mathrm{Sp}}}}
\def\bR{\mathbb{R}}
\def\cR{{\mathcal{R}}}
\def\tG{{\widetilde{G}}}
\def\cY{\mathcal{Y}}
\def\cN{\mathcal{N}}
\section{Theta correspondence}
\subsection{Represnetation of double cover of Symplectic group}
\frame{
   \frametitle{\subsecname}
   For simplectic group $Sp$ (except complex case),
   for each central charactor $\chi$ there is 
   a {\em oscillator representation} $\omega=\omega_\chi$
   of the double cover $\tSp$ of $\Sp$.
   
   For any subgroup $G$ of $Sp$, let $\tG$ be the perimage of projecetion 
   $\pi\colon \tSp \to \Sp$. Let
   \[
   \cR(\tG) = \Set{\text{countinous irr. adm. representation }
     \rho\text{ of }\tG| \rho\cong \cY^{\infty}/\cN_\rho }/\text{ininitesimal equivalence}
   \]
   
   We have
   $\cR(\tG_1\times \tG_2) \cong \cR(\tG_1)\times \cR(\tG_2)$.
}


\subsection{Reductive dual pair in Symplectic group}
\frame{
 \frametitle{\subsecname}
 $(G,G')$ called a reductive dual pair if $G$, $G'$ are reductive 
 subgroups of some 
 symplectic group $Sp$ and are mutual centralizers.
 
 Roughly, we want to know the repsentation of $\tG$ by the represntation of 
 $\tG'$. 
 
 
 We can consider irreducible dual pairs. 
 \begin{enumerate}[Type I]
 \item $\Sp=\Sp(V\otimes V')$, $G$ and $G'$ 
   are group of isometries of $(,)$ $(,)'$
   respectively. $(V,(,))$ hermitian $(V',(,)')$ skew-hermitian.
 \item $\Sp=\Sp(U\oplus U^*)$, $G=\mathrm{GL}(U)$, $G'=\mathrm{GL}(U')$ 
 \end{enumerate}
}


\subsection{Theta correspondence}
\frame{
 \frametitle{\subsecname}
 We follows~\cite{Howe1989}.
 There is a one-one correspondence between $\cR(\tG)$ and $\cR(\tG')$ 
 and $\cR(\tG \tG')$ is the graph of this bijection.
 
 Steps to give this correspondence:
 \begin{enumerate}[(1)]
 \item Given $\rho' \in \cR(\tG')$, 
 \item Define
   \[
   \cY_{\rho'}=\bigcap_{\rho \cong \cY^{\infty}/\cY_1} \cY_1.
   \]
 \item Consider $\cY^\infty/\cY_{\rho'}$ be a $\tG\tG'$ module, then
   \[
   \cY^\infty/\cY_{\rho'} = \rho_1\otimes \rho'
   \]
   $\rho_1$ is a $\tG$ module
 \item $\rho_1$ has a unique irreducible quotiont $\rho'$.
 \item $\rho'\leftrightarrow \rho$ gives the correspondence. 
 \end{enumerate}

 The proof is purely algebraically by using the classical invariant theory.
}

\subsection{Unitarity version}
\frame{
 \frametitle{\subsecname}
 Want to know when this correspondence preserve unitarity. Li~\cite{Li1989}
 given results in {\em stable range} for Type I reductive dual pair.
 Li introduct a different approch to the correspondence:
 \begin{enumerate}[(1)]
 \item Fix realizaitions $\cY^\infty$ and $H_\omega^\infty$ 
   of $\omega$ and $\sigma\in \widehat{\tG'}(\varepsilon)$
 \item Consider $\cY^\infty\otimes H_\sigma^\infty$.
 \item Define form:
   \[
   (\Phi,\Phi')_\sigma = \int_{G'}(\Phi,(\omega\otimes \sigma)(g)\Phi')dg
   \]
 \item make sense in {\em stable range}: Witt index of $V$
   $\geq \dim V'$.
 \item Let 
   \[
   R = \Set{\Phi \in \cY^\infty\otimes H_\sigma^\infty|(\Phi,\Phi')_\sigma =0 
   \text{ for all} \Phi'}
   \]
 \item $H(\sigma) \triangleq (\cY^\infty\otimes H_\sigma^\infty)/R$ is irreducible 
   unitary representation denoted by
   $\phi(\sigma)$.
 \end{enumerate}

 This correspondence is 
 same as Howe's by $\rho'\to \pi(\rho'^*)$.
 
}

\subsection{Unitary dual}

\frame{
 \frametitle{\subsecname}
 A central problem in represnetaion theory is to know the unitary dual of 
 group $G$. 
 
 For real reductive group, 
 we can construct unitary representations by parabolic induction. 
 This gives {\rm principle series}.
 
 But they are not all. There are also some {\rm discrete series}
 (can be constructed by cohomological induction or other ways).
 
 Theta-correspondence give another way to construct. 
 
 It seems some realation between cohomological induction 
 and theta-correspondence \cite{Zhu2004}. 
}


%\frame{
%  \frametitle{References}
%  \bibstyle{apa}
%  %\bibliographystyle{plain}
%  \bibliography{bib/reppapers}	
%}

\section{References}
\begin{frame}[allowframebreaks]{Bibliography}
\bibliographystyle{alpha}
\bibliography{bib/reppapers}
\end{frame}

\end{document}
