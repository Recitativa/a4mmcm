%\documentclass[final]{beamer} % beamer 3.10: do NOT use option hyperref={pdfpagelabels=false} !
\documentclass[final,hyperref={pdfpagelabels=false}]{beamer} % beamer 3.07: get rid of beamer warnings
 \mode<presentation>  %% check http://www-i6.informatik.rwth-aachen.de/~dreuw/latexbeamerposter.php for example
 \usetheme{SMS2011}
 %% you should define your own theme e.g. for big headlines using your own logos 

%\usepackage[english]{babel}
\usepackage[latin1]{inputenc}
\usepackage{amsmath,amsthm, amssymb, latexsym}
\usepackage{braket}
%\usepackage{paralist}
\usepackage{eufrak}
\usepackage[all,cmtip]{xy}
\usepackage{diagxy}


\usepackage{array}

\usefonttheme[onlymath]{serif}
\boldmath

\def\emph#1{{\em \underline{#1}}}

\usepackage[orientation=portrait,size=a1,scale=1.52,debug]{beamerposter}   
% e.g. for DIN-A1 poster, with optional grid and debug output

\def\sign{{\rm sign}}
\def\Ker{{\rm Ker}}
\def\Im{{\rm Im}}
\def\Hom{{\rm Hom}}
\def\End{{\rm End}}
\def\Mat{{\rm Mat}}
\def\Ind{{\rm Ind}}
\def\bR{{\mathbb{R}}}
\def\bN{{\mathbb{N}}}
\def\bZ{{\mathbb{Z}}}
\def\bC{{\mathbb{C}}}
\def\bQ{{\mathbb{Q}}}
\def\bB{{\mathbb{B}}}
\def\bA{{\mathbb{A}}}
\def\bs{{\mathbf{s}}}
\def\bd{{\mathbf{d}}}
\def\bT{{\mathbb{T}}}
\def\bt{{\mathbf{t}}}
\def\br{{\mathbf{r}}}
\def\vv{{\vec{v}}}
\def\vw{{\vec{w}}}
\def\vx{{\vec{x}}}
\def\vy{{\vec{y}}}
\def\v0{{\vec{0}}}
\def\ol{\overline}
\def\sspan{\rm{span}}
\def\sl2{{\mathfrak{sl}(2)}}
\def\slc{{\mathfrak{sl}(2,\bC)}}
\def\sp{{\mathfrak{sp}}}
\def\gg{{\mathfrak{g}}}
\def\kk{{\mathfrak{k}}}
\def\pp{{\mathfrak{p}}}
\def\qq{{\mathfrak{q}}}
\def\sP{\mathcal{P}}
\def\sH{\mathcal{H}}
\def\sU{\mathcal{U}}
\def\sC{\mathcal{C}}
\def\sS{\mathcal{S}}
\def\Stab{{\rm Stab}}
\def\ad{{\rm ad}}
\def\Ad{{\rm Ad}}
\def\id{{\rm id}}
\def\sgn{{\rm sgn}}
\def\gcd{{\rm gcd}}
\def\inn#1#2{\left\langle{#1},{#2}\right\rangle}
\def\abs#1{\left|{#1}\right|}
\def\norm#1{{\left\|{#1}\right\|}}
\def\Sp{{\rm Sp}}
\def\SL{{\rm SL}}
\def\O{{\rm O}}
\def\SO{{\rm SO}}
\def\det{{\rm det}}

\def\tG{{\widetilde{G}}}
\def\tK{{\widetilde{K}}}
\def\tM{{\widetilde{M}}}
\def\tJ{{\widetilde{J}}}
\def\trho{{\widetilde{\rho}}}
\def\tsigma{{\widetilde{\sigma}}}
\def\tDelta{\widetilde{\Delta}}
\def\barZ{{\overline{Z}}}
\def\bz{{\overline{z}}}

\def\Ad{\mathrm{Ad}}
\def\fgl{\mathfrak{gl}}
\def\fsl{\mathfrak{sl}}
\def\fso{\mathfrak{so}}
\def\diag#1{\mathrm{diag}(#1)}
\def\lww{\mathcal{W}}
\def\lxx{\mathcal{X}}
\def\lyy{\mathcal{Y}}
\def\fbb{\mathfrak{b}}
\def\fhh{\mathfrak{h}}
\def\fnn{\mathfrak{n}}
\def\fuu{\mathfrak{u}}
\def\fll{\mathfrak{l}}
\def\foo{\mathfrak{o}}
\def\fpp{\mathfrak{p}}
\def\fqq{\mathfrak{q}}
\def\ftt{\mathfrak{t}}
\def\fgg{\mathfrak{g}}
\def\fkk{\mathfrak{k}}
\def\caa{\mathcal{A}}
\def\ccc{\mathcal{C}}
\def\cdd{\mathcal{D}}
\def\chh{\mathcal{H}}
\def\cjj{\mathcal{J}}
\def\crr{\mathcal{R}}
\def\css{\mathcal{S}}
\def\cpp{\mathcal{P}}
\def\cww{\mathcal{W}}
\def\cnn{\mathcal{N}}
\def\cuu{\mathcal{U}}
\def\cxx{\mathcal{X}}
\def\cyy{\mathcal{Y}}
\def\cff{\mathcal{F}}
\def\czz{\mathcal{Z}}
\def\cug{\cuu(\fgg)}
\def\fmm{\mathfrak{m}}
\def\GL{\mathrm{GL}}
\def\SO{\mathrm{SO}}
\def\OO{\mathrm{O}}
\def\rad{\mathrm{rad}}
\def\SL{\mathrm{SL}}
\def\tr{\mathrm{tr}}
\def\Mat{\mathrm{Mat}}
\def\U{\mathrm{U}}
\def\Gr{\mathrm{Gr\,}}
\def\tU{{\widetilde{U}}}
\def\pz#1{\partial z_{#1}}
\def\ddt{\left.\frac{d}{dt}\right|_{t=0}}
\def\csigma\
\def\Ind{{\rm Ind}}


\def\cmm#1#2{\left[{#1},{#2}\right]}
\def\acmm#1#2{\left\{{#1},{#2}\right\}}

\def\seesawpair#1#2#3#4{
\xymatrix{
{#1} \ar@{-}[dr]& {#2} \\
{#3} \ar@{-}[ur] & {#4}}
}

\def\real{{\rm Re\,}}
\def\imag{{\rm Im\,}}

\def\rmk{{\bf Remark:}}

\def\cR{{\mathcal{R}}}
\def\cC{\mathcal{C}}

 
\title[$\cuu(\fgg)^K$ actions]{Dual pairs and $\cuu(\fgg)^K$ actions}
\author[Ma]{Ma Jia Jun}
\date{3 May 2011}
\institute[NUS]{National University of Singapore}


\begin{document}
  \begin{frame}
    %\begin{minipage}[t][150cm][t]{.999\linewidth}
    \begin{columns}[t]
    \begin{column}{.48\linewidth}
      \begin{block}{Introduction and Motivation}
        \begin{itemize}
        \item Construct representations for real classical groups:
          \begin{itemize}
          \item Parabolic induction
          \item Cohomological induction / Derived Functor
          \item Theta correspondence
          \end{itemize}
        \item Identify Unitary representations
          \begin{itemize}
          \item Signature character for cohomological induction,
          \item Theta lifting in stable range preserves the untarity.
          \end{itemize}
        \item Relations between parabolic induction and theta:\\
           \hskip2em Kudla's induction principle
        \item Question: \\
          \hskip2em 
          What is the relationship between derived functor and theta?
        \item Agenda
          \begin{itemize}
          \item Notation is introduced in left column,
          \item The main result and exmaples are given 
            in right column.
          \end{itemize}
        \end{itemize}
      \end{block}
      \vfill
      \begin{block}{Oscillator representation}
        \begin{itemize}
        \item Fix a symplectic space $W$, define \emph{Heisenberg group} $\mathrm{H} = W\oplus \bR$.
        \item Fix a nontrivial unitary character $\psi$ of $\bR$. 
        \item \emph{Oscillator representation}: the unqiue (up to isomorphism) irreducible unitary represntation of $\mathrm{H}$ with centeral character $\psi$.
        %\item These reprenstation related to simple harmonic oscillator in quantum mechanics.
        \item This representation can be extended to a representation $\omega$ of 
          $\widetilde{\Sp(W)}\ltimes \mathrm{H}$, where
          $\widetilde{\Sp(W)}$ is  the unquie nontrivial double cover of the symplectic group $\Sp(W)$.
        \end{itemize}
      \end{block}
      \vfill
      \begin{block}{Reductive dual pairs}
        $(G,G')$ called a \emph{reductive dual pair} in $\Sp(W)$ if
        \begin{itemize}
        \item $G$, $G'$  are subgroups of $\Sp(W)$ and act on $W$ reductively,
        \item $G$ is the centralizer of $G'$ in $\Sp(W)$ and vice versa. 
        \end{itemize}
        List of irreducible real reductive dual pairs:
       \begin{center}\small
        \begin{tabular}{ l |lll }
           & $G$ & $G'$ & $\Sp(W)$\\
          \hline
           & $\mathrm{O}(p,q)$ & $\Sp(2n,\bR))$ &$\Sp(2(p+q)n,\bR)$\\
           Type I & $\mathrm{O}(p,\bC)$ & $\Sp(2n,\bC))$ &$\Sp(4pn,\bR)$\\
          & $\mathrm{U}(p,q)$ & $\mathrm{U}(r,s)$ & $\Sp(2(p+q)(r+s),\bR)$\\ 
          & $\Sp(p,q)$& $\mathrm{O}^*(n)$ & $\Sp(4n(p+q),\bR$\\
          \hline
          &$\GL(m,\bR)$ & $\GL(n,\bR)$& $\Sp(2nm,\bR)$\\
         Type II &$\GL(m,\bC)$& $\GL(n,\bC))$& $\Sp(4nm,\bR)$ \\
          &$\GL(m,\bH)$& $\GL(n,\bH))$& $\Sp(8nm,\bR)$
        \end{tabular}
      \end{center}
    \end{block}

    \begin{block}{Theta/Howe correspondence}
      \begin{columns}
        \begin{column}{0.7\textwidth}
      \begin{itemize}
        \item Let $\tG$ denote the preimage of $G$ in the double cover 
          $\widetilde{\Sp(W)}$ for a subgroup $G$ of $\Sp$.
        \item Let $\cR(\tG,\omega)$ be the set of isomorphism classes 
          of irreducible smooth representations of $\tG$,
          which can be realized as a quotation of $\omega$.
        \item There is a one-one correspondence $\theta$
          between $\cR(\tG,\omega)$ and $\cR(\tG',\omega)$.
        \item This correspondence (over $\bR$) is proved by {\em Roger E. Howe} 
          using invariant theory.
        \item Howe also conjectured the general local and global theorems of theta correspondence.
  %      \item It is initialed by the studying theta series and have prefound 
      \end{itemize}
    \end{column}
    \begin{column}{0.26\textwidth}
      \includegraphics[width=\textwidth]{howe.jpg}
    \end{column}
  \end{columns}
\end{block}

\begin{block}{Derived functor}
  \begin{itemize}
     \item $\cC(\fgg,K)$: category of admissible $(\fgg, K)$-modules ($K$ compact)
      \begin{itemize}
      \item Category of $\bC$-vector spaces  with compatible $\fgg$ and $K$ action,
      \item $\sspan\Set{Kv}$ is finite for any vector $v$,
      \item each $K$-type occurs finite many times.
      \end{itemize}
    \item \emph{Zuckerman functor}  $\Gamma_{\fgg,M}^{\fgg,K}
      \colon \cC(\fgg,M)\to \cC(\fgg,K)$, 
      where $M\subset K$.
      \begin{itemize}
      \item realized as taking the subspace of $K$ finite vectors,
      \item left exact.
      \item has an analog called {\em Bernstein functor} (taking cofinite vectors, right exact).
      \end{itemize}
    \item $\Gamma^j\triangleq\left(\Gamma_{\fgg,M}^{\fgg,K}\right)^j
      \colon \cC(\fgg,M)\to \cC(\fgg,K)$: \\
      the $j$-th right derived functor of $\Gamma$.
    \item $\Gamma^j$ can be realized on $H^j(\fkk,M; \underline{\hskip1em} \otimes \mathcal{H}(K))$.
    \end{itemize}
\end{block}
\end{column}
\begin{column}{.48\linewidth}
    \begin{block}{The main result}
      Let
      \begin{itemize}
      \item $(G_1,G'_1)$ and $(G_2,G'_2)$ be real reductive dual pairs,
  %    \item ($G_\bC$,$G'_\bC$) be complex reductive dual pairs,
      %\item $G_1$ and $G_2$($G'_1$ and $G'_2$) 
      %  be two different real form of $G_\bC$($G'_\bC$),
      \item $G_1$ and $G_2$ ($G'_1$ and $G'_2$) sit in the same complexification $G_\bC$ ($G'_\bC$),
      \item the Cartan involutions of $G_1$ and $G_2$  commute,
      \item $K_1$ ($K_2$) be the maximal compact subgroups of $G_1$ ($G_2$), so
        \begin{itemize}
        \item we can view $(\fgg, \tK_1)$-module 
          as $(\fgg, \tK_1\cap \tK_2)$-module
        \item $\left(\Gamma_{\fgg,\tK_1\cap \tK_2}^{\fgg,\tK_2}\right)^j$ provide functors 
          from $\cC(\fgg, \tK_1)$ to $\cC(\fgg,\tK_2)$.
        \end{itemize}
     \end{itemize}
     Suppose
     \begin{itemize}
        \item $\chi_1$ and $\chi_2$ are characters of $\tG'_1$ and $\tG'_2$.
      \item $\Gamma^j\theta(\chi_1)$ and  $\theta(\chi_2)$ 
        have same infinitesimal character
        and both contain a same multiplicity free $\tK_2$-type $\tau$.
      \end{itemize}
      Then 
      $\Gamma^j\theta(\chi_1)$ 
      has a subquotion isomorphic to $\theta(\chi_2)$.
    \end{block}
    
    \begin{block}{See-saw pair and  joint action of $\cuu(\fgg)^\tH$ 
and $\cuu(\fhh')^{\tG'}$}


Consider following See-Saw pair:
\vskip.5ex
\begin{minipage}[t]{0.65\textwidth}
$
\xymatrix{
\hskip4em\Omega(\sigma)& G \ar@{-}[dr] & H'& \Omega(\tau)\hskip4em\\
\cR(\tH,\omega) \ni \tau &H\ar@{-}[ur] & G' & \sigma  \in \cR(\tG',\omega)
}
$
\end{minipage}
\begin{minipage}[t]{0.3\textwidth}\small
$H < G$,  $H'> G'$,\\
$\Omega(\tau)$ and $\Omega(\sigma)$\\ are Howe quotients
\end{minipage}
\vskip.5ex
Then:
\[
\Hom_{\tH}(\Omega(\sigma), \tau) 
\cong  \Hom_{\tH\times \tG'}(\omega, \tau\otimes \sigma)
\cong  \Hom_{\tG'}(\Omega(\tau), \sigma)
\]

Observation:
\begin{itemize}
\item $\cuu(\fgg)^\tH$ and $\cuu(\fhh')^{\tG'}$ 
  have a joint action on $\Hom_{\tH\times \tG'}(\omega,\tau\otimes\sigma)$,
\item $\displaystyle \omega(\cuu(\fgg)^\tH) = \omega(\cuu(\fhh')^{\tG'})$.
\end{itemize}

\emph{Key fact:} the $\cuu(\fgg)^\tH$ action on $\Hom_\tH(\Omega(\sigma),\tau)$ and the $\cuu(\fhh')^{\tG'}$  action on $\Hom_{\tG'}(\Omega(\tau),\sigma)$ determine  each other.
\end{block}

\begin{block}{A surjectivity theorem by Helgason}
  Let $\fgg$ be a complex classical Lie algebra.  
  $\fgg = \fkk+\fpp$ under an involution. 
  $\chi$ is a character of $\fkk$.
   Then the following map is surjective:
  \[
  \czz(\fgg) \to \cuu(\fgg)^\fkk/(\Ker(\chi|_{\cuu(\fkk)})\cuu(\fgg)\cap\cuu(\fgg)^\fkk)
  \]
\end{block}

\begin{block}{Sketch of the proof}
  \begin{itemize}
  \item 
  Note that the $\cuu(\fgg)^{\tK_2}$ action commute with the derived functor.
\item 
  The surjectivity theorem will imply that the $\cuu(\fgg)^{\tK_2}$ action is determined by
  infinitesimal characters via above joint action.
\item Use the well-known theorem by Harish-Chandra: $\cuu(\fgg)^{K}$ action on a $K$-isotypic space determines the  irreducible $G$-representation.
\item 
  The main result follows from comparing the $\cuu(\fgg)^{\tK_2}$ actions on $\tau$.
\end{itemize}
% 
%  and the well-known fact that $\cuu(\fgg)^{K_2}$ action determines the 
%  irreducible representation by Harish-Chandra.
\end{block}

\begin{block}{Examples}
      Following examples are constructed by Wallach and Zhu.
      %\cite{WallachZhu2004}.

      Let $p\leq s$, $q\leq r$ and $p+q=m\leq n$.  
      $G_1,G_2 = \Sp(2n,\bR)$, $G_1' = O(m)$, $G_2'=O(p,q)$.
      $\theta^{p,q}$ is the theta lifting from $O(p,q)$ to
      $\Sp(2n,\bR)$. Then 
      \[
      \displaystyle
      \Gamma_{p,q}(\theta^{0,m}(1)) \cong \theta^{p,q}(1^{\xi,\eta}).\]
      Here $ \xi \equiv r-q, \eta\equiv s-p \pmod{2}$,
      $1^{\xi,\eta}$ is the character of $O(p,q)$
      whose restriction to $O(p)\times O(q)$ is
      $\det^\xi\otimes \det^\eta$ and
      $\Gamma_{p,q}\theta^{0,m}(1)$ is certain submodule of
      \[
      \left(\Gamma_{\sp(2n),U(r)\times U(s)}^{\sp(2n),
          U(r+s)}\right)^{rs-(r-q)(s-p)}\theta^{0,m}(1)
      \]
    \end{block}
 
     \begin{block}{Future Directions}
         \begin{itemize}
           \item  Find  examples and similar results 
             for represetations other than characters.
           \item Find an explicit intertwining map of above isomorphism.
           \item Find applications of above result.
         \end{itemize}
     \end{block}

     
 %     \begin{block}{Bibliography}
%        \bibliographystyle{plain} 
%        \bibliography{bib/reppapers}{} 
%      \end{block}
   \end{column}
 \end{columns}
%\end{minipage}
\end{frame}

\end{document}
