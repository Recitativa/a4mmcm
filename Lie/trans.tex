\documentclass[12pt]{amsart}
\usepackage[margin=3cm]{geometry}

\usepackage{hyperref}

\usepackage{amssymb}
\usepackage{graphicx}
%\usepackage{amscd}
\usepackage{braket}
\usepackage{paralist}
\usepackage{eufrak}
%\usepackage{calrsfs}
%\usepackage[small,nohug,heads=littlevee]{diagrams}
%\diagramstyle[labelstyle=\scriptstyle]
%\usepackage{diagrams}
\usepackage[all,cmtip]{xy}
\usepackage{diagxy}
%\usepackage{pictexwd,dcpic}
%\usepackage{mathrsfs}


\newtheorem{lemma}{Lemma}
\newtheorem{thm}{Theorem}
\newtheorem{prop}{Proposition}
\newtheorem{cor}{Corollary}
\newenvironment{expl}{\it}{\color{black}\normalsize}
\DeclareMathAlphabet{\mathpzc}{OT1}{pzc}{m}{it}



\def\Ker{{\rm Ker}}
\def\Im{{\rm Im}}
\def\Hom{{\rm Hom}}
\def\End{{\rm End}}
\def\Mat{{\rm Mat}}
\def\Ind{{\rm Ind}}
\def\bR{{\mathbb{R}}}
\def\bN{{\mathbb{N}}}
\def\bZ{{\mathbb{Z}}}
\def\bC{{\mathbb{C}}}
\def\bQ{{\mathbb{Q}}}
\def\bB{{\mathbb{B}}}
\def\bA{{\mathbb{A}}}
\def\bs{{\mathbf{s}}}
\def\bd{{\mathbf{d}}}
\def\bT{{\mathbb{T}}}
\def\bt{{\mathbf{t}}}
\def\br{{\mathbf{r}}}
\def\vv{{\vec{v}}}
\def\vw{{\vec{w}}}
\def\vx{{\vec{x}}}
\def\vy{{\vec{y}}}
\def\v0{{\vec{0}}}
\def\ol{\overline}
\def\sspan{\rm{span}}
\def\sl2{{\mathfrak{sl}(2)}}
\def\slc{{\mathfrak{sl}(2,\bC)}}
\def\sp{{\mathfrak{sp}}}
\def\gg{{\mathfrak{g}}}
\def\kk{{\mathfrak{k}}}
\def\pp{{\mathfrak{p}}}
\def\qq{{\mathfrak{q}}}
\def\sP{\mathcal{P}}
\def\sH{\mathcal{H}}
\def\sU{\mathcal{U}}
\def\sC{\mathcal{C}}
\def\Stab{{\rm Stab}}
\def\ad{{\rm ad}}
\def\Ad{{\rm Ad}}
\def\id{{\rm id}}
\def\sgn{{\rm sgn}}
\def\gcd{{\rm gcd}}
\def\inn#1#2{\left\langle{#1},{#2}\right\rangle}
\def\abs#1{\left|{#1}\right|}
\def\norm#1{{\left\|{#1}\right\|}}
\def\Sp{{\rm Sp}}
\def\det{{\rm det}}

\def\tJ{{\widetilde{J}}}
\def\barZ{{\overline{Z}}}


\def\Ad{\mathrm{Ad}}
\def\agl{\mathfrak{gl}}
\def\asl{\mathfrak{sl}}
\def\aso{\mathfrak{so}}
\def\asp{\mathfrak{sp}}
\def\au{\mathfrak{u}}
\def\diag#1{\mathrm{diag}(#1)}
\def\lww{\mathcal{W}}
\def\lxx{\mathcal{X}}
\def\lyy{\mathcal{Y}}
\def\fhh{\mathfrak{h}}
\def\fnn{\mathfrak{n}}
\def\fuu{\mathfrak{u}}
\def\foo{\mathfrak{o}}
\def\fpp{\mathfrak{p}}
\def\fqq{\mathfrak{q}}
\def\ftt{\mathfrak{t}}
\def\fgg{\mathfrak{g}}
\def\fkk{\mathfrak{k}}
\def\chh{\mathcal{H}}
\def\crr{\mathcal{R}}
\def\cpp{\mathcal{P}}
\def\fmm{\mathfrak{m}}
\def\GL{\mathrm{GL}}
\def\SO{\mathrm{SO}}
\def\rad{\mathrm{rad}}
\def\SL{\mathrm{SL}}
\def\tr{\mathrm{tr}}
\def\Mat{\mathrm{Mat}}
\def\U{\mathrm{U}}
\def\tU{{\widetilde{U}}}
\def\ddt{\frac{d}{dt}}
\def\pz#1{\partial z_{#1}}

\def\cmm#1#2{\left[{#1},{#2}\right]}
\def\acmm#1#2{\left\{{#1},{#2}\right\}}


\def\real{{\rm Re\,}}
\def\imag{{\rm Im\,}}

\hypersetup{
    bookmarks=false,         % show bookmarks bar?
    unicode=false,          % non-Latin characters in Acrobat’s bookmarks
    pdftoolbar=true,        % show Acrobat’s toolbar?
    pdfmenubar=false,        % show Acrobat’s menu?
    pdffitwindow=false,      % page fit to window when opened
    pdftitle={Qualify Examination Answers - Algebra},    % title
    pdfauthor={Ma Jia Jun},     % author
    pdfsubject={Subject},   % subject of the document
    pdfcreator={Creator},   % creator of the document
    pdfproducer={Producer}, % producer of the document
    pdfkeywords={keywords}, % list of keywords
    pdfnewwindow=true,      % links in new window
    colorlinks=true,       % false: boxed links; true: colored links
    linkcolor=blue,          % color of internal links
    citecolor=green,        % color of links to bibliography
    filecolor=magenta,      % color of file links
    urlcolor=cyan           % color of external links
}


\newcounter{ssection}
\setcounter{ssection}{0}
\renewcommand{\subsection}{
  \addtocounter{ssection}{1}{\bf  \arabic{ssection}.\  }}

\title{transfer of $K$ type}

\begin{document}
\maketitle

\section{Heisenberg groups}
Consider $H = W\oplus Z$, $W = X\oplus Y$ 
with group law: 
\[
h(x,y,z)h(x',y',z') = h(x+x', y+y', z+z'+xy').
\]
This law is from 
\[
\begin{pmatrix}
1 & x & z\\
0 & I & y\\
0 & 0 & 1
\end{pmatrix}
\begin{pmatrix}
1 & x' & z'\\
0 & I & y'\\
0 & 0 & 1
\end{pmatrix}
=
\begin{pmatrix}
1 & x+x' & z+z'+ x y'\\
0 & I & y+y'\\
0 & 0 & 1
\end{pmatrix}.
\]

Let $w = (x,y)$. 

Define exponation map $\exp$ from the lie algebra $\fhh$ of $H$ to $H$. 
\[
\fhh = \lww\oplus F= \Set{\begin{pmatrix}
    0 & a & t\\
    0 & 0 & b\\
    0 & 0 & 0
  \end{pmatrix}}
\]
Note
\[
\begin{pmatrix}
0 & a & c\\
0 & 0 & b\\
0 & 0 & 0
\end{pmatrix}
\begin{pmatrix}
0 & a' & c'\\
0 & 0 & b'\\
0 & 0 & 1
\end{pmatrix}
=
\begin{pmatrix}
0 & 0 & a b'\\
0 & 0 & 0\\
0 & 0 & 0
\end{pmatrix}.
\]
Hence $X^3 = 0$ for any $X\in \fhh$. 
Then 
\[
\exp(X) = I + X + \frac{1}{2}X^2= h(a,b, c+ \frac{1}{2} ab) \in H. \]
and
\[ 
\log(h) = (x,y, z-\frac{1}{2}xy\in \fhh. 
\]

Using above formula, we have
\[
h^{-1} = \exp(-\log(h)) = h(-x,-y,-z+x y).
\]

\[
\begin{split}
hgh^{-1}g^{-1} &= h(x,y,z)h(x',y',z') h(-x,-y,-z+xy) h(-x',y',-z'+x'y')\\
&= h(x+x',y+y', z+z'+xy')h(-(x+x'),-(y+y'), -(z+z')+xy+x'y'+xy')\\
&= h(0,0,-(x+x')(y+y')+xy + x'y' +xy')
= h(0,0,xy'-x'y)\\ 
&= h(0,0,\inn{w}{w'}).
\end{split}
\]

\[
[X,Y] = (0,0,ab'-a'b) = (0,0,\inn{w}{w'}) \in \lww.
\]

\[
\begin{split}
\exp(w,c)\exp(w',c') &= h(w,c+\frac{1}{2}ab)h(w',c'+\frac{1}{2}a'b')
= h(w+w',c+c'+\frac{1}{2}(ab+a'b')+ab')\\
&= \exp(w+w', c+c'+\frac{1}{2}(ab'-b'a)) = 
\exp(w+w',c+c'+\frac{1}{2}\inn{w}{w'}).
\end{split}
\]
This give an alternateve construction of Heisenberg groups. 

The Haar measure on $H$ is $dzdxdy = dzdw$.


Choose a unitary character $\chi$ of $F$, extend to $S = Y\oplus Z$, by 
$Y$ act trivially.
Let 
\[
\chh_\chi = \Ind_S^H \chi 
=  \Set{f\colon H\to F|f(sh) = \chi(s)f(h)\, \forall s\in S\text{ and }
   \int_{S\backslash H} |f(h)|^2 d\dot h <\infty}
\]
The right translation give a $H$-module structure on $\chh_\chi$.  

We can consider $f\in \chh_\chi$ as a function on $X$ by 
\[
\phi(x) =f(h(x,0,0)) = f(\exp(x,0,0)).
\]
This give a isometry from $\chh_\chi$ to $L^2(X)$. 

Then 
\[
\begin{split}
\rho(h(x',y',z'))\phi(x) =& f(h(x,0,0)h(x',y',z')) =
f(h(x',y',z'+xy'))\\
 =& f(h(0,y', z'+xy')h(x+x',0,0))\\
=& \chi(z'+xy')\phi(x+x').
\end{split}
\]
This is the Schr\"odinger representation. Moreover
\[
\int_{S\backslash H} |f(h)|^2 d\dot h = \int_X|\phi(x)|^2 dx
\]


The oscillator representation, choose polarlization $W=X\oplus Y$. 
Let $P = P(Y) = \Stab_\Sp Y$, $M = P(Y)\cap P(X) \cong \GL(X)$,
$N=\cap_{y\in Y} \Stab_\Sp y$, i.e the kernel of map $P(Y) \to \GL(Y)$. 
There is a isomorphism from $N$ to $S^{2*}(X)$ given by 
\[
n \mapsto (\beta_n \colon v,v' \mapsto \inn{v}{n v'}
\]
For $\beta\in S^{2*}(X)$, $\beta(v,v')$ give a map $\gamma_\beta\colon X\to Y$
such that 
\[
\inn{v}{\gamma_\beta(v')} = \beta(v,v')
\]
The inverse map then defined by 
\[
\beta \mapsto n\colon \begin{cases}
v & \text{for } v\in Y\\
v+\gamma_\beta(v) & \text{for } v\in X
\end{cases}
\]
One can verify that $\beta_n^{-1} = -\beta_n$


Extand $\chi$ to $P\ltimes S$ by $P$ act trivially. 
Let $\chh_\chi = \Ind_{P\ltimes S}^{P\ltimes H}\chi$ be the unitary 
induced representation. $P\ltimes H$ act by left translation.
Again let 
\[
\phi(x) = f(h(x,0,0))
\]
This compactable with $\Ind_S^H \chi$, and has same Hermitian inner product:
\[
\int_{X}\abs{\phi(x)}^2 dx 
\]

Then for $m\in M=\GL(X)$
\[
f(h(x,0,0)g)=f(m m^{-1}h(x,0,0)m) = f(h(m^{-1}x,0,0))  = \phi(m^{-1}x)
\]
Since 
\[
|\phi(m^{-1}x)|^2 dx = |\phi(x)|^2 d mx = |\phi(x)|^2 \abs{\det_{\GL(X)} m} dx,
\]
define 
\[
\omega(m)\phi(x) = \abs{\det_{\GL(X)} m}^{-1/2}f(mx).
\]

\[
\begin{split}
f(h(x,0,0)n) =& f(n n^{-1}h(x,0,0)n) = f(e(n^{-1}(x,0),0)) \\
=& f(e(x,\gamma_{\beta_{n^{-1}}}(x),0)) = 
f(h(x, \gamma_{\beta_{n^{-1}}}(x), \frac{1}{2}\inn{x}{\gamma_{\beta_{n^{-1}}}(x)}))\\
=& f(h(x, \gamma_{\beta_{n^{-1}}}(x),  -\frac{1}{2}\beta_n(x,x))) \\
=& \chi(-\frac{1}{2}\beta_n(x,x))f(h(x,0,0))
\end{split}
\]
Hence 
\[
\omega(n)\phi(x) = \chi(-\frac{1}{2}\beta_n(x,x))\phi(x)
\]

The special element is 
\[
\sigma = \begin{pmatrix}
0 & 1\\
1 & 0
\end{pmatrix} \in \Sp(W)
\]
It exchange the role of $x, y$. It is reasonable to believe the intertwining
map is a Fourier transform. 


\section{Fock model}
Let $J\in \Sp(W)\cap \sp(W)$ such that $J^2 = -\id$.
Then $J$ given a complex structure on $W$.
Then 
\[
B_J(w,w') = \inn{Jw}{w'}
\]
is a (non-degenerate) symmetric bilinear form.

\[
H_J(w,w') = \inn{Jw}{w'} + i\inn{w}{w'}
\]
become a Hermitian bilinear form(inner product) on $W$ where the 
complex structure on $W$ defined by $J$, i.e. $i$ act on $W$ by $J$. 
In fact,
\[
H_J(w',w) = \inn{Jw'}{w} + i \inn{w'}{w} = -\inn{w'}{Jw} + i\inn{w'}{w}
= \inn{Jw}{w'} - i \inn{w}{w'}= \overline{H_J(w,w')}
\]
and
\[
H_J(Jw,w') = \inn{JJw}{w'} + i\inn{Jw}{w'} = i\inn{Jw}{w'} - \inn{w}{w'} 
= i(\inn{Jw}{w'} + i\inn{w}{w'} = i H_J(w,w').
\]

Complexify $W$ to $W_\bC$. Extend $\inn{}{}$ to $W_\bC$.  
 
View $H = W\oplus Z$ from the exponatial map of Lie algebra $\fhh$ of $H$. 
Then the group law becomes
\[
\exp(w, t) \exp(w',t) = \exp(w+w', t+t'+\frac{1}{2}\inn{w}{w'}).
\]
Similary define $H_\bC$. 
Let $V^\pm$ be the $\pm i$ eigen space of $J$ on $W_\bC$. 
Then 
\[
V^\pm = \Set{w \mp iJw | w\in W },
\] 
since $J(w - iJw) = iw+Jw = i(w-iJw)$ and
$J(w+iJw) = -iw + Jw = -i(w+iJw)$.


Then consider the subgroup $P = V^-\oplus Z  < H_\bC$.
Consider the following digram
\[
\bfig
\square/^{(}->`->>`->>`^{(}->/[
H`H_\bC`Z\backslash H`P\backslash H_\bC;{}`{}`{}`{}]
\efig
\]

Hence $W = Z\backslash H$ have a complex structure given by $G_\bC$. 
In fact, by
\[
iw \equiv Jw +i(w+iJw) \equiv Jw \pmod{P}.
\]
Hence the complex structure given by $J$ is same as the complex structure given
by above commutative diagram. 


Fix a character $\chi(t)=e^{2\pi i\lambda t}$ of $\bR$ for some $\lambda\in \bR$. 
Extend it into a character of $\bC$ then $P$ in a obvious way. 
Define
\[
\chh_\chi = \Set{f\colon G_\bC \to \bC|\begin{array}{rl}
    (a) & f \text{ is holomorphic};\\
    (b) & f(pg) = \chi(p)f(g) \quad \forall p\in P, g\in G_\bC;\\
    (c) & \int_{Z\backslash G} \abs{f(g)}^2 dg < \infty
  \end{array}
}
\]
Inner product is given by 
\[
(f,f') = \int_{Z\backslash G} f(g)\overline{f'(g)} dg.
\]
$G$ act on $\chh_\chi$ by right translation.

For $X \in W$.
\[
\begin{split}
f(\exp w+tiX) =& f(\exp w+ tJX + ti(X+iJX))\\
=& f(e^{-\frac{1}{2}\inn{ti(X+iJX)}{w+tJX}}\exp ti(X+iJX) \exp(w+tJX))  \\
=& e^{-2\pi i \lambda\frac{1}{2}\inn{ti(X+iJX)}{w+tJX}}f(\exp (w+tJX))\\
= & e^{2\pi \lambda \frac{1}{2} t \inn{X+iJX}{w} + t^2 \inn{X+iJX}{JX}}f(\exp(w+tJW))
\end{split}
\]
Then 
\[
\ddt f(\exp w+tiX) =  \ddt f(\exp w+tJX) + 2\pi \lambda \frac{1}{2} 
\inn{X+iJX}{w} f(\exp(w+tJX))
\]
Then the holomorphic condition is,
\[
\lim_{t\to 0} \frac{f(\exp(w+tX)) -f(\exp(w))}{t} = 
\lim_{t\to 0} \frac{f(\exp(w+tiX))- f(\exp(w))}{ti} 
\]
 i.e.
\[
i \ddt f(\exp w+tX) = \ddt f(\exp w+tJX) + \pi \lambda
\inn{X+iJX}{w} f(\exp(w+tJX)).
\]

Choose $J$ such that $\inn{w}{Jw} \geq 0$, this $J$ called ``positive''. 


For $f\in \chh_\chi$, define
\[
\phi(w) = e^{\pi \lambda \inn{w}{Jw}/2} f(\exp w) 
\] 

We check the holomorphic condition becomes 
\[
\theta_{JX} \phi = i \theta_{X} \phi.
\]
where 
\[
(\theta_X \phi)(w) = \frac{d}{dt}\phi(w+tX) \quad \for X\in W.
\]

In fact, 
\[
\begin{split}
\theta_X \phi(w) =& \ddt e^{\pi \lambda \inn{w+tX}{J(w+tX)}/2} f(\exp(w+tX))\\
= &  e^{\pi \lambda \inn{w}{Jw}}(\ddt f(\exp(w+tX)) 
+ \frac{1}{2}\pi\lambda (\inn{X}{Jw}+\inn{w}{JX}) f(\exp(w+tX))) \\
= &  e^{\pi \lambda H_J(w,w)}(\ddt f(\exp(w+tX)) + \pi\lambda \inn{X}{Jw} f(\exp(w+tX))) \\
\end{split}
\] 
Then 
\[
\begin{split}
&\theta_{JX}\phi - i\theta_{X}\phi \\
= & \ddt f(\exp(w+tJX)) + \pi\lambda \inn{JX}{Jw} f(\exp(w+tJX))\\
&-i \left[ \ddt f(\exp(w+tX)) + \pi\lambda \inn{X}{Jw} f(\exp(w+tX))\right]\\
=& \ddt f(\exp(w+tJX)  + \pi \lambda \inn{X+iJX}{w} f(\exp(w+tJX)
- i \ddt f(\exp(w+tX))= 0 
\end{split}
\]

Finiteness condition becomes:
\[
\int_W e^{-\pi \lambda \inn{w}{Jw}}\abs{\phi(w)}^2dw < \infty.
\]

Then
\[
\begin{split}
(\rho(\exp(w,t))\phi)(w') =& e^{\pi \lambda \inn{w'}{Jw'}/2} f(\exp(w') \exp(w,t))\\
=& e^{\pi \lambda \inn{w'}{Jw'}/2} f(\exp(w'+w,t+ \frac{1}{2}\inn{w'}{w}))\\
=& e^{\pi \lambda \inn{w'}{Jw'}/2} \chi(t+ \frac{1}{2}\inn{w'}{w}) f(\exp(w'+w))\\
=& \chi(t) e^{\frac{1}{2}\pi \lambda (\inn{w'}{Jw'}+2i \inn{w'}{w})}
 f(\exp(w+w')) \\
=& \chi(t) e^{\frac{1}{2}\pi \lambda (\inn{w'}{Jw'}+2i \inn{w'}{w})} 
e^{-\pi \lambda \inn{w'+w}{J(w'+w)}/2} \phi(w+w')\\
=& \chi(t) e^{-\pi \lambda (\inn{w}{Jw}/2+ \inn{w}{Jw'}+i\inn{w}{w'}))} 
 \phi(w+w') 
\end{split}
\]


So the infinitesimal representaition is 
\[
\rho(X) \phi = \theta_X(\phi) -\pi\lambda H_X \phi
\]
Where $H_X(w') =\inn{X}{Jw'} + i\inn{X}{w'} = -H_J(w',X)$ is a linear function.

Take $Y = \frac{1}{2}(X-iJX)$. Then $X = Y + \overline{Y}$, 
and 
\begin{align*}
\rho(Y) \phi &= \theta_X \phi \\
\rho(\overline{Y}) \phi &= -\pi \lambda H_X \phi
\end{align*}

The space 
\[
\Set{\phi \in \chh_\chi |\rho(V^+) \phi = 0}
\]
is one dimension, this is since $\phi$ is holomorphic, 
all derivative equal to zero mean $\phi$ is constant. 

Hence $\rho$ is irreducible. 

Choose complex basis of $W$ $\Set{P_1, \cdots, P_n}$
respect to $J$ such that
$H_J(P_k, P_l) = \delta_{kl}$ and let $Q_j = J P_j$. 
Then $\Set{z, P_j, Q_j}$ form a basis of $\fhh$. 
Let $z_k(P_l) = H_J(P_l, P_k) =\delta_{kl}$ be complex linear function on 
$W$. Then $z_j$ generate $\chh_\chi$. 

\[
\rho(P_j-iQ_j) = 2 \theta_{P_j} =  2\theta_{z_j}
\]
\[
\rho(P_j+iQ_j) = -2\pi\lambda (-H_J(\cdot,P_j)) 
 = 2\pi \lambda z_j
\]

Above is from Pierre Cartier 

\section{Fock model II}
Let $J\in \Sp(W)\cap \sp(W)$ such that $J^2 = -\id$.
Then $J$ given a complex structure on $W$.
Then 
\[
B_J(w,w') = \inn{Jw}{w'}
\]
is a (non-degenerate) symmetric bilinear form.

\[
H_J(w,w') = \inn{Jw}{w'} + i\inn{w}{w'}
\]
become a Hermitian bilinear form(inner product) on $W$ where the 
complex structure on $W$ defined by $J$, i.e. $i$ act on $W$ by $J$. 
In fact,
\[
H_J(w',w) = \inn{Jw'}{w} + i \inn{w'}{w} = -\inn{w'}{Jw} + i\inn{w'}{w}
= \inn{Jw}{w'} - i \inn{w}{w'}= \overline{H_J(w,w')}
\]
and
\[
H_J(Jw,w') = \inn{JJw}{w'} + i\inn{Jw}{w'} = i\inn{Jw}{w'} - \inn{w}{w'} 
= i(\inn{Jw}{w'} + i\inn{w}{w'} = i H_J(w,w').
\]

Complexify $W$ to $W_\bC$. Extend $\inn{}{}$ to $W_\bC$.  
 
View $H = W\oplus Z$ from the exponatial map of Lie algebra $\fhh$ of $H$. 
Then the group law becomes
\[
\exp(w, t) \exp(w',t) = \exp(w+w', t+t'+\frac{1}{2}\inn{w}{w'}).
\]
Similary define $H_\bC$. 
Let $V^\pm$ be the $\pm i$ eigen space of $J$ on $W_\bC$. 
Then 
\[
V^\pm = \Set{w \mp iJw | w\in W },
\] 
since $J(w - iJw) = iw+Jw = i(w-iJw)$ and
$J(w+iJw) = -iw + Jw = -i(w+iJw)$.


Then consider the subgroup $P = V^-\oplus Z  < H_\bC$.
Consider the following digram
\[
\bfig
\square/^{(}->`->>`->>`^{(}->/[
H`H_\bC` H/Z`H_\bC/P;{}`{}`{}`{}]
\efig
\]

Hence $W = H/Z$ have a complex structure given by $H_\bC$. 
In fact, by
\[
iw \equiv Jw +i(w+iJw) \equiv Jw \pmod{P}.
\]
Hence the complex structure given by $J$ is same as the complex structure given
by above commutative diagram. 


Fix a character $\chi(t)=e^{2\pi i\lambda t}$ of $\bR$ for some $\lambda\in \bR$. 
Extend it into a character of $\bC$ then $P$ in a obvious way. 
Define
\[
\chh_\chi = \Set{f\colon G_\bC \to \bC|\begin{array}{rl}
    (a) & f \text{ is holomorphic};\\
    (b) & f(gp) = \chi(p^{-1})f(g) \quad \forall p\in P, g\in G_\bC;\\
    (c) & \int_{G/Z} \abs{f(g)}^2 dg < \infty
  \end{array}
}
\]
Inner product is given by 
\[
(f,f') = \int_{G/Z} f(g)\overline{f'(g)} dg.
\]
$G$ act on $\chh_\chi$ by left translation.

For $X \in W$.
\[
\begin{split}
f(\exp w+tiX) =& f(\exp w+ tJX + ti(X+iJX))\\
=& f(e^{\frac{1}{2}\inn{ti(X+iJX)}{w+tJX}} \exp(w+tJX)\exp ti(X+iJX))  \\
=& e^{2\pi i \lambda\frac{1}{2}\inn{ti(X+iJX)}{w+tJX}}f(\exp (w+tJX))\\
= & e^{-2\pi \lambda \frac{1}{2} t \inn{X+iJX}{w} - t^2 \inn{X+iJX}{JX}}f(\exp(w+tJW))
\end{split}
\]
Then 
\[
\ddt f(\exp w+tiX) =  \ddt f(\exp w+tJX) - \pi \lambda  
\inn{X+iJX}{w} f(\exp w)
\]
Then the holomorphic condition is,
\[
\lim_{t\to 0} \frac{f(\exp(w+tX)) -f(\exp(w))}{t} = 
\lim_{t\to 0} \frac{f(\exp(w+tiX))- f(\exp(w))}{ti} 
\]
 i.e.
\[
i \ddt f(\exp w+tX) = \ddt f(\exp w+tJX) - \pi \lambda
\inn{X+iJX}{w} f(\exp w).
\]

Choose $J$ such that $\inn{Jw}{w}=H_J(w,w) \geq 0$, this $J$ called ``positive''. 


For $f\in \chh_\chi$, define
\[
\phi(w) = e^{\pi \lambda \inn{Jw}{w}/2} f(\exp w) 
\] 

We check the holomorphic condition becomes 
\[
\theta_{JX} \phi = i \theta_{X} \phi.
\]
where 
\[
(\theta_X \phi)(w) = \frac{d}{dt}\phi(w+tX) \quad \for X\in W.
\]

In fact, 
\[
\begin{split}
\theta_X \phi(w) =& \ddt e^{\pi \lambda \inn{J(w+tX)}{w+tX}/2} f(\exp(w+tX))\\
= &  e^{\pi \lambda \inn{Jw}{w}}(\ddt f(\exp(w+tX)) 
+ \frac{1}{2}\pi\lambda (\inn{JX}{w}+\inn{Jw}{X}) f(\exp w )) \\
= &  e^{\pi \lambda H_J(w,w)}(\ddt f(\exp(w+tX)) + \pi\lambda \inn{JX}{w} 
f(\exp w)) \\
\end{split}
\] 
Then 
\[
\begin{split}
& e^{-\pi \lambda H_J(w,w)}\cdot (\theta_{JX}\phi - i\theta_{X}\phi) \\
= & \ddt f(\exp(w+tJX)) + \pi\lambda \inn{JJX}{w} f(\exp w)\\
&-i \left[ \ddt f(\exp(w+tX)) + \pi\lambda \inn{JX}{w} f(\exp w)\right]\\
=& \ddt f(\exp(w+tJX)  - \pi \lambda \inn{X+iJX}{w} f(\exp w)
- i \ddt f(\exp(w+tX))= 0 
\end{split}
\]

Finiteness condition becomes:
\[
\int_W e^{-\pi \lambda \inn{Jw}{w}}\abs{\phi(w)}^2dw < \infty.
\]

Then
\[
\begin{split}
(\rho(\exp(w,t))\phi)(w') 
=& e^{\pi \lambda \inn{Jw'}{w'}/2} f(\exp(w,t)^{-1}\exp(w'))\\
=& e^{\pi \lambda \inn{Jw'}{w'}/2} f(\exp(-w,-t)\exp(w'))\\
=& e^{\pi \lambda \inn{Jw'}{w'}/2} f(\exp(w'-w,-t+\frac{1}{2}\inn{w'}{w}))\\
=& e^{\pi \lambda \inn{Jw'}{w'}/2} \chi(-t+ \frac{1}{2}\inn{w'}{w}) f(\exp(w'-w))\\
=& \chi(t) e^{\frac{1}{2}\pi \lambda (\inn{Jw'}{w'}+2i \inn{w'}{w})}
 f(\exp(w'-w)) \\
=& \chi(-t) e^{\frac{1}{2}\pi \lambda (\inn{Jw'}{w'}+2i \inn{w'}{w})} 
e^{-\pi \lambda \inn{J(w'-w)}{w'-w}/2} \phi(w'-w)\\
=& \chi(-t) e^{-\pi \lambda \inn{Jw}{w}/2+ \pi\lambda (\inn{Jw'}{w}+i\inn{w'}{w})} 
 \phi(w'-w) \\
=& \chi(-t) e^{-\pi\lambda H_J(w,w)/2} e^{\pi\lambda H_J(w',w)} \phi(w'-w)
\end{split}
\]


So the infinitesimal representaition is 
\[
\rho(X) \phi = -\theta_X(\phi) +\pi\lambda H_X \phi
\]
Where $H_X(w') = H_J(w',X)$ is a (complex-)linear function on $W$.
Note that $H_{JX} = -i H_X$ and $\theta_{JX} = i\theta_X$.

Take $Y = \frac{1}{2}(X-iJX)$. Then $X = Y + \overline{Y}$, 
and 
\[
\begin{split}
\rho(Y) \phi =& \frac{1}{2}[(-\theta_X + \pi \lambda H_X) 
-i(-\theta_{JX}+\pi\lambda H_{JX})]\\
=& -\theta_X \phi 
\end{split}
\]
\[
\begin{split}
\rho(\overline{Y}) \phi =& \frac{1}{2}[(-\theta_X + \pi \lambda H_X) 
+i(-\theta_{JX}+\pi\lambda H_{JX})]\\
=& \pi\lambda H_X \\
\end{split}
\]

The space 
\[
\Set{\phi \in \chh_\chi |\rho(V^+) \phi = 0}
\]
is one dimension, this is since $\phi$ is holomorphic, 
all derivative equal to zero mean $\phi$ is constant. 

Hence $\rho$ is irreducible. 

Choose complex basis of $W$ $\Set{P_1, \cdots, P_n}$
respect to $J$ such that
$H_J(P_k, P_l) = \delta_{kl}$ and let $Q_j = J P_j$. 
Then $\Set{z, P_j, Q_j}$ form a basis of $\fhh$. 
Let $z_k(P_l) = H_J(P_l, P_k) =\delta_{kl}$ be complex linear function on 
$W$. Then $z_j$ generate $\chh_\chi$. 

\[
\rho(P_j-iQ_j) = -2 \theta_{P_j} =  -2\theta_{z_j}
\]
\[
\rho(P_j+iQ_j) = 2\pi\lambda (H_J(\cdot,P_j)) 
 = 2\pi \lambda z_j
\]

Above ALMOST from Pierre Cartier 


\section{dual pair}

Consider $U$ a complex vector space , and the symmetric algebra $S(U)$.

Define $M_u$ be the left mutiplication on $S(U)$ for any $u$, 
and $D_{u^*}$ be the differential operator act on $S(U)$ by
\[
D_{u^*}(xy) = \mu^*(x) y + xD_{u^*}(y) \quad \forall x\in U, y\in S(U)
\]
Clearly
\[
[D_{u^*}, M_u] = \mu^*(u) \id \quad \forall u\in U, u^*\in U^*.
\]

Define 
\[
\tU = U\oplus U^*, 
\]
and a symplectic form 
\[
\inn{u_1+u_1^*}{u_2+u_2^*} = u_1^*(u_2) - u_2^*(u_1)
\]
By above we have a canonical embedding 
\[
i\colon \tU \hookrightarrow \End(S(U))   
\]
call the image $i(x)$ of $x$ again $x$. 
Then 
\[
[x,y] = \inn{x}{y} \id \quad \forall x,y \in \tU.
\]

Consider 
$\End^\circ$ be the algebra generated by $i(\tU)$. It has a natrue filtration.
Then $S^2(\tU) \hookrightarrow \End^{\circ (2)}$ by $\{a,b\}$.

Since $[,c]$ is a representation on $\End(S(U))$, 
\[
\cmm{\acmm{a}{b}}{c}
= \acmm{\cmm{a}{c}}{b} + \acmm{a}{\cmm{b}{c}}
= 2b\inn{a}{c} + 2a \inn{b}{c}
\]
Hence
\[
\inn{\cmm{\acmm{a}{b}}{c}}{d}
= 2\inn{b}{d}\inn{a}{c} + 2\inn{a}{d}\inn{b}{c}
= -2\inn{c}{b}\inn{a}{d} -2\inn{c}{a}\inn{b}{d}
=-\inn{c}{\cmm{\acmm{a}{b}}{d}},
\]
i,e, $S^2(\tU) \subset \sp(\tU)$. Since both side has dimension $n(2n+1)$,
they are equal. 

\section{computation}
Take 
\[
J = \begin{pmatrix} 0 & -1\\1 & 0
\end{pmatrix}
\]
Define 
\[
\inn{w}{v} = v^T J w = ad - cb \quad \forall w = (a,b), v = (c,d) \in \bR^n
\]

\[
\tJ = -J = J^T = 
\begin{pmatrix}
  0 & 1 \\ -1 & 0
\end{pmatrix}
\]
satisfy 
\[
\inn{\tJ w}{w} = w^T w \geq 0.
\]

Choose a basis of $W$ respect to $H_\tJ$.
Let $e_1, \cdots, e_{2n}$ be the standard basis of $\bR^2$.
Noting that 
\[
H_\tJ(e_k,e_l) = \inn{\tJ e_k}{e_l}+i\inn{e_k}{e_l}
= e_l^T e_k =\delta_{kl} \quad \forall k,l \leq n
\]
Hence we set $P_k = e_k$ and $Q_k = \tJ P_k = -e_{n+k}$
Then $z_k = e_k - i e_{n+k}$ and $\overline{z_k} = e_k +i e_{n+k}$
by $z_k(P_l) = \delta_{kl}$ and $z_k(\tJ P_l)= i\delta_{kl}$.

We can compute $K$ which is the isometric group of $H_\tJ$.
In fact compute the Lie algebra $\fkk$ of $K$. 
\[
K = \set{k  \in \GL(2n,\bR)|k^tk = I, k^TJk = J }
\]
Then the lie algebra is,
\[
\fkk = \Set{X\in \agl(2n,\bR)|X^T + X  = 0, X^TJ + JX = 0}
\]
If
\[
X = \begin{pmatrix} A & B \\ C & D \\
\end{pmatrix}
\]

Then 
\[
\begin{pmatrix}
A^T & C^T \\
B^T & D^T 
\end{pmatrix} =
\begin{pmatrix}
-A & -B \\
-C & -D 
\end{pmatrix}
\]
and 
\[
\begin{pmatrix}
C^T & -A^T \\
D^T & -B^T 
\end{pmatrix} =
\begin{pmatrix}
C & D \\
-A & -B 
\end{pmatrix}.
\]
Hence $A^T = -A$, $D = -A^T= A$, $B = B^T$, $C = -B^T = -B$,
i.e. 
\[
X = \begin{pmatrix}
A & B\\
-B & A
\end{pmatrix}
\]
$A$ antisymmetric and $B$ symmetric.
Note that we have a natrue isomorphism of lie algebra
$\fkk \cong U(n)$ by 
\[
X = \begin{pmatrix}
A & B\\
-B & A
\end{pmatrix}
\mapsto
A + iB.
\]
By the isomorphism $\bR^2 \simeq \bC$ with $(a,b) \to a-ib$, 
we have $H_\tJ((a,b),(c,d)) = c^ta + b^td +i(d^t a- b^t c)
= (c+id)^t(a-ib)$, hence identify $H_\tJ$ with the usual Hermitian inner
product on $\bC$. This is compatible with above lie algebra isomorphism.
In fact
\[
X\begin{pmatrix}a\\b
\end{pmatrix}
= \begin{pmatrix} Aa+Bb\\
-Ba+ Ab
\end{pmatrix} 
\]
and compatible with
\[
(A+iB)(a-ib) = Aa+Bb -i(-Ba+Bb)
\]

We can use 
\begin{align*}
Z_k  &= \frac{1}{2}(P_k-iQ_k) = \frac{1}{2}(e_k + i e_{n+k})\\
\barZ_k &= \frac{1}{2}(P_k+iQ_k) = \frac{1}{2}(e_k - i e_{n+k})
\end{align*}
be the basis of $W_{\bC}$. Where $Z_k$ span the $i$-eigen space of $\tJ$.
Then 
\begin{align*}
\rho(Z_k) &= -\pz{k} \\
\rho(\barZ_k) & = \pi\lambda z_k
\end{align*}

Now compute the action of $\fhh$ on the oscillator representation. 
In fact 
\[
\begin{split}
e_k &= \frac{1}{2}[(P_k-iQ_k) + (P_k + iQ_k)] =
Z_k + \barZ_k =  -\pz{j}  + \pi \lambda z_k\\
e_{n+k} & = i\frac{1}{2}[-(P_k-iQ_k) + (P_k + iQ_k)] = i(-Z_k+\barZ_k)
= i\pz{k} +i \pi \lambda z_k
\end{split}
\]
Now the matrix of change of basis is: $Z_k, \barZ_j \mapsto e_k, e_{n+j}$
by 
\begin{equation}\label{eq:tZtoe}
\frac{1}{2}\begin{pmatrix}
1 & 1\\
i & -i 
\end{pmatrix}
\end{equation}
And the inverse is $e_k, e_{n+j} \mapsto Z_k, \barZ_j$
\[
\begin{pmatrix}
1 & -i\\
1 & i
\end{pmatrix}
\]

now consider the symplectic lie algebra $\sp \simeq S^2(W_\bC)$. 
Note that 
\[
\cmm{\pz{k}}{z_j} = \delta_{kj}
\]
\[
\cmm{Z_k}{\barZ_j} = \cmm{-\pz{k}}{\pi \lambda z_j} = -\pi\lambda \delta_{kj}
\]
\[
\cmm{Z_k}{Z_j} = \cmm{\barZ_k}{\barZ_j} =0.
\]

For all $0\leq k,j \leq n$
\[
\cmm{e_k}{e_{j}}= \cmm{Z_k+\barZ_k}{Z_j+\barZ_j} = \pi\lambda (\delta_{kj}-\delta_{jk}) 
=0 
\]
\[
\cmm{e_{n+k}}{e_{n+j}} = \cmm{i(-Z_k+\barZ_k)}{i(-Z_j+\barZ_j)} = 0
\]
\[
\cmm{e_k}{e_{n+j}} = \cmm{Z_k + \barZ_k}{i(-Z_j+\barZ_j)} = -2\pi\lambda i \delta_{kj} 
\]
Hence for $w,v\in W$, we have 
\[
\cmm{w}{v}= - 2\pi \lambda i \inn{w}{v} 
\]

Now consider how $\sp = S^2(W_\bC)$ act on $W_\bC$ or $W$ by adjoint.
\begin{align*}
\cmm{\acmm{Z_k}{Z_j}}{Z_l} &= \acmm{\cmm{Z_k}{Z_l}}{Z_j} 
+ \acmm{Z_k}{\cmm{Z_j}{Z_l}} = 0\\
\cmm{\acmm{Z_k}{Z_j}}{\barZ_l} &= \acmm{\cmm{Z_k}{\barZ_l}}{Z_j} 
+ \acmm{Z_k}{\cmm{Z_j}{\barZ_l}} = -2\pi \lambda(\delta_{kl}Z_j+\delta_{jl}Z_k)
\end{align*}

\begin{align*}
 \cmm{\acmm{Z_k}{\barZ_j}}{Z_l} &= \acmm{\cmm{Z_k}{Z_l}}{\barZ_j} 
+ \acmm{Z_k}{\cmm{\barZ_j}{Z_l}} = 2\pi \lambda \delta_{jl} Z_k\\
 \cmm{\acmm{Z_k}{\barZ_j}}{\barZ_l} &= \acmm{\cmm{Z_k}{\barZ_l}}{\barZ_j} 
+ \acmm{Z_k}{\cmm{\barZ_j}{\barZ_l}} = -2\pi \lambda \delta_{jl} \barZ_j
\end{align*}

\begin{align*}
 \cmm{\acmm{\barZ_k}{\barZ_j}}{Z_l} &= \acmm{\cmm{\barZ_k}{Z_l}}{\barZ_j} 
+ \acmm{\barZ_k}{\cmm{\barZ_j}{Z_l}} 
= 2\pi \lambda (\delta_{kl} \barZ_j+ \delta_{jl}\barZ_k)\\
 \cmm{\acmm{\barZ_k}{\barZ_j}}{\barZ_l} 
&= \acmm{\cmm{\barZ_k}{\barZ_l}}{\barZ_j} 
+ \acmm{\barZ_k}{\cmm{\barZ_j}{\barZ_l}} 
= 0
\end{align*}

In the basis $Z_k, \barZ_k$, (assume $k<=j$)
Hence 
\begin{align}\label{eq:matZ}
\acmm{Z_k}{Z_j} \mapsto& -2\pi \lambda (E_{j,n+k}+E_{k,n+j}) = 2\pi\lambda
\begin{pmatrix}
 & & & -1\\
 & &-1 & \\
 & &  & \\
 & &  &  
\end{pmatrix}
\\
\acmm{Z_k}{\barZ_j} \mapsto& 2\pi\lambda (E_{k,j}-E_{n+j,n+k}) 
= 2\pi\lambda
\begin{pmatrix}
 &1 &  & \\
 & &  & \\
 & &  & \\
 & &-1  &  
\end{pmatrix}
\\
\acmm{\barZ_k}{\barZ_j} \mapsto & 2\pi\lambda (E_{n+j,k} + E_{n+k,j}) 
= 2\pi\lambda
\begin{pmatrix}
 & &  & \\
 & &  & \\
 & 1&  & \\
 1& &  &  
\end{pmatrix}
\end{align}

If $k<j$,
\begin{align*}
\acmm{e_k}{e_j} =& \acmm{Z_k+\barZ_k}{Z_j+\barZ_j} = 
\acmm{Z_k}{Z_j} +\acmm{\barZ_k}{Z_j} +
\acmm{Z_k}{\barZ_j} + \acmm{\barZ_k}{\barZ_j}\\
=& 2 \pi \lambda[-(E_{j,n+k}+E_{k,n+j})+(E_{n+j,k} + E_{n+k,j}) +
(E_{k,j}+E_{j,k}-E_{n+k,n+j}-E_{n+j,n+k})] \\
=& 2\pi\lambda
\begin{pmatrix}
  &1 &  &-1 \\
1 &  & -1 & \\
  & 1&  &-1 \\
1 &  &-1  &  
\end{pmatrix}\\
\acmm{e_k}{e_{n+j}} =& \acmm{Z_k+\barZ_k}{-iZ_j+i\barZ_j} = 
-i\acmm{Z_k}{Z_j} -i\acmm{\barZ_k}{Z_j} +
i\acmm{Z_k}{\barZ_j} + i\acmm{\barZ_k}{\barZ_j}\\
=& 2 \pi \lambda[i(E_{j,n+k}+E_{k,n+j})+i(E_{n+j,k} + E_{n+k,j}) +
(-iE_{j,k}+iE_{n+k,n+j} +iE_{k,j}-iE_{n+j,n+k})] \\
=& 2\pi\lambda
\begin{pmatrix}
  & i&   &i \\
-i&  & i & \\
  & i&   &-i \\
i &  & i &  
\end{pmatrix}\\
\acmm{e_{n+k}}{e_{n+j}} =& \acmm{-iZ_k+i\barZ_k}{-iZ_j+i\barZ_j} 
=-\acmm{Z_k-\barZ_k}{Z_j-\barZ_j} 
=& 
-\acmm{Z_k}{Z_j} +\acmm{\barZ_k}{Z_j} +
\acmm{Z_k}{\barZ_j} - \acmm{\barZ_k}{\barZ_j}\\
=& 2 \pi \lambda[(E_{j,n+k}+E_{k,n+j})-(E_{n+j,k} + E_{n+k,j}) +
(E_{k,j}+E_{j,k}-E_{n+k,n+j}-E_{n+j,n+k})] \\
=& 2\pi\lambda
\begin{pmatrix}
  &1 &  &1 \\
1 &  & 1 & \\
  & -1&  &-1 \\
-1&  &-1  &  
\end{pmatrix}\\
\end{align*}

\section{$\chh(U(n))$}
$\chh(K)$ means all $K$-finite elements in the represnetaion of $K\times K$ on 
$C(K)$.  Now assume $K=U(n)$. We claim that 
\[
\crr(\GL(n, \bC)) \simeq \chh(K),
\]
where $\crr(\GL(n,\bC))$ is the coordinate ring of affine variety $\GL(n,\bC)$,
or the subring of functions on $\GL(n,\bC)$ generated by entries and $\det$. 

Consider the restriction map:
\[
\pi:\crr(\GL(n,\bC)) \to C(K).
\]
It is injective.
Not so ``rigid'': since element in $\crr(\GL(n,\bC))$ are holomorphic functions
on $\GL(n,\bC)$ hence determined by restrict to $K$, a infinite compact set. 

Suppose that $f(g) = \sum p_k(g) \det^k(g) \in \crr(\GL(n,\bC))$, where
$p_k$ are polynomials on $M_n(\bC)$. Since it is a 
finite sum $f\det^k$ is a polynomial on $M_n(\bC)$ 
for sufficient large $k\in \bZ$.
Since  $\det^k(g)\neq 0$ on $\GL(n,\bC)$,
$f(g) = 0$ for all $g\in \GL(n,\bC)$ eqivalent to $f(g)\det^k(g) = 0$
for all $g\in \GL(n,\bC)$. 
Since $\GL(n,\bC)$ is (Zariski) dense in $M_n(\bC)$, above eqivalent to 
$f\det^k=0$ as polynomial on $M_n(\bC)$. 
Assume $\pi(f)=0$, i.e. $f(g) = 0$ for all $g\in U(n)$. Hence 
$f(g)\det^k(g)=0$ for all $g\in U(n)$. 
Now decomposite $f$ in to homogenous component, i.e. $f=\sum p_l$. 
Then $0=f(au)=\sum p_l(au) =\sum a^l p_l(u)$ for all $a\in U(1)$. Hence
$p_l(u) =0$. Since any $g\in \GL(n,\bC)$ can written as $\lambda u$ for 
$\lambda \in \bC$. We have $f(g) = f(\lambda u) = \sum \lambda^l p_l(u) = 0$. 
Hence $f=0$ in $\crr(GL(n,\bC)$.

It is obvious that $\Im \pi \subset \chh(K)$
 since the $K$ action on $\crr(\GL(n,\bC))$ is finite. 
For the subjectivity, we only have to check the $\Im \pi$ is dense in $C(K)$. 
Then by Peter-Weyl theorem, $\Im \pi=\chh(K)$. 
The image is dense by Stone-Weierstrass theorem. 
Fist $K$ is a compact Hausdorff space, 
We check:
\begin{enumerate}[(1)]
\item obviously, $\Im \pi$ is a subalgebra of $C(K)$.
\item obviously, it contains non-zero constant function.
\item the matrix entries $z_{ij}$ separate points.
\item Closed by conjugation is by the following observation: 
$u\in U(k)$ then $u^* = u^{-1} = {\rm Adj}(u) \det^{-1}(u)$, hence 
$\overline{z_{ij}} \in \Im\pi$. Moreover $\det(u)\in U(1)$, hence
$\overline{\det^{-1}(u)} = \det(u) \in \Im\pi$.
Combine these evidences, we see for any $f = \sum p_l \det^{-l} \in \Im \pi$,
\[
\overline{f}(u) = \sum \overline{p_l(u)}\overline{\det(u)}^{-l}
= \sum \widetilde{p}_l(\overline{u})\det(u)^l \in \Im \pi
\]  
\end{enumerate}

We can summerize the result:

\begin{lemma}
\[
\crr(\GL(n,\bC)) \simeq \chh(U(n))
\]
as representation of $K$. 
\end{lemma}

\subsection{Lie algebra actions on $\chh(K)$ }
Now take $\fkk_\bC = \agl(V)$, 

\[
r(X) = \frac{d}{dt} f(M \exp(tX)) 
= \sum_{kl} \frac{\partial}{\partial m_{kl}} f (M) (MX)_{kl}
= \sum_{kl} \frac{\partial}{\partial m_{kl}} f (M) m_{ks} x_{sl}
\]
Hence 
\[
E_{sl} \mapsto \sum_{k} m_{ks} \frac{\partial}{\partial m_{kl}}.
\]

Define 
\[
D_v(f)(x) = \frac{d}{dt}f(x+tv)
\]
\[
M_{v^*}(f) = v^* f.
\]

Then 
\[
v^*\otimes u \in \agl(V) \to  M_{v^*} D_v.
\]

Here identify $E_{sl} = $

\[
\frac{d}{dt} \det(M\exp tX) = \frac{d}{dt}\det(M)\det(tX) 
 = \det(M)  \frac{d}{dt} e^{t\tr(X)} = \tr(X) \det(M) 
\]
Same,
\[
\frac{d}{dt} \det^{-1}(M\exp tX) = \det^{-1}(M) \frac{d}{dt}e^{-t\tr(X)}
=-\tr(X) \det^{-1}(M).
\]

Hence for function $f\det^{-n} \in \chh(K)$ where $f\in \cpp(M_{n,n})$ and
$X \in \agl$, we have, 
\[
\begin{split}
r(X) f\det^{-n}(M) &= \det^{-n}(M) r(X) f(M) + f(M) r(X) \det^{-n}(M)\\
 &= \det^{-n}(M) 
(\sum_{sl} x_{sl} (\sum_{k} m_{ks} \frac{\partial}{\partial m_{kl}} f(M) 
- \delta_{sl}f(M)))
\end{split}
\]
 

\section{sp}

$\sp_\bC(n)$ has roots $\pm \epsilon_k \pm \epsilon_l$, $k\neq l$ and 
and $\pm 2\epsilon_k$, $k$.
Simple roots are 
\[
\alpha_1 = 2\epsilon_n, \alpha_2 = \epsilon_1 -\epsilon_2, \cdots, 
\alpha_n = \epsilon_{n-1}-\epsilon_{n}.
\]

Take 
\[
\begin{cases}
\eta_k = \epsilon_k & 1\leq k \leq r,\\
\eta_{r+k}= -\epsilon_{n-k+1} & 1\leq k \leq s
\end{cases}
\]
Take 
\[
H = \frac{1}{2} \sum_{k=1}^n \eta_k, = \frac{1}{2} \sum_{k=1}^r \epsilon_k 
- \frac{1}{2} \sum_{k=r+1}^n \epsilon_k
\]
\[
H' = \frac{1}{2} \sum_{k=1}^r \eta_k + \frac{1}{2} \sum_{k=r+1}^n \eta_k
= \frac{1}{2} \sum_{k=1}^n \epsilon_k
\]
Here
$\epsilon_k$ correspond to a element in $\fhh$, the usual Cartan subalgebra
-- diagnal matrix.
In fact, 
\[
\epsilon_k \leftrightarrow H_{\epsilon_k}= E_{k,k}-E_{n+k,n+k}
\] 

Here we realize $\sp_\bC$ by $2n\times 2n$ matrix
\[
\begin{pmatrix}
A & B \\
C & -A^T
\end{pmatrix}
\]
such that $B,C$ symmetric, i.e. $B=B^T$ and $C= C^T$.
Follows Howe\cite{Howe1985}, we take negative $\dagger$, where 
\[
\begin{pmatrix}
A & B\\
C & -A^T 
\end{pmatrix}^\dagger
=
\begin{pmatrix}
A^* & -C^*\\
-B^* & -\overline{A}
\end{pmatrix}
=\begin{pmatrix}
A^* & -\overline{C}\\
-\overline{B} & -\overline{A}
\end{pmatrix}
\]
(by (\ref{eq:matZ})) be the involution defining the realform of $\sp_\bC$.
The the element in realform is 
\[
\begin{pmatrix}
A & B\\
\overline{B} & \overline{A}
\end{pmatrix}
\]
since $ -A^* = A$ and $C=\overline{B}$.

Now from (\ref{eq:tZtoe})  conjugate by 
\[c = \frac{1}{2}
\begin{pmatrix}
1 & 1\\
i & i
\end{pmatrix},
\]
we have 
\[
\begin{split}
(\Ad c)
\begin{pmatrix}
A & B\\
\overline{B} & \overline{A}
\end{pmatrix}
=& \frac{1}{2}
\begin{pmatrix}
(A + \overline{A}) + (B +\overline{B}) &-i(A-\overline{A}) +i(B-\overline{B})\\
i(A-\overline{A}) + i(B-\overline{B})& -(A + \overline{A}) + (B +\overline{B})
\end{pmatrix}\\
=&\begin{pmatrix}
\real A  + \real B & \imag A - \imag B\\
-\imag A -\imag B & -(\real A+\real B)  
\end{pmatrix}
\in \sp_\bR
\end{split}
\]
since $\real A$ is antisymmetric and $\imag A, \real B, \imag B$ is symmetric. 
Hence this gives the Lie algebra isomorphism from the fix point of $\dagger$ 
to $\sp_\bR$. 

Hence 
\[
H = \frac{1}{2} \begin{pmatrix}
I_r & & & \\
 & -I_s & & \\
 & & -I_r & \\
 & & & I_s 
\end{pmatrix}
\text{ and }
H' = \frac{1}{2} 
\begin{pmatrix}
I_r & & & \\
 & I_s & & \\
 & & -I_r & \\
 & & & -I_s 
\end{pmatrix}
\]
Then 
\[
\theta = e^{\pi i \ad H} =\Ad \begin{pmatrix}
iI_r & & & \\
 & -iI_s & & \\
 & & -iI_r & \\
 & & & iI_s 
\end{pmatrix}
\]
and
\[
\theta' = e^{\pi i \ad H'} = \Ad \begin{pmatrix}
iI_r & & & \\
 & iI_s & & \\
 & & -iI_r & \\
 & & & -iI_s 
\end{pmatrix}
\]

For any element 
\[ X = 
\begin{pmatrix}
A & B \\
C & -A^T 
\end{pmatrix} \in \sp_\bC(n)
\]
We have 
\[
\theta'(X) = \begin{pmatrix}
A & -B\\
-C & -A^T
\end{pmatrix}
\]
Hence 
\[
\fkk_\bC' = \Set{\begin{pmatrix}
A & 0 \\
0 & -A^T 
\end{pmatrix}} \simeq \agl_{n, \bC}  
\]
in a obvious way. 

For the fix point of $\theta$, we consider the matrix in the following form
\[
X = \begin{pmatrix}
  A & B & C & D\\
  E & F & {D}^{T} & G\\
  K & L & -{A}^{T} & -{E}^{T}\\
  {L}^{T} & M & -{B}^{T} & -{F}^{T}
\end{pmatrix}
\]
Then 
\[
\theta(X) = 
\begin{pmatrix}
  A & -B & -C & D\\
  -E & F & {D}^{T} & -G\\
  -K & L & -{A}^{T} & {E}^{T}\\
  {L}^{T} & -M & {B}^{T} & -{F}^{T}
\end{pmatrix}
\]
Hence $\fkk_\bC$ consisting elements in the following form
\[
\begin{pmatrix}
  A & 0 & 0 & D\\
  0 & F & {D}^{T} & 0\\
  0 & L & -{A}^{T} & 0\\
 {L}^{t} & 0 & 0 & -{F}^{T}
\end{pmatrix}
\Rightarrow
\begin{pmatrix}
  A & D & 0 & 0\\
  {L}^{T} & -{F}^{T} & 0 & 0\\
  0 & 0 & -{A}^{T} & -L\\
  0 & 0 & -{D}^{T} & F
\end{pmatrix}
\]
by conjugating the permutation matrix
\[
\begin{pmatrix}
  1 & 0 & 0 & 0\\
  0 & 0 & 0 & 1\\
  0 & 0 & -1 & 0\\
  0 & 1 & 0 & 0
\end{pmatrix}
\]
This give a nature way to identify $\fkk_\bC$. to $\agl_n(\bC)$.
Moreover 
\[
\fpp_\bC = \Set{
\begin{pmatrix}
  0 & B & C & 0\\
 E & 0 & 0 & G\\
 K & 0 & 0 & -{E}^{T}\\
 0 & M & -{B}^{T} & 0
\end{pmatrix}
|
}
\]


We can see that $\fmm_\bC = \fkk_\bC \cap \fkk_\bC'$, i.e. 
the fixed point of $\theta$ in $\ftt_\bC'$ is 
\[
\Set{\begin{pmatrix}
    A & & & \\
    & B & & \\
    & & -A^T &\\
    & & & -B^T 
  \end{pmatrix}|
  A\in \Mat_{r,r}(\bC), B\in \Mat_{s,s}(\bC)
}
\]

Also , $\foo = \fpp_\bC'$ consists matrix in form 
\[
\begin{pmatrix}
0 &  B \\
C & 0 
\end{pmatrix}
\]
and $B= B^T$, $C= C^T$.
And 
\[
\foo^+ = \Set{\begin{pmatrix}
0 &  B \\
0 & 0 
\end{pmatrix}
| B = B^T \in \Mat_{n,n}(\bC)
}
\]
\[
\foo^- = \Set{\begin{pmatrix}
0 &  0 \\
C & 0 
\end{pmatrix}
| C = C^T \in \Mat_{n,n}(\bC)
}
\]
\[
\foo_k^+ =  \Set{\begin{pmatrix}
0 &  0 &  0 &D \\
0 & 0  & D^T & 0 \\
0 & 0 & 0 & 0 \\
0 & 0 & 0 & 0 
\end{pmatrix}
| B = B^T \in \Mat_{n,n}(\bC)}
\]




\section{$W_{p,q}$}
Define $\theta'$-stable parabolic $\fqq$ of $\fkk_\bC$ by weight
\[
\lambda_{p,q} = -(\eta_{r-q+1} +  \cdots + \eta_r)+(\eta_{r+1} + \eta_{r+p})
 = \sum_{k=r-q+1}^{r+p} - \epsilon_{k}
\]
So the non-compact nilpotent part is 
\[
\fnn\cap \fpp = \Set{g_{\alpha}|\alpha = -(\epsilon_k+\epsilon_l), 
  k\geq r+1, l \leq r
  \text{ and } (
  k \leq r+p \text{ or } l \geq r-q+1)
}
\]

\section{notation}



Give $(\fgg_\bC, M)$-module $(\pi, V)$.

In this situation. 
Consider 
\[
\Gamma^i (V) = \Hom_{M}(\bigwedge^i(\fkk_\bC/\fmm_\bC), V\otimes \chh(K))
\]


The $\fkk$-module structure on $V\otimes \chh(K)$ given when compute the
$\Gamma^i$ is $\pi\otimes l_K$
The nature $K$ structure on $\Gamma^i$ is given by $I\otimes r_K$.

Look $V\otimes \chh(K)$ as $\chh(K, V)$ by $v\otimes f \mapsto F$, 
$F(k) = f(k)v$. 
 
Give $\fgg$ action on $\chh(K,V)$ by 
\[
(\mu(X)F)(k) = \pi(\Ad(k) X) F(k)
\]

Choose a basis $\Set{X_1, \cdots, X_n}$ of $\fgg$ and 
let $\lambda_1, \cdots,\lambda_n$ be the dual basis of $\fgg^*$.

Then (denote $\pi(X)$ $X$ here). 
\[
\mu(X) (v\otimes f)(k) = \Ad(k) X v f(k) = 
\sum_{l} \lambda_l(\Ad(k)X) X_l v f(k) 
= \sum_{l} (X_l v \otimes c_{\lambda_l,X} f)(k),
\]
where $c_{\lambda,X} = \lambda(\Ad(k)X)$.
Same as \cite{BorelWallach2000} p. 25, I.8.1,
\[
\mu(X) (v\otimes f) = \sum_l X_l v \otimes c_{\lambda_l,X} f. 
\]

This $\fgg$ structure compatible with the $K$ structure on $\Gamma^i$, i.e. 
\[
\pi(\Ad(k) X) F(k) = (r_X F)(k)\quad \forall X\in \fkk, k\in K.
\]

Let $Y=\Ad(k)X$, then $X = \Ad(k^{-1})Y$,
\[
YF(k) = \frac{d}{dt} F(k\exp(t\Ad k^{-1}Y))
= \frac{d}{dt} F(\exp(tY)k)
= l(Y)F (k). 
\]

This means the image of $\Hom_\fmm(W_{p,q},V\otimes \chh(K))$ is in 
$(V\otimes \chh(K))^\fkk$, where the $\fkk$ act by $\pi\otimes l$.

But, if restrict everyting in $(V\otimes \chh(K))^\fkk$, 
\[
\Gamma_{p,q} = \Hom_M(W_{p,q}, (V\otimes \chh(K))^\fkk)
=(W_{p,q}^*)^M \otimes (V\otimes \chh(K))^\fkk=0?
\] 


\bibliography{bib/reppapers}{}
\bibliographystyle{alpha}

\end{document}

