\documentclass[12pt]{amsart}
\usepackage[margin=3cm]{geometry}

\usepackage{hyperref}

\usepackage{amssymb}
\usepackage{graphicx}
%\usepackage{amscd}
\usepackage{braket}
\usepackage{paralist}
\usepackage{eufrak}
%\usepackage{calrsfs}
%\usepackage[small,nohug,heads=littlevee]{diagrams}
%\diagramstyle[labelstyle=\scriptstyle]
%\usepackage{diagrams}
\usepackage[all,cmtip]{xy}
\usepackage{diagxy}
%\usepackage{pictexwd,dcpic}
%\usepackage{mathrsfs}


\newtheorem{lemma}{Lemma}
\newtheorem{thm}{Theorem}
\newtheorem{prop}{Proposition}
\newtheorem{cor}{Corollary}
\newenvironment{expl}{\it}{\color{black}\normalsize}
\DeclareMathAlphabet{\mathpzc}{OT1}{pzc}{m}{it}



\def\Ker{\rm{Ker}}
\def\Im{\rm{Im}}
\def\Hom{\rm{Hom}}
\def\Mat{\rm{Mat}}
\def\Ind{{\rm Ind}}
\def\bR{{\mathbb{R}}}
\def\bN{{\mathbb{N}}}
\def\bZ{{\mathbb{Z}}}
\def\bC{{\mathbb{C}}}
\def\bQ{{\mathbb{Q}}}
\def\bB{{\mathbb{B}}}
\def\bA{{\mathbb{A}}}
\def\bs{{\mathbf{s}}}
\def\bd{{\mathbf{d}}}
\def\bT{{\mathbb{T}}}
\def\bt{{\mathbf{t}}}
\def\br{{\mathbf{r}}}
\def\vv{{\vec{v}}}
\def\vw{{\vec{w}}}
\def\vx{{\vec{x}}}
\def\vy{{\vec{y}}}
\def\v0{{\vec{0}}}
\def\ol{\overline}
\def\sspan{\rm{span}}
\def\sl2{{\mathfrak{sl}(2)}}
\def\slc{{\mathfrak{sl}(2,\bC)}}
\def\sp{{\mathfrak{sp}}}
\def\gg{{\mathfrak{g}}}
\def\kk{{\mathfrak{k}}}
\def\pp{{\mathfrak{p}}}
\def\qq{{\mathfrak{q}}}
\def\sP{\mathcal{P}}
\def\sH{\mathcal{H}}
\def\sU{\mathcal{U}}
\def\sC{\mathcal{C}}
\def\Stab{{\rm Stab}}
\def\ad{{\rm ad}}
\def\Ad{{\rm Ad}}
\def\id{{\rm id}}
\def\sgn{{\rm sgn}}
\def\gcd{{\rm gcd}}
\def\inn#1#2{\left\langle{#1},{#2}\right\rangle}
\def\abs#1{\left|{#1}\right|}
\def\norm#1{{\left\|{#1}\right\|}}
\def\Sp{{\rm Sp}}

\def\Ad{\mathrm{Ad}}
\def\agl{\mathrm{gl}}
\def\asl{\mathrm{sl}}
\def\aso{\mathrm{so}}
\def\asp{\mathrm{sp}}
\def\au{\mathrm{u}}
\def\diag#1{\mathrm{diag}(#1)}
\def\lww{\mathcal{W}}
\def\lxx{\mathcal{X}}
\def\lyy{\mathcal{Y}}
\def\fhh{\mathfrak{h}}
\def\ftt{\mathfrak{t}}
\def\fgg{\mathfrak{g}}
\def\fkk{\mathfrak{k}}
\def\chh{\mathcal{H}}
\def\fmm{\mathfrak{m}}
\def\GL{\mathrm{GL}}
\def\SO{\mathrm{SO}}
\def\rad{\mathrm{rad}}
\def\SL{\mathrm{SL}}
\def\tr{\mathrm{tr}}
\def\Mat{\mathrm{Mat}}
\def\U{\mathrm{U}}


\hypersetup{
    bookmarks=false,         % show bookmarks bar?
    unicode=false,          % non-Latin characters in Acrobat’s bookmarks
    pdftoolbar=true,        % show Acrobat’s toolbar?
    pdfmenubar=false,        % show Acrobat’s menu?
    pdffitwindow=false,      % page fit to window when opened
    pdftitle={Qualify Examination Answers - Algebra},    % title
    pdfauthor={Ma Jia Jun},     % author
    pdfsubject={Subject},   % subject of the document
    pdfcreator={Creator},   % creator of the document
    pdfproducer={Producer}, % producer of the document
    pdfkeywords={keywords}, % list of keywords
    pdfnewwindow=true,      % links in new window
    colorlinks=true,       % false: boxed links; true: colored links
    linkcolor=blue,          % color of internal links
    citecolor=green,        % color of links to bibliography
    filecolor=magenta,      % color of file links
    urlcolor=cyan           % color of external links
}


\newcounter{ssection}
\setcounter{ssection}{0}
\renewcommand{\subsection}{
  \addtocounter{ssection}{1}{\bf  \arabic{ssection}.\  }}

\title{transfer of $K$ type}

\begin{document}
\maketitle

\section{Heisenberg groups}
Consider $H = W\oplus Z$, $W = X\oplus Y$ 
with group law: 
\[
h(x,y,z)h(x',y',z') = h(x+x', y+y', z+z'+xy').
\]
This law is from 
\[
\begin{pmatrix}
1 & x & z\\
0 & I & y\\
0 & 0 & 1
\end{pmatrix}
\begin{pmatrix}
1 & x' & z'\\
0 & I & y'\\
0 & 0 & 1
\end{pmatrix}
=
\begin{pmatrix}
1 & x+x' & z+z'+ x y'\\
0 & I & y+y'\\
0 & 0 & 1
\end{pmatrix}.
\]

Let $w = (x,y)$. 

Define exponation map $\exp$ from the lie algebra $\fhh$ of $H$ to $H$. 
\[
\fhh = \lww\oplus F= \Set{\begin{pmatrix}
    0 & a & t\\
    0 & 0 & b\\
    0 & 0 & 0
  \end{pmatrix}}
\]
Note
\[
\begin{pmatrix}
0 & a & c\\
0 & 0 & b\\
0 & 0 & 0
\end{pmatrix}
\begin{pmatrix}
0 & a' & c'\\
0 & 0 & b'\\
0 & 0 & 1
\end{pmatrix}
=
\begin{pmatrix}
0 & 0 & a b'\\
0 & 0 & 0\\
0 & 0 & 0
\end{pmatrix}.
\]
Hence $X^3 = 0$ for any $X\in \fhh$. 
Then 
\[
\exp(X) = I + X + \frac{1}{2}X^2= h(a,b, c+ \frac{1}{2} ab) \in H. \]
and
\[ 
\log(h) = (x,y, z-\frac{1}{2}xy\in \fhh. 
\]

Using above formula, we have
\[
h^{-1} = \exp(-\log(h)) = h(-x,-y,-z+x y).
\]

\[
\begin{split}
hgh^{-1}g^{-1} &= h(x,y,z)h(x',y',z') h(-x,-y,-z+xy) h(-x',y',-z'+x'y')\\
&= h(x+x',y+y', z+z'+xy')h(-(x+x'),-(y+y'), -(z+z')+xy+x'y'+xy')\\
&= h(0,0,-(x+x')(y+y')+xy + x'y' +xy')
= h(0,0,xy'-x'y)\\ 
&= h(0,0,\inn{w}{w'}).
\end{split}
\]

\[
[X,Y] = (0,0,ab'-a'b) = (0,0,\inn{w}{w'}) \in \lww.
\]

\[
\begin{split}
\exp(w,c)\exp(w',c') &= h(w,c+\frac{1}{2}ab)h(w',c'+\frac{1}{2}a'b')
= h(w+w',c+c'+\frac{1}{2}(ab+a'b')+ab')\\
&= \exp(w+w', c+c'+\frac{1}{2}(ab'-b'a)) = 
\exp(w+w',c+c'+\frac{1}{2}\inn{w}{w'}).
\end{split}
\]
This give an alternateve construction of Heisenberg groups. 

The Haar measure on $H$ is $dzdxdy = dzdw$.


Choose a unitary character $\chi$ of $F$, extend to $S = Y\oplus Z$, by 
$Y$ act trivially.
Let 
\[
\chh_\chi = \Ind_S^H \chi 
=  \Set{f\colon H\to F|f(sh) = \chi(s)f(h)\, \forall s\in S\text{ and }
   \int_{S\backslash H} |f(h)|^2 d\dot h <\infty}
\]
The right translation give a $H$-module structure on $\chh_\chi$.  

We can consider $f\in \chh_\chi$ as a function on $X$ by 
\[
\phi(x) =f(h(x,0,0)) = f(\exp(x,0,0)).
\]
This give a isometry from $\chh_\chi$ to $L^2(X)$. 

Then 
\[
\begin{split}
\rho(h(x',y',z'))\phi(x) =& f(h(x,0,0)h(x',y',z')) =
f(h(x',y',z'+xy'))\\
 =& f(h(0,y', z'+xy')h(x+x',0,0))\\
=& \chi(z'+xy')\phi(x+x').
\end{split}
\]
This is the Schr\"odinger representation. Moreover
\[
\int_{S\backslash H} |f(h)|^2 d\dot h = \int_X|\phi(x)|^2 dx
\]


The oscillator representation, choose polarlization $W=X\oplus Y$. 
Let $P = P(Y) = \Stab_\Sp Y$, $M = P(Y)\cap P(X) \cong \GL(X)$,
$N=\cap_{y\in Y} \Stab_\Sp y$, i.e the kernel of map $P(Y) \to \GL(Y)$. 
There is a isomorphism from $N$ to $S^{2*}(X)$ given by 
\[
n \mapsto (\beta_n \colon v,v' \mapsto \inn{v}{n v'}
\]
For $\beta\in S^{2*}(X)$, $\beta(v,v')$ give a map $\gamma_\beta\colon X\to Y$
such that 
\[
\inn{v}{\gamma_\beta(v')} = \beta(v,v')
\]
The inverse map then defined by 
\[
\beta \mapsto n\colon \begin{cases}
v & \text{for } v\in Y\\
v+\gamma_\beta(v) & \text{for } v\in X
\end{cases}
\]
One can verify that $\beta_n^{-1} = -\beta_n$


Extand $\chi$ to $P\ltimes S$ by $P$ act trivially. 
Let $\chh_\chi = \Ind_{P\ltimes S}^{P\ltimes H}\chi$ be the unitary 
induced representation. $P\ltimes H$ act by left translation.
Again let 
\[
\phi(x) = f(h(x,0,0))
\]
This compactable with $\Ind_S^H \chi$, and has same Hermitian inner product:
\[
\int_{X}\abs{\phi(x)}^2 dx 
\]

Then for $m\in M=\GL(X)$
\[
f(h(x,0,0)g)=f(m m^{-1}h(x,0,0)m) = f(h(m^{-1}x,0,0))  = \phi(m^{-1}x)
\]
Since 
\[
|\phi(m^{-1}x)|^2 dx = |\phi(x)|^2 d mx = |\phi(x)|^2 \abs{\det_{\GL(X)} m} dx,
\]
define 
\[
\omega(m)\phi(x) = \abs{\det_{\GL(X)} m}^{-1/2}f(mx).
\]

\[
\begin{split}
f(h(x,0,0)n) =& f(n n^{-1}h(x,0,0)n) = f(e(n^{-1}(x,0),0)) \\
=& f(e(x,\gamma_{\beta_{n^{-1}}}(x),0)) = 
f(h(x, \gamma_{\beta_{n^{-1}}}(x), \frac{1}{2}\inn{x}{\gamma_{\beta_{n^{-1}}}(x)}))\\
=& f(h(x, \gamma_{\beta_{n^{-1}}}(x),  -\frac{1}{2}\beta_n(x,x))) \\
=& \chi(-\frac{1}{2}\beta_n(x,x))f(h(x,0,0))
\end{split}
\]
Hence 
\[
\omega(n)\phi(x) = \chi(-\frac{1}{2}\beta_n(x,x))\phi(x)
\]

The special element is 
\[
\sigma = \begin{pmatrix}
0 & 1
1 & 0
\end{pmatrix} \in \Sp(W)
\]
It exchange the role of $x, y$. It is reasonable to believe the intertwining
map is a Fourier transform. 


\section{Fock model}
Let $J\in \Sp(W)\cap \sp(W)$ such that $J^2 = -\id$.
Then $J$ given a complex structure on $W$.
Then 
\[
B_J(w,w') = \inn{Jw}{w'}
\]
is a (non-degenerate) symmetric bilinear form.

\[
H_J(w,w') = \inn{Jw}{w'} + i\inn{w}{w'}
\]
become a Hermitian bilinear form(inner product) on $W$ where the 
complex structure on $W$ defined by $J$. 
I.e. $i w = J(w)$. 
 
Complexify $W$ to $W_\bC$. Extend $\inn{}{}$ to $W_\bC$.  
 
View $H = W\oplus Z$ from the exponatial map of Lie algebra $\fhh$ of $H$. 
Then the group law becomes
\[
\exp(w, t) \exp(w',t) = \exp(w+w', t+t'+\frac{1}{2}\inn{w}{w'}).
\]
Similary define $H_\bC$. 
Let $V^\pm$ be the $\pm i$ eigen space of $J$ on $W_\bC$. 
Then consider the subgrou $P = V^-\oplus Z  < H_\bC$.
Consider the following digram
\[
\bfig
\square/^{(}->`->>`->>`^{(}->/[G`G_\bC`Z\backslash G`P\backslash G_\bC;{}`{}`{}`{}]
\efig
\]



\section{sp}

$X\in \sp$, 
\section{notation}



Give $(\fgg_\bC, M)$-module $(\pi, V)$.

In this situation. 
Consider 
\[
\Gamma^i (V) = \Hom_{\fkk}(\fkk/\fmm, V\otimes \chh(K))
\]


The $\fkk$-module structure on $V\otimes \chh(K)$ given when compute the
$\Gamma^i$ is $\pi\otimes l_K$
The nature $K$ structure on $\Gamma^i$ is given by $I\otimes r_K$.

Look $V\otimes \chh(K)$ as $\chh(K, V)$ by $v\otimes f \mapsto F$, 
$F(k) = f(k)v$. 
 
Give $\fgg$ action on $\chh(K,V)$ by 
\[
(\mu(X)F)(k) = \pi(\Ad(k) X) F(k)
\]
This $\fgg$ structure compact with the $K$ structure, i.e. 
\[
\pi(\Ad(k) X) F(k) = (r_X F)(k).
\]
Take $k=1$, 
this means 
\[
\pi(x)v \otimes f = v \otimes r_x f
\]
for any $x\in U(\fkk)$.


\end{document}

