\documentclass{beamer} 

\usepackage{pdfsync}

\usepackage{hyperref}
\usepackage{amssymb}
\usepackage{amsmath, amsthm}
\usepackage{braket}
\usepackage{paralist}
\usepackage{eufrak}
\usepackage[all,cmtip]{xy}
\usepackage{diagxy}

\usetheme{Boadilla} 

\begin{document}

\def\sign{{\rm sign}}
\def\Ker{{\rm Ker}}
\def\Im{{\rm Im}}
\def\Hom{{\rm Hom}}
\def\End{{\rm End}}
\def\Mat{{\rm Mat}}
\def\Ind{{\rm Ind}}
\def\bR{{\mathbb{R}}}
\def\bN{{\mathbb{N}}}
\def\bZ{{\mathbb{Z}}}
\def\bC{{\mathbb{C}}}
\def\bQ{{\mathbb{Q}}}
\def\bB{{\mathbb{B}}}
\def\bA{{\mathbb{A}}}
\def\bs{{\mathbf{s}}}
\def\bd{{\mathbf{d}}}
\def\bT{{\mathbb{T}}}
\def\bt{{\mathbf{t}}}
\def\br{{\mathbf{r}}}
\def\vv{{\vec{v}}}
\def\vw{{\vec{w}}}
\def\vx{{\vec{x}}}
\def\vy{{\vec{y}}}
\def\v0{{\vec{0}}}
\def\ol{\overline}
\def\sspan{\rm{span}}
\def\sl2{{\mathfrak{sl}(2)}}
\def\slc{{\mathfrak{sl}(2,\bC)}}
\def\sp{{\mathfrak{sp}}}
\def\gg{{\mathfrak{g}}}
\def\kk{{\mathfrak{k}}}
\def\pp{{\mathfrak{p}}}
\def\qq{{\mathfrak{q}}}
\def\sP{\mathcal{P}}
\def\sH{\mathcal{H}}
\def\sU{\mathcal{U}}
\def\sC{\mathcal{C}}
\def\sS{\mathcal{S}}
\def\Stab{{\rm Stab}}
\def\ad{{\rm ad}}
\def\Ad{{\rm Ad}}
\def\id{{\rm id}}
\def\sgn{{\rm sgn}}
\def\gcd{{\rm gcd}}
\def\inn#1#2{\left\langle{#1},{#2}\right\rangle}
\def\abs#1{\left|{#1}\right|}
\def\norm#1{{\left\|{#1}\right\|}}
\def\Sp{{\rm Sp}}
\def\SL{{\rm SL}}
\def\O{{\rm O}}
\def\SO{{\rm SO}}
\def\det{{\rm det}}

\def\tG{{\widetilde{G}}}
\def\tK{{\widetilde{K}}}
\def\tM{{\widetilde{M}}}
\def\tJ{{\widetilde{J}}}
\def\trho{{\widetilde{\rho}}}
\def\tsigma{{\widetilde{\sigma}}}
\def\tDelta{\widetilde{\Delta}}
\def\barZ{{\overline{Z}}}
\def\bz{{\overline{z}}}

\def\Ad{\mathrm{Ad}}
\def\fgl{\mathfrak{gl}}
\def\fsl{\mathfrak{sl}}
\def\fso{\mathfrak{so}}
\def\diag#1{\mathrm{diag}(#1)}
\def\lww{\mathcal{W}}
\def\lxx{\mathcal{X}}
\def\lyy{\mathcal{Y}}
\def\fbb{\mathfrak{b}}
\def\fhh{\mathfrak{h}}
\def\fnn{\mathfrak{n}}
\def\fuu{\mathfrak{u}}
\def\fll{\mathfrak{l}}
\def\foo{\mathfrak{o}}
\def\fpp{\mathfrak{p}}
\def\fqq{\mathfrak{q}}
\def\ftt{\mathfrak{t}}
\def\fgg{\mathfrak{g}}
\def\fkk{\mathfrak{k}}
\def\caa{\mathcal{A}}
\def\ccc{\mathcal{C}}
\def\cdd{\mathcal{D}}
\def\chh{\mathcal{H}}
\def\cjj{\mathcal{J}}
\def\crr{\mathcal{R}}
\def\css{\mathcal{S}}
\def\cpp{\mathcal{P}}
\def\cww{\mathcal{W}}
\def\cnn{\mathcal{N}}
\def\cuu{\mathcal{U}}
\def\cxx{\mathcal{X}}
\def\cyy{\mathcal{Y}}
\def\cff{\mathcal{F}}
\def\czz{\mathcal{Z}}
\def\cug{\cuu(\fgg)}
\def\fmm{\mathfrak{m}}
\def\GL{\mathrm{GL}}
\def\SO{\mathrm{SO}}
\def\OO{\mathrm{O}}
\def\rad{\mathrm{rad}}
\def\SL{\mathrm{SL}}
\def\tr{\mathrm{tr}}
\def\Mat{\mathrm{Mat}}
\def\U{\mathrm{U}}
\def\Gr{\mathrm{Gr\,}}
\def\tU{{\widetilde{U}}}
\def\pz#1{\partial z_{#1}}
\def\ddt{\left.\frac{d}{dt}\right|_{t=0}}
\def\csigma\
\def\Ind{{\rm Ind}}


\def\cmm#1#2{\left[{#1},{#2}\right]}
\def\acmm#1#2{\left\{{#1},{#2}\right\}}

\def\seesawpair#1#2#3#4{
\xymatrix{
{#1} \ar@{-}[dr]& {#2} \\
{#3} \ar@{-}[ur] & {#4}}
}

\def\real{{\rm Re\,}}
\def\imag{{\rm Im\,}}

\def\rmk{{\bf Remark:}}


\title[$\cuu(\fgg)^K$ in dual pairs]{$\cuu(\fgg)^K$ actions in dual pairs}
\author{Ma Jia Jun}
\institute[NUS]{National University of Singapore}

\begin{frame}[plain]
  \titlepage
\end{frame}

\begin{frame}{Notions}
For any real reducitve group $G$, 
we will consider a subgroup $H$ of $G$ such that $K_H = H\cap K_G$ is a maximal 
compact subgroup of $H$, where $K_G$ is the maximal compact subgroup of $G$.

Now a $(\fgg,K)$ module is also a $(\fhh,K_H)$ module.

All the module we considered later will in this case, 
so we suppress the maximal compact subgroup and work with 
the $(\fgg,K_G)$-module category.
A $H$-module just means $(\fhh,K_H)$-module and write,
\[
\Hom_{H}(V,W) = \Hom_{\fhh,K_H}(V,W),
\]
for any $(\fhh,K_H)$-module $V$ and $W$.
\end{frame}

\begin{frame}{A Generization of $\cuu(\fgg)^K$}


Suppose $H\subset G$,  $V$ is a $G$ module, $W$ is a irreducible $H$ module
 such that
$\Hom_{H}(V,W)\neq 0$.
Note that $\cuu(\fgg)^H \subset \Hom_H(V,V)$. $\cuu(\fgg)^H$ act on right 
of $\Hom_H(V,W)$ by compositon.

To make the action on left, define
Define a $H$-module \[
\Omega_V(W) = V / \cnn_{V,W},
\quad \text{where }
\cnn_{V,W} = \bigcap_{T\in \Hom_H(V,W)}\Ker(T).
\]
Now $\cuu(\fgg)^H$ act on $\Omega_V(W)$ via the quotient map.

\begin{lemma}
Suppose that $\dim \Hom_{H}(V,W)=1$,
Then $S \in \cuu(\fgg)^H$ act on $\Omega_{V,W}$ by a  scaler $c_{V,W}(S)$.
\end{lemma}
\rmk This scaler is determined by the $S$ action on any vector $v\in V$ with 
non-zero image under any $H$-equivariant projection to $W$ .

\end{frame}


\begin{frame}{Howe quotient.}
Suppose $(G,G')$ is a reducitive dual pair with 
Weil representation $\omega$.

Let $R(\omega,\widetilde{G})$
be the set of irreducible  representation of $\widetilde{G}$
which can be realize as a quotient of $\omega$.

We will always work under the double covers of groups, so
we just suppress the $\widetilde{\ \  }$ and think all groups 
appears later actually are some double cover.

For $G'$ representatoion $\sigma \in R(\omega,G')$, we have
\[
 \Omega_\omega(\sigma) = \rho(\sigma)\otimes \sigma.
\]
 as $G\times G'$ representation, where $\rho(\sigma)$ is some $G$-module.
Similary for $G$ representation $\tau\in R(\omega,G)$,
\[
 \Omega_\omega(\tau) = \tau\otimes \rho(\tau).
\]

\end{frame}

\begin{frame}{See-saw pair and  joint action of $\cuu(\fgg)^H$ 
and $\cuu(\fhh')^{G'}$}


Suppose we have following See-Saw pair:
\[
\xymatrix{
\rho(\sigma)& G \ar@{-}[dr] & H'& \rho(\tau)\\
\tau &H\ar@{-}[ur] & G' & \sigma
}
\]
where $H < G$, $H'> G'$. Let $\sigma\in R(\omega, G')$, $\tau\in R(\omega,H)$.
Then:
\[
\Hom_{H}(\rho(\sigma), \tau) 
\cong  \Hom_{H\times G'}(\omega, \tau\otimes \sigma)
\cong  \Hom_{G'}(\rho(\tau), \sigma)
\]

{\color{blue}
 $\cuu(\fgg)^H$ and $\cuu(\fhh')^{G'}$ 
has a joint action on $\Hom_{H\times G'}(\omega,\tau\otimes\sigma)$

we can link the $\cuu(\fgg)^H$ action on $\Hom_H(\rho(\sigma),\tau)$ and $\cuu(\fhh')^{G'}$  action on $\Hom_{G'}(\rho(\tau),\sigma)$.
}
\end{frame}

\begin{frame}{Relation between $\cuu(\fgg)^H$ and $\cuu(\fhh')^{G'}$}
\begin{lemma}\label{lemma:ugkcorr}
For any see-saw pair, 
\[
\omega(\cuu(\fgg)^H) = \omega(\cuu(\fhh')^{G'}).
\]
Hence for any $x\in \cuu(\fgg)^H$ we can choose $y\in \cuu(\fhh')^{G'}$ 
such that $\omega(x) = \omega(y)$ and this choice is independent of real form.
\end{lemma}

It is followed from the model of oscillate representation (c.f. \cite{Adams2007} 
and \cite{Howe1989Rem}).
In particular, let $\End^\circ$ be the set of all polynomial coefficient differential operators.
Then the image of $\omega(\cuu(\fgg))$ is inside $\End^\circ$ and is precisely the $G'$ inveriant, i.e.:
\[
\omega(\cuu(\fgg)) = \left(\End^\circ\right)^{G'}.
\]

So 
\[
\omega(\cuu(\fgg)^H) = \omega(\cuu(\fgg))^H = \omega(\cuu(\fgg))\cap \omega(\cuu(\fhh')) = \omega(\cuu(\fhh')^{G'})
\]
\end{frame}

\begin{frame}{A lemma in the Multiplicity one case}
\begin{lemma}
Suppose that $\dim \Hom_H(\rho(\sigma),\tau)) =1$.
The character $c_{\rho(\tau),\sigma}$ of $\cuu(\fgg)^H$ and
 the character $c_{\rho(\sigma),\tau}$ of $\cuu(\fhh')^{G'}$ determine each other 
and is independent of real form.
\end{lemma}

\end{frame}

\begin{frame}{A subjectivity result}
The following result is crucial for the later discussion. It is known by \cite{Shimura1990} and \cite{Wallach1992real}.
Moreover it is used in \cite{Zhu2003} to study the representations with scaler $K$-type.
\begin{lemma}\label{lemma:scalerk}
  Suppose that $\fgg$ be the complexification of 
  a classical Lie algebra.
  Consder symmetric pair $(G,H)$ where
  $H$ is a subgroup $G$ meet all the component of $G$.
  Moreover there is a involution, such that $\fgg=Lie(G)_\bC$ decomposite into 
  $\fgg = \fkk+\fpp$ where 
  $\fkk=Lie(H)_\bC$.
  
  Let $\sigma$ be a one-dimensional representation of $H$.
  Let $\cjj = \Ker(\sigma|_{\cuu(\fkk)})$.
  Then the following map is surjective:
  \[
  \czz(\fgg) \to \cuu(\fgg)^H/(\cjj\cuu(\fgg)\cap\cuu(\fgg)^H)
  \]
\end{lemma}
\end{frame}

\begin{frame}
\begin{lemma}
In above setting, let $V$ be a $G$ module and $W$ be a character of $H$
such that $\dim \Hom_H(V,W)=1$. 
Then the $\cuu(\fgg)^H$ action on $\Omega_{V,W}$ is determined by the
$Z(\fgg)$ action on $V$ and the  ideal $\cjj\subset \cuu(\fkk)$ of 
$\rho$.
\end{lemma}
Tthi is because $\cjj\cuu(\fgg)\cap\cuu(\fgg)^H$ act trivially on $\Omega_{V,W}$.


\begin{theorem}\label{thm:ugkcor}
Let $\sigma \in R(G', \omega)$ be a character of $G'$. 
Suppose that $\tau \in R(H, \omega)$ such that 
$\dim \Hom_{H}(\rho(\sigma),\tau) = 1$.
Then the $\cuu(\fgg)^H$ action on $\Omega_{\rho(\sigma),\tau}$ 
is determined by the infinitesimal character of $\tau$ and annilater ideal 
$\cjj = \Ker(\sigma|_{\cuu(\fgg')})$  
\end{theorem}
\end{frame}

\begin{frame}{An application on derived functor modules}

Let $\Gamma=\Gamma_M^{K^c}$ be the Zuckerman functor picking up the $\fkk$-finite space in any $(\fgg,M)$-module, where $M$ is 
the maximal compact subgroup of $H$.
$K^c$ is a compact real form of $\fkk$ such that $\cuu(\fgg)^H = \cuu(\fgg)^{K^c}$.
%We abuse of notation and write $\Gamma = \Gamma_{\fkk,M}^{\fkk,K^c}$ when it apply to $(\fkk,M)$-module.

From the realization of derived functor module (c.f. \cite{Wallach1992real} Chapter 6), we have
\begin{lemma}\label{lemma:derugkact}
Let $V$ be a $(\fgg,M)$-module, 
$s\in \cuu(\fgg)^{K^c}$ act on 
$\Gamma^iV$ by $\Gamma^i s$, where view $s$ as a $(\fkk,M)$-module endomorphism
of $V$. 
\end{lemma}
\end{frame}

\begin{frame}{The main result}
\begin{theorem}
  Suppose that 
  $G_1$ and $G_2$ are two real forms of a reductive Lie algebra $\fgg$.
  There are Cartan involutions of $G_i$ commute to each other(c.f. \cite{WallachZhu2004}).
  $G_1'$ and $G_2'$ are two real forms of a reductive Lie algebra $\fgg'$.

  $(G_1,G_1')$ and $(G_2,G_2')$ form reductive dual pairs .
  
  Suppose 
  $\fgg=\fkk_i \oplus \fpp_i$ under a Cartan involution of $G_i$.
  $K_i$ be the maximal compact subgroup of $G_i$ with Lie algebre $\fkk_i$.
 
  Let $\Gamma=\Gamma_{K_1\cap K_2}^{K_2}$ be the Zuckerman functor.
  Suppose 
  \begin{itemize}
    \item $\sigma_i$ are characters of $G_i'$, where $i=1,2$. 
  \item  Let $\cjj_i = \Ker(\sigma_i|_{\cuu(\fgg')}) \subset \cuu(\fgg')$.
    $\cjj_1 = \cjj_2$.
  \item $\tau_1$ is a $(\fkk_2,K_1\cap K_2)$-module such that
    $\dim \Hom_{(\fkk_2,K_1\cap K_2)}(\rho(\sigma_1),\tau_1) = 1$.
  \item $\tau_2$ is a $K_2$-module occure in the
    image of map $\Gamma^i \rho(\sigma_1) \to \Gamma^i \tau$ 
    such that 
    $\dim \Hom_{K_2}(\rho(\sigma_2),\tau_2) =1$.
  \end{itemize}
  Then $\Gamma^i \rho(\sigma_1)$ and 
  $\rho(\sigma_2)$ has a isomrophic irreducible subquotient 
  with common $K_2$-type $\tau_2$.
\end{theorem}
\end{frame}

\begin{frame}{Examples}
Following examples are constructed in \cite{WallachZhu2004}.

Let $p\leq s$, $q\leq r$ and
$p+q=m\leq n$. 
$G_1,G_2 = \Sp(2n,\bR)$, $G_1' = O(m)$, $G_2'=O(p,q)$ and
$H = U(r,s)$.
$\theta^{p,q}$ is the theta lifting from $O(p,q)$ to $\Sp(2n,\bR)$.

Then we have
\[
\Gamma_{p,q}(\theta^{0,m}(1)) \cong \theta^{p,q}(1^{\xi,\eta}),
\]
where $\xi \equiv r-q, \eta\equiv s-p \pmod{2}$.


$\Gamma_{p,q}\theta^{0,m}(1)$ is certain submodule of 
\[
\left(\Gamma_{\sp(2n),U(r)\times U(s)}^{\sp(2n), U(r+s)}\right)^{rs-(r-q)(s-p)}\theta^{0,m}(1)
\]
 

For more examples, see \cite{Enright1985}.
\end{frame}

\bibliography{bib/reppapers}{}
\bibliographystyle{alpha}


\end{document}

