\documentclass[11pt]{amsart}
\usepackage[margin=2cm]{geometry}
\usepackage{amssymb}
\usepackage{graphicx}
\usepackage{braket}
\usepackage{paralist}

\newtheorem{Thm}{Theorem}
\newtheorem{Cor}[Thm]{Corollary}
\newtheorem{Lem}[Thm]{Lemma}
\newtheorem{Def}[Thm]{Definition}
\newtheorem{Rmk}[Thm]{Remark}


\def\cA{{\mathcal{A}}}
\def\cAg{{\mathcal{A}_G}}
\def\cL{{\mathcal{L}}}
\def\cLE{\cL^\cE}
\def\cE{{\mathcal{E}}}
\def\bC{{\mathbb{C}}}
\def\bA{{\mathbb{A}}}
\def\bAg{{\mathbb{A}_\mathfrak{g}}}
\def\Fg{{F_{\fgg}}}
\def\fgg{{\mathfrak{g}}}
\def\bZ{{\mathbb{Z}}}
\def\dg{{d_{\fgg}}}
\def\End{{\mathrm{End}}}
\def\Str{\mathop{\mathrm{Str}}}
\def\kw#1{{\em #1}}
\def\ch{\mathrm{ch}}
\def\Ah{{\hat{A}}}
\def\chg{\ch_\fgg}
\def\Ahg{\Ah_\fgg}

\title{Index}
\author{Ma Jia Jun, HT071232M}


\begin{document}
\maketitle


\section{Notation and useful facts}
Let $G$ be a Lie group and $\fgg$ is the Lie algebra of $G$.
\begin{Def}
We say that $\pi:\cE\to M$ is a  $G$-equivariant fiber bundle if $G$ act smoothly on left both on $\cE$ and $M$ such that 
\[
\gamma \circ \pi = \pi \circ \gamma \quad \forall \gamma\in G.
\]
When $\cE$ is a vector bundle, we require in addition that the $G$
action on fiber, $\gamma^\cE:\cE_x \to \cE_{\gamma\cdot x}$,  is linear.

Now $G$ act on the space of sections $\Gamma(M,\cE)$ by 
\[
(\gamma\cdot s)(x) = \gamma^\cE\cdot s(\gamma^{-1}\cdot x) \quad
\forall x\in M, s\in \Gamma(M,\cE). 
\]
For $X \in \fgg$, we denote $\cL^\cE(X)$ the infinitesimal action of
$\Gamma(M,\cE)$, 
\[
\cL^\cE(X) s =\left.\frac{d}{dt}\right|_{t=0} exp(tX)\cdot s,
\]
which is called the Lie derivative.
\end{Def}
For any vector field $X$ on $M$, it defines a famliy of diffeomorphisms $\phi_t$
on $M$ for sufficient small $t$. This diffeomorphism will induce the
Lie derivative on tensor bundles, denote by $\cL(X)$. 

Now consider the bundle of exterior differentials $\bigwedge
T^*M$, denote $\cA(M)=\Gamma(M,\bigwedge T^*M)$  the space of
differential forms, which is a graded algebra by nature grading
inherited from $\bigwedge T^*M$.
Denote $\cA(M,\cE)=\Gamma(M,\bigwedge T^*M\otimes \cE)$  the space
of $\cE$ valued differential forms. We can take $\cE$ be the trivial
line bundle on $M$, then $\cA(M) \cong \cA(M,\cE)$ as $\cA(M)$ module.  

For any differential form $\alpha$, denote the
left exterior multiplication  by $\alpha$ operator on $\cA(M, \cE)$ by
$\varepsilon(\alpha)$.
For any vector field $X$ on $M$, denote the contraction map induced
from the natural paring between $TM$ and $T^*M$ by 
$\tau(X):\cA^\bullet(M,\cE)\to
\cA^{\bullet-1}(M,\cE)$. Also denote the exterior differential
operator by $d:\cA^\bullet(M,\cE) \to \cA^{\bullet+1}(M,\cE)$. The
  following E.Cartan's homotopy formula is useful:
\[
\cL(X) = d\circ \tau(X) + \tau(X)\circ d.
\]
For any $X\in \fgg$ it induce a vector feild $X_M$ on $M$, we will
abuse the notation still call this vector feild $X$.

\section{Equivariant vector bundle}


\section{Equivariant Cohomology}
Let $G$ be a Lie group. 
$\cE$ be a equivariant vector bundle. 

Consdier the space $\bC[\fgg]\otimes \cA(M,\cE)$, which is the space of polynomial maps from $\fgg$ to $\cA(M,\cE)$. View this space as a $G$-space by action
\[
(g\cdot \alpha)(X) = g\cdot (\alpha (g^{-1}\cdot X)) \quad \forall g\in G,
X\ in \fgg, \alpha \in \bC[\fgg]\otimes \cA(M,\cE).
\]
Give a $\bZ$-grading of this space by 
\[
\deg(P\otimes \alpha) = 2\deg(P)+\deg(\alpha), 
\quad \forall P\in \bC[\fgg], \alpha \in \cA(M,\cE)
\]

\begin{Def}
Let 
\[
\cA_G(M,\cE) = \left(\bC[\fgg]\otimes \cA(M,\cE)\right)^G
\]
be the $G$-invariant subspace of $\bC[\fgg]\otimes \cA(M,\cE)$, vectors in this space called equivariant differential forms with value in $\cE$.

Define operator $\dg$ on $\bC[\fgg]\otimes \cA(M,\cE)$ by
\[
\dg \alpha(X) = d(\alpha(X)) - \tau(X)(\alpha(X)).
\]
It can be show that $\dg$ increase the degree by one and restrict on 
$\cAg(M)$ we have $\dg^2= 0$. Hence we can define
the equivariant cohomology $H_G^*(M)$ to be the cohomology of complex $\left(\cAg(M),\dg\right)$.
\end{Def}

\begin{Rmk}
By homotopy formula,
\[
(\dg^2 \alpha)(X) = - \cL(X)\alpha(X)\quad \forall \alpha
\in \bC[\fgg]\otimes \cA(M)
\]
The infinitesimal condition to define a $G$-equivariant form is 
\[
\cL(Y)\alpha(X) - \alpha([Y,X]) = 0 \quad \forall X,Y \in \fgg.
\]
Hence $\dg^2 = 0$ on $\cAg(M)$.
\end{Rmk}

\begin{Def}
Suppose a superbundle $\cE = \cE^+\oplus \cE^-$ 
with both even and odd part are $G$-equvariant.
A superconnection on $\cE$ called $G$-invariant if 
\[
[\bA,\cL^\cE(X)]=0.
\]
The equivariant superconnection $\bAg$ is the operator given by 
\[
(\bAg \alpha)(X) = (\bA-\tau(X))(\alpha(X)) \quad \forall  \bC[\fgg]\otimes \cA(M,\cE).
\]
It satisfies \[
\bAg(\alpha\wedge \theta) = \dg \alpha \wedge \theta +
(-1)^{|\alpha|}\alpha\wedge \bAg\theta \quad \forall\alpha \in \cA(M),
\theta \in \cA(M,\cE).
\]
Define the \kw{equivariant curvature} $\Fg$ by
\[
\varepsilon(\Fg(X))\alpha(X) = \bAg^2\alpha(X) + \cL^\cE(X)\alpha(X),
\] 
which can be show is in $\cAg^+(M,\End(\cE))$.
Moreover the \kw{equivariant Bianchi formula} holds:
\[
\bAg\Fg \triangleq [\bAg,\varepsilon(\Fg)] = 0.
\]
Define the \kw{moment} of $X\in \fgg$ by 
\[
\mu(X) = \cLE(X) - [\tau(X),\bA] = \Fg(X) - F \in \cAg^+(M,\End(\cE)),
\]
where $F = \bA^2$ is the usual curvature for superconnection $\bA$.

If $f(z)$ is a polynomial in the indeterminate $z$, we define an
\kw{equivariant characteristic form} by taking supertrace,
i.e. $\Str(f(\Fg)) \in \cAg^+(M)$. In paticular, we have
\kw{equivariant Chern character} 
\[\chg(\bA) = \Str(\exp(-\Fg))
\]
 and
\kw{equivariant $\Ah$-genus} 
\[
\Ahg(\nabla)(X) =
{\det}^{1/2}\left(\frac{\Fg(X)/2}{\sinh(\Fg(X)/2)}\right)
\]
for $G$-equvariant connection $\nabla$.
\end{Def}

\begin{Rmk}
We extend $\cL^\cE(X)$ to act on $\cA(M,\cE)$ by 
\[
\cLE(X)(\alpha\wedge \theta) =\cL(X) \alpha \wedge \theta +\alpha
\wedge \cLE(X)\theta \quad \forall \alpha\in \cA(M), \theta\in \cA(M,\cE).
\]

As in the non-equivariant case, we have $\Str(f(\Fg))$ is
equivariantly closed, i.e. $\dg \Str(f(\Fg)) = 0$ and its equivariant
cohomology class is independent of the choice of the $G$-invariant
superconnection $\bA$.
\end{Rmk}


\subsection{symplectic manifold}
A symplectic manifold is a manifold $M$ with symplectic two form
$\Omega$, where $\Omega\in \cA^2(M)$ such that $d\Omega=0$ and
the bilinear form $\Omega_x(X,Y)$ on $T_xM$ is non-degenerate.
For any $f\in C^\infty(M)$ one can define the \kw{Hamiltonian vector
  field} generated by $f$ by 
\[
df = \tau(H_f)\Omega.
\] 
$G$ acts on $M$ Hamiltonian means that there is a $G$-equivariant linear
map $\mu:\fgg\to C^\infty(M)$ such that 
\[
d\mu(X) = \tau(X) \Omega \quad \forall X\in \fgg.
\]
One define the \kw{symplectic moment map} of the action be the
$C^\infty$ map $\mu\colon M\to \fgg^*$ defined by $(X,\mu(m)) =
\mu(X)(m)$.

\subsection{Coadjoint Orbit}
The most importent example of symplectic manifold with Hamiltonian $G$
action is the Coadjoint Orbit. 
Let $\fgg^*$ be the dual vector space to the vector space of Lie
algebra $\fgg$ of $G$. $G$ act on $\fgg$ by adjoint action and on
$\fgg^*$ by corresponding dual action.

Let $M_\lambda$ be an orbit of coadjoint represention of $G$ on $\fgg^*$.

\subsection{The Localization Formula}

\begin{Thm} Let $\alpha$ be an equivariantly closed differential form
  on $M$. Assume $X_M$ has only isolated zeros. Then
\[
\int_M \alpha(X) = (-2\pi)^{\dim(M)/2}\sum_{p\in M_0(X)}\frac{\alpha(X)(p)}{\det^{1/2}(L_p)},
\]
where $\alpha(X)(p)$ means the value of the function $\alpha(X)_{[0]}$
at point $p\in M$.
\end{Thm}

\section{The Fourier Transform of Coadjoint Orbits}
The 

\end{document}
