\documentclass[12pt]{amsart}
\usepackage[margin=3cm]{geometry}
\usepackage[hyperindex=true]{hyperref}

\usepackage{amssymb}
\usepackage{amsxtra}
\usepackage{graphicx}

\usepackage{braket}
\usepackage{paralist}
\usepackage{eufrak}
\usepackage{eucal}

\usepackage{dsfont}

\newtheorem{thm}{Theorem}
\newtheorem{lem}[thm]{Lemma}
\newtheorem{prop}[thm]{Proposition}
\newtheorem{cor}[thm]{Corollary}

\linespread{1.5}

\title{$L^2$ duality for stable reductive dual pairs}
\author{Roger Howe}

\def\inn#1#2{\left\langle{#1},{#2}\right\rangle}
\def\abs#1{\left|{#1}\right|}
\def\emp#1{\underline{\em #1}}
\def\Sp{{\mathrm{Sp}}}
\def\tSp{{\widetilde{\mathrm{Sp}}}}
\def\GL{{\mathrm{GL}}}
\def\tGL{{\widetilde{\mathrm{GL}}}}
\def\Hom{{\mathrm{Hom}}}
\def\End{{\mathrm{End}}}
\def\im{{\mathrm{im\,}}}
\def\tP{{\widetilde{P}}}
\def\tM{{\widetilde{M}}}
\def\tG{{\widetilde{G}}}
\def\tg{{\widetilde{g}}}
\def\tm{{\widetilde{m}}}
\def\htG{{\widetilde{G}}}
\def\tH{{\widetilde{H}}}
\def\htH{\widehat{\widetilde{H}}}
\def\kk{\kappa}
\def\inv{{\natural}}
\def\ind{{\mathrm{ind}}}
\def\tr{{\mathrm{tr\,}}}
\def\rank{{\mathrm{rank\,}}}
\begin{document}
\maketitle

 Let $F$ be a local field not of characteristic $2$. 
Let $w$ be a vector space over $F$ equipped with a symlectic form 
$\inn{}{}$. 
Let $\Sp(W,\inn{}{})=\Sp$ be the isometry group of $\inn{}{}$.
Let $(G,G')$ be an irreducible reductive dual pair in $\Sp$. 
(See \cite{Howe1979} for terminology.) From the classification 
\cite{Howe1979} of such pairs, we know there is division algebra $D$ over
$F$ (not necessarily central), with an ivolution $\inv$ 
(which reduces to the identity in the case of a type II pair), 
and vector spaces $U$ and $U'$ over $D$, with $\inv$-Hermitian 
or $\inv$-anti Hermitian form $(\, ,\, )$ and $(\, , \,)'$
(which are non-degenerate in the type I case and trivial in the 
type II case), such that $G$ and $G'$ are isomorphic to the isometry groups 
of these forms (and they are simply $\GL(U)$ and $GL(U')$  respectively in 
the type II case). For convenience we may assume $\dim U \geq \dim U'$,
and we will refer to $G$ as the larger number of the pair, and 
to $G'$ as the smaller member. We will call $(G,G')$ stable if, 
in the type I case $(\,,\,)$ admits isotropic subspaces of dimension at least
$\dim U'$, or in the type II case, if simply $\dim_D U\geq 2\dim_D U'$
(this condition being implicit in the type I case).

The duality phenomena exhibited by reductive dual pairs \cite{Howe1979}
are considerably simpler for stable pairs than in general,
and computations are easier to make in the range of dimensions implied
by stability (\cite{Gelbart1979}, \cite{Gross1977}, \cite{Weil1965}). 
The main purpose of this paper is to illustrats this assertion convincingly
by giving a relatively simple proof of $L^2$ duality for stable pairs.
Let $\tSp$ be the two-fold cover of $\Sp$, and let $\omega$ be a fixed
oscillator representation of $\tSp$. then $\omega$ is a unitary representation 
of $\tSp$ acting on some Hilbert space to be specified later. 
Let $\kk$ be the kernel of the projection from $\tSp$ to $\Sp$. 
Then $\kk$ has oreder $2$ and $\omega|_\kk$ is a multiple of the unique
non-trivial character $\epsilon$ of $\kk$.
If $H\subseteq \Sp$ is a subgroup, Let $\tH$ be the inverse image of $H$
in $\tSp$. Then $\tH$ projects on to $H$ with kernel $\kk$ and 
any representation of $\tH$ occuring in $\omega|_{\tH}$ will 
restrict to a multiple of $\epsilon$ on $\kk$. 
Denote by $\htH_t(\epsilon)$ the subset of the unitary dual of 
$\tH$ whose elements restrict to multiples of $\epsilon$ on $\kk$.
Let $\htH_t(\epsilon)$ be the support of the Plancherel measure in 
$\htH(\epsilon)$. We will call $\htH_t(\epsilon)$ the tempered
dual of $\tH$. It may also be described as the set of (irreducible unitary)
representations of $\tH$ weakly contained in $\ind_\kk^{\tH}\epsilon$.
The representation $\ind_\kk^{\tH}\epsilon$ is roughly half of 
the regular representation of $\tH$.
We will refer to it as the $\epsilon$-regular representation.

Our main result is 

\begin{thm}\label{thm:1}
Let $(G,G')\subseteq \Sp$ be a stable irreducible reductive dual pair. 
Then the von Neumann algebras generated by $\omega(\tG)$ and by $\omega(\tG')$
are mutual commutants. 
More precisely, suppose $Y\subseteq U$ is an isotropic subspace such that
$Y$ and $U\setminus Y$ have dimension $\geq \dim U$. Let $P$ be 
the parabolic subgroup of $G$ preserving $Y$. 
Then $\omega(\tP)$ already generates the commutant of $\omega(\tG')$. 
In particular almost all the representations of $\tG$ 
occurring in $\omega|_\tG$ are already irreducible on $\tP$.

Moreover, $\omega|_\tG'$ is quasi-equivalent to the $\epsilon$-regular 
representation of $\tG'$. Therefor we have the direct integral decomposition
\begin{equation}\label{eq:1}
\omega|_{\tG\cdot\tG'}\cong \int_{\widehat{\tG}_t(\epsilon)}\sigma\otimes \sigma' 
d\mu(\sigma')
\end{equation}
where $\sigma'$ ranges over $\widehat{\tG'}_t(\epsilon)$ and $\sigma$
varies in $\widehat{\tG}_t(\epsilon)$, and $d\mu$ in Plancherel measure on 
$\widehat{\tG'}_t(\epsilon)$. In particular, the discrete spectrum of 
$\omega|_{\tG\cdot\tG'}$ is parametrized by the discrete series of $\tG'$
which restrict to $\epsilon$ on $\kk$. The correspondence
\[
\sigma'\to\sigma 
\]
defined by $\ref{eq:1}$ is an injection of $\htG'_t(\epsilon)$ 
into $\htG(\epsilon)$ up to a set of Plancherel measure zero.
\end{thm}

{\bf Remark:} there seems a good chance that the discrete spectrum of 
$\omega|_{\tG\cdot \tG'}$ would give rise to representations constructed 
infinitesimally by an ``orbit quantization'' procedure using sheaf cohomology 
by G. Zuckerman. If so, then $\omega$ would provide global realizations of
these representations and establish their unitarity.

\proof: The procedure of the proof is to consider a convenient particular
realization of $\omega$ in which the operators coming from
$G$ and $G'$ are easy to analyze. Eventrually we can reduce the proof to the
classical fact \cite{Dixmier1982} that the right and left actions of a 
group on its own $L^2$-spaces are mutual commutants.

The cases when $(G,G')$ is type II or type I with split $G$ are 
easier than the general stable case. Since type I and type II must be
treated separately anyway, we begin with type II for purposes 
of illustrating the general idea. So let $(G,G')$ be an irreducible 
type II pair, and let $D$ be the associated division algebra. 
It will be conveniant to alter slightly the description of 
$(G,G')$ given in \cite{Howe1979}, by mixing the contragredient 
and co-gredient actions of $G$ and $G'$. 
Specifically, we can find two left $D$ vector spaces $U$ and $U'$ such that 
\begin{equation}\label{eq:2}
W\cong \Hom_D(U,U')\oplus\Hom_D(U',U)
\end{equation}
in such fashion that $G\cong \GL(U)$ acts by permultiplication 
(by $g^{-1}$) on $\Hom_D(U,U')$ and by postmultiplication
on $\Hom_D(U',U)$, and vice verse for $G'\cong \GL(U')$. 
The form $\inn{}{}$ is given by 
\begin{equation}\label{eq:3}
\inn{(S_1,T_1)}{(S_2,T_2)} = \tr T_1S_2 - \tr S_1T_2
\end{equation}
where $S_i\in \Hom_D(U,U')$ and $T_i\in \Hom_D(U',U)$, and 
where $\tr$ is the trace map of $\End_D(U)$ or $\End_D(U')$
as algebra over $F$. 

We have supposed $\dim_DU\geq 2\dim_D U'$. choose in $U$ two subspaces
$U_1$ and $U_2$ such that $\dim U_i \geq \dim U'$ and $U=U_1\oplus U_2$. 
Set 
\begin{equation}\label{eq:4}
\begin{split}
X &= \Hom_D(U_1, U') \oplus \Hom_D(U',U_2) \\
Y &= \Hom_D(U_2, U') \oplus \Hom_D(U',U_1) \\
\end{split}
\end{equation}
We see that $(X,Y)$ form a complete polarization for $W$, and we can 
form the Schrodinger model \cite{Cartier1966}, \cite{Gelbart1979} of 
$\omega$ associated to $(X,Y)$. The space on which $\omega(\tSp)$
acts is then $L^2(X)$.

Since $G'$ preserves both $X$ and $Y$, the action of $\omega(\tG')$ will
come from the linear action of $G'$ on $X$ (obtained by restricting the
action of $G'$ on $W$). Then \cite{Gelbart1979} for $\tg\in \tG'$, 
with image $g\in G'$,
\begin{equation}\label{eq:5}
\omega(\tg)(f)(x) = \sigma(\tg)\abs{\det g}^b f(g^{-1}(x))
\quad x\in X, f\in L^2(X)
\end{equation}
where $\sigma(\tg)$ is an eighth root of unity, and 
$b = 2^{-1}(\dim U_2 -\dim U_1)$, and $\abs{\det g}$ 
is most conveniently defined by
\begin{equation}\label{eq:6}
\abs{\det g}\int_{U'} f(g(u')) du'=\int_{U'}f(u') du'
\end{equation}
where $du'$ is Haar measure on $U'$.

The isotropy group $M$ of $X$ and $Y$ is $G$ is just the isotropy group
of $U_1$ and $U_2$, and is isomorphic to $\GL(U_1)\times\GL(U_2)$.
$\tM$ will also act on $L^2(X)$ esentially through the action of $M$ on $X$, 
by a formula analogous to $(\ref{eq:5})$. 
The isotropy group in $G$ of $Y$ alone is a parabolic subgroup $P$ of $G$ 
containing $M$ as Levi component. Let $N$ be the unipotent redical of $P$. 
Then $P=M\cdot N$, and $N\cong \Hom_D(U_2,U_1)$. 
Also $N$ may be lifted uniquely to a subgroup of $\tSp$, so 
we will not speak of $\widetilde{N}$, but regard $\omega$ directly 
as a representation of $N$, which we identify with 
$\Hom_D(U_2,U_1)$. In our present realization of $\omega$, the group 
$\omega(N)$ consists of multiplication operators. 
Explicitly, 
\begin{equation}\label{eq:7}
\omega(n)f(x) = \chi(\frac{1}{2}B_n(x,x))f(x)\quad n\in N
\end{equation}
where $\chi$ is some unitary character of (the additive group of) $F$
determined by $\omega$, and $B_n$ is a symmetric bilinear form on $X$, 
defined by 
\begin{equation}\label{eq:8}
B_n((S_1,T_1),(S_2,T_2)) = \tr(S_1n T_2 + S_2n T_1) 
\end{equation}
where $S_i\in \Hom_D(U_1,U')$ and $T_i\in \Hom_D(U',U_2)$, and 
$\tr$ is, as above is (\ref{eq:3}), the trace on $\End_D(U')$
as algebra over $F$.

Having described the actions of $\omega(\tG')$ and of $\omega(\tP)$
on $L^2(X)$, we next investigate how they are related. 
The action of $G'$ on $X$ partitions $X$ into $G$-orbits and 
it is clear from $(\ref{eq:5})$ that $\omega|_{\tG'}$ decomposes into
a direct integral of what are essentially permutation 
representations on the various $G$-orbits. 
These orbits may be described as follows. 
Given $(S,T)$ in $X$, with $S\in \Hom_D(U_1,U')$ and 
$T\in \Hom_D(U',U_2)$, we see that the subspace $\ker S$ of
$U_1$ and the subspace $\im T\in U_2$ are invariant under $G'$, 
and $TS$ is the the map 
\[
TS: U_1/\ker S \to \im T.
\]
Conversely, Witt's Theorem (see \cite{Jacobson1953} and \cite{HoweOsc1}
says these $3$ invariants:
$\ker S$, $\im T$ and $TS$, characterize the $G'$ orbit of $(S,T)$ in $X$. 
In particular, we get a mapping 
\begin{equation}\label{eq:9}
\begin{split}
\tau:X/G' &\to \Hom_D(U_1,U_2)\\
\tau(S,T) &= TS
\end{split}
\end{equation}
from $G'$ orbits to $\Hom_D(U_1,U_2)$. We will call $\tau$ the 
\emp{orbit parameter map}.

Since $\tau(S,T)$ factors through $U'$, it can have rank at most
$\dim U'$. If $\tau(S,T)$ has rank exactly $\dim U'$, we will call 
$(S,T)$ \emp{generic}. The $G'$ orbit of a generic point
will be called a \emp{generic $G'$-orbit}.
evidently $(S,T)$ will be generic if and only if $\rank 
ST=\rank S = \rank T =\dim U'$ so that $S$ is surjective and $T$ 
is injective. It follows that $\ker S = \ker TS$ and $\im T=\im TS$, 
so that the orbit parameter map is a bijection from the generic $G'$-orbits
to the subvariety of $\Hom_D(U_1,U_2)$ of maps of rank $\dim U'$. 
Also, if we fix a surjection $S_0:U_1\to U'$, then 
the points $(S_0,T)$ form a cross-section to the generic $G'$-orbits 
whose orbit parameter has fixed kernel equal to $\ker S_0$ as
$T$ ranges over the injections from $U'$ to $U_2$. Also the generic points
are open and form a set of full measure in $X$, 
i.e., its complement is of measure zero and closed. 
Finally, observe that $G'$ acts freely on 
each generic orbit. Hence $\omega|_{\tG'}$ is seen to be 
the direct integral over the generic $G'$-orbits of copies of 
the $\epsilon$-regular representation of $\tG'$.
In particular, $\omega|_{\tG'}$ is qusi-equivalent to the $\epsilon$-regular
representation of $\tG'$. 

Next consider $\omega(\tP)$. The space $\Hom_D(U_1,U_2)$, the range of
the orbit parameter maps, is canonically the dual vector space
to $N\cong \Hom_D(U_2,U_1)$. 
By a standard construction, given a unitary character $\chi$ of $F$,
one may identify $N$ to the Pontrajagin dual of $\Hom_D(U_1,U_2)$, 
and in particular, points of $N$ to characters of $\Hom_D(U_1,U_2)$, 
by the recipy
\begin{equation}\label{eq:10}
\psi_n(x) = \chi(\tr(xn)) \quad n\in N, x\in \Hom_D(U_1,U_2)
\end{equation}
let $\chi$ be as in $(\ref{eq:7})$. Then the formual $(\ref{eq:7})$,
$(\ref{eq:8})$ and $(\ref{eq:9})$ say that the multiplication operator 
$\omega(n)$ is simply multiplication $\psi_n\circ \tau$.
In particular, since $\tau$ is injective on the generic $G'$-orbits, 
the multiplication operators of $\omega(N)$ separate these orbits.

The effect of $M$ acting on $X$ is to permute the $G'$-orbits
among themselves. 
Since $M\cong \GL(U_1)\times \GL(U_2)$, it acts in an obvious way on 
$\Hom_D(U_1,U_2)$. It is simple to check that 
the orbit parameter map $\tau$ of $(\ref{eq:9})$ is equivariant 
for the actions of $M$ on domain and range. 
We may also observe that $M$ acts transitively on the set of generic points, 
from which we concluded that:
\begin{enumerate}[i)]
\item $M$ acts transitively on the set of generic $G'$-orbits, and
\item the isotropy group in $M$ of a given generic $G'$-orbit acts 
transitively on the orbit.
\end{enumerate}

We now have the following data. 
The representation $\omega|_{\tG'}$ is the direct integral of 
copies of the right $\epsilon$-regular representaion over a 
certain parameter space. The group $\omega(\tP)$ provides multiplicatoin 
operators separating points in the parameter space and 
permutation operators permuting transitively the points of the parameter space, 
and such that the isotropy group at any point generates the 
left $\epsilon$-regular representation which, 
as is well-known \cite{Dixmier1982}, generates the full commutant of
the right regular representation. It is not hard to convince oneself
(it is a straightforward exercise of moderate length to deduce) 
from this data that $\omega(\tP)$ indeed generates the full commutant of 
$\omega(\tG')$. One first shows, by considering sums of the permutation 
operators truncated by characteristic functions,
that the full algebra generated by multiplications on 
the parameter space and the left $\epsilon$-regular representation 
at each point is in the algebra generated by $\omega(\tP)$. Then
the full commutant of $\omega(\tG')$ is generated by this algebra 
and the permutations in $\omega(\tP)$, since these act 
transitively on the parameter space. we omit the details. 
This concludes the proof of Theorem~\ref{thm:1} in the type II stable case.


We proceed to consider the type I case. If the form $(\,,\,)$ on $U$ 
is completely split, i.e., is a sum of hyperbolic planes, 
the proof proceeds in almost complete analogy with the proof for 
type II pairs, but in general it is slightly more involved. 
The computations take place not in a Schr\"odinger model, 
but in a mixed Fock-Schr\"odinger model.
These mixed models have received less attention then pure 
Fock or Schrodinger models, so for reference we will write down some
of the basic formulas for realizing $\rho$ by these mixed models.

Recall that $W$ is our basic symplectic vector space. Let $Y\subseteq W$
be an isotropic subspace, not necesarily maximal. 
Let $Y^\perp$ be the annihilator in $W$ with respect to 
$\inn{ }{ }$. Let $X$ be an isotropic complement to $Y^\perp$ in $W$, 
and let $W_0$ be the orthogonal complement of $X\oplus Y$.
We then have the direct sum decomposition
\begin{equation}\label{eq:11}
W = X\oplus W_0\oplus Y.
\end{equation}

Let $P$ be the parabolic subgroup of $\Sp = \Sp(W,\inn{}{})$ 
where elements preserve $Y$. 
Let $M\subseteq P$ be the subgroup whose elements also preserve $X$,
hence $W_0$.  Then $M$ is a Levi component for $P$, 
and $P=MN$ where $N$ is the unipotent radical of $P$. 
We may also describe $N$ as the subgroup of $P$ which acts 
trivially on $Y$, on $Y^\perp/Y$, and on $W/Y^\perp$.
By contrast, the restriction of $M$ to $X\oplus W_0$ is faithful, 
and induce an isomorphism 
\begin{equation}\label{eq:12}
\alpha: M\stackrel{\sim}{\to} \GL(X)\times \Sp(W_0)
\end{equation}
where $\Sp(W_0)$ is, as you would expect, the isometry group of 
the restriction of $\inn{}{}$ to $W_0$. (This restriction is non-degenerate.)

The group $N$ is two step nilpotent. The center $Z(N)$ of $N$ is the subgroup
of $N$ that acts trivially on $Y^\perp$, or equivalently on $W/Y$.
It is canonically isomorphic to $B^2(X)$, the space of symmetric 
bilinear forms via the mapping
\begin{equation}\label{eq:13}
\begin{split}
\beta: z &\mapsto \beta_z\\
\beta_z(x_1,x_2)&=\inn{zx_2}{x_1}=\inn{x_1}{(1-z)x_2}
\end{split}
\end{equation}
The quotient $N/Z(N)$ is canonically isomorphic to $\Hom(X,W_0)$, 
by the recipe
\begin{equation}\label{eq:14}
\begin{split}
\gamma:n & \mapsto \gamma_n\\
\gamma_n(x)& = nx-x \text{ modulo } Y
\end{split}
\end{equation}
There is also a convenient cross section to $\gamma$, 
which by abuse of notation we will denote $\gamma^{-1}$. 
Write a general element in $W$ as $w=(x,w_0,y)$
according to the decomposition $(\ref{eq:11})$.
Vie $\inn{}{}$, the dual of of $X$ is identified to $Y$, 
and $W_0$ is self-dual. 
Thus given a map $T\in \Hom(X,W_0)$, 
we may consider the adjoint $T^*$ to be the map from $W_0$ 
to $Y$ satisfying
\[
\inn{Tx}{w_0} +\inn{x}{T^*w_0} = 0.
\]
With this conventio, define for such $T$
\begin{equation}\label{eq:16}
\gamma^{-1}(T)(x,w_0,y) = (x,w_0+T(x),y+T^*(w_0)+ \frac{1}{2} T^*T(x)).
\end{equation}
Then $\gamma^{-1}(T)$ is in $N$ and $\gamma(\gamma^{-1}(T))=T$, 
as one easilly verifies. 

By vitrue of the decomposition we have the product 
$H=\Sp(W_0)\times \Sp(X\oplus Y)$ embedded as a subgroup 
of $\Sp(W)$. The restriction of the oscillator representation 
$\omega$ to $\tH$ and pulled back to $\tSp(W_0)\times \tSp(X\oplus Y)$
is just the outer tensor product of oscillator representations $\omega_1$
of $\tSp(W_0)$ and $\omega_2$ of $\tSp(X\oplus Y)$. The mixed model 
of $\omega$ we will be working 
with has as space the tensor product of the space of $\omega_1$ realized by 
an anisotropic or Fock model and the space of the Schr\"odinger model 
for $\omega_2$ corresponding to $(X,Y)$. 
The space of $\omega_q$ is an appropriate subspace $F$ of $L^2(W_0)$. 
(See \cite{HoweOsc2}) Its precise form will not be of great concern to us. 
The space of $\omega$ will thus be $L^2(X)\otimes F \cong L^2(X,F)$, 
the space of $F$-valued functions on $X$ with square integrable norm. 
The subgroups $\Sp(W_0)$, $\GL(X)$ and $Z(N)$ are all subgroups
of $\Sp(W_0)\times \Sp(X\oplus Y)$, so they act according to their
known action on one or the other factors of $\omega$. 
We record the formulas.
\begin{equation}
\begin{split}
a)\; & \omega(\tg)(\phi)(x)=\omega_1(\tg)(\phi(x))\quad \tg\in \tSp(W_0), 
x\in X, \phi\in L^2(X,F)\\
b)\; & \omega(z)(\phi)(x) = \chi(\frac{1}{2}B_z(x,x))\phi(x)\quad z\in Z(N)\\
c)\; & \omega(\tm)(\phi)(x) = \sigma(m)\abs{\det m}^{-1/2}\phi(m^{-1}(x))
\quad \tm\in \tGL(X)
\end{split}
\end{equation}
where $\sigma$, $\abs{\det m}$ and $\chi$ have essentially the same 
meanings as in $(\ref{eq:5})$, $(\ref{eq:6})$ and $(\ref{eq:7})$.

The new and interesting feature of this mixed model is a formula for 
the ation of a general element of $N$. By our discussion above, 
we see it will be enough to write this formula for the 
elements $\gamma^{-1}(T)$, where $T\in \Hom(X,W_0)$. 
There is a map $\rho$ of $W_0$ into the group of unitary operators of $F$ 
satisfying
\begin{equation}
\begin{split}
\rho(w_1)\rho(w_2)=&\rho(w_1+w_2)\chi(\frac{1}{2}\inn{w_1}{w_2}) 
\quad w_i \in W_0\\
\omega_1(\tg)\rho(w)\omega_1(\tg)^{-1} =& \rho(g(w))\quad \tg \in \tSp(W_0), 
w\in W_0
\end{split}
\end{equation}
It will be recognized that $\rho$ is the restriction to $W_0$ of
the representation with central character $\chi$ of the Heisenberg group
attached to $W_0$. 
See for example \cite{Howe1979}. A computation now shows that for 
$T\in \Hom(X,W_0)$
\begin{equation}\label{eq:19}
\omega(\gamma^{-1}(T))\phi(x) = \rho(-T(x))(\phi(x)).
\end{equation}

With these formulas established, we return to our stable irreducible type I
reductive dual pair $(G,G')$. Let $D$ be the division algebra
associated to $(G,G')$ with involution $\inv$. Let $U$ and $U'$
be the basic $D$-module associated to $G$ and $G'$ respectively, 
and let $(\,,\,)$ and $(\,,\,)'$ be the $\inv$-Hermitian or
anti-Hermitian (one of each) forms on $U$ and $u'$ of 
which $G$ and $G'$ are the isometry groups. Then we may take
\begin{equation}\label{eq:20}
W\cong \Hom_D(U,U')
\end{equation}
Moreover there is a natural identification
\[
\begin{split}
\* \colon \Hom_D(U,U') &\to \Hom_D(U',U)\\
T&\mapsto T^*
\end{split}
\]
defined by the identity
\begin{equation}\label{eq:21a}
(u,T^*u') = (Tu,u')' \quad u\in U, u'\in U', T\in \Hom_D(U,U').
\end{equation}
There is an analogous map 
\[
\begin{split}
\*' \colon \Hom_D(U',U) &\to \Hom_D(U,U')\\
S&\mapsto S^{*'}
\end{split}
\]
defined by the analogous identity
\begin{equation}\label{eq:21b}
(u',S^*u)' = (Tu',u).
\end{equation}
Using the fact that one of $(\,,\,)$ and $(\,,\,)'$ is $\inv$-Hermitian and
the other $\inv$-anti-Hermitian we see that 
\begin{equation}
\begin{split}
T^{**'} &= -T \quad T\in \Hom_D(U,U')\\
S^{*'*} &= -S \quad S\in \Hom_D(U',U)
\end{split}
\end{equation}


\bibliographystyle{alpha}
\bibliography{bib/reppapers}

\end{document}
