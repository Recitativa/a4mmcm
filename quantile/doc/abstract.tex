\chapter{Abstract}
Options are one of the most popular derivatives traded in the market. 
%So evaluate the value of options is
%a very important question in finicial industry.
%However the return of options can only be determined when it is exercised.
%So option pricing became a very difficult problem.
In the Black-Scholes model, one of the most important models for option pricing, the price of the underlying asset is assumed to satisfy the geometric Brownian motion, which is a very interesting object itself. In this thesis, we provide a quick review about the theory of option pricing in the first part of this thesis. It is well known that the Black-Scholes formula solves the pricing problem for a European option. However, for other types of options, usually there is no such closed-form formula. Numerical methods are developed to solve these problems. In this thesis, we consider the $\alpha$-quantile options. Partially because the $\alpha$-quantiles of a Brownian motion are highly path-dependent, many fundamental problems are still open. One problem is the discretization error between the $\alpha$-quantile of a Brownian motion and that of the Gaussian random walk. This is the first step to connect the price of continuously and discretely monitored $\alpha$-quantile options. We have found a difference between the strong order of convergence of the discretization error for genuine $\alpha$-quantiles $(0< \alpha < 1)$ and that for the maximum ($\alpha=1$) by simulation. Another problem is with the pricing of $\alpha$-quantile options. Although the risk-neutral pricing formula for European-style $\alpha$-quantile options is given in \cite{Dassios1995}, it still needs a numerical method such as the forward shooting method proposed by \cite{Kwok2001} and a Monte Carlo method proposed by \cite{Laura2001}. However, these existing methods cannot be extended to price the American-style $\alpha$-quantile options. In this thesis, we propose a tree method which, to our knowledge, is the first solution to price American-style $\alpha$-quantile options. We show how Richardson extrapolation can be applied to improve the accuracy of our lattice method. 
\vspace{1cm}

\noindent{\bf Keywords}: Option Pricing, $\alpha$-quantile Options, Discretization Error, Tree Method, Euler Scheme, Richardson Extrapolation
