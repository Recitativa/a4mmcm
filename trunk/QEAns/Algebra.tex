\documentclass[12pt]{amsart}
\usepackage[margin=3cm]{geometry}
\usepackage{hyperref}

\usepackage{amssymb}
\usepackage{graphicx}
%\usepackage{amscd}
\usepackage{braket}
\usepackage{paralist}
\usepackage{eufrak}
%\usepackage{calrsfs}
%\usepackage[small,nohug,heads=littlevee]{diagrams}
%\diagramstyle[labelstyle=\scriptstyle]
%\usepackage{diagrams}
\usepackage{amscd}
%\usepackage{pictexwd,dcpic}
%\usepackage{mathrsfs}

\newtheorem{exec}{Question}[section]
\newenvironment{subproof}{\newline\footnotesize\it}{\normalsize}
\DeclareMathAlphabet{\mathpzc}{OT1}{pzc}{m}{it}


\def\Ker{\rm{Ker}}
\def\Im{\rm{Im}}
\def\Hom{\rm{Hom}}
\def\Mat{\rm{Mat}}
\def\bR{{\mathbb{R}}}
\def\bN{{\mathbb{N}}}
\def\bZ{{\mathbb{Z}}}
\def\bC{{\mathbb{C}}}
\def\bQ{{\mathbb{Q}}}
\def\vv{{\vec{v}}}
\def\vw{{\vec{w}}}
\def\vx{{\vec{x}}}
\def\vy{{\vec{y}}}
\def\v0{{\vec{0}}}
\def\ol{\overline}
\def\sspan{\rm{span}}
\def\sl2{{\mathfrak{sl}(2)}}
\def\slc{{\mathfrak{sl}(2,\bC)}}
\def\sP{\mathcal{P}}
\def\sH{\mathcal{H}}
\def\sU{\mathcal{U}}
\def\sC{\mathcal{C}}
\def\kbar{{k\makebox[-0.06em][r]{\raisebox{0.5ex}{--}}}}
\def\ad{{\rm ad}}
\def\Ad{{\rm Ad}}
\def\id{{\rm id}}
\def\sgn{{\rm sgn}}
\def\gcd{{\rm gcd}}

\hypersetup{
    bookmarks=false,         % show bookmarks bar?
    unicode=false,          % non-Latin characters in Acrobat��s bookmarks
    pdftoolbar=true,        % show Acrobat��s toolbar?
    pdfmenubar=false,        % show Acrobat��s menu?
    pdffitwindow=false,      % page fit to window when opened
    pdftitle={Qualify Examination Answers - Algebra},    % title
    pdfauthor={Ma Jia Jun},     % author
    pdfsubject={Subject},   % subject of the document
    pdfcreator={Creator},   % creator of the document
    pdfproducer={Producer}, % producer of the document
    pdfkeywords={keywords}, % list of keywords
    pdfnewwindow=true,      % links in new window
    colorlinks=true,       % false: boxed links; true: colored links
    linkcolor=blue,          % color of internal links
    citecolor=green,        % color of links to bibliography
    filecolor=magenta,      % color of file links
    urlcolor=cyan           % color of external links
}


\title{Qualify Examination Answers - Algebra }

\begin{document}
\maketitle

\section{Sem2, 2000/2001}
\begin{exec}
Let $G$ be a finite group with a unique maximal subgroup. 
Show that $G$ is cyclic.
\end{exec}
\proof
Let $M$ be the maximal subgroup of $G$. For any $g\in G\setminus M$.
$\braket{g} = G$. Since otherwise $\braket{g}$ should contained in the 
maximal subgroup $M$, a contradiction. 
\qed

\begin{exec}
Let $A$ be a subgroup of index $n$ of a finite group $G$ and let 
\[
\Set{g_1A, g_2A, \cdots, g_nA}
\]
be a set of coset representatives of $G/A$. For each $g\in G$, define
\[
f_g\colon G/A\to G/A
\]
by $f_g(g_i A)=gg_iA$. Prove that $f_g$ is a bijection. Define
$\chi\colon G\to S_n$ by 
\[
\chi(g)=f_g
\]
Prove that $\chi$ is a group homomorphism. Determine the kernel of $\chi$.
\end{exec}
\proof
Since $f_g \circ f_{g^{-1}} = \id_{G/A}$, $f_{g^{-1}} \circ f_g = \id_{G/A}$. $f_g$ is
bijective.
It's easy to see that 
$\chi(gh)(g_iA) = f_{gh}(g_iA) = ghg_iA = f_g\circ f_h(g_iA) 
= (\chi(g)\chi(h))(g_iA)$.
So $\chi$ is a group homomorphism.

$\chi_{g} = 1$ iff $f_g=\id_{G/A}$ iff $gg_iA = g_iA$ for all $g_i$. $g_i$ just 
representative element of $g_iA$. So it's equivalent to
 $ghA = hA$ for all $h\in G$.
 So $\Ker\chi = \Set{g\in G|h^{-1}gh\in A \forall h\in G}$.
\qed

\begin{exec}
Let $R$ be a commutative ring with identity and let $\chi\colon R\to F$ be 
a nontrivial ring homomorphism, where $F$ is an integral domain. Prove
that kernel of $\chi$ is a prime ideal.
\end{exec}
\proof
$F$ is integral domain then $\Im \chi$ is integral domain by the definition.
So $\Ker\chi$ is prime. (If $ab\in \Ker \chi$, 
$\Ker\chi = ab+\Ker\chi = (a+\Ker\chi)(b+\Ker\chi)$. 
So $a+\Ker\chi = \Ker\chi$  or $b+\Ker\chi = \Ker\chi$ by definition
 of integral 
domain. So either $a$ or $b$ in $\Ker \chi$.)
\qed
\begin{exec}\label{Q:5_00b}
Let $V$ be a vector space of finite dimension over a field $F$. Suppose that
$V$ is a integral domain. Prove that $V$ is a field.
\end{exec}
\proof
Note that all right ideal of $V$ is $F$ vector subspace of $V$
($xf = x (1_V f)\in I$ for any right ideal $I$ of $V$ and $x\in I$, $f\in F$). 
Since $V$ is finite dimension, $V$ is right Artinian ring. 
So for any $0\neq a\in V$, exists $k\in\bN$, $b\in V$, s.t. $a^k = a^{k+1}b$(
$(a)\supset (a^2)\supset (a^3) \cdots $ terminate).
Since $V$ is integral domain, $ab = 1_V$. So $V$ is a field. 
\qed

\begin{exec}
Let $E/F$ be a field extension and let $a,b\in E$ be algebraic over $F$. Prove
that every element in $F(a,b)$ is algebraic over $F$. 
\end{exec}
\proof
For any $v\in F(a,b)$, $F(v)\subset F(a,b)$. 
So $[F(v):F] \leq [F(a,b):F] = [F(a)(b):F(a)][F(a),F]\leq [F(b):F][F(a):F] 
<\infty$.
Hence $v$ is algebraic over $F$. 
\qed

\section{Sem1, 2001/2002}
\begin{exec}\label{Q:1_02a}
\begin{enumerate}[(a)]
\item Show that if $R$ is a commutative ring with identity, then
every maximal ideal of $R$ is a prime ideal.
\item Show that if $R$ is a Principal Ideal Domain, then every Prime ideal
of $R$ is a maximal ideal.
\item Give an example of a ring $R$ which has a prime ideal that is not maximal.
\end{enumerate}
\end{exec}
\proof
\begin{enumerate}[(a)]
\item $I$ maximal $\Leftrightarrow$ $R/I$ is field, So $R/I$ is integral domain
$\Leftrightarrow$ $I$ prime.
\item 
  \begin{enumerate}[(i)]
  \item $I=(p)$ is prime iff $p$ is prime
    ($p$ nonunit and $p|ab$ gives $p|a$ or $p|b$).
    It's easy since $p|a\Leftrightarrow a\in (p)$.
  \item $p$ is prime the $p$ is irreducible($r$ nonunit and $r=ab$ gives 
    $a$ or $b$ is unit).\\
    If $p=ab$, then $p|ab$. WLOG, suppose that $p|a$ then $a= ps$. So $p=psb$,
    then $1=sb$ since $R$ is integral domian. So $b$ is unit.
  \item $r$ is irreducible iff $(r)$ is maximal in the set of all proper
    principle ideals.\\
    If $r$ is irreducible, $(r)\subset (s)$. Then $r=sb$. If $s$ is unit, 
    $(s)=R$, if $b$ is unit, $s = rb^{-1}$, i.e. $(s) \subset (r)$. So 
    $(r)$ is maximal in all proper  principle ideals. 
    If $(r)$ is maximal in all proper principle ideals, $r=ab$, 
    $(r) \subset (a)$. Then if $(a)=R$, $a$ is unit. If $(a)=(r)$,
    $a = rs$. So $r=rsb$, so $sb=1$ i.e. $b$ is unit.
  \end{enumerate}
  In the PID, every ideal is principle, so if $I$ is prime, $I$ is maximal.
\item See \hyperref[Q:1_03a]{Question~\ref{Q:1_03a}}
\end{enumerate}
\qed
\begin{exec}
\begin{enumerate}[(a)]
\item Let $G$ and $H$ be finite groups with relatively prime orders. 
Let $\theta\colon G\to H$ be a group homomorphism. 
Whtat can conclude about $\theta$ why?
\item Let $H$ be a subgroup of a group $G$ with index $2$. Prive that $H\lhd G$.
\item Give an example to show that $H$ may not be a normal subgroup of $G$
if $[G:H] = 3$.
\end{enumerate}
\end{exec}
\proof
\begin{enumerate}[(a)]
\item $\theta$ is trivial. $\Im(\theta)$ is a subgroup of $H$ so 
$|H|$ divided by $|\Im(\theta)|$. 
By $|\Im(\theta)| = \frac{|G|}{|\Ker(\theta)}$, $|G|$ 
divided by $|\Im(\theta)|$.
Hence $\Im(\theta)$ is trivial, since $|G|$, $|H|$ coprime.
\item Note that if $g\notin H$,  $\Set{H, gH}$ forms a partition of $G$.
Also $\Set{H, Hg}$ is a partition of $G$. So $gH=Hg$ since $H=H$. 
(Then $gHg^{-1}=H$).
So $H$ is normal in $G$. 
\item Consider $S_3$ which is a non-abelian order $6$ group. 
It has a order $2$ subgroup $H$ and order $3$ subgroup $N$.
Then $H\cap N = 1$, $N\lhd G$. So $H$ can not be normal in $S_3$ otherwise 
$S_3 = H\times N$ is abelian. 
\end{enumerate}
\qed

\begin{exec}
If $L$ is a field extension of $K$ such that $[L:K]=p$ where $p$ is a prime
number, show that $L=K(a)$ for every $a\in L$ that is not in $K$.
\end{exec}
\proof
Note that $[L:K]= [L:K(a)][K(a),K]$, $[K(a),K]>1$ since $a\notin K$. 
So $[L:K(a)]$ i.e. $L=K(a)$.
\qed

\section{Sem2, 2001/2002}
\begin{exec}
Classify all groups of order $n$ up to isomorphism.
\begin{enumerate}[(a)]
\item $n$ is the square of a prime integer.
\item $n=pq$ where $p,Q$ are primes with $p>q$ and $q$ does not divide $p-1$.
\end{enumerate}
\end{exec}
\proof
\begin{enumerate}[(a)]
\item Suppose that $|G|=p^2$. 
  We claim that $G$ is abelian.
  First, the center $Z(G)$ of $G$ is non-trivial.
  Consider $G$ act on itself by conjugation. If $x\in Z(G)$, the orbit length
  is $1$. Let $S=\Set{x_i}$ be the representitive set of different orbit.
  Then we get the class equation:
  \[
  |G| = |Z(G)|+\sum_{i}[G:H_i]
  \]
  where $H_i=C_G(x_i)$. Note that $p|[G:H_i]$. So $|Z(G)\equiv 0 \pmod{p}$.
  $|Z(G)|$ non-tivial since $e \in Z(G)$.
  
  If $Z(G)=G$, then $G$ is abelian. 
  If $Z(G)<G$, then $|G/Z(G)|=p$ so $G/Z(G)$ is cyclic. 
  Let $a\in G\setminus Z(G)$ and $a+Z(G)$ is a generator of $G/Z(G)$.
  Then every element in $G$ can written as $a^ng$ for $n\in \bZ$, $g\in Z(G)$. 
  So $a^{n_1}g_1 a^{n_2}g_2 = a^{n_1}a^{n_2}g_1g_2 = a^{n_2}g_2a^{n_1}g_1$.
  Hence $G$ is abelian. 

  So $G$ can be $\bZ_{p^2}$ and $\bZ_p\times \bZ_p$ by the classification of 
  finite abelian group. 
\item Condier Sylow $q$-subgroup $Q$. The number of Sylow $q$-subgroup 
  is $kq+1$ and divides $pq$. Since $q\nmid p-1$, only $1$ Sylow $q$-subgroup.
  So $Q$ is normal in $G$. On the other hand, there is a 
  Sylow $p$-subgroup $P$. It is normal in $G$ since $p>q$, the number of Sylow 
  $p$-group is $1$. Clearly $p\cap Q = \braket{e}$, $PQ=G$. $P\cong \bZ_p$
  and $Q\cong \bZ_q$.
  So $G \cong \bZ_p\oplus \bZ_q = \bZ_{pq}$. 
\end{enumerate}
\qed
\section{Sem1, 2002/2003}
\begin{exec}
\begin{enumerate}[(a)]
\item Determine whether each of the following pairs of groups are isomorphic:
\begin{enumerate}[(i)]
\item $\bZ_2\times \bZ_2\times \bZ_2$, $\bZ_8$;
\item $\bZ$, $\bQ$;
\item $\bR^*$, $\bC^*$;
\item $\bR^*$, $\bQ^*$;
\item $\bQ$, $\bQ\times\bQ$.
\end{enumerate}
\item $\bZ[i]=\Set{a+bi|a,b\in \bZ}$ is a Euclidean domain
  with respect to the Euclidean distance $d$, where
  \[
  d(a+bi)=a^2+b^2
  \]
  \begin{enumerate}[(i)]
    \item Find $\alpha,\beta \in \bZ[i]$ such that 
      \[
      1-5i = (1+2i)\alpha + \beta,
      \]
      where $|\beta|<5$.
    \item Decide, with reasons, which of the following elements are irreducible
      in $\bZ[i]$:
      \[
      1+i,2+3i,1+3i.
      \]
    \end{enumerate}
  \end{enumerate}
\end{exec}
\proof
\begin{enumerate}[(a)]
\item 
\begin{enumerate}[(i)]
\item No. $\bZ_8$ have order $8$ element, 
  but all element in $\bZ_2\times \bZ_2\times \bZ_2$ at most order $2$.
\item No. $\bQ$ is divisible, say any $x\in \bQ$, $n\in \bZ$ exists $y\in bQ$
  s.t. $ny=x$. But $\bZ$ is not divisible.
\item No. $\bC^*$ have any order $n$ subgroup say $\left<e^{2\pi i/n}\right>$. 
  But the only finite subgroup in $\bR^*$ is $\Set{\pm 1}$.
\item No. $\bR^*$ and $\bQ^*$ have different cardinal number.
\item Consider $\bQ$ and $\bQ\times \bQ$ as $\bZ$-module, 
  If $\bQ$ isomorphism to $\bQ\times\bQ$ by homomorphism $\phi$. 
  We have exact sequence:
  \[
  0\rightarrow \bQ \xrightarrow{\phi} \bQ\times \bQ \rightarrow 0.
  \]
  localization by 
  tensor product with $\bQ$. It gives a $\bQ$-module( $\bQ$-vector space)
  exact sequence:
  \[
  0\rightarrow \bQ \xrightarrow{\phi\otimes \id_{\bQ}} 
  \bQ\times \bQ \rightarrow 0.
  \]
  Since $\bQ$ have IBN, $\bQ$ can not isomorphism to $\bQ\times \bQ$.
\end{enumerate}
\item a
\end{enumerate}

\begin{exec}
  \begin{enumerate}[(a)]
  \item
    If $p$ is a prime number, show that the symmetric group $S_p$ has 
    exactly $(p-2)!$ Sylow $p$-subgroups. Deduce that $(p-1)!+1$ 
    is divisible by $p$.
  \item Prove that a ring wit a prime number of elements is either a field
    or a zero ring (i.e. a ring in which all products are zero).
  \end{enumerate}
\end{exec}
\proof
\begin{enumerate}[(a)]
\item a
\item b
\end{enumerate}
\qed

\section{Sem 2, 2002/2003}

\begin{exec}
  Let $F$ be a finite field with $p^n$ elements. Prove that 
  \begin{enumerate}[(a)]
  \item the multiplicative group $F^\times = F\setminus \Set{0}$ is cyclic.
  \item $F$ contains a subfield with $p^m$ elements if and only if $m|n$.
  \end{enumerate}
\end{exec}
\proof
\begin{enumerate}[(a)]
\item It's the same as \hyperref[Q:5_04b]{Question~\ref{Q:5_04b}}.
\item Suppose that $K$ is a subfield of $F$ then $K^\times$ is a subgroup 
  of $F^\times$. So $|K^\times|$ diviedes $F^\times$.
  As we know finite field with charactor $p$ is a extention field of $\bZ_p$
  So have $p^m$ elements.
  It's clearly that $p^m-1 | p^n-1$ if and only if $m|n$.

  On the other hand, if $m|n$, $p^m-1|p^n-1$.
  Let $a$ be a generator of $F^\times$, then consider 
  \[
  K=\Set{0}\cup \braket{a^{\frac{p^n-1}{p^m-1}}}
  \]
  Clearly $|K|=p^n$. We will prove that $K$ is a field. 
  Consider the map 
  \begin{align*}
    \phi_m \colon F &\to F \\
    x&\mapsto x^{p^m}
  \end{align*}
  Note that $\phi$ is a field homomorphism since
  $(x+y)^{p^m} = x^{p^m} + y^{p^m}$.
  
  It's also easy to see that every element $x$ in $K$ satisify 
  $\phi_m(x) = x$ ($0$ is trivial; it's ture for other elements in $K$
  since $|\braket{a^{\frac{p^n-1}{p^m-1}}}| = p^m-1$.)
  But equation $x^{p^m}-x =0$ have at most $p^m$ solutions.
  So $K$ is the set of all elements s.t. $\phi_k(x)=x$.
  So $K$ is a field since the set $\Set{x\in F|\phi_k(x)=x}$ form a field
  ($x+y,xy, x^{-1} \in K$ if $x,y\in K$).
\end{enumerate}
\qed
\section{Sem 1, 2003/2004}
\begin{exec}\label{Q:1_03a}
  \begin{enumerate}[(a)]
  \item Let $G$ be the additive grou $\bQ/\bZ$. 
    Show that any finite subgroup of $G$ is cyclic.
  \item For the ring $R=\bZ\times \bZ$, give an example for each 
    of the following:
    \begin{enumerate}[(i)]
    \item a maximal ideal of $R$.
    \item a prime ideal of $R$ that is not maximal.
    \end{enumerate}
  \end{enumerate}
\end{exec}

\begin{exec}\label{Q:2_03a}
  \begin{enumerate}[(a)]
  \item Let $G$ be a finite group, and $H$ be a subgroup of index $2$. 
    Show that $x^2\in H$ for any $x\in G$
    and hence deduce that $H$ contains all elements of $G$ of odd order.
  \item Let $n>3$ be an integer, and let $G$ be a subgroup of $S_n$. 
    Assume that $G$ has an odd permutation. Show that $G$ has a normal 
    subgroup of index $2$.
  \item Let $A_4$ be the subgroup of even permutations in $S_4$. Show that 
    $A_4$ has no subgroup of index $2$.
  \end{enumerate}
\end{exec}
\proof
\begin{enumerate}[(a)]
\item Clearly $H$ is normal in $G$. $G/H$ is order $2$. So 
  $\pi(x^2) = \pi(x)^2 = e$ where $\pi$ is the canonical map. Then $x^2\in H$.
  Suppose that $x$ have odd order, then $x^2$ is the generator of $<x>$ (A
  result of cyclic group, $\braket{x^r} = \braket{x}$ for any $(r, |x|) = 1$).
  Since $x^2\in H$, $\braket{x}\subset H$. So $x\in H$.  
\item There is a natrual sign map form $\sgn\colon S_n\to \Set{\pm 1}$.
  restric on $G$. If $G$ has odd permutation, $\sgn\vert_{G}$ is epimorphism.
  Then $\Ker \sgn_{G}$ is a normal subgroup of $G$ with index $2$.
\item Note that $\braket{(123)}$, $\braket{(124)}$, $\braket{(134)}$
and $\braket{(234)}$ gives $4$-distinct order $3$ subgroup of $A_4$.
If $A_4$ have index $2$ subgroup $H$. then $|H|=6$. But there already have
$8$ odd order element. They should be in $H$, a contradiction. 
\end{enumerate}
\qed

\section{Sem 2, 2003/2004}
\begin{exec}
Show that if $a$ and $b$ are elements in a group $G$, then $ab$ and $ba$ 
have the same order.
\end{exec}
\proof
Suppose $o(ba)$ is finite.
Note that $(ab)^n = a(ba)^n a^{-1}$. If $n=o(ab)$, $(ab)^n=1$.
So $o(ab)|o(ba)$. In the same way $o(ba)|o(ab)$.
It's also easy to see that $ab$ and $ba$ should both have finite order.
\qed

\begin{exec}
\begin{enumerate}[(a)]
\item Let $H$ and $K$ be subgroups of a group $G$ with $H$ normal in $G$.
  Show that 
  \[
  HK:=\Set{hk:h\in H, k\in K}
  \]
  is a subgroup of $G$ and show that $H$ is normal in $HK$.
\item Show that $(H\cap K)$ is normal in $K$ and that 
  \[
  K/(H\cap K) \cong HK/H
  \]
\item Show that if $H$ is a normal subgroup of $G$ such that 
  \[
  gcd(|H|,[G:H])=1
  \]
  then $H$ is the unique subgroup of $G$ of order $|H|$.
\end{enumerate}
\end{exec}
\proof
$(a)$,$(b)$ are trivial. 
If $K$ is a order $|H|$ subgroup of $G$. 
 Let $n = |K/(H\cap K)|$,
then $n$ divides $|H|$. We have $K/(H\cap K)\cong HK/H$.
So $n = \frac{|G/H|}{[G/H:HK/H]}$. So $n$ divides $[G:H]$. 
Hence $n=1$. $H\cap K=K$ i.e. $K=H$ by $|K|=|H|$.  
\qed

\begin{exec}
\begin{enumerate}[(a)]
\item Show that if $R$is a finite integral domain with a unit element, 
  then $R$ is a field.
\item 
  Show that if $R$ is a finite commutative ring with a unit element, 
  then every prime ideal of $R$ is a maximal ideal
\end{enumerate}
\end{exec}
\proof
\begin{enumerate}[(a)]
\item $R$ is finite, so $R$ is right Artinian ring. right Artinian integral 
  domain is field. It's the same as \hyperref[Q:5_00b]{Question~\ref{Q:5_00b}}.
\item g
\end{enumerate}
\qed

\begin{exec}
Let $R$ is a ring with a unit element, $1_R$, in which 
\[
(ab)^2 = a^2 b^2
\]
for all $a,b\in R$. Prove that $R$ must be commutative.
\end{exec}
\proof
(From 
\href{http://sci.tech-archive.net/Archive/sci.math/2006-08/msg02505.html}{
sci.math}.)
$((a+1)b)^2=(a+1)^2b^2$ gives $(ab)^2 + ab^2 + bab + b^2 = a^2b^2 + 2ab^2 + b^2$.
So $bab = ab^2$.
Then $(b+1)a(b+1) = a(b+1)^2$ gives
$bab+ ba+ ab + a = ab^2+ 2ab + a$. 
Hence $ba = ab$. So $R$ commutative.
\qed

\section{Sem 1, 2004/2005}
\begin{exec}
Classify all groups of order $8$ up to isomorphism.
\end{exec}
\proof
If $G$ is abelian, then $G$ can be $\bZ_2\times \bZ_2 \times \bZ_2$, 
$\bZ_4\times \bZ_2$ and $\bZ_8$. 
If $G$ is non-abelian, $G$ has a order. 
\qed

\section{Sem 2, 2004/2005}
\subsection{Ring Theory}
\begin{exec}
Prove that every integral domain can be imbedded in a field.
\end{exec}

\begin{exec}
Let $D$ be an itegral domain and let 
$F=\Set{x\in D|xd=1\text{ for some }d\in D}$. 
Suppose that $D$ is a finite dimensional vector space over $F$.
Prove that $D$ is a field.
\end{exec}
\proof
I don't think $F$ is a field. 
If $F$ is a field, it's the same as \hyperref[Q:5_00b]{Question~\ref{Q:5_00b}}.
\qed

\subsection{Group Theory}
\begin{exec}
Let $G$ be a group of order $56$. Suppose that $G$ has $2$ or more 
subgroups of order $7$. Prove that $G$ has a subgroup isomorphic to 
$\bZ_2\times \bZ_2\times \bZ_2$. 
\end{exec}
\proof
By Sylow's theorem $G$ has $7$ different Sylow $7$-subgroups. 
So there is unique Sylow $2$-subgroups containing all element of $G$ not in
Sylow $7$-subgroups. 
\qed

\subsection{Field Theory}
\begin{exec}\label{Q:5_04b}
Let $F$ be a finite field. Prove that $F-\Set{0}$ under multiplication is a
cyclic group.
\end{exec}
\proof
It's well know that finite multiplication group of field is cyclic. 
Clearly $F-\Set{0}$ form a group under multiplication, then it is cyclic.

We can prove this result as following.
Let $G$ be a finite multiplication subgroup of field $F$. 
Let the primary decomposition of $G$  be $\bigoplus_{i=1}^nG_{p_i}$, 
where $n\in \bN$ and $p_i$ are prime number and 
$G_{p_i} = \oplus_{j=1}^{n_i} \bZ_{p^{\alpha_j}}$, $n_i\in bN$, $alpha_{i,j}\geq 1$.
We claim that $n_i=1$, i.e. $G_{p_i} = \bZ_{p^{\alpha_i}}$, then $G$ is cyclic
($\bZ_a\oplus \bZ_b = \bZ_{ab}$ if $\gcd(a,b)=1$). 
If $n_i>1$, for some $i$, $G$ has two distinct order $p_i$ subgroup, then 
there are more than $p_i$ element is $G$ satisfying the equation $x^{p_i}-1=0$.
Since $F$ is a field, there at most $p_i$ different solution of the equation, 
a contradicition.  
\qed

\begin{exec}
Prove that $\sqrt{2}+\sqrt{3}+\sqrt{5}+\sqrt{7}$ is irrational.
\end{exec}

\section{Sem 1, 2005/2006}

\section{Sem 2, 2005/2006}
\begin{exec}\label{Q:1_05b}
\begin{enumerate}[(a)]
\item Prove that a group of order $12$ either has a normal subgroup of order
  $3$, or is isomorphic to $A_4$, the alternating group on $4$ letters.
\item Show that any simple group acting on a set of $n$ elements 
  is isomorphic to a subgroup of $A_n$, the alternating group
  on $n$ letters.
\end{enumerate}
\end{exec}
\proof
\begin{enumerate}[(a)]
\item If $|G|=12$ and has no order $3$ normal subgroup. Then the number 
  of it's Sylow $3$-subgroups is $4$. 
  %Then by counting number of order $3$ 
  %element, we know all the elements which are not order $3$ form 
  %the unique Sylow-$2$ subgroup of $G$. 
  Let $G$ act on the set of Sylow $3$-subgroups  $S$ by conjugation.
  It gives a homomorphism $\phi\colon G\to S_4$. The same as
  \hyperref[Q:2_03a]{Question~\ref{Q:2_03a} (c)} $G$ has no order $6$ subgroup. 
  So $\Im\phi \subset A_4$. 
  Since $G$ act on $S$ transitively. $|\Im\phi|\geq |S| =4$. 
  So $|\Ker\phi| \leq 3$. Since $G$ have no order $3$ subgroup, 
  $|\Ker\phi|\neq 3$. 
  Since $A_4$ have no order $6$ subgroup, $|\Ker\phi|\neq 2$. 
  So $\phi$ is monomorphism, then isomorphisom from $G$ to $A_4$.
\item See \hyperref[Q:1_08a]{Question~$\ref{Q:1_08a}$}
\end{enumerate}
\qed

\begin{exec}
Let $R$ be a ring, not necessarily commutative and not 
necessarily containing the multiplicative identity. 
Prove that if $R[X]$ is a principal ideal domain, then $R$ is a field.
\end{exec}
\proof
First we can embed $R$ in $R[X]$. Since $R[X]$ is integral domain 
(commutative, no zero divisor),
$R$ is integral domain.  
Consider the evaluation $\phi\colon R[X]\to R$ by $f \mapsto f(0)$.
$\phi$ is surjective. Since $R$ is integral domain, $\Ker\phi$ is 
prime ideal in $R[X]$, then it is maximal ideal by 
\hyperref[Q:1_02a]{Question~\ref{Q:1_02a}~(b)}. So $R$ is a field. 
\qed

\section{Sem 1, 2007/2008}
\begin{exec}
Prove that a simple group of order $60$ is isomorphic to $A_5$.
\end{exec}
\proof
Note that, if there is a action of $G$ on set $S$ with $|S|=n$, then
there is a injective from $G$ to $A_n$. 
(See \hyperref[Q:1_05b]{Question~\ref{Q:1_05b} (b)})
Since $|G|=60$, $|A_n|\geq |G|$ i.e. $n \geq 5$.
$60=3\times 4\times 5$.
Consider $2$-Sylow group. There is two approachs. 
\begin{enumerate}[(a)]
\item 
  If there are two $2$-Sylow subgroup $P$,$Q$ with non-trivial intersection.
  Clearly $H=P\cap Q$ is order $2$. Choose $e\neq x\in H$.
  Then $P\cap Pq \subset C_G(x)$(order $4$ group are all abelian), where
  $q\in Q\setminus H$. So $|C_G(x)|\geq 8$. Cearly $C_G(x)\neq G$,
  if so $C(G)$ is a non-trivial normal subgroup of $G$.  
  So $|C_G(x)| \leq 12$, by looking the left action of $G$ on $G/C_G(x)$ 
  ($[G:C_G(x)]\geq 5$).
  Now $|C_G(x)|$ divides $60$ and $|P|$ divids $C_G(x)$($P < C_G(x)$). 
  So $|C_G(x)| = 12$. Hence it gives a isomorphism from $G$ to $A_5$ 
  by looking the left action of $G$ on $G/C_G(x)$.
  
  If all $2$-Sylow subgroup have no non-trivial intersection, fix a $2$-Sylow
  subgroup $P$. Consider the normalizer $N_G(P)$. We will prove that 
  $N_G(P) \neq P$. If so, the only possible is $|N_G(P)| = 12$, 
  then $G\cong A_5$.
  
  Suppose that $N_G(P) = P$, then $|N_G(P)|=4$. So there is fifteen differen
  $2$-Sylow subgroup of $G$ since $N_G(P)$ is the stabilizer of the action
  of $G$ on the set $S$ of all $2$-Sylow subgroups
  and $G$ act on $S$ transitively. 
  Note that $G$ is simple, so there is six different $5$-Sylow subgroups
  of $G$. 
  Clearly the interesection of different $5$-Sylow subgroups is trivial. 
  Also the interrsection of $5$-Sylow subgroup and $2$-Sylow subgroup is 
  trivial since $\gcd(4,5)=1$.
  Then there at least $1+(4-1)*15+(5-1)*6 = 70 > 60$ differenet element in $G$,
  a contradiction.
\item 
 The number of Sylow-$2$ subgroup can be
$3,5,15$. Now consider $G$ act on Sylow-$2$ by conjugation. 
\begin{enumerate}[(i)]
\item $3$ is impossible.
\item If it has $5$ Sylow-$2$ subgroup, 
  it gives a isomorphism $G$ to $A_5$ since $|A_5|=60$.
\item $15$ is impossible in the proof of (a).
\end{enumerate}
\end{enumerate}
\qed

\section{Sem 2, 2007/2008}

\section{Sem 1, 2008/2009}
\begin{exec}\label{Q:1_08a}
\begin{enumerate}[(a)]
\item Let $G$ be a finite simple group, and suppose that $H$ is proper 
  subgroup of $G$ of index $k$. Show that there exists an injective group 
  homomorphism from $G$ to the alternating group $A_k$ of degree $k$.
\item Show that a group of order $120$ is not simple. 
\end{enumerate}
\end{exec}
\proof
\begin{enumerate}[(a)]
\item
Consider $G$ act on the set of left cosets $\Set{gH|g\in G}$ by left 
multiplication, i.e. $x\cdot gH = (xg)H$. It gives a map $\phi$ from
$G\to S_k$ since $\# \Set{gH|g\in H} = k$. Clearly $\phi$ is nontrival 
since $H$ is proper subgroup of $G$ ($\exists g$ s.t. $gH \neq H$). 
So $\phi$ is monomorphism since $G$ is simple ($\Ker\phi$ is normal in $G$).
Moreover $\Im \phi\subset A_n$. If not $\sgn \phi\colon G \to \Set{\pm 1}$
 is epimorphism. Then $G$ have a nontrivial index $2$ normal subgroup 
$\Ker\,\sgn\phi$($|G|>2$).
\item If $|G|=120=8\times 5\times 3$ 
  and $G$ is simple.  By Sylow's theorem, $|G|$ has a order $8$
subgroup $H$, it's normalizer $N(H)$ can not be $G$ since $G$ is simple.
If the index of $N(H)$ is $5$ or $3$ for any cases it's impossible
since $60=|A_5|, 6=|A_3| < 120=|G|$. But by (a) $G$ can be embedded into 
$A_5$ or $A_3$, a contradiction.
If the index of $N(H)$ is $15$, then there is $15$ different 
$2$-Sylow subgroup of $G$. 

\end{enumerate}
\qed

\begin{exec}
  \begin{enumerate}[(a)]
  \item Let $R$ and $S$ be integral domains with $R\subseteq S$. 
    Prove or disprove the following:
    \begin{enumerate}[(i)]
    \item If $R$ is a Euclidean domain,
      then $S$ is a unique factorisation domain.
    \item If $S$ is a Euclidean domain, 
      then $R$ is a unique factorisation domain.
    \end{enumerate}
  \item Let $\phi\colon T \to U$ be a surjective ring homomorphism between two
    integral domains $T$ and $U$. Prove or disprove the following:
    \begin{enumerate}[(i)]
    \item If $T$ is a principal ideal domain, then $U$ is a principal ideal
      domain.
    \item If $T$ is a unique factorisation domain, then $U$ is a unique 
      factorisation domain.
    \end{enumerate}
  \end{enumerate}
\end{exec}
\proof
\begin{enumerate}[(a)]
\item a
\item 
  \begin{enumerate}[(i)]
  \item For any ideal $I \subset U$, $\phi^{-1}(I)$ is a ideal in $T$. 
    Then $\phi^{-1}(I) = (r)$ for some $r$. Then $I = (\phi(r))$.
  \item Not true!
    \href{http://en.wikipedia.org/wiki/Unique_factorization_domain#Counterexamples}{A example on wikipedia.} $F[X,Y, Z,W]$ is UFD for any field $F$. 
    But $F[X,Y,Z,W]/(XY-ZW)$ is not UFD. 
  \end{enumerate}
\end{enumerate}
\qed


\end{document}
