\documentclass[cjk]{beamer}
\usepackage{CJK}
\usepackage{latexsym, amsfonts, amssymb, amsmath,  amsthm}
\usepackage{amsopn, amstext, amscd,pifont}
\usepackage{amssymb,bm}
\usepackage{braket}
\usepackage{cases}
\usepackage{paralist}
\usepackage{eufrak}
\usepackage[all]{xy}
\usepackage[overload]{textcase}
\usetheme{Singapore}
\usefonttheme{professionalfonts}

\newtheorem{thm}[theorem]{Theorem}
\newtheorem{exa}[theorem]{Example}

\begin{document}

\title{On quantiles of Brownian motion and \\
quantile options}
\author{Zhu Yong Ting}
\date{} \frame{\titlepage}

\section{Agenda}
\begin{frame}
\frametitle{Organization of this Thesis}

\begin{itemize}
\item Introduction  $\color{red}\surd$
\item Principles of option pricing 
\item Numerical methods
\item Quantile and quantile options $\color{red}\surd$
\item Conclusion and future work $\color{red}\surd$
\end{itemize}
\end{frame}

\section{Introduction} 
\begin{frame}
\frametitle{Definition of quantile of Brownian motion}
For a stochastic process \{$X_t$\} on $(\Omega, \mathbb Q, \mathcal F)$, the $\alpha$-quantile $(0 \leq \alpha \leq 1)$
is defined by
\[
M(\alpha,t)(\omega) = \inf\Set{x:\int_0^t ds1_{(X_s (\omega)\leq x)} > \alpha t}.
\]
Special cases: 

$\displaystyle\inf_{0\leq s \leq t}  X_s = \lim_{\alpha\to 0}M(\alpha,t)$ and $\displaystyle\sup_{0\leq s \leq t} X_s = \lim_{\alpha\to 1} M(\alpha, t)$.
\end{frame}

\section{quantile}



\end{document} 