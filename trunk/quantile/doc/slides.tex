\documentclass[cjk]{beamer}
\usepackage{CJK}
\usepackage{latexsym, amsfonts, amssymb, amsmath,  amsthm}
\usepackage{amsopn, amstext, amscd,pifont}
\usepackage{amssymb,bm}
\usepackage{braket}
\usepackage{cases}
\usepackage{paralist}
\usepackage{eufrak}
\usepackage[all]{xy}
\usepackage[overload]{textcase}
\usetheme{Singapore}
\usefonttheme{professionalfonts}

\newtheorem{thm}[theorem]{Theorem}
\newtheorem{exa}[theorem]{Example}

\begin{document}

\title{On quantiles of Brownian motion and \\
quantile options}
\author{Zhu Yong Ting}
\date{} \frame{\titlepage}

\section{Agenda}
\begin{frame}
\frametitle{Organization of this Thesis}

\begin{itemize}
\item Introduction  $\color{red}\surd$
\item Principles of option pricing 
\item Numerical methods
\item Quantile and quantile options $\color{red}\surd$
\item Conclusion and future work $\color{red}\surd$
\end{itemize}
\end{frame}

\section{Introduction} 
\begin{frame}
\frametitle{Introduction}
Definition of quantiles:
for a stochastic process \{$X_t$\} on $(\Omega, \mathbb Q, \mathcal F)$, \\
\[
M(\alpha,t)(\omega) = \inf\Set{x:\int_0^t ds1_{(X_s (\omega)\leq x)} > \alpha t}.
\]
the $\alpha$-quantile $(0 \leq \alpha \leq 1)$ of this process at time $t$.\newline


\pause
Quantile options:
Just replace the spot price by the quantiles.

\pause
E.g. pay off function of 
\begin{itemize}
\item 
$\alpha$-quantile Eurpean call option: $(S(\alpha,T)-K)^+$;
\item 
$\alpha$-quantile Eurpean call option: $(S(\alpha,\tau)-K)^+$.
% when execrice the option at time \tau
\end{itemize}
\end{frame}

\section{Quantile}
\begin{frame}{Quantiles of Brownian motion}
Properties:
\begin{itemize}
\item $\displaystyle\inf_{0\leq s \leq t}  X_s = \lim_{\alpha\to 0}M(\alpha,t)$
\item $\displaystyle\sup_{0\leq s \leq t} X_s = \lim_{\alpha\to 1} M(\alpha, t)$.
\end{itemize}
\vspace{1em}
\pause

Now assume $X_t$ is a Brwonian motion.

Key observation:
\[
(M(\alpha,T),X_T) 
{\stackrel{\text{(law)}}{=}}
 (\max_{t\leq \alpha T}X_t+\min_{t\leq (1-\alpha)T}X'_t, X_{\alpha T}+X'_{(1-\alpha)T}),
\]
where $X'_t$ is an independent copy of $X_t$. 
\end{frame}

\section{Discrete v.s. Continous}



\section{Quantile Options}

\subsection{European}

\subsection{American}

\section{Further directions}


\end{document} 