\documentclass[12pt]{amsart}
\usepackage[margin=3cm]{geometry}
\usepackage[hyperindex=true]{hyperref}

\usepackage{amssymb}
\usepackage{amsxtra}
\usepackage{graphicx}

\usepackage{braket}
\usepackage{paralist}
\usepackage{eufrak}
\usepackage{eucal}

\usepackage{dsfont}

\newtheorem{thm}{Theorem}[section]
\newtheorem{lem}[thm]{Lemma}
\newtheorem{prop}[thm]{Proposition}
\newtheorem{cor}[thm]{Corollary}

%\usepackage{makeidx} 
\makeindex

\def\Ker{\rm{Ker}}
\def\Im{\rm{Im}}
\def\Hom{\rm{Hom}}
\def\End{\rm{End}}
\def\Mat{\rm{Mat}}
\def\bR{{\mathbb{R}}}
\def\bN{{\mathbb{N}}}
\def\bZ{{\mathbb{Z}}}
\def\bC{{\mathbb{C}}}
\def\bQ{{\mathbb{Q}}}
\def\vv{{\vec{v}}}
\def\vw{{\vec{w}}}
\def\vx{{\vec{x}}}
\def\vy{{\vec{y}}}
\def\v0{{\vec{0}}}
\def\ol{\overline}
\def\sspan{\rm{span}}
\def\sl2{{\mathfrak{sl}(2)}}
\def\slc{{\mathfrak{sl}(2,\bC)}}
\def\fg{{\mathfrak{g}}}
\def\fs{{\mathfrak{s}}}
\def\fk{{\mathfrak{k}}}
\def\Sp{\mathrm{Sp}}
\def\GL{\mathrm{GL}}
\def\sP{\mathcal{P}}
\def\sH{\mathcal{H}}
\def\sU{\mathcal{U}}
\def\sC{\mathcal{C}}
\def\kbar{{k\makebox[-0.06em][r]{\raisebox{0.5ex}{--}}}}
\def\ad{{\rm ad\,}}
\def\Ad{{\rm Ad\,}}
\def\id{{\rm id\,}}
\def\sgn{{\rm sgn\,}}
\def\gcd{{\rm gcd\,}}
\def\inn#1#2{\left<{#1}\,,\,{#2}\right>}
\def\bform#1#2{\left({#1}\,,\,{#2}\right)}
\def\cform#1#2{\left\{{#1}\,,\,{#2}\right\}}
\def\isoarrow{\tilde{\rightarrow}}
\def\tr{{\rm tr}\,}
\def\invl{{\natural}}
\def\Gal{{\rm Gal}\,}
\def\tB{\widetilde{B}}
\def\ttau{\widetilde{\tau}}
\def\oO{{\mathcal{O}}}


\hypersetup{
    bookmarks=false,         % show bookmarks bar?
    unicode=false,          % non-Latin characters in Acrobat��s bookmarks
    pdftoolbar=true,        % show Acrobat��s toolbar?
    pdfmenubar=false,        % show Acrobat��s menu?
    pdffitwindow=false,      % page fit to window when opened
    pdftitle={Qualify Examination Answers - Algebra},    % title
    pdfauthor={Ma Jia Jun},     % author
    pdfsubject={Subject},   % subject of the document
    pdfcreator={Creator},   % creator of the document
    pdfproducer={Producer}, % producer of the document
    pdfkeywords={keywords}, % list of keywords
    pdfnewwindow=true,      % links in new window
    colorlinks=true,       % false: boxed links; true: colored links
    linkcolor=blue,          % color of internal links
    citecolor=green,        % color of links to bibliography
    filecolor=magenta,      % color of file links
    urlcolor=cyan           % color of external links
}


\title{}

\begin{document}
\maketitle
\section{Heisenberg groups}
\begin{prop}\label{prop:1.1}
\end{prop}
\begin{prop}\label{prop:1.2}
Give $W$ the structure of $A$-module as in proposition~\ref{prop:1.1} d). 
Then $\inn{}{}\colon W\times W \to A$ is an $A$-bilinear form which 
is skew-symmetric and non-degenerate in the strong sense that the map 
$\alpha\colon W \to \Hom_A(W,A)$ defined by 
\begin{equation}\label{eq:1.5}
\alpha(w)(w') = \inn{w}{w'}
\end{equation}
is an isomorphism of $A$-modules.
\end{prop}

\section{Structure of $\Sp(W)$; free polarizations}
\begin{prop}\label{prop:5.2}
\begin{enumerate}[a)]
\item $\Sp$ acts transitively on the set of free polarizations of rank $m$.
\item Let $U\subseteq W$ be a free polarization, and let $P(U,W)= P(U) = P$
be the subgroup of $\Sp$ leaving $U$ invariant. Then the restrictio map 
\begin{equation}\label{eq:5.2}
r\colon P\rightarrow \GL(U)
\end{equation}
is surjective.
\item Let $N(U,W) = N(U)=N$ be the kernel of the map $r$ of $(\ref{eq:5.2})$.
Then $N$ acts simply transitively on the set of complete polarizations with
$U$ as first member. Equivalently $N$ acts simply transitively on 
the set of free polarizations of $W$ complementray to $U$.
\item Fix a free polarization $U'$ complementry to $U$. 
Then there is an isomorphism 
\begin{equation}\label{eq:5.3}
\beta\colon N \isoarrow S^{2*}(U'),
\end{equation}
where $S^{2*}(U')$ denots the space of symmetric bilinear forms on $U'$.
\item Let $M=P(U)\cap P(U')$. Then $M$ is a complement to $N$ 
in $P$, so that $P$ is a semidirect product.
\begin{equation}\label{eq:5.4}
P\simeq M \ltimes N
\end{equation}
\end{enumerate}
\end{prop}

\section{Reductive dual pairs, definition and classification}
\label{sec:6}
Let $\Gamma$ be a group $(G, G')$ be a pair of subgroups of $\Gamma$. 
We will say $(G,G')$ form a {\em dual pair}\index{dual pair} 
of subgroups of $\Gamma$ if
$G$ is the centralizer in $\Gamma$ of $G'$ and vice-versa. 

It is not hard to finde dual pairs of subgroups in a group. Start with any 
subgroup $G\subseteq \Gamma$. Let $G'$ be the centralizer of 
$G$ in $\Gamma$ and let $G''$ be the centralizer of $G'$. Then $(G'', G')$ is
a dual pair in $\Gamma$.

When we take $\Gamma=\Sp(W)$ for some symplectic vector space, we can refine 
the concept slightly. We say $(G, G') \subseteq \Sp(W)$ form a
{\em reductive dual pair}\index{dual pair!reductive} if first 
$(G,G')$ is a dual pair in $\Sp$, and moreover $G$ and $G'$ are reductive in 
the sense that they act (absolutely) reductively on $W$.

The goal of this section is to describe reductive dual pairs in $\Sp$. 
Let $(G, G')$ be such a pair. Suppose $W=W_1\oplus W_2$ is an orthognal 
direct sum, and that each $W_i$ is invariant by $G$ and by $G'$. Let
$G_i = G|_{W_i}$ be the group of transformations of $W_i$ obtained by 
restricting elements of $G$. Define $G_i$ similarly. 
Let $r_i\colon G\to G_i$, and $r'_i$ be the obvious maps. 
Then half a moment's thought convinces one that 
$r_1\times r_2\colon G\to G_1\times G_2$ is an isomorphism\footnote{
it is injective by the vector space decomposition $W = W_1\oplus W_2$, 
it is surjective since $G$ is the comutator of $G'$, 
and $G|W\times \Set{1}$ comutes the action of $G'$. }, and similarly for 
$r'_1\times  r'_2$. Moreover, $(G_i,G'_i)$ will be a reductive dual pair
in $\Sp(W_i)$. To describe this situation we will say that $(G,G')$ 
is the {\em direct sum}\index{direct sum} of $(G_i, G'_i)$.
If $(G,G')$ has no non-trivial direct sum decomposition, then we will call 
$(G,G')$ {\em irreducible}\index{irreducible}. 

\begin{prop}\label{prop:6.1}
\begin{enumerate}[a)]
\item Every reductive dual pair is the direct sum of irreducible subpairs in an 
essentially unique way (i.e., up to numbering of the pairs)
\item If $(G,G')$ is irreducible, then either:
\begin{enumerate}[i)]
\item $G\dot G'$ acts irreducible on $W$, and $W$ consists of a single 
isotypic component (which is self-dual) for $G$ or for $G'$; or
\item $W= U_1 \oplus U_2$, where each $U_i$ is invariant and irreducible for 
$G\dot G'$, and is maximal isotropic in $W$. The restriction map then take
$(G,G')$ to a dual pair in $\GL(U_i)$.
\end{enumerate}
\end{enumerate}
\end{prop}
\proof
Consider first the action of $G$ alone on $W$. Let $V_1$ and $V_2$ be 
irreducible $G$-subspaces of $W$. If the pairing between $V_1$ and $V_2$ 
induced by $\inn{}{}$ is non-trivial, then it must be non-degenerate,
by irreducibility of the $V_i$, and thus will induce a $G$-equivariant 
isomorphism between $V_2$ and $\Hom(V_1, F)=V_1^*$. 
(We use the conventional $*$ to denote dual space now that we are working over
a field.) Thus $V_1$ must be orthogonal to any $G$-submodule of $W$ except 
one isomorphic to $V_1^*$; and since $W$ is the direct sum of 
irreducible $G$-modules, since $G$ is assumed to act reductively on $W$, 
there will be a $G$-submodule of $W$ isomorphic to $V_1^*$ paired 
non-degenerately with $V_1$. 

Consider the decomposition $W=\bigoplus_i U_i$ of $W$ into isotypic components 
for $G$. That is, each $U_i$ is the sum of all $G$-submodules of some given 
isomorphism type. By the preceding paragraph, we see that either
the isomorphism type defining $U_i$ is self-dual and that $\inn{}{}|_{U_i}$
will be non-degenerate; or there is another isotypic component $U_j$
containing the dual isomorphism type to that of $U_i$, 
and that each of $U_i$ and $U_j$ are isotropic, but $\inn{}{}$ is 
non-degenerate on $U_i\oplus U_j$.

Thus we may write $W=\bigoplus_j U'_j$ where each $U'_j$ is either self-dual
isotypic for $G$, or the sum of two mutually contragredient isotypic 
components. This sum is orthogonal. The $U'_j$ are clearly determined uniquely 
up to order by $G$ and are invariant by $G'$. Therefore 
they give a direct sum decomposition of $(G, G')$. To prove the proposition, 
therefor, it will sufficent to show that each $G$-isotypic component is irreducible under $G\cdot G'$. 

Consider a subspace $U\subseteq W$ which is irreducible for $G\cdot G'$. 
Then $U$ must consist of a single isotypic component for either $G$ 
or $G'$. The form $\inn{}{}$ is either trivial or non-degenerate on $U$. 
Suppose first $\inn{}{}$ is non-degenerate. Then $W= U\oplus U^\perp$ is a 
decomposition of $(G,G')$.  Since the operator which is the identity on $U$
and minus the identity on $U^\perp$ commutes with both $G$ and $G'$, it must 
belong to both. 
Therefore $U$ and $U^\perp$ contain no isomorphic $G$-submodules\footnote{if has
there exits an non-trivial intertwining map $T\colon U\to U^\perp$, 
then $-Tu=gTu =Tgu = Tu$, a contradicitoion},
so $U$ is a full isotypic component in $W$ for $G$ and $G'$. This is case 
$b)i)$. 

Secondly, suppose $\inn{}{}$ is trivial on $U$. 
Then there is another $G\cdot G'$-irreducible subspace $V$ which is paired 
non-trivially, hence non-degenerately, with $U$. I claim $V$ must also be 
isotropic. Otherwise, we could write $W=V\oplus V^\perp$ and reason as just 
above. But then $U$ would be the graph of a non-trivial $G\cdot G'$ 
intertwining morphism form $V^\perp$ to $V$, which, we saw above, does not 
exist. Thus $U$ and $V$ are both isotropic. 
Write $W=(U\oplus V) \oplus (U\oplus V)^\perp$. Observe that multiplicatio by 
a scalar $t$ on $U$, by $t^-1$ on $V$ and by $1$ on $(U\oplus V)^\perp$
defines an element of $\Sp$ commuting with both $G$ and $G'$, so belonging
to both. Thus again $U$ must be a full $G$-isotypic component in $W$, 
and the proposition is proved. 
\qed

Let $(G,G')$ be a reductive dual pair in $\Sp$. We will say $(G,G')$
is of {\em type I}\index{type I} or {\em type II}\index{type II} according as
possibility i) or possibility ii) of proposition~\ref{prop:6.1} b) obtains. 
We proceed to describe more precisely the pairs of the two types. 
We begin with type II pairs, as these are somewhat simpler than type I pairs.

\begin{prop}\label{prop:6.2}
Let $(G,G')\subseteq \Sp$ be an irreducible type II reductive dual pair. 
Let $(U_1, U_2)$ be the complete polarization of $W$ invariant by $G\cdot G'$,
so that $W=U_1\oplus U_2$ and each of $U_1$ and $U_2$ are isotropic and 
irreducible under $G\cdot G'$. Then restriction to $U_i$ embeds 
$(G,G')$ as a reductive dual pair in $GL(U_i)$. Thus there is 
\begin{enumerate}[i)]
\item a division algebra $D$, and
\item a right vector space $V$ and a left vector space $V'$ over $D$, 
\end{enumerate}
such that $U_1$ is isomorphic to $V\otimes_D V'$ in such fashion 
that $G$ is identified to $\GL_D(V)\otimes I_{V'}$ and $G'$ 
is identified to $I_V\otimes GL_D(V')$ where $I_V$ and $I_{V'}$
are the identity operators on $V$ and $V'$ respectively. 
\end{prop}
\proof
Both $G$ and $G'$ are in the subgroup $M$ preserving $U_1$ and $U_2$, 
as described in proposition~\ref{prop:5.2} e). By that result $M$ is identified 
by restriction with $\GL(U_i)$. Clearly, if $G'$ is the centralizer 
of $G$ in $\Sp$, it is the centralizer of $G$ in $M$ also. 
We now may apply the classical description of reductive dual pairs 
in the general linear groups, essentially amounting to the
Double Commutant Theorem of linear algebra, to obtain the existence 
of $D$ and the decompositio of $U_1$ as a tensor product\footnote{
Here view $U_1$ as a $G$ representation which has only one isotypic component,
hence $U_1 = \bigoplus_{h} V_h$ wherr all $V_h \simeq V_\sigma$, where 
$\sigma \in \hat{G}$. Then $D = \Hom_G(V_\sigma, V_\sigma)$.
}.

Remark: Since $U_2$ may be identified to $U_1^*$ as in $(\ref{eq:1.5})$,
we may write $W \simeq V\otimes_D V' \oplus V'^*\otimes_D V^*$ in such 
a way that $(G,G')$ is identified to $(\GL_D(V), \GL_D(V'))$
acting in the obvious way. 
Sometimes a slight variant of this construction is useful.
Namely, we can take two right $D$ vector space $V$ and $V'$ and identify
$U_1$ with $\Hom_D(V,V')$. Then $G$ acts by right multiplication by 
inverses and $G'$ acts by left multiplication. 
Then 
\begin{equation}\label{eq:6.1}
W\simeq \Hom_D(V,V') \oplus \Hom_D(V',V)
\end{equation}
and if $S\in \Hom_D(V,V')$ and $T\in \Hom_D(V',V)$, and
$g\in G$; $g'\in G'$, then 
\begin{equation}\label{eq:6.2}
g\cdot g'(S,T) = (g'Sg^{-1}, gTg'^{-1})
\end{equation}
This formulation has the advantage that there is a simple formula
for $\inn{}{}$. 
Assuming the identification $(\ref{eq:6.1})$ is normalized properly, 
we have
\begin{equation}\label{eq:6.3}
\inn{(S_1,T_1)}{(S_2,T_2)} = \tr(S_1T_2-S_2T_1)
\end{equation}
where notation is parallel to $(\ref{eq:6.2})$, and $\tr$ here is
reduced trace\footnote{
the usual trace get a element in $D$, than consider $D$ as a linear operator 
on $F$ vector space $D$, it has a trace, this is the value of $\tr$ here. 
} over $F$ on $End_D(V')$. 
Of course $\tr(T_2S_1 - T_1S_2)$ gives the same answer.

We turn to type $I$ pairs, which present mor variety.

\begin{prop}\label{prop:6.3}
Let $(G,G')$ be an irreducible type I reductive dual pair. Then there exist
\begin{enumerate}[i)]
\item a division algebra $D$,
\item with involution (i.e., involutory antiautomorphism) $\invl$,
\item vector spaces $V$ and $V'$ over $D$
\item with forms $\bform{}{}$ and $\bform{}{}'$, which are $D$-linear
in the first variable and either $\invl$-hermitian or $\invl$-skew hermitian
(one of each type)
\end{enumerate}
such that $W\simeq V\otimes_DV'$ in such fashion that $G$ is identified to 
the isometry group of $\bform{}{}$ and $G'$ to the isometry group of 
$\bform{}{}'$.
\end{prop}

Remarks: 
\begin{enumerate}[a)]
\item Just as in the type II case, we can slightly less symmetric and write
\begin{equation}\label{eq:6.4}
W\simeq \Hom_D(V,V')
\end{equation}
Then the action of $G$ is by permultiplication (by inverses) and
that of $G'$ is by postmultiplication. Of course we could also 
write $W\cong \Hom_D(V',V)$. 
Note that there are nice maps mediating between these two alternatives. 
Namely, define
\begin{align}\label{eq:6.5}
*\colon \Hom_D(V,V')&\to \Hom_D(V,V'), & \text{and} \notag \\
*'\colon \Hom_D(V',V)&\to \Hom_D(V',V), & \text{by} \notag \\
\bform{Tv}{v'}' &= \bform{v},{T^* v'} &\text{for } T\in \Hom_D(V,V') \\
\bform{Sv'}{v}' &= \bform{v'},{S^{*'} v} 
& \text{for } S\in \Hom_D(V',V) \notag
\end{align}
Since $\bform{}{}$ and $\bform{}{}'$ have opposite parties, we see that
\begin{equation}\label{eq:6.6}
T^{**'} = -T \quad S^{*'*} = -S
\end{equation}
Again, this slightly assymetric approach allows one to give a convenient
formula for $\inn{}{}$.
\begin{equation}\label{eq:6.7}
\inn{T_1}{T_2} = \tr(T_2^* T_1) \quad \text{for } T_i \in \Hom_D(V,V')
\end{equation}
where $\tr$ again is reduced trace on $\End_D(V)$. Implicit 
in $(\ref{eq:6.7})$ is the fact that $\tr(T^*T)=0$. This is indeed the case. 
For the moment we merely record $(\ref{eq:6.7})$. We will discuss it more
hully in $\S\ref{sec:7}$
\item The classification of irreducible reductive dual pairs given by this 
proposition and the previous one is a reworking from a new viewpoint 
of classical results, going back through Weil [] and Siegel to Albert.
These results can also be dug out of Satake [], and propositoin~\ref{prop:6.4}
following is stated in Shimure []. What seems to be new here is the 
insistence on the equality of statues of $G$ and $G'$, and of the mutuality 
of their relation to one another.
\end{enumerate}

\proof
We must begin with some generalities on $D$-semi-linear forms 
where $D$ is a division algebra with involution $\invl$ over some field $F$.
Consider $D$ as a left vector space over itself, 
and define
\begin{equation}\label{eq:6.8}
\bform{x}{y}_\circ = x y^\invl
\end{equation}
for $x,y$ in $D$. Then $\inn{}{}_\circ$ satisfies
\begin{align}\label{eq:6.9}
i) \quad \bform{ax}{y}_\circ &= a\bform{x}{y}_\circ,  &
 \bform{x}{ay}_\circ&=\bform{x}{y}_\circ a^\invl\quad \text{ for } a,x,y\in D\\
ii)\quad \bform{y}{x}_\circ  &= \bform{x}{y}_\circ^\invl. \notag
\end{align}

If $E$ is any left $D$ vector space, call a form $\bform{}{}$
satisfying $(\ref{eq:6.9})$ i) 
{\em $\invl$-sesquilinear}\index{$\invl$-sesquilinear}.
If it also setisfies $(\ref{eq:6.9})$ ii) call it 
{\em $\invl$-hermitian}\index{$\invl$-hermitian}.
Evidently, if we choose a basis for $E$ and so identify 
$E$ with $D^k$, then the $k$-fold direct sum
of $\bform{}{}_\circ$ as in $(\ref{eq:6.8}$ will define 
a non-degenerate $\invl$-hermitian form on $V$.

Let $\tr$ denote the reduced trace map from $D$ to $F$. 
We recall 
\begin{align}\label{eq:6.10}
\tr(xy) &= \tr(yx) & \tr(x^\invl)&=\tr(x) & \text{for } x,y\in D 
\end{align}
The $D$ vector space $E$ may be regarded as an $F$ vector space, 
denoted $E_F$, by restriction of scalars. 
If $\bform{}{}$ is an $\invl$-hermitian form on $E$, 
then 
\[
\tr\bform{}{}\colon x,y \mapsto \tr(\bform{x}{u}) \quad x,y \in E = E_F
\]
defines a symmetric bilinear form on $E_F$.
Moreover $(\ref{eq:6.9})$ i) and $(\ref{eq:6.10})$ imply 
\begin{equation}\label{eq:6.11}
\tr\bform{ax}{y} = \tr\bform{x}{a^\invl y} \quad \text{for } a\in D, x,y,\in E_F
\end{equation}

Let $\bform{}{}_1$ be a fixed non-degenerate $\invl$-hermitian form on $E$.
Let $\cform{}{}$ ba an $F$-bilinear form on $E_F$ satisfying $(\ref{eq:6.11})$.
Since $\tr\bform{}{}_1$ is non-degenerate we can write
\begin{equation}\label{eq:6.12}
\cform{}{}=\tr\bform{Tx}{u} \quad x,y\in E_F
\end{equation}
for some $T\in \Hom_F(E_F)$. We compute
\[
\tr\bform{Tax}{y} = \cform{ax}{y} =\cform {x}{a^\invl y}_1 
= \tr\bform{Tx}{a^\invl y}_1 = \tr\bform{aTx}{y}_1
\]
Therefor actually $T$ in in $\Hom_D(E)$, so that if
\begin{equation}\label{eq:6.13}
\bform{x}{y}_2 = \cform{Tx}{y}_2
\end{equation}
then $\cform{}{}_2$ is a $\invl$-sesquilinear form on $E$ such that 
\begin{equation}\label{eq:6.14}
\cform{}{} = \tr\bform{}{}
\end{equation}
Thus we have shown

\begin{prop}\label{prop:6.4}
If $D$ is a division algebra over a field $F$, with involution $\invl$, 
and $E$ is a (left) vector space over $D$, then the map 
\[
\bform{}{} \mapsto \tr\bform{}{}
\]
establishes an isomorphism between $\invl$-sesquilinear form on $E$
and bilinear forms on $E_F$ satisfying $(\ref{eq:6.11})$. Under this 
map $\invl$-(anti) hermitian form correspond to (anti) symmetric forms.
\end{prop}

Remark: this shows in particular we may always lift forms satisfying 
$(\ref{eq:6.11})$ to bilinear forms over the $\invl$-fixed subfield 
of the center of $D$.

Return to our reductive dual pair $(G,G')$ acting irreducibly on 
$W$. From the standard double commutant theory we know we can finde 
a division algebra $D$ and a left $D$-vector space $V$ and $V'$ 
such that $W\cong \Hom_D(V,V')$ in such a manner that the action of $G$ 
is identified to right multiplication by $D$ linear mappings of $V$ 
and the action of $G'$ is given by left multiplication by $D$ linear 
maps of $V'$. To prove 
proposition~\ref{prop:6.3} we have to find an involution $\invl$ of $D$ 
and $\invl$-sesquilinear forms $\bform{}{}$ and $\bform{}{}'$ on $V$
and $V'$ respectively, invariant by $G$ and $G'$ respectively, 
one $\invl$-hermitian and the other $\invl$-antihermitian.

Suppose we find two involutions $\invl$ and $\sharp$ for $D$ which 
have the same restriction to the center of $D$. Then $\invl\sharp$ 
is an automorphism of $D$ over its center, so by Skolem-Nother\footnote{
Skolem-Noether theorem:
let $A$ and $B$ be central simple algebras over $K$, 
Suppose that the dimension of $B$ over the field $K$ is finite,
Then if
\[f,g \colon A \to B\]
are $K$-algebra homomorphisms, 
there exists a unit $b$ in $B$ such that
\[g(a) = b��f(a)b^{-1}\]
for all $a$ in $A$.
}
\begin{equation}\label{eq:6.15}
\invl\sharp(d) = \delta d \delta^{-1} \quad \text{for } d\in D.
\end{equation}
for some $\delta\in D$. Thus involutions on $D$ having a given restriction 
to the center differ by inner automorphisms of $D$.

Consider a vector space $U$ over $F$, and a non-degenrate bilinear form
$\bform{}{}$ on $F$, either symmetric or antisymmetric. 
Then $\bform{}{}$ induces an $\End(U)$ an involution $\sharp$ defined by 
\begin{equation}\label{eq:6.16}
\bform{Tu_1}{u_2} = \bform{u_1}{T^\sharp u_2} \quad u_i \in U, T\in \End(U)
\end{equation}
and satisfying the familiar rules:
\begin{equation}\label{eq:6.17}
\begin{split}
i)& \quad(ST)^\sharp = T^\sharp S^\sharp \quad  (S+T)^\sharp = S^\sharp + T^\sharp\\
ii)&\quad T^{\sharp\sharp} = T \\
iii)& \quad \text{The isometries of } \bform{}{} \text{ are } 
\Set{g\in GL(U)|g^\sharp = g^{-1}}
\end{split}
\end{equation}

Supppose $G$ is a group acting irreducibly on $U$. Let $L$ be
the span of $G$ in $End(U)$ and let $D$ be the commuting algebra. 
Then $L$ is a simple algebra, $D$ is a division algebra and 
$D\cap L$ is their common center\footnote{
$C_{L}L = D \cap L$ by definition.
by von Neumann bicommutant theorem,
$L$ is the commutator of $D$. So $C_D D = L\cap D$.}. 
Suppose $G$ preserves the form $\bform{}{}$ of the last paragraph.
Then clearly, from $(\ref{eq:6.17})$ iii) we see $\sharp$ preserves $G$,
hence $L$ hence $D$. 
Thus $D$ is a division algebra with involution. 
If $\bform{}{}_1$ is another $G$-invariant form inducing the involution 
$\sharp_1$, then again by $(\ref{eq:6.17})$ iii), 
both $\sharp$ and $\sharp_1$ agree on $L$, hence they agree on the center
of $D$.  We can say more. Since $G$ acts irreducibly, the form $\bform{}{}$
must be non-degenerate. Hence we may write
\[
\bform{u_1}{u_2}_1 = \bform{Tu_1}{u_2}
\]
for some $T$ in $\End(U)$. Since $\bform{}{}_1$ is also $G$-invariant, 
we must have $T\in D$. The relation between $\sharp$ and $\sharp_1$
is easily seen to be 
\[
S^{\sharp_1} = (T^\sharp)^{-1} S^\sharp T^\sharp.
\]
Hence we can demonstrate directly the conjugacy between $\sharp$
and $\sharp_1$ in this situation.

Now take $G$ to be the first member of our pair $(G,G')$ and take
$U\subseteq W$ an irreducible subspace for $G$. Then the $D$ 
of the above paragraph is isomorphic to the $D$ in the 
isomorphism $W\simeq \Hom(V,V')$. 
Further $U$ is isomorphic to $V$ as a joint $G$ and $D$ module. 
On $U$ there are many bilinear forms invariant by $G$, 
which we may transfer to $V$. Namely, for $g'_1, g'_2$ in $G'$, 
consider 
\begin{equation}\label{eq:6.18}
B_{g_1,g_2}(u_1,u_2) = \inn{g'_1 (u_1)}{g'_2(u_2)}
\end{equation}
Some of these forms will be non-degenerate\footnote{
Consider cannoical map $W\to U^*$ by $w \mapsto \inn{}{w}$. 
it is a $G$-intertwing map, hence, there is a subspace $U'$ of $W$
isomorphism to $U^*$, since $W$ has only one $G$-type, $U'\simeq U$,
Since $U\cap U'$ is $G$ invariant, hence two cases $U=U'$ or $U\cap U'=0$.
Then choose $g'$ be identity map, or extend the isomorphism $U\cong U'$. 
}, and so 
will their symmetric or antisymmetric parts.
Thus we may find at least one form $B$, either symmetric or 
antisymmetric on $V$ invariant by $G$. 
If $\invl$ is the involution of $D$ induced by $B$, then according 
to proposition~\ref{prop:6.4}, there is a $\invl$-hermitian 
or $\invl$-antihermitian form $\bform{}{}$ on $V$ invariant by $G$\footnote{
passing to $\tr\inn{}{}$, $\invl$ action are same,
 by $\tr(\delta d \delta^{-1})=\tr(d)$.
}.

By similar reasoning, and by adjusting our involutions as explained above, 
we can also find a form $\bform{}{}'$ on $V'$ either 
$\invl$-hermitian or $\invl$-antihermitian, 
and invariant by $G'$. Thus we have forms on $V$ and $V'$. 
It remains only to show we can take exactly one to 
be $\invl$-hermitian and the other to be $\invl$-antihermitian.
To do this, it is more convenient to use the involution 
on $D$ to regard $V'$ as a right vector space and write 
$W\simeq V\otimes_D V'$. Take $v_i\in V$ and $v'_i\in V'$, 
and consider the quantity
\begin{equation}\label{eq:6.19}
\cform{v_1\otimes v'_1}{v_2\otimes v'_2} 
= \tr(\bform{v_1}{v_2}\bform{v'_2}{v'_1}')
\end{equation}
If $d\in D$, we compute
\[
\begin{split}
&\cform{(dv_1)\otimes v'_1}{v_2\otimes v'_2}
=\tr(\bform{dv_1}{v_2}\bform{v'_2}{v'_1}')\\
=\,& \tr(d\bform{v_1}{v_2}\bform{v'_2}{v'_1}')
= \tr(\bform{v_1}{v_2}\bform{v'_2}{v'_1}'d)\\
=\,& \tr(\bform{v_1}{v_2}\bform{v'_2}{d^\invl v'_1}')
= \cform{(v_1)\otimes d^\invl v'_1}{v_2\otimes v'_2}
\end{split}
\]
A similar relation holds in the second variable. It follows that 
$\cform{}{}$ factors to define a $F$-bilinear form on $V\otimes_D V'$.
Moreover $\cform{}{}$ will be symmetric if $\bform{}{}$
and $\bform{}{}'$ have the sam parity, and 
will be anti-symmetric if they have opposite parties. 
Since $\cform{}{}$ is obviously $G\cdot G'$ invariant, 
we have $\cform{w_1}{w_2} = \inn{c w_1}{w_2}$ where 
$c$ commutes with $G\cdot G'$. 
That is, $c$ is in the center of $D$. If we replace $\bform{}{}$
with the form 
\[
v_1, v_2\mapsto \bform{c^{-1} v_1}{v_2}
\]
befor we perform the construction $(\ref{eq:6.19})$, then we 
will have $c=1$, or $\cform{}{}=\inn{}{}$.
Since $\inn{}{}$ is antisymmetric, the symmetry properties of 
$\bform{}{}$ and $\bform{}{}'$ are as desired. 

Finally, we see that the full isometry group of $\bform{}{}$
perserves $\inn{}{}$, commutes $G'$ and contains $G$, 
so it is equal to $G$. Similarly, 
$G'$ is the full isometry group of $\bform{}{}'$.
This concludes proposition~\ref{prop:6.3}.

To round out this section, we will review the rudiments of the 
classification theory for division algebras with involution. 
See [] for a more complete discussion. 
Given a field $F$, the set of isomorphism classes of simple algebras 
central over $F$ forms a semigroup under tensor product and 
if one factors out by the matrix algebras over $F$, one obtains 
a torsion group called the {\em Brauer group}\index{Brauer group} of $F$.
Given a simple algebra $M$ over $F$, let $\Set{M}$ denote its class in
the Brauer group. Then $\Set{M}^{-1}$ is represented by 
the {\em opposed algebra}\index{opposed algebra} $\widetilde{M}$ 
of $M$ -- the same space with reversed order of multiplication.

Let $\overline{F}$ be a separable algebraic closure of $F$,
with multiplicative group $\overline{F}^\times$. Let $\Gamma$
be the Galois group of $\overline{F}$ over $F$. 
A basic result  []  is that the Brauer group of $F$ 
has an alternative description as the Galois cohomology 
group $H^2(\Gamma;\overline{F}^\times)$. Suppose $F'\subseteq \overline{F}$
is a finite Galois extension of $F$. 
Then $\Gal(F'/F)$ acts in a natural way (via its action on cocycles) 
on the Brauer group $H^2(\Gamma'; \overline{F}^\times)$, 
where $\Gamma'$ is the Galois group of $\overline{F}$ over $F'$. 
Call this action $\alpha$. Let $M_1$ and $M_2$ be simple algebras 
central over $F'$ and let $\varphi\colon M_1\to M_2$ be an $F$-linear
isomorphism. 
If $\varphi|_{F'} = h \in \Gal(F'/F)$, then $\Set{M_2}=\alpha(h) \Set{M_1}$.

Suppose now $M$ is a simple algebra over $F'$ with involution $\invl$,
and let $F\subseteq F'$ be the fixed field of $\invl$.
Then $F'=F$ or $F'$ is quadratic over $F$. In any case, 
the restriction $\invl|_{F'}$ is a generator $\sigma$ of 
$\Gal(F'/F)$. Combining $\invl$ with the standard antiautomorphism of
$L$ with $\tilde{L}$, the opposite algebra, we conclude that 
$\alpha(\sigma) \Set{L} = \Set{\tilde{L}} = \Set{L}^{-1}$, 
or in other words,
\begin{equation}\label{eq:6.20}
(\alpha(\sigma)+1)\Set{L} = 0
\end{equation}
If $\alpha(\sigma)$ reduces to the identity, then $(\ref{eq:6.20})$
just says $2\Set{L}=0$, 
or $\Set{L}$ is of order $2$ in the Brauer group. 
More detailed statements depend on the finer structure of $F'$.

Let us now take $F'$ to be a local field. The Brauer groups 
of local fields are know []. If $F'$ is non-Archimedean, 
then its Brauer group is $\bQ/\bZ$. The Brauer group of $\bR$
is $\bZ/2$ and that of $\bC$ is trivial. 
Shafarevich's Theorem [] shows the Galois acton is trivial.
Therefore in these cases, the only possibilities for division algebras
over $F$ such that $F$ is the fixed field of the center are
a) $F$ itself; b) a quadratic extension $F'$ over $F$; 
c) the unique quaternion algebra over $F$;
d) the quaternion algebra over a quadratic extension of $F$. 
However, as it turns out, possibility d) is also impossible. 
Hence there are in fact only $3$ possibilities for $D$.

The situation for global fields is more complicated because the 
Galois action on the Brauer group can be non-trivial. 
See [] for some discussion of this.

\section{Lie algebras of the classical groups; Cayley transform}  
\label{sec:7}
The groups which emerged from the discussion of \S\ref{sec:6}, 
that is, the general linear group of a vector space over a division algebra,
and the isometry group of a hermitian or antihermitian sesquilinear form
over a division algebra with involution, are often referred to as
{\em classical groups}\index{classical groups}. 
Sometimes other groups closely related to these, e.g., special linear groups,
etc., are also called classical groups. 
However, in the present paper, a classical group will be precisely 
one of the groups specified in proposition~\ref{prop:6.3} and \ref{prop:6.4},
in other words, one member of a reductive dual pair in $\Sp$.

In many parts of the development of the theory of reductive dual pairs,
the type I and type II groups, e.g., isometry groups and $\GL$,
must be discussed separately, a circumstance which predictably leads at 
times to considerable tedium. Sometimes explicit separate discussion of
$\GL$ can be avoided if we make the convention that $\GL$ is the isometry
group of the zero form. That is, results stated for isometry groups 
are formally correct for $\GL$ under this interpretation. Thus below and
elsewhere where there is no explicit treatment of $\GL$ separate from 
isometry groups, the results are to be interpreted as holding for $\GL$ 
as isometries of the trivial form (and ignoring the fact that $D$ 
should have an involution), unless the results are stated explicitly for type I
classical groups. 

Let $G$ be a classical group, the isometry group of the hermitian 
or antihermitian form $\bform{}{}$ on the vector space $V$ over the 
division algebra $D$ with involution $\invl$. We will call 
$(V, D, \invl, \bform{}{})$ the {\em basic data}\index{basic data}
of $G$ and will always consider $G$ coming with this data attached.
Thus $G$ is not simply an isomorphism class of groups, but is acting
on a particular space in a particular way. Regarding the form $\bform{}{}$, 
it will be said to be of {\em type $(D,\invl, +)$}\index{type!$(D,\invl, +)$}
or {\em type $(D,\invl, -)$}\index{type!$(D,\invl, -)$} depending on 
whether it is $\invl$-hermitian or $\invl$-antihermitian. 
The type $(D, \invl, -)$ will be said to bo of {\em dual type}\index{dual type}
to the type $(D, \invl,+)$, and vice versa. 

The {\em Lie algebra}\index{Lie algebra} $\fg$ of $G$ is defined heuristically
as the collection $T\in \End_D(V)$ such that, if $\epsilon$ is an infinitesimal
-- a non-zero quantity so small its square is zero, then $I+\epsilon T$
is an isometry of $\bform{}{}$. Formally this amouts to the identity
\begin{equation}\label{eq:7.1}
\bform{Tv}{v'} + \bform{v}{Tv'}= 0 \quad T\in \fg, v, v'\in V
\end{equation}
It is seay to check that $\fg$ as defined by $(\ref{eq:7.1})$ is indeed a 
Lie algebra, i.e., is closed under taking commutators. 
Also if $T\in \fg$ and $g\in G$, we define
\begin{equation}\label{eq:7.2}
\Ad g (T) = g T g^{-1}
\end{equation}
It is easy to compute that $\Ad G$ preserves $\fg$, so $\Ad$ defines an action
of $G$ on $\fg$.

Let $\tB(V)$ denote the space of forms on $V$ of the type dual to $\bform{}{}$.
There is a natural action $\sigma$ of $\GL_D(V)$ on $\tB(V)$ defined 
by 
\begin{equation}\label{eq:7.3}
\sigma(A) \beta(v,v') = \beta(A^{-1}v, A^{-1}v) \quad \text{for } 
A\in \GL_D(V), \beta \in \tB(V)
\end{equation} 
For $(\ref{eq:7.1})$ it is straightforward to verify the follwoing fact. 

\begin{prop}\label{prop:7.1}
Define $\beta\colon \fg \to \tB(V)$ by 
\begin{equation}\label{eq:7.4}
\beta_T(v,v') = \bform{Tv}{v'}
\end{equation}
Then $\beta$ is an isomorphism form $\fg$ to $\tB(V)$ and is equivariant 
for the actions $\Ad$ and $\sigma$ of $G$.\footnote{
$\beta_T(v,v')=(Tv,v') = -(v,Tv') = - \epsilon (Tv',v) 
= -\epsilon \beta_T(v',v)$
where $\epsilon=\pm 1$ is the type of $\bform{}{}$. Hence the map is well 
defined, it is injective by non-degenerate of $\bform{}{}$.
Same argument as above, $\beta$ is surjective. 
Hence isomorphism as vector space.
It is easy to check $\beta$ is an intertwining map.
}
\end{prop}

For $G=\GL_D(V)$, we ignore proposition~\ref{prop:7.1} and simply note
that $\fg$ is all of $\End_D(V)$. 

Let $G$ and $G'$ be classical groups with basic data $(V, D, \invl, \bform{}{})$
and $(V', D, \invl, \bform{}{}')$ respectively. Take $\bform{}{}$
and $\bform{}{}'$ to be of dual type, so that $(G,G')$ acting on 
$\Hom_D(V,V')$ by right and left multiplication form an irreducible type I
reductive dual pair. We have defined the isomorphisms 
$*\colon \Hom_D(V,V')\to \Hom_D(V',V)$ and $*'$ in the reverse direction 
in $(\ref{eq:6.5})$.

\begin{prop}\label{prop:7.2}
The map 
\begin{equation}\label{eq:7.5}
\ttau\colon T\mapsto T^*T \quad \Hom_D(V,V')
\end{equation}
has image in $\fg$. Similarly the map 
\begin{equation}\notag
\ttau'\colon S\mapsto S^{*'}S \quad \Hom_D(V',V)
\end{equation}
has image in $\fg'$. Moreover
\begin{equation}\label{eq:7.5a}\tag{\ref{eq:7.5}) a}
\begin{split}
\ttau(g'Tg^{-1}) &= \Ad g (\tau(T))\\
\ttau'(gS g'^{-1}) &= \Ad g'(\tau'(S))
\end{split}
\end{equation}
Hence the image under $\ttau$ of a $G\cdot G'$ orbit in $\Hom_D(V,V')$ 
is an $\Ad G$ orbit in $\fg$, and the image of a $G'$ orbit is a single point;
and similary for $\ttau'$.
\end{prop}

Remark: By this proposition we see that to each $G\cdot G'$ orbit $\oO$ in
$W=\Hom_D(V,V')\simeq \Hom_D(V',V)$ we can attach a pair of orbits 
$(\ttau(\oO), \ttau'(\oO)$ in $\fg$ and $fg'$. Thus we obtain a correspondence
$\ttau(\oO)\leftrightarrow \ttau'(\oO)$ between certain $\Ad G$ 
orbits in $\fg$ and certain $\Ad G'$ orbits in $G'$. We will see 
in \S\ref{sec:8} that this correspondence is ``generically'',
i.e., for $\oO$ in some Zariski open set, bijective. 
This phenomenon was to my knowledge first made explicit in [].

\proof: We recall from $(\ref{eq:6.5})$ the definition of $*$
\[
\bform{Tv}{v'}' = \bform{v}{T^*v} \quad T\in \Hom_D(V,V')
\]
Therefore
\[
\begin{split}
& \bform{T^*Tv_1}{v_2} = \pm\bform{v_2}{T^*Tv_1}^\invl = 
\pm\bform{Tv_2}{Tv_1}'^\invl \\
=& -\bform{Tv_1}{Tv_2}' = -\bform{v_1}{T^*Tv_2}
\end{split}
\]
Comparing with $(\ref{eq:7.1})$ we find that $T^*T$ satisfies the 
condition to belong to $\fg$. We also compute 
\[
\begin{split}
\bform{g'Tg^{-1}v}{v'}' &= \bform{Tg^{-1}v}{g'^{-1} v'}' = 
\bform{g^{-1} v}{T^* g'^{-1}v'} \\
&= \bform{v}{gTg'^{-1} v'},
\end{split}
\]
whence
\begin{equation}\label{eq:7.6}
(g'Tg^{-1})^* = g T g'^{-1}.
\end{equation}
Equation $(\ref{eq:7.5a})$ is immediate form $(\ref{eq:7.6})$.

Remark: Let $F$ be a subfield of the center of $D$, in the fixed field of 
$\invl$, such that $D$ is a finite dimensional over $F$ and its 
center is separable over $F$. Then $\tr=\tr(D/F)$, 
the reduced trace of $D$ over $F$ is defined on $D$ and also on 
$\Hom_D(V,V)$. Since $\fg$ is the lie algebra of the isometries of some 
non-degenerate form, $\tr$ vanishes on $\fg$. 
Hence $\tr(T^*T)=0$, for $T\in \Hom_D(V,V')$, which shows that 
$(\ref{eq:6.7})$ does define an alternating form on $W=\Hom_D(V,V')$.
Formula $(\ref{eq:7.5a})$ shows $(\ref{eq:6.7})$ is invariant 
by $G$ and by $G'$. Hence altering the identification of $\Hom_D(V,V')$
with $W$ by an element of the center of $D$ if necessary, 
we can indeed arrange that the form $\inn{}{}$ on $W$ be given by
$(\ref{eq:6.7})$.

Let $F$ continue as in the above remark. Although $\fg \subseteq \Ker\,\tr$,
the bilinear form
\begin{equation}\label{eq:7.7}
T, S \mapsto \tr TS \quad T,S\in \fg
\end{equation}
is a non-degenerate symmetric, $\Ad$-invariant bilinear form on $\fg$.
Indeed symmetry and $\Ad$-invariance are standard facts. As to non-degeneracy,
recall that the form $\bform{}{}$ an involution $\sharp$ on $\End_D(V)$ 
according to the recipe $(\ref{eq:6.16})$. In terms of $\sharp$, 
formula $(\ref{eq:7.1})$ can be written 
\begin{equation}\label{eq:7.8}
\fg = \Set{T\in \End_D(V)| T^\sharp = -T}
\end{equation}
Therefore the map $T\mapsto \frac{1}{2}(T- T^\sharp)$ projects $\End_D(V)$ 
onto $\fg$, and we have 
\begin{equation}\label{eq:7.9}
\End_D(V) = \fg \oplus \fs
\end{equation}
where
\begin{equation}\label{eq:7.10}
\fs = \Set{T\in \End_D(V)| T=T^\sharp}
\end{equation}
Since $\tr T^\sharp = \tr T$, the decomposition $(\ref{eq:7.9})$ is orthogonal 
for the pairing $(\ref{eq:7.7})$. Since $\tr TS$ is non-degenerate on
$\End_D(V)$, it is also non-degenerate on $\fg$.

\section{Witt's Theorem and orbits}\label{sec:8}
\printindex

\end{document}
