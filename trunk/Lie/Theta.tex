\documentclass[12pt]{amsart}
\usepackage[margin=3cm]{geometry}
\usepackage[hyperindex=true]{hyperref}

\usepackage{amssymb}
\usepackage{amsxtra}
\usepackage{graphicx}

\usepackage{braket}
\usepackage{paralist}
\usepackage{eufrak}

\usepackage{dsfont}

\newtheorem{thm}{Theorem}[section]
\newtheorem{lem}[thm]{Lemma}
\newtheorem{prop}[thm]{Proposition}
\newtheorem{cor}[thm]{Corollary}

%\usepackage{makeidx} 
\makeindex

\def\Ker{\rm{Ker}}
\def\Im{\rm{Im}}
\def\Hom{\rm{Hom}}
\def\Mat{\rm{Mat}}
\def\bR{{\mathbb{R}}}
\def\bN{{\mathbb{N}}}
\def\bZ{{\mathbb{Z}}}
\def\bC{{\mathbb{C}}}
\def\bQ{{\mathbb{Q}}}
\def\vv{{\vec{v}}}
\def\vw{{\vec{w}}}
\def\vx{{\vec{x}}}
\def\vy{{\vec{y}}}
\def\v0{{\vec{0}}}
\def\ol{\overline}
\def\sspan{\rm{span}}
\def\sl2{{\mathfrak{sl}(2)}}
\def\slc{{\mathfrak{sl}(2,\bC)}}
\def\Sp{\mathrm{Sp}}
\def\GL{\mathrm{GL}}
\def\sP{\mathcal{P}}
\def\sH{\mathcal{H}}
\def\sU{\mathcal{U}}
\def\sC{\mathcal{C}}
\def\kbar{{k\makebox[-0.06em][r]{\raisebox{0.5ex}{--}}}}
\def\ad{{\rm ad}}
\def\Ad{{\rm Ad}}
\def\id{{\rm id}}
\def\sgn{{\rm sgn}}
\def\gcd{{\rm gcd}}
\def\inn#1#2{\left<{#1}\,,\,{#2}\right>}
\def\bform#1#2{\left({#1}\,,\,{#2}\right)}
\def\cform#1#1{\left\{{#1}\,,\,{#2}\right\}}
\def\isoarrow{\tilde{\rightarrow}}
\def\tr{{\rm tr}}
\def\invl{{\natural}}

\hypersetup{
    bookmarks=false,         % show bookmarks bar?
    unicode=false,          % non-Latin characters in Acrobat��s bookmarks
    pdftoolbar=true,        % show Acrobat��s toolbar?
    pdfmenubar=false,        % show Acrobat��s menu?
    pdffitwindow=false,      % page fit to window when opened
    pdftitle={Qualify Examination Answers - Algebra},    % title
    pdfauthor={Ma Jia Jun},     % author
    pdfsubject={Subject},   % subject of the document
    pdfcreator={Creator},   % creator of the document
    pdfproducer={Producer}, % producer of the document
    pdfkeywords={keywords}, % list of keywords
    pdfnewwindow=true,      % links in new window
    colorlinks=true,       % false: boxed links; true: colored links
    linkcolor=blue,          % color of internal links
    citecolor=green,        % color of links to bibliography
    filecolor=magenta,      % color of file links
    urlcolor=cyan           % color of external links
}


\title{}

\begin{document}
\maketitle
\section{Heisenberg groups}
\begin{prop}\label{prop:1.1}
\end{prop}
\begin{prop}\label{prop:1.2}
Give $W$ the structure of $A$-module as in proposition~\ref{prop:1.1} d). 
Then $\inn{}{}\colon W\times W \to A$ is an $A$-bilinear form which 
is skew-symmetric and non-degenerate in the strong sense that the map 
$\alpha\colon W \to \Hom_A(W,A)$ defined by 
\begin{equation}\label{eq:1.5}
\alpha(w)(w') = \inn{w}{w'}
\end{equation}
is an isomorphism of $A$-modules.
\end{prop}

\section{Structure of $\Sp(W)$; free polarizations}
\begin{prop}\label{prop:5.2}
\begin{enumerate}[a)]
\item $\Sp$ acts transitively on the set of free polarizations of rank $m$.
\item Let $U\subseteq W$ be a free polarization, and let $P(U,W)= P(U) = P$
be the subgroup of $\Sp$ leaving $U$ invariant. Then the restrictio map 
\begin{equation}\label{eq:5.2}
r\colon P\rightarrow \GL(U)
\end{equation}
is surjective.
\item Let $N(U,W) = N(U)=N$ be the kernel of the map $r$ of $(\ref{eq:5.2})$.
Then $N$ acts simply transitively on the set of complete polarizations with
$U$ as first member. Equivalently $N$ acts simply transitively on 
the set of free polarizations of $W$ complementray to $U$.
\item Fix a free polarization $U'$ complementry to $U$. 
Then there is an isomorphism 
\begin{equation}\label{eq:5.3}
\beta\colon N \isoarrow S^{2*}(U'),
\end{equation}
where $S^{2*}(U')$ denots the space of symmetric bilinear forms on $U'$.
\item Let $M=P(U)\cap P(U')$. Then $M$ is a complement to $N$ 
in $P$, so that $P$ is a semidirect product.
\begin{equation}\label{eq:5.4}
P\simeq M \ltimes N
\end{equation}
\end{enumerate}
\end{prop}

\section{Reductive dual pairs, definition and classification}
Let $\Gamma$ be a group $(G, G')$ be a pair of subgroups of $\Gamma$. 
We will say $(G,G')$ form a {\em dual pair}\index{dual pair} 
of subgroups of $\Gamma$ if
$G$ is the centralizer in $\Gamma$ of $G'$ and vice-versa. 

It is not hard to finde dual pairs of subgroups in a group. Start with any 
subgroup $G\subseteq \Gamma$. Let $G'$ be the centralizer of 
$G$ in $\Gamma$ and let $G''$ be the centralizer of $G'$. Then $(G'', G')$ is
a dual pair in $\Gamma$.

When we take $\Gamma=\Sp(W)$ for some symplectic vector space, we can refine 
the concept slightly. We say $(G, G') \subseteq \Sp(W)$ form a
{\em reductive dual pair}\index{dual pair!reductive} if first 
$(G,G')$ is a dual pair in $\Sp$, and moreover $G$ and $G'$ are reductive in 
the sense that they act (absolutely) reductively on $W$.

The goal of this section is to describe reductive dual pairs in $\Sp$. 
Let $(G, G')$ be such a pair. Suppose $W=W_1\oplus W_2$ is an orthognal 
direct sum, and that each $W_i$ is invariant by $G$ and by $G'$. Let
$G_i = G|_{W_i}$ be the group of transformations of $W_i$ obtained by 
restricting elements of $G$. Define $G_i$ similarly. 
Let $r_i\colon G\to G_i$, and $r'_i$ be the obvious maps. 
Then half a moment's thought convinces one that 
$r_1\times r_2\colon G\to G_1\times G_2$ is an isomorphism\footnote{
it is injective by the vector space decomposition $W = W_1\oplus W_2$, 
it is surjective since $G$ is the comutator of $G'$, 
and $G|W\times \Set{1}$ comutes the action of $G'$. }, and similarly for 
$r'_1\times  r'_2$. Moreover, $(G_i,G'_i)$ will be a reductive dual pair
in $\Sp(W_i)$. To describe this situation we will say that $(G,G')$ 
is the {\em direct sum}\index{direct sum} of $(G_i, G'_i)$.
If $(G,G')$ has no non-trivial direct sum decomposition, then we will call 
$(G,G')$ {\em irreducible}\index{irreducible}. 

\begin{prop}\label{prop:6.1}
\begin{enumerate}[a)]
\item Every reductive dual pair is the direct sum of irreducible subpairs in an 
essentially unique way (i.e., up to numbering of the pairs)
\item If $(G,G')$ is irreducible, then either:
\begin{enumerate}[i)]
\item $G\dot G'$ acts irreducible on $W$, and $W$ consists of a single 
isotypic component (which is self-dual) for $G$ or for $G'$; or
\item $W= U_1 \oplus U_2$, where each $U_i$ is invariant and irreducible for 
$G\dot G'$, and is maximal isotropic in $W$. The restriction map then take
$(G,G')$ to a dual pair in $\GL(U_i)$.
\end{enumerate}
\end{enumerate}
\end{prop}
\proof
Consider first the action of $G$ alone on $W$. Let $V_1$ and $V_2$ be 
irreducible $G$-subspaces of $W$. If the pairing between $V_1$ and $V_2$ 
induced by $\inn{}{}$ is non-trivial, then it must be non-degenerate,
by irreducibility of the $V_i$, and thus will induce a $G$-equivariant 
isomorphism between $V_2$ and $\Hom(V_1, F)=V_1^*$. 
(We use the conventional $*$ to denote dual space now that we are working over
a field.) Thus $V_1$ must be orthogonal to any $G$-submodule of $W$ except 
one isomorphic to $V_1^*$; and since $W$ is the direct sum of 
irreducible $G$-modules, since $G$ is assumed to act reductively on $W$, 
there will be a $G$-submodule of $W$ isomorphic to $V_1^*$ paired 
non-degenerately with $V_1$. 

Consider the decomposition $W=\bigoplus_i U_i$ of $W$ into isotypic components 
for $G$. That is, each $U_i$ is the sum of all $G$-submodules of some given 
isomorphism type. By the preceding paragraph, we see that either
the isomorphism type defining $U_i$ is self-dual and that $\inn{}{}|_{U_i}$
will be non-degenerate; or there is another isotypic component $U_j$
containing the dual isomorphism type to that of $U_i$, 
and that each of $U_i$ and $U_j$ are isotropic, but $\inn{}{}$ is 
non-degenerate on $U_i\oplus U_j$.

Thus we may write $W=\bigoplus_j U'_j$ where each $U'_j$ is either self-dual
isotypic for $G$, or the sum of two mutually contragredient isotypic 
components. This sum is orthogonal. The $U'_j$ are clearly determined uniquely 
up to order by $G$ and are invariant by $G'$. Therefore 
they give a direct sum decomposition of $(G, G')$. To prove the proposition, 
therefor, it will sufficent to show that each $G$-isotypic component is irreducible under $G\cdot G'$. 

Consider a subspace $U\subseteq W$ which is irreducible for $G\cdot G'$. 
Then $U$ must consist of a single isotypic component for either $G$ 
or $G'$. The form $\inn{}{}$ is either trivial or non-degenerate on $U$. 
Suppose first $\inn{}{}$ is non-degenerate. Then $W= U\oplus U^\perp$ is a 
decomposition of $(G,G')$.  Since the operator which is the identity on $U$
and minus the identity on $U^\perp$ commutes with both $G$ and $G'$, it must 
belong to both. 
Therefore $U$ and $U^\perp$ contain no isomorphic $G$-submodules\footnote{if has
there exits an non-trivial intertwining map $T\colon U\to U^\perp$, 
then $-Tu=gTu =Tgu = Tu$, a contradicitoion},
so $U$ is a full isotypic component in $W$ for $G$ and $G'$. This is case 
$b)i)$. 

Secondly, suppose $\inn{}{}$ is trivial on $U$. 
Then there is another $G\cdot G'$-irreducible subspace $V$ which is paired 
non-trivially, hence non-degenerately, with $U$. I claim $V$ must also be 
isotropic. Otherwise, we could write $W=V\oplus V^\perp$ and reason as just 
above. But then $U$ would be the graph of a non-trivial $G\cdot G'$ 
intertwining morphism form $V^\perp$ to $V$, which, we saw above, does not 
exist. Thus $U$ and $V$ are both isotropic. 
Write $W=(U\oplus V) \oplus (U\oplus V)^\perp$. Observe that multiplicatio by 
a scalar $t$ on $U$, by $t^-1$ on $V$ and by $1$ on $(U\oplus V)^\perp$
defines an element of $\Sp$ commuting with both $G$ and $G'$, so belonging
to both. Thus again $U$ must be a full $G$-isotypic component in $W$, 
and the proposition is proved. 
\qed

Let $(G,G')$ be a reductive dual pair in $\Sp$. We will say $(G,G')$
is of {\em type I}\index{type I} or {\em type II}\index{type II} according as
possibility i) or possibility ii) of proposition~\ref{prop:6.1} b) obtains. 
We proceed to describe more precisely the pairs of the two types. 
We begin with type II pairs, as these are somewhat simpler than type I pairs.

\begin{prop}\label{prop:6.2}
Let $(G,G')\subseteq \Sp$ be an irreducible type II reductive dual pair. 
Let $(U_1, U_2)$ be the complete polarization of $W$ invariant by $G\cdot G'$,
so that $W=U_1\oplus U_2$ and each of $U_1$ and $U_2$ are isotropic and 
irreducible under $G\cdot G'$. Then restriction to $U_i$ embeds 
$(G,G')$ as a reductive dual pair in $GL(U_i)$. Thus there is 
\begin{enumerate}[i)]
\item a division algebra $D$, and
\item a right vector space $V$ and a left vector space $V'$ over $D$, 
\end{enumerate}
such that $U_1$ is isomorphic to $V\otimes_D V'$ in such fashion 
that $G$ is identified to $\GL_D(V)\otimes I_{V'}$ and $G'$ 
is identified to $I_V\otimes GL_D(V')$ where $I_V$ and $I_{V'}$
are the identity operators on $V$ and $V'$ respectively. 
\end{prop}
\proof
Both $G$ and $G'$ are in the subgroup $M$ preserving $U_1$ and $U_2$, 
as described in proposition~\ref{prop:5.2} e). By that result $M$ is identified 
by restriction with $\GL(U_i)$. Clearly, if $G'$ is the centralizer 
of $G$ in $\Sp$, it is the centralizer of $G$ in $M$ also. 
We now may apply the classical description of reductive dual pairs 
in the general linear groups, essentially amounting to the
Double Commutant Theorem of linear algebra, to obtain the existence 
of $D$ and the decompositio of $U_1$ as a tensor product\footnote{
Here view $U_1$ as a $G$ representation which has only one isotypic component,
hence $U_1 = \bigoplus_{h} V_h$ wherr all $V_h \simeq V_\sigma$, where 
$\sigma \in \hat{G}$. Then $D = \Hom_G(V_\sigma, V_\sigma)$.
}.

Remark: Since $U_2$ may be identified to $U_1^*$ as in $(\ref{eq:1.5})$,
we may write $W \simeq V\otimes_D V' \oplus V'^*\otimes_D V^*$ in such 
a way that $(G,G')$ is identified to $(\GL_D(V), \GL_D(V'))$
acting in the obvious way. 
Sometimes a slight variant of this construction is useful.
Namely, we can take two right $D$ vector space $V$ and $V'$ and identify
$U_1$ with $\Hom_D(V,V')$. Then $G$ acts by right multiplication by 
inverses and $G'$ acts by left multiplication. 
Then 
\begin{equation}\label{eq:6.1}
W\simeq \Hom_D(V,V') \oplus \Hom_D(V',V)
\end{equation}
and if $S\in \Hom_D(V,V')$ and $T\in \Hom_D(V',V)$, and
$g\in G$; $g'\in G'$, then 
\begin{equation}\label{eq:6.2}
g\cdot g'(S,T) = (g'Sg^{-1}, gTg'^{-1})
\end{equation}
This formulation has the advantage that there is a simple formula
for $\inn{}{}$. 
Assuming the identification $(\ref{eq:6.1})$ is normalized properly, 
we have
\begin{equation}\label{eq:6.3}
\inn{(S_1,T_1)}{(S_2,T_2)} = \tr(S_1T_2-S_2T_1)
\end{equation}
where notation is parallel to $(\ref{eq:6.2})$, and $\tr$ here is
reduced trace\footnote{
the usual trace get a element in $D$, than consider $D$ as a linear operator 
on $F$ vector space $D$, it has a trace, this is the value of $\tr$ here. 
} over $F$ on $End_D(V')$. 
Of course $\tr(T_2S_1 - T_1S_2)$ gives the same answer.

We turn to type $I$ pairs, which present mor variety.

\begin{prop}\label{prop:6.3}
Let $(G,G')$ be an irreducible type I reductive dual pair. Then there exist
\begin{enumerate}[i)]
\item a division algebra $D$,
\item with involution (i.e., involutory antiautomorphism) $\invl$,
\item vector spaces $V$ and $V'$ over $D$
\item with forms $\bform{}{}$ and $\bform{}{}'$, which are $D$-linear
in the first variable and either $\invl$-hermitian or $\invl$-skew hermitian
(one of each type)
\end{enumerate}
such that $W\simeq V\otimes_DV'$ in such fashion that $G$ is identified to 
the isometry group of $\bform{}{}$ and $G'$ to the isometry group of 
$\bform{}{}'$.
\end{prop}

Remarks: 
\begin{enumerate}[a)]
\item Just as in the type II case, we can slightly less symmetric and write
\begin{equation}\label{eq:6.4}
W\simeq \Hom_D(V,V')
\end{equation}
Then the action of $G$ is by permultiplication (by inverses) and
that of $G'$ is by postmultiplication. Of course we could also 
write $W\cong \Hom_D(V',V)$. 
Note that there are nice maps mediating between these two alternatives. 
Namely, define
\begin{align}
*\colon \Hom_D(V,V')&\to \Hom_D(V,V'), & \text{and} \notag \\
*'\colon \Hom_D(V',V)&\to \Hom_D(V',V), & \text{by} \notag \\
\bform{Tv}{v'}' &= \bform{v},{T^* v'} &\text{for } T\in \Hom_D(V,V') \\
\bform{Sv'}{v}' &= \bform{v'},{S^{*'} v} 
& \text{for } S\in \Hom_D(V',V) \notag
\end{align}
\end{enumerate}

\proof
We must begin with some generalities on $D$-semi-linear forms 
where $D$ is a division algebra with involution $\invl$ over some field $F$.
Consider $D$ as a left vector space over itself, 
and define
\begin{equation}\label{eq:6.8}
\bform{x}{y}_\circ = x y^\invl
\end{equation}
for $x,y$ in $D$. Then $\inn{}{}_\circ$ satisfies
\begin{align}\label{eq:6.9}
i) \quad \bform{ax}{y}_\circ &= a\bform{x}{y}_\circ,  &
 \bform{x}{ay}_\circ&=\bform{x}{y}_\circ a^\invl\quad \text{ for } a,x,y\in D\\
ii)\quad \bform{y}{x}_\circ  &= \bform{x}{y}_\circ^\invl. \notag
\end{align}

If $E$ is any left $D$ vector space, call a form $\bform{}{}$
satisfying $(\ref{eq:6.9})$ i) 
{\em $\invl$-sesquilinear}\index{$\invl$-sesquilinear}.
If it also setisfies $(\ref{6.9})$ ii) call it 
{\em $\invl$-hermitian}\index{$\invl$-hermitian}.
Evidently, if we choose a basis for $E$ and so identify 
$E$ with $D^k$, then the $k$-fold direct sum
of $\bform{}{}_\circ$ as in $(\ref{eq:6.8}$ will define 
a non-degenerate $\invl$-hermitian form on $V$.

Let $\tr$ denote the reduced trace map from $D$ to $F$. 
We recall 
\begin{align}\label{eq:6.10}
\tr(xy) &= \tr(yx) & \tr(x^\invl)&=\tr(x) & \text{for } x,y\in D 
\end{align}
The $D$ vector space $E$ may be regarded as an $F$ vector space, 
denoted $E_F$, by restriction of scalars. 
If $\bform{}{}$ is an $\invl$-hermitian form on $E$, 
then 
\[
\tr\bform{}{}\colon x,y \mapsto \tr(\bform{x}{u}) \quad x,y \in E = E_F
\]
defines a symmetric bilinear form on $E_F$.
Moreover $(\ref{eq:6.9})$ i) and $(\ref{eq:6.10})$ imply 
\begin{equation}\label{eq:6.11}
\tr\bform{ax}{y} = \tr\bform{x}{a^\invl y} \quad \text{for } a\in D, x,y,\in E_F
\end{equation}

Let $\bform{}{}_1$ be a fixed non-degenerate $\invl$-hermitian form on $E$.
Let $\cform{}{}$ ba an $F$-bilinear form on $E_F$ satisfying $(\ref{eq:6.11})$.
Since $
\printindex

\end{document}
