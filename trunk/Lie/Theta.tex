\documentclass[12pt]{amsart}
\usepackage[margin=3cm]{geometry}
\usepackage[hyperindex=true]{hyperref}

\usepackage{amssymb}
\usepackage{amsxtra}
\usepackage{graphicx}

\usepackage{braket}
\usepackage{paralist}
\usepackage{eufrak}

\usepackage{dsfont}

\newtheorem{thm}{Theorem}[section]
\newtheorem{lem}[thm]{Lemma}
\newtheorem{prop}[thm]{Proposition}
\newtheorem{cor}[thm]{Corollary}

%\usepackage{makeidx} 
\makeindex

\def\Ker{\rm{Ker}}
\def\Im{\rm{Im}}
\def\Hom{\rm{Hom}}
\def\Mat{\rm{Mat}}
\def\bR{{\mathbb{R}}}
\def\bN{{\mathbb{N}}}
\def\bZ{{\mathbb{Z}}}
\def\bC{{\mathbb{C}}}
\def\bQ{{\mathbb{Q}}}
\def\vv{{\vec{v}}}
\def\vw{{\vec{w}}}
\def\vx{{\vec{x}}}
\def\vy{{\vec{y}}}
\def\v0{{\vec{0}}}
\def\ol{\overline}
\def\sspan{\rm{span}}
\def\sl2{{\mathfrak{sl}(2)}}
\def\slc{{\mathfrak{sl}(2,\bC)}}
\def\Sp{\mathrm{SP}}
\def\GL{\mathrm{GL}}
\def\sP{\mathcal{P}}
\def\sH{\mathcal{H}}
\def\sU{\mathcal{U}}
\def\sC{\mathcal{C}}
\def\kbar{{k\makebox[-0.06em][r]{\raisebox{0.5ex}{--}}}}
\def\ad{{\rm ad}}
\def\Ad{{\rm Ad}}
\def\id{{\rm id}}
\def\sgn{{\rm sgn}}
\def\gcd{{\rm gcd}}
\def\inn#1#2{\left<{#1},{#2}\right>}

\hypersetup{
    bookmarks=false,         % show bookmarks bar?
    unicode=false,          % non-Latin characters in Acrobat��s bookmarks
    pdftoolbar=true,        % show Acrobat��s toolbar?
    pdfmenubar=false,        % show Acrobat��s menu?
    pdffitwindow=false,      % page fit to window when opened
    pdftitle={Qualify Examination Answers - Algebra},    % title
    pdfauthor={Ma Jia Jun},     % author
    pdfsubject={Subject},   % subject of the document
    pdfcreator={Creator},   % creator of the document
    pdfproducer={Producer}, % producer of the document
    pdfkeywords={keywords}, % list of keywords
    pdfnewwindow=true,      % links in new window
    colorlinks=true,       % false: boxed links; true: colored links
    linkcolor=blue,          % color of internal links
    citecolor=green,        % color of links to bibliography
    filecolor=magenta,      % color of file links
    urlcolor=cyan           % color of external links
}


\title{}

\begin{document}
\maketitle

\section{reductive dual pairs}
Let $\Gamma$ be a group $(G, G')$ be a pair of subgroups of $\Gamma$. 
We will say $(G,G')$ form a {\em dual pair}\index{dual pair} 
of subgroups of $\Gamma$ if
$G$ is the centralizer in $\Gamma$ of $G'$ and vice-versa. 

It is not hard to finde dual pairs of subgroups in a group. Start with any 
subgroup $G\subseteq \Gamma$. Let $G'$ be the centralizer of 
$G$ in $\Gamma$ and let $G''$ be the centralizer of $G'$. Then $(G'', G')$ is
a dual pair in $\Gamma$.

When we take $\Gamma=\Sp(W)$ for some symplectic vector space, we can refine 
the concept slightly. We say $(G, G') \subseteq \Sp(W)$ form a
{\em reductive dual pair}\index{dual pair!reductive} if first 
$(G,G')$ is a dual pair in $\Sp$, and moreover $G$ and $G'$ are reductive in 
the sense that they act (absolutely) reductively on $W$.

The goal of this section is to describe reductive dual pairs in $\Sp$. 
Let $(G, G')$ be such a pair. Suppose $W=W_1\oplus W_2$ is an orthognal 
direct sum, and that each $W_i$ is invariant by $G$ and by $G'$. Let
$G_i = G|_{W_i}$ be the group of transformations of $W_i$ obtained by 
restricting elements of $G$. Define $G_i$ similarly. 
Let $r_i\colon G\to G_i$, and $r'_i$ be the obvious maps. 
Then half a moment's thought convinces one that 
$r_1\times r_2\colon G\to G_1\times G_2$ is an isomorphism\footnote{
it is injective by the vector space decomposition $W = W_1\oplus W_2$, 
it is surjective since $G$ is the comutator of $G'$, 
and $G|W\times \Set{1}$ comutes the action of $G'$. }, and similarly for 
$r'_1\times  r'_2$. Moreover, $(G_i,G'_i)$ will be a reductive dual pair
in $\Sp(W_i)$. To describe this situation we will say that $(G,G')$ 
is the {\em direct sum}\index{direct sum} of $(G_i, G'_i)$.
If $(G,G')$ has no non-trivial direct sum decomposition, then we will call 
$(G,G')$ {\em irreducible}\index{irreducible}. 

\begin{prop}
\begin{enumerate}[a)]
\item Every reductive dual pair is the direct sum of irreducible subpairs in an 
essentially unique way (i.e., up to numbering of the pairs)
\item If $(G,G')$ is irreducible, then either:
\begin{enumerate}[i)]
\item $G\dot G'$ acts irreducible on $W$, and $W$ consists of a single 
isotypic component (which is self-dual) for $G$ or for $G'$; or
\item $W= U_1 \oplus U_2$, where each $U_i$ is invariant and irreducible for 
$G\dot G'$, and is maximal isotropic in $W$. The restriction map then take
$(G,G')$ to a dual pair in $\GL(U_i)$.
\end{enumerate}
\end{enumerate}
\end{prop}
\proof
Consider first the action of $G$ alone on $W$. Let $V_1$ and $V_2$ be 
irreducible $G$-subspaces of $W$. If the pairing between $V_1$ and $V_2$ 
induced by $\inn{}{}$ is non-trivial, then it must be non-degenerate,
by irreducibility of the $V_i$, and thus will induce a $G$-equivariant 
isomorphism between $V_2$ and $\Hom(V_1, F)=V_1^*$. 
(We use the conventional $*$ to denote dual space now that we are working over
a field.) Thus $V_1$ must be orthogonal to any $G$-submodule of $W$ except 
one isomorphic to $V_1^*$; and since $W$ is the direct sum of 
irreducible $G$-modules, since $G$ is assumed to act reductively on $W$, 
there will be a $G$-submodule of $W$ isomorphic to $V_1^*$ paired 
non-degenerately with $V_1$. 

Consider the decomposition $W=\bigoplus_i U_i$ of $W$ into isotypic components 
for $G$. That is, each $U_i$ is the sum of all $G$-submodules of some given 
isomorphism type. By the preceding paragraph, we see that either
the isomorphism type defining $U_i$ is self-dual and that $\inn{}{}|_{U_i}$
will be non-degenerate; or there is another isotypic component $U_j$
containing the dual isomorphism type to that of $U_i$, 
and that each of $U_i$ and $U_j$ are isotropic, but $\inn{}{}$ is 
non-degenerate on $U_i\oplus U_j$.

Thus we may write $W=\bigoplus_j U'_j$ where each $U'_j$ is either self-dual
isotypic for $G$, or the sum of two mutually contragredient isotypic 
components. This sum is orthogonal. The $U'_j$ are clearly determined uniquely 
up to order by $G$ and are invariant by $G'$. Therefore 
they give a direct sum decomposition of $(G, G')$. To prove the proposition, 
therefor, it will sufficent to show that each $G$-isotypic component is irreducible under $G\cdot G'$. 

Consider a subspace $U\subseteq W$ which is irreducible for $G\cdot G'$. 
Then $U$ must consist of a single isotypic component for either $G$ 
or $G'$. The form $\inn{}{}$ is either trivial or non-degenerate on $U$. 
Suppose first $\inn{}{}$ is non-degenerate. Then $W= U\oplus U^\prep$ is a 
decomposition of $(G,G')$.  Since the operator which is the identity on $U$
and minus the identity on $U^\prep$ commutes with both $G$ and $G'$, it must 
belong to both. 
Therefore $U$ and $U^\prep$ contain no isomorphic $G$-submodules\footnote{if has
there exits an non-trivial intertwining map $T\colon U\to U^\prep$, 
then $-Tu=gTu =Tgu = Tu$, a contradicitoion},
so $U$ is a full isotypic component in $W$ for $G$ and $G'$. This is case 
$b)i)$. 

Secondly, suppose $\inn{}{}$ is trivial on $U$. 
Then there is another $G\cdot G'$-irreducible subspace $V$ which is paired 
non-trivially, hence non-degenerately, with $U$. I claim $V$ must also be 
isotropic. Otherwise, we could write $W=V\oplus V^\prep$ and reason as just 
above. But then $U$ would be the graph of a non-trivial $G\cdot G'$ 
intertwining morphism form $V^\prep$ to $V$, which, we saw above, does not 
exist. Thus $U$ and $V$ are both isotropic. 
Write $W=(U\oplus V) \oplus (U\oplus V)^\prep$. Observe that multiplicatio by 
a scalar $t$ on $U$, by $t^-1$ on $V$ and by $1$ on $(U\oplus V)^\prep$
defines an element of $\Sp$ commuting with both $G$ and $G'$, so belonging
to both. Thus again $U$ must be a full $G$-isotypic component in $W$, 
and the proposition is proved. 
\qed



\printindex

\end{document}
