\documentclass[12pt]{article}
\usepackage[margin=2cm]{geometry}
\usepackage{pdfsync}

\usepackage{hyperref}



\usepackage{amssymb}
\usepackage{amsmath, amsthm}
\usepackage{graphicx}
%\usepackage{amscd}
\usepackage{braket}
\usepackage{paralist}
\usepackage{eufrak}
%\usepackage{calrsfs}
%\usepackage[small,nohug,heads=littlevee]{diagrams}
%\diagramstyle[labelstyle=\scriptstyle]
%\usepackage{diagrams}
\usepackage[all,cmtip]{xy}
\usepackage{diagxy}
%\usepackage{pictexwd,dcpic}
%\usepackage{mathrsfs}


\newtheorem{lemma}{Lemma}
\newtheorem{thm}{Theorem}
\newtheorem{prop}{Proposition}
\newtheorem{cor}{Corollary}
\newenvironment{expl}{\it}{\color{black}\normalsize}
\DeclareMathAlphabet{\mathpzc}{OT1}{pzc}{m}{it}

\def\sign{{\rm sign}}
\def\Ker{{\rm Ker}}
\def\Im{{\rm Im}}
\def\Hom{{\rm Hom}}
\def\End{{\rm End}}
\def\Mat{{\rm Mat}}
\def\Ind{{\rm Ind}}
\def\bR{{\mathbb{R}}}
\def\bN{{\mathbb{N}}}
\def\bZ{{\mathbb{Z}}}
\def\bC{{\mathbb{C}}}
\def\bQ{{\mathbb{Q}}}
\def\bB{{\mathbb{B}}}
\def\bA{{\mathbb{A}}}
\def\bs{{\mathbf{s}}}
\def\bd{{\mathbf{d}}}
\def\bT{{\mathbb{T}}}
\def\bt{{\mathbf{t}}}
\def\br{{\mathbf{r}}}
\def\vv{{\vec{v}}}
\def\vw{{\vec{w}}}
\def\vx{{\vec{x}}}
\def\vy{{\vec{y}}}
\def\v0{{\vec{0}}}
\def\ol{\overline}
\def\sspan{\rm{span}}
\def\sl2{{\mathfrak{sl}(2)}}
\def\slc{{\mathfrak{sl}(2,\bC)}}
\def\sp{{\mathfrak{sp}}}
\def\gg{{\mathfrak{g}}}
\def\kk{{\mathfrak{k}}}
\def\pp{{\mathfrak{p}}}
\def\qq{{\mathfrak{q}}}
\def\sP{\mathcal{P}}
\def\sH{\mathcal{H}}
\def\sU{\mathcal{U}}
\def\sC{\mathcal{C}}
\def\sS{\mathcal{S}}
\def\Stab{{\rm Stab}}
\def\ad{{\rm ad}}
\def\Ad{{\rm Ad}}
\def\id{{\rm id}}
\def\sgn{{\rm sgn}}
\def\gcd{{\rm gcd}}
\def\inn#1#2{\left\langle{#1},{#2}\right\rangle}
\def\abs#1{\left|{#1}\right|}
\def\norm#1{{\left\|{#1}\right\|}}
\def\Sp{{\rm Sp}}
\def\SL{{\rm SL}}
\def\O{{\rm O}}
\def\SO{{\rm SO}}
\def\det{{\rm det}}

\def\tG{{\widetilde{G}}}
\def\tK{{\widetilde{K}}}
\def\tM{{\widetilde{M}}}
\def\tJ{{\widetilde{J}}}
\def\trho{{\widetilde{\rho}}}
\def\tDelta{\widetilde{\Delta}}
\def\barZ{{\overline{Z}}}
\def\bz{{\overline{z}}}

\def\Ad{\mathrm{Ad}}
\def\agl{\mathfrak{gl}}
\def\asl{\mathfrak{sl}}
\def\aso{\mathfrak{so}}
\def\asp{\mathfrak{sp}}
\def\au{\mathfrak{u}}
\def\diag#1{\mathrm{diag}(#1)}
\def\lww{\mathcal{W}}
\def\lxx{\mathcal{X}}
\def\lyy{\mathcal{Y}}
\def\fbb{\mathfrak{b}}
\def\fhh{\mathfrak{h}}
\def\fnn{\mathfrak{n}}
\def\fuu{\mathfrak{u}}
\def\fll{\mathfrak{l}}
\def\foo{\mathfrak{o}}
\def\fpp{\mathfrak{p}}
\def\fqq{\mathfrak{q}}
\def\ftt{\mathfrak{t}}
\def\fgg{\mathfrak{g}}
\def\fkk{\mathfrak{k}}
\def\caa{\mathcal{A}}
\def\ccc{\mathcal{C}}
\def\cdd{\mathcal{D}}
\def\chh{\mathcal{H}}
\def\cjj{\mathcal{J}}
\def\crr{\mathcal{R}}
\def\css{\mathcal{S}}
\def\cpp{\mathcal{P}}
\def\cww{\mathcal{W}}
\def\cnn{\mathcal{N}}
\def\cuu{\mathcal{U}}
\def\cxx{\mathcal{X}}
\def\cyy{\mathcal{Y}}
\def\cff{\mathcal{F}}
\def\czz{\mathcal{Z}}
\def\cug{\cuu(\fgg)}
\def\fmm{\mathfrak{m}}
\def\GL{\mathrm{GL}}
\def\SO{\mathrm{SO}}
\def\OO{\mathrm{O}}
\def\rad{\mathrm{rad}}
\def\SL{\mathrm{SL}}
\def\tr{\mathrm{tr}}
\def\Mat{\mathrm{Mat}}
\def\U{\mathrm{U}}
\def\Gr{\mathrm{Gr\,}}
\def\tU{{\widetilde{U}}}
\def\pz#1{\partial z_{#1}}
\def\ddt{\left.\frac{d}{dt}\right|_{t=0}}

\def\Ind{{\rm Ind}}


\def\cmm#1#2{\left[{#1},{#2}\right]}
\def\acmm#1#2{\left\{{#1},{#2}\right\}}

\def\seesawpair#1#2#3#4{
\xymatrix{
{#1} \ar@{-}[dr]& {#2} \\
{#3} \ar@{-}[ur] & {#4}}
}

\def\real{{\rm Re\,}}
\def\imag{{\rm Im\,}}

\hypersetup{
    bookmarks=false,         % show bookmarks bar?
    unicode=false,          % non-Latin characters in Acrobat’s bookmarks
    pdftoolbar=true,        % show Acrobat’s toolbar?
    pdfmenubar=false,        % show Acrobat’s menu?
    pdffitwindow=false,      % page fit to window when opened
    %pdftitle={},    % title
    pdfauthor={Ma Jia Jun},     % author
    %pdfsubject={Subject},   % subject of the document
    pdfcreator={Hoxide},   % creator of the document
    %pdfproducer={Producer}, % producer of the document
    %pdfkeywords={keywords}, % list of keywords
    pdfnewwindow=true,      % links in new window
    colorlinks=true,       % false: boxed links; true: colored links
    linkcolor=blue,          % color of internal links
    citecolor=green,        % color of links to bibliography
    filecolor=magenta,      % color of file links
    urlcolor=cyan           % color of external links
}


\title{transfer of $K$ type}

\begin{document}
\maketitle

\section{Heisenberg groups}
Consider $H = W\oplus Z$, $W = X\oplus Y$ 
with group law: 
\[
h(x,y,z)h(x',y',z') = h(x+x', y+y', z+z'+xy').
\]
This law is from 
\[
\begin{pmatrix}
1 & x & z\\
0 & I & y\\
0 & 0 & 1
\end{pmatrix}
\begin{pmatrix}
1 & x' & z'\\
0 & I & y'\\
0 & 0 & 1
\end{pmatrix}
=
\begin{pmatrix}
1 & x+x' & z+z'+ x y'\\
0 & I & y+y'\\
0 & 0 & 1
\end{pmatrix}.
\]

Let $w = (x,y)$. 

Define exponation map $\exp$ from the lie algebra $\fhh$ of $H$ to $H$. 
\[
\fhh = \lww\oplus F= \Set{\begin{pmatrix}
    0 & a & t\\
    0 & 0 & b\\
    0 & 0 & 0
  \end{pmatrix}}
\]
Note
\[
\begin{pmatrix}
0 & a & c\\
0 & 0 & b\\
0 & 0 & 0
\end{pmatrix}
\begin{pmatrix}
0 & a' & c'\\
0 & 0 & b'\\
0 & 0 & 1
\end{pmatrix}
=
\begin{pmatrix}
0 & 0 & a b'\\
0 & 0 & 0\\
0 & 0 & 0
\end{pmatrix}.
\]
Hence $X^3 = 0$ for any $X\in \fhh$. 
Then 
\[
\exp(X) = I + X + \frac{1}{2}X^2= h(a,b, c+ \frac{1}{2} ab) \in H. \]
and
\[ 
\log(h) = (x,y, z-\frac{1}{2}xy\in \fhh. 
\]

Using above formula, we have
\[
h^{-1} = \exp(-\log(h)) = h(-x,-y,-z+x y).
\]

\[
\begin{split}
hgh^{-1}g^{-1} &= h(x,y,z)h(x',y',z') h(-x,-y,-z+xy) h(-x',y',-z'+x'y')\\
&= h(x+x',y+y', z+z'+xy')h(-(x+x'),-(y+y'), -(z+z')+xy+x'y'+xy')\\
&= h(0,0,-(x+x')(y+y')+xy + x'y' +xy')
= h(0,0,xy'-x'y)\\ 
&= h(0,0,\inn{w}{w'}).
\end{split}
\]

\[
[X,Y] = (0,0,ab'-a'b) = (0,0,\inn{w}{w'}) \in \lww.
\]

\[
\begin{split}
\exp(w,c)\exp(w',c') &= h(w,c+\frac{1}{2}ab)h(w',c'+\frac{1}{2}a'b')
= h(w+w',c+c'+\frac{1}{2}(ab+a'b')+ab')\\
&= \exp(w+w', c+c'+\frac{1}{2}(ab'-b'a)) = 
\exp(w+w',c+c'+\frac{1}{2}\inn{w}{w'}).
\end{split}
\]
This give an alternateve construction of Heisenberg groups. 

The Haar measure on $H$ is $dzdxdy = dzdw$.


Choose a unitary character $\chi$ of $F$, extend to $S = Y\oplus Z$, by 
$Y$ act trivially.
Let 
\[
\chh_\chi = \Ind_S^H \chi 
=  \Set{f\colon H\to F|f(sh) = \chi(s)f(h)\, \forall s\in S\text{ and }
   \int_{S\backslash H} |f(h)|^2 d\dot h <\infty}
\]
The right translation give a $H$-module structure on $\chh_\chi$.  

We can consider $f\in \chh_\chi$ as a function on $X$ by 
\[
\phi(x) =f(h(x,0,0)) = f(\exp(x,0,0)).
\]
This give a isometry from $\chh_\chi$ to $L^2(X)$. 

Then 
\[
\begin{split}
\rho(h(x',y',z'))\phi(x) =& f(h(x,0,0)h(x',y',z')) =
f(h(x',y',z'+xy'))\\
 =& f(h(0,y', z'+xy')h(x+x',0,0))\\
=& \chi(z'+xy')\phi(x+x').
\end{split}
\]
This is the Schr\"odinger representation. Moreover
\[
\int_{S\backslash H} |f(h)|^2 d\dot h = \int_X|\phi(x)|^2 dx
\]


The oscillator representation, choose polarlization $W=X\oplus Y$. 
Let $P = P(Y) = \Stab_\Sp Y$, $M = P(Y)\cap P(X) \cong \GL(X)$,
$N=\cap_{y\in Y} \Stab_\Sp y$, i.e the kernel of map $P(Y) \to \GL(Y)$. 
There is a isomorphism from $N$ to $S^{2*}(X)$ given by 
\[
n \mapsto (\beta_n \colon v,v' \mapsto \inn{v}{n v'}
\]
For $\beta\in S^{2*}(X)$, $\beta(v,v')$ give a map $\gamma_\beta\colon X\to Y$
such that 
\[
\inn{v}{\gamma_\beta(v')} = \beta(v,v')
\]
The inverse map then defined by 
\[
\beta \mapsto n\colon \begin{cases}
v & \text{for } v\in Y\\
v+\gamma_\beta(v) & \text{for } v\in X
\end{cases}
\]
One can verify that $\beta_n^{-1} = -\beta_n$


Extand $\chi$ to $P\ltimes S$ by $P$ act trivially. 
Let $\chh_\chi = \Ind_{P\ltimes S}^{P\ltimes H}\chi$ be the unitary 
induced representation. $P\ltimes H$ act by left translation.
Again let 
\[
\phi(x) = f(h(x,0,0))
\]
This compactable with $\Ind_S^H \chi$, and has same Hermitian inner product:
\[
\int_{X}\abs{\phi(x)}^2 dx 
\]

Then for $m\in M=\GL(X)$
\[
f(h(x,0,0)g)=f(m m^{-1}h(x,0,0)m) = f(h(m^{-1}x,0,0))  = \phi(m^{-1}x)
\]
Since 
\[
|\phi(m^{-1}x)|^2 dx = |\phi(x)|^2 d mx = |\phi(x)|^2 \abs{\det_{\GL(X)} m} dx,
\]
define 
\[
\omega(m)\phi(x) = \abs{\det_{\GL(X)} m}^{-1/2}f(mx).
\]

\[
\begin{split}
f(h(x,0,0)n) =& f(n n^{-1}h(x,0,0)n) = f(e(n^{-1}(x,0),0)) \\
=& f(e(x,\gamma_{\beta_{n^{-1}}}(x),0)) = 
f(h(x, \gamma_{\beta_{n^{-1}}}(x), \frac{1}{2}\inn{x}{\gamma_{\beta_{n^{-1}}}(x)}))\\
=& f(h(x, \gamma_{\beta_{n^{-1}}}(x),  -\frac{1}{2}\beta_n(x,x))) \\
=& \chi(-\frac{1}{2}\beta_n(x,x))f(h(x,0,0))
\end{split}
\]
Hence 
\[
\omega(n)\phi(x) = \chi(-\frac{1}{2}\beta_n(x,x))\phi(x)
\]

The special element is 
\[
\sigma = \begin{pmatrix}
0 & 1\\
1 & 0
\end{pmatrix} \in \Sp(W)
\]
It exchange the role of $x, y$. It is reasonable to believe the intertwining
map is a Fourier transform. 


\section{Fock model}
Let $J\in \Sp(W)\cap \sp(W)$ such that $J^2 = -\id$.
Then $J$ given a complex structure on $W$.
Then 
\[
B_J(w,w') = \inn{Jw}{w'}
\]
is a (non-degenerate) symmetric bilinear form.

\[
H_J(w,w') = \inn{Jw}{w'} + i\inn{w}{w'}
\]
become a Hermitian bilinear form(inner product) on $W$ where the 
complex structure on $W$ defined by $J$, i.e. $i$ act on $W$ by $J$. 
In fact,
\[
H_J(w',w) = \inn{Jw'}{w} + i \inn{w'}{w} = -\inn{w'}{Jw} + i\inn{w'}{w}
= \inn{Jw}{w'} - i \inn{w}{w'}= \overline{H_J(w,w')}
\]
and
\[
H_J(Jw,w') = \inn{JJw}{w'} + i\inn{Jw}{w'} = i\inn{Jw}{w'} - \inn{w}{w'} 
= i(\inn{Jw}{w'} + i\inn{w}{w'} = i H_J(w,w').
\]

Complexify $W$ to $W_\bC$. Extend $\inn{}{}$ to $W_\bC$.  
 
View $H = W\oplus Z$ from the exponatial map of Lie algebra $\fhh$ of $H$. 
Then the group law becomes
\[
\exp(w, t) \exp(w',t) = \exp(w+w', t+t'+\frac{1}{2}\inn{w}{w'}).
\]
Similary define $H_\bC$. 
Let $V^\pm$ be the $\pm i$ eigen space of $J$ on $W_\bC$. 
Then 
\[
V^\pm = \Set{w \mp iJw | w\in W },
\] 
since $J(w - iJw) = iw+Jw = i(w-iJw)$ and
$J(w+iJw) = -iw + Jw = -i(w+iJw)$.


Then consider the subgroup $P = V^-\oplus Z  < H_\bC$.
Consider the following digram
\[
\bfig
\square/^{(}->`->>`->>`^{(}->/[
H`H_\bC`Z\backslash H`P\backslash H_\bC;{}`{}`{}`{}]
\efig
\]

Hence $W = Z\backslash H$ have a complex structure given by $G_\bC$. 
In fact, by
\[
iw \equiv Jw +i(w+iJw) \equiv Jw \pmod{P}.
\]
Hence the complex structure given by $J$ is same as the complex structure given
by above commutative diagram. 


Fix a character $\chi(t)=e^{2\pi i\lambda t}$ of $\bR$ for some $\lambda\in \bR$. 
Extend it into a character of $\bC$ then $P$ in a obvious way. 
Define
\[
\chh_\chi = \Set{f\colon G_\bC \to \bC|\begin{array}{rl}
    (a) & f \text{ is holomorphic};\\
    (b) & f(pg) = \chi(p)f(g) \quad \forall p\in P, g\in G_\bC;\\
    (c) & \int_{Z\backslash G} \abs{f(g)}^2 dg < \infty
  \end{array}
}
\]
Inner product is given by 
\[
(f,f') = \int_{Z\backslash G} f(g)\overline{f'(g)} dg.
\]
$G$ act on $\chh_\chi$ by right translation.

For $X \in W$.
\[
\begin{split}
f(\exp w+tiX) =& f(\exp w+ tJX + ti(X+iJX))\\
=& f(e^{-\frac{1}{2}\inn{ti(X+iJX)}{w+tJX}}\exp ti(X+iJX) \exp(w+tJX))  \\
=& e^{-2\pi i \lambda\frac{1}{2}\inn{ti(X+iJX)}{w+tJX}}f(\exp (w+tJX))\\
= & e^{2\pi \lambda \frac{1}{2} t \inn{X+iJX}{w} + t^2 \inn{X+iJX}{JX}}f(\exp(w+tJW))
\end{split}
\]
Then 
\[
\ddt f(\exp w+tiX) =  \ddt f(\exp w+tJX) + 2\pi \lambda \frac{1}{2} 
\inn{X+iJX}{w} f(\exp(w+tJX))
\]
Then the holomorphic condition is,
\[
\lim_{t\to 0} \frac{f(\exp(w+tX)) -f(\exp(w))}{t} = 
\lim_{t\to 0} \frac{f(\exp(w+tiX))- f(\exp(w))}{ti} 
\]
 i.e.
\[
i \ddt f(\exp w+tX) = \ddt f(\exp w+tJX) + \pi \lambda
\inn{X+iJX}{w} f(\exp(w+tJX)).
\]

Choose $J$ such that $\inn{w}{Jw} \geq 0$, this $J$ called ``positive''. 


For $f\in \chh_\chi$, define
\[
\phi(w) = e^{\pi \lambda \inn{w}{Jw}/2} f(\exp w) 
\] 

We check the holomorphic condition becomes 
\[
\theta_{JX} \phi = i \theta_{X} \phi.
\]
where 
\[
(\theta_X \phi)(w) = \frac{d}{dt}\phi(w+tX) \quad \forall X\in W.
\]

In fact, 
\[
\begin{split}
\theta_X \phi(w) =& \ddt e^{\pi \lambda \inn{w+tX}{J(w+tX)}/2} f(\exp(w+tX))\\
= &  e^{\pi \lambda \inn{w}{Jw}}(\ddt f(\exp(w+tX)) 
+ \frac{1}{2}\pi\lambda (\inn{X}{Jw}+\inn{w}{JX}) f(\exp(w+tX))) \\
= &  e^{\pi \lambda H_J(w,w)}(\ddt f(\exp(w+tX)) + \pi\lambda \inn{X}{Jw} f(\exp(w+tX))) \\
\end{split}
\] 
Then 
\[
\begin{split}
&\theta_{JX}\phi - i\theta_{X}\phi \\
= & \ddt f(\exp(w+tJX)) + \pi\lambda \inn{JX}{Jw} f(\exp(w+tJX))\\
&-i \left[ \ddt f(\exp(w+tX)) + \pi\lambda \inn{X}{Jw} f(\exp(w+tX))\right]\\
=& \ddt f(\exp(w+tJX)  + \pi \lambda \inn{X+iJX}{w} f(\exp(w+tJX)
- i \ddt f(\exp(w+tX))= 0 
\end{split}
\]

Finiteness condition becomes:
\[
\int_W e^{-\pi \lambda \inn{w}{Jw}}\abs{\phi(w)}^2dw < \infty.
\]

Then
\[
\begin{split}
(\rho(\exp(w,t))\phi)(w') =& e^{\pi \lambda \inn{w'}{Jw'}/2} f(\exp(w') \exp(w,t))\\
=& e^{\pi \lambda \inn{w'}{Jw'}/2} f(\exp(w'+w,t+ \frac{1}{2}\inn{w'}{w}))\\
=& e^{\pi \lambda \inn{w'}{Jw'}/2} \chi(t+ \frac{1}{2}\inn{w'}{w}) f(\exp(w'+w))\\
=& \chi(t) e^{\frac{1}{2}\pi \lambda (\inn{w'}{Jw'}+2i \inn{w'}{w})}
 f(\exp(w+w')) \\
=& \chi(t) e^{\frac{1}{2}\pi \lambda (\inn{w'}{Jw'}+2i \inn{w'}{w})} 
e^{-\pi \lambda \inn{w'+w}{J(w'+w)}/2} \phi(w+w')\\
=& \chi(t) e^{-\pi \lambda (\inn{w}{Jw}/2+ \inn{w}{Jw'}+i\inn{w}{w'}))} 
 \phi(w+w') 
\end{split}
\]


So the infinitesimal representaition is 
\[
\rho(X) \phi = \theta_X(\phi) -\pi\lambda H_X \phi
\]
Where $H_X(w') =\inn{X}{Jw'} + i\inn{X}{w'} = -H_J(w',X)$ is a linear function.

Take $Y = \frac{1}{2}(X-iJX)$. Then $X = Y + \overline{Y}$, 
and 
\begin{align*}
\rho(Y) \phi &= \theta_X \phi \\
\rho(\overline{Y}) \phi &= -\pi \lambda H_X \phi
\end{align*}

The space 
\[
\Set{\phi \in \chh_\chi |\rho(V^+) \phi = 0}
\]
is one dimension, this is since $\phi$ is holomorphic, 
all derivative equal to zero mean $\phi$ is constant. 

Hence $\rho$ is irreducible. 

Choose complex basis of $W$ $\Set{P_1, \cdots, P_n}$
respect to $J$ such that
$H_J(P_k, P_l) = \delta_{kl}$ and let $Q_j = J P_j$. 
Then $\Set{z, P_j, Q_j}$ form a basis of $\fhh$. 
Let $z_k(P_l) = H_J(P_l, P_k) =\delta_{kl}$ be complex linear function on 
$W$. Then $z_j$ generate $\chh_\chi$. 

\[
\rho(P_j-iQ_j) = 2 \theta_{P_j} =  2\theta_{z_j}
\]
\[
\rho(P_j+iQ_j) = -2\pi\lambda (-H_J(\cdot,P_j)) 
 = 2\pi \lambda z_j
\]

Above is from Pierre Cartier 

\section{Fock model II}
Let $J\in \Sp(W)\cap \sp(W)$ such that $J^2 = -\id$.
Then $J$ given a complex structure on $W$.
Then 
\[
B_J(w,w') = \inn{Jw}{w'}
\]
is a (non-degenerate) symmetric bilinear form.

\[
H_J(w,w') = \inn{Jw}{w'} + i\inn{w}{w'}
\]
become a Hermitian bilinear form(inner product) on $W$ where the 
complex structure on $W$ defined by $J$, i.e. $i$ act on $W$ by $J$. 
In fact,
\[
H_J(w',w) = \inn{Jw'}{w} + i \inn{w'}{w} = -\inn{w'}{Jw} + i\inn{w'}{w}
= \inn{Jw}{w'} - i \inn{w}{w'}= \overline{H_J(w,w')}
\]
and
\[
H_J(Jw,w') = \inn{JJw}{w'} + i\inn{Jw}{w'} = i\inn{Jw}{w'} - \inn{w}{w'} 
= i(\inn{Jw}{w'} + i\inn{w}{w'} = i H_J(w,w').
\]

Complexify $W$ to $W_\bC$. Extend $\inn{}{}$ to $W_\bC$.  
 
View $H = W\oplus Z$ from the exponatial map of Lie algebra $\fhh$ of $H$. 
Then the group law becomes
\[
\exp(w, t) \exp(w',t) = \exp(w+w', t+t'+\frac{1}{2}\inn{w}{w'}).
\]
Similary define $H_\bC$. 
Let $V^\pm$ be the $\pm i$ eigen space of $J$ on $W_\bC$. 
Then 
\[
V^\pm = \Set{w \mp iJw | w\in W },
\] 
since $J(w - iJw) = iw+Jw = i(w-iJw)$ and
$J(w+iJw) = -iw + Jw = -i(w+iJw)$.


Then consider the subgroup $P = V^-\oplus Z  < H_\bC$.
Consider the following digram
\[
\bfig
\square/^{(}->`->>`->>`^{(}->/[
H`H_\bC` H/Z`H_\bC/P;{}`{}`{}`{}]
\efig
\]

Hence $W = H/Z$ have a complex structure given by $H_\bC$. 
In fact, by
\[
iw \equiv Jw +i(w+iJw) \equiv Jw \pmod{P}.
\]
Hence the complex structure given by $J$ is same as the complex structure given
by above commutative diagram. 


Fix a character $\chi(t)=e^{2\pi i\lambda t}$ of $\bR$ for some $\lambda\in \bR$. 
Extend it into a character of $\bC$ then $P$ in a obvious way. 
Define
\[
\chh_\chi = \Set{f\colon G_\bC \to \bC|\begin{array}{rl}
    (a) & f \text{ is holomorphic};\\
    (b) & f(gp) = \chi(p^{-1})f(g) \quad \forall p\in P, g\in G_\bC;\\
    (c) & \int_{G/Z} \abs{f(g)}^2 dg < \infty
  \end{array}
}
\]
Inner product is given by 
\[
(f,f') = \int_{G/Z} f(g)\overline{f'(g)} dg.
\]
$G$ act on $\chh_\chi$ by left translation.

For $X \in W$.
\[
\begin{split}
f(\exp w+tiX) =& f(\exp w+ tJX + ti(X+iJX))\\
=& f(e^{\frac{1}{2}\inn{ti(X+iJX)}{w+tJX}} \exp(w+tJX)\exp ti(X+iJX))  \\
=& e^{2\pi i \lambda\frac{1}{2}\inn{ti(X+iJX)}{w+tJX}}f(\exp (w+tJX))\\
= & e^{-2\pi \lambda \frac{1}{2} t \inn{X+iJX}{w} - t^2 \inn{X+iJX}{JX}}f(\exp(w+tJW))
\end{split}
\]
Then 
\[
\ddt f(\exp w+tiX) =  \ddt f(\exp w+tJX) - \pi \lambda  
\inn{X+iJX}{w} f(\exp w)
\]
Then the holomorphic condition is,
\[
\lim_{t\to 0} \frac{f(\exp(w+tX)) -f(\exp(w))}{t} = 
\lim_{t\to 0} \frac{f(\exp(w+tiX))- f(\exp(w))}{ti} 
\]
 i.e.
\[
i \ddt f(\exp w+tX) = \ddt f(\exp w+tJX) - \pi \lambda
\inn{X+iJX}{w} f(\exp w).
\]

Choose $J$ such that $\inn{Jw}{w}=H_J(w,w) \geq 0$, this $J$ called ``positive''. 


For $f\in \chh_\chi$, define
\[
\phi(w) = e^{\pi \lambda \inn{Jw}{w}/2} f(\exp w) 
\] 

We check the holomorphic condition becomes 
\[
\theta_{JX} \phi = i \theta_{X} \phi.
\]
where 
\[
(\theta_X \phi)(w) = \frac{d}{dt}\phi(w+tX) \quad \forall X\in W.
\]

In fact, 
\[
\begin{split}
\theta_X \phi(w) =& \ddt e^{\pi \lambda \inn{J(w+tX)}{w+tX}/2} f(\exp(w+tX))\\
= &  e^{\pi \lambda \inn{Jw}{w}}(\ddt f(\exp(w+tX)) 
+ \frac{1}{2}\pi\lambda (\inn{JX}{w}+\inn{Jw}{X}) f(\exp w )) \\
= &  e^{\pi \lambda H_J(w,w)}(\ddt f(\exp(w+tX)) + \pi\lambda \inn{JX}{w} 
f(\exp w)) \\
\end{split}
\] 
Then 
\[
\begin{split}
& e^{-\pi \lambda H_J(w,w)}\cdot (\theta_{JX}\phi - i\theta_{X}\phi) \\
= & \ddt f(\exp(w+tJX)) + \pi\lambda \inn{JJX}{w} f(\exp w)\\
&-i \left[ \ddt f(\exp(w+tX)) + \pi\lambda \inn{JX}{w} f(\exp w)\right]\\
=& \ddt f(\exp(w+tJX)  - \pi \lambda \inn{X+iJX}{w} f(\exp w)
- i \ddt f(\exp(w+tX))= 0 
\end{split}
\]

Finiteness condition becomes:
\[
\int_W e^{-\pi \lambda \inn{Jw}{w}}\abs{\phi(w)}^2dw < \infty.
\]

Then
\[
\begin{split}
(\rho(\exp(w,t))\phi)(w') 
=& e^{\pi \lambda \inn{Jw'}{w'}/2} f(\exp(w,t)^{-1}\exp(w'))\\
=& e^{\pi \lambda \inn{Jw'}{w'}/2} f(\exp(-w,-t)\exp(w'))\\
=& e^{\pi \lambda \inn{Jw'}{w'}/2} f(\exp(w'-w,-t+\frac{1}{2}\inn{w'}{w}))\\
=& e^{\pi \lambda \inn{Jw'}{w'}/2} \chi(-t+ \frac{1}{2}\inn{w'}{w}) f(\exp(w'-w))\\
=& \chi(t) e^{\frac{1}{2}\pi \lambda (\inn{Jw'}{w'}+2i \inn{w'}{w})}
 f(\exp(w'-w)) \\
=& \chi(-t) e^{\frac{1}{2}\pi \lambda (\inn{Jw'}{w'}+2i \inn{w'}{w})} 
e^{-\pi \lambda \inn{J(w'-w)}{w'-w}/2} \phi(w'-w)\\
=& \chi(-t) e^{-\pi \lambda \inn{Jw}{w}/2+ \pi\lambda (\inn{Jw'}{w}+i\inn{w'}{w})} 
 \phi(w'-w) \\
=& \chi(-t) e^{-\pi\lambda H_J(w,w)/2} e^{\pi\lambda H_J(w',w)} \phi(w'-w)
\end{split}
\]


So the infinitesimal representaition is 
\[
\rho(X) \phi = -\theta_X(\phi) +\pi\lambda H_X \phi
\]
Where $H_X(w') = H_J(w',X)$ is a (complex-)linear function on $W$.
Note that $H_{JX} = -i H_X$ and $\theta_{JX} = i\theta_X$.

Take $Y = \frac{1}{2}(X-iJX)$. Then $X = Y + \overline{Y}$, 
and 
\[
\begin{split}
\rho(Y) \phi =& \frac{1}{2}[(-\theta_X + \pi \lambda H_X) 
-i(-\theta_{JX}+\pi\lambda H_{JX})]\\
=& -\theta_X \phi 
\end{split}
\]
\[
\begin{split}
\rho(\overline{Y}) \phi =& \frac{1}{2}[(-\theta_X + \pi \lambda H_X) 
+i(-\theta_{JX}+\pi\lambda H_{JX})]\\
=& \pi\lambda H_X \\
\end{split}
\]

The space 
\[
\Set{\phi \in \chh_\chi |\rho(V^+) \phi = 0}
\]
is one dimension, this is since $\phi$ is holomorphic, 
all derivative equal to zero mean $\phi$ is constant. 

Hence $\rho$ is irreducible. 

Choose complex basis of $W$ $\Set{P_1, \cdots, P_n}$
respect to $J$ such that
$H_J(P_k, P_l) = \delta_{kl}$ and let $Q_j = J P_j$. 
Then $\Set{z, P_j, Q_j}$ form a basis of $\fhh$. 
Let $z_k(P_l) = H_J(P_l, P_k) =\delta_{kl}$ be complex linear function on 
$W$. Then $z_j$ generate $\chh_\chi$. 

\[
\rho(P_j-iQ_j) = -2 \theta_{P_j} =  -2\theta_{z_j}
\]
\[
\rho(P_j+iQ_j) = 2\pi\lambda (H_J(\cdot,P_j)) 
 = 2\pi \lambda z_j
\]

Above ALMOST from Pierre Cartier 


\section{dual pair}

Consider $U$ a complex vector space , and the symmetric algebra $S(U)$.

Define $M_u$ be the left mutiplication on $S(U)$ for any $u$, 
and $D_{u^*}$ be the differential operator act on $S(U)$ by
\[
D_{u^*}(xy) = \mu^*(x) y + xD_{u^*}(y) \quad \forall x\in U, y\in S(U)
\]
Clearly
\[
[D_{u^*}, M_u] = \mu^*(u) \id \quad \forall u\in U, u^*\in U^*.
\]

Define 
\[
\tU = U\oplus U^*, 
\]
and a symplectic form 
\[
\inn{u_1+u_1^*}{u_2+u_2^*} = u_1^*(u_2) - u_2^*(u_1)
\]
By above we have a canonical embedding 
\[
i\colon \tU \hookrightarrow \End(S(U))   
\]
call the image $i(x)$ of $x$ again $x$. 
Then 
\[
[x,y] = \inn{x}{y} \id \quad \forall x,y \in \tU.
\]

Consider 
$\End^\circ$ be the algebra generated by $i(\tU)$. It has a natrue filtration.
Then $S^2(\tU) \hookrightarrow \End^{\circ (2)}$ by $\{a,b\}$.

Since $[,c]$ is a representation on $\End(S(U))$, 
\[
\cmm{\acmm{a}{b}}{c}
= \acmm{\cmm{a}{c}}{b} + \acmm{a}{\cmm{b}{c}}
= 2b\inn{a}{c} + 2a \inn{b}{c}
\]
Hence
\[
\inn{\cmm{\acmm{a}{b}}{c}}{d}
= 2\inn{b}{d}\inn{a}{c} + 2\inn{a}{d}\inn{b}{c}
= -2\inn{c}{b}\inn{a}{d} -2\inn{c}{a}\inn{b}{d}
=-\inn{c}{\cmm{\acmm{a}{b}}{d}},
\]
i,e, $S^2(\tU) \subset \sp(\tU)$. Since both side has dimension $n(2n+1)$,
they are equal. 

\section{computation}
Take 
\[
J = \begin{pmatrix} 0 & -1\\1 & 0
\end{pmatrix}
\]
Define 
\[
\inn{w}{v} = v^T J w = ad - cb \quad \forall w = (a,b), v = (c,d) \in \bR^n
\]

\[
\tJ = -J = J^T = 
\begin{pmatrix}
  0 & 1 \\ -1 & 0
\end{pmatrix}
\]
satisfy 
\[
\inn{\tJ w}{w} = w^T w \geq 0.
\]

Choose a basis of $W$ respect to $H_\tJ$.
Let $e_1, \cdots, e_{2n}$ be the standard basis of $\bR^2$.
Noting that 
\[
H_\tJ(e_k,e_l) = \inn{\tJ e_k}{e_l}+i\inn{e_k}{e_l}
= e_l^T e_k =\delta_{kl} \quad \forall k,l \leq n
\]
Hence we set $P_k = e_k$ and $Q_k = \tJ P_k = -e_{n+k}$
Then $z_k = e_k - i e_{n+k}$ and $\overline{z_k} = e_k +i e_{n+k}$
by $z_k(P_l) = \delta_{kl}$ and $z_k(\tJ P_l)= i\delta_{kl}$.

We can compute $K$ which is the isometric group of $H_\tJ$.
In fact compute the Lie algebra $\fkk$ of $K$. 
\[
K = \set{k  \in \GL(2n,\bR)|k^tk = I, k^TJk = J }
\]
Then the lie algebra is,
\[
\fkk = \Set{X\in \agl(2n,\bR)|X^T + X  = 0, X^TJ + JX = 0}
\]
If
\[
X = \begin{pmatrix} A & B \\ C & D \\
\end{pmatrix}
\]

Then 
\[
\begin{pmatrix}
A^T & C^T \\
B^T & D^T 
\end{pmatrix} =
\begin{pmatrix}
-A & -B \\
-C & -D 
\end{pmatrix}
\]
and 
\[
\begin{pmatrix}
C^T & -A^T \\
D^T & -B^T 
\end{pmatrix} =
\begin{pmatrix}
C & D \\
-A & -B 
\end{pmatrix}.
\]
Hence $A^T = -A$, $D = -A^T= A$, $B = B^T$, $C = -B^T = -B$,
i.e. 
\[
X = \begin{pmatrix}
A & B\\
-B & A
\end{pmatrix}
\]
$A$ antisymmetric and $B$ symmetric.
Note that we have a natrue isomorphism of lie algebra
$\fkk \cong U(n)$ by 
\[
X = \begin{pmatrix}
A & B\\
-B & A
\end{pmatrix}
\mapsto
A + iB.
\]
By the isomorphism $\bR^2 \simeq \bC$ with $(a,b) \to a-ib$, 
we have $H_\tJ((a,b),(c,d)) = c^ta + b^td +i(d^t a- b^t c)
= (c+id)^t(a-ib)$, hence identify $H_\tJ$ with the usual Hermitian inner
product on $\bC$. This is compatible with above lie algebra isomorphism.
In fact
b\[
X\begin{pmatrix}a\\b
\end{pmatrix}
= \begin{pmatrix} Aa+Bb\\
-Ba+ Ab
\end{pmatrix} 
\]
and compatible with
\[
(A+iB)(a-ib) = Aa+Bb -i(-Ba+Bb)
\]

We can use 
\begin{align*}
Z_k  &= \frac{1}{2}(P_k-iQ_k) = \frac{1}{2}(e_k + i e_{n+k})\\
\barZ_k &= \frac{1}{2}(P_k+iQ_k) = \frac{1}{2}(e_k - i e_{n+k})
\end{align*}
be the basis of $W_{\bC}$. Where $Z_k$ span the $i$-eigen space of $\tJ$.
Then 
\begin{align*}
\rho(Z_k) &= -\pz{k} \\
\rho(\barZ_k) & = \pi\lambda z_k
\end{align*}

Now compute the action of $\fhh$ on the oscillator representation. 
In fact 
\[
\begin{split}
e_k &= \frac{1}{2}[(P_k-iQ_k) + (P_k + iQ_k)] =
Z_k + \barZ_k =  -\pz{j}  + \pi \lambda z_k\\
e_{n+k} & = i\frac{1}{2}[-(P_k-iQ_k) + (P_k + iQ_k)] = i(-Z_k+\barZ_k)
= i\pz{k} +i \pi \lambda z_k
\end{split}
\]
Now the matrix of change of basis is: $Z_k, \barZ_j \mapsto e_k, e_{n+j}$
by 
\begin{equation}\label{eq:tZtoe}
\frac{1}{2}\begin{pmatrix}
1 & 1\\
i & -i 
\end{pmatrix}
\end{equation}
And the inverse is $e_k, e_{n+j} \mapsto Z_k, \barZ_j$
\[
\begin{pmatrix}
1 & -i\\
1 & i
\end{pmatrix}
\]

now consider the symplectic lie algebra $\sp \simeq S^2(W_\bC)$. 
Note that 
\[
\cmm{\pz{k}}{z_j} = \delta_{kj}
\]
\[
\cmm{Z_k}{\barZ_j} = \cmm{-\pz{k}}{\pi \lambda z_j} = -\pi\lambda \delta_{kj}
\]
\[
\cmm{Z_k}{Z_j} = \cmm{\barZ_k}{\barZ_j} =0.
\]

For all $0\leq k,j \leq n$
\[
\cmm{e_k}{e_{j}}= \cmm{Z_k+\barZ_k}{Z_j+\barZ_j} = \pi\lambda (\delta_{kj}-\delta_{jk}) 
=0 
\]
\[
\cmm{e_{n+k}}{e_{n+j}} = \cmm{i(-Z_k+\barZ_k)}{i(-Z_j+\barZ_j)} = 0
\]
\[
\cmm{e_k}{e_{n+j}} = \cmm{Z_k + \barZ_k}{i(-Z_j+\barZ_j)} = -2\pi\lambda i \delta_{kj} 
\]
Hence for $w,v\in W$, we have 
\[
\cmm{w}{v}= - 2\pi \lambda i \inn{w}{v} 
\]

Now consider how $\sp = S^2(W_\bC)$ act on $W_\bC$ or $W$ by adjoint.
\begin{align*}
\cmm{\acmm{Z_k}{Z_j}}{Z_l} &= \acmm{\cmm{Z_k}{Z_l}}{Z_j} 
+ \acmm{Z_k}{\cmm{Z_j}{Z_l}} = 0\\
\cmm{\acmm{Z_k}{Z_j}}{\barZ_l} &= \acmm{\cmm{Z_k}{\barZ_l}}{Z_j} 
+ \acmm{Z_k}{\cmm{Z_j}{\barZ_l}} = -2\pi \lambda(\delta_{kl}Z_j+\delta_{jl}Z_k)
\end{align*}

\begin{align*}
 \cmm{\acmm{Z_k}{\barZ_j}}{Z_l} &= \acmm{\cmm{Z_k}{Z_l}}{\barZ_j} 
+ \acmm{Z_k}{\cmm{\barZ_j}{Z_l}} = 2\pi \lambda \delta_{jl} Z_k\\
 \cmm{\acmm{Z_k}{\barZ_j}}{\barZ_l} &= \acmm{\cmm{Z_k}{\barZ_l}}{\barZ_j} 
+ \acmm{Z_k}{\cmm{\barZ_j}{\barZ_l}} = -2\pi \lambda \delta_{jl} \barZ_j
\end{align*}

\begin{align*}
 \cmm{\acmm{\barZ_k}{\barZ_j}}{Z_l} &= \acmm{\cmm{\barZ_k}{Z_l}}{\barZ_j} 
+ \acmm{\barZ_k}{\cmm{\barZ_j}{Z_l}} 
= 2\pi \lambda (\delta_{kl} \barZ_j+ \delta_{jl}\barZ_k)\\
 \cmm{\acmm{\barZ_k}{\barZ_j}}{\barZ_l} 
&= \acmm{\cmm{\barZ_k}{\barZ_l}}{\barZ_j} 
+ \acmm{\barZ_k}{\cmm{\barZ_j}{\barZ_l}} 
= 0
\end{align*}

In the basis $Z_k, \barZ_k$, (assume $k<=j$)
Hence 
\begin{align}\label{eq:matZ}
\acmm{Z_k}{Z_j} \mapsto& -2\pi \lambda (E_{j,n+k}+E_{k,n+j}) = 2\pi\lambda
\begin{pmatrix}
 & & & -1\\
 & &-1 & \\
 & &  & \\
 & &  &  
\end{pmatrix}
\\
\acmm{Z_k}{\barZ_j} \mapsto& 2\pi\lambda (E_{k,j}-E_{n+j,n+k}) 
= 2\pi\lambda
\begin{pmatrix}
 &1 &  & \\
 & &  & \\
 & &  & \\
 & &-1  &  
\end{pmatrix}
\\
\acmm{\barZ_k}{\barZ_j} \mapsto & 2\pi\lambda (E_{n+j,k} + E_{n+k,j}) 
= 2\pi\lambda
\begin{pmatrix}
 & &  & \\
 & &  & \\
 & 1&  & \\
 1& &  &  
\end{pmatrix}
\end{align}

If $k<j$,
\begin{align*}
\acmm{e_k}{e_j} =& \acmm{Z_k+\barZ_k}{Z_j+\barZ_j} = 
\acmm{Z_k}{Z_j} +\acmm{\barZ_k}{Z_j} +
\acmm{Z_k}{\barZ_j} + \acmm{\barZ_k}{\barZ_j}\\
=& 2 \pi \lambda[-(E_{j,n+k}+E_{k,n+j})+(E_{n+j,k} + E_{n+k,j}) +
(E_{k,j}+E_{j,k}-E_{n+k,n+j}-E_{n+j,n+k})] \\
=& 2\pi\lambda
\begin{pmatrix}
  &1 &  &-1 \\
1 &  & -1 & \\
  & 1&  &-1 \\
1 &  &-1  &  
\end{pmatrix}\\
\acmm{e_k}{e_{n+j}} =& \acmm{Z_k+\barZ_k}{-iZ_j+i\barZ_j} = 
-i\acmm{Z_k}{Z_j} -i\acmm{\barZ_k}{Z_j} +
i\acmm{Z_k}{\barZ_j} + i\acmm{\barZ_k}{\barZ_j}\\
=& 2 \pi \lambda[i(E_{j,n+k}+E_{k,n+j})+i(E_{n+j,k} + E_{n+k,j}) +
(-iE_{j,k}+iE_{n+k,n+j} +iE_{k,j}-iE_{n+j,n+k})] \\
=& 2\pi\lambda
\begin{pmatrix}
  & i&   &i \\
-i&  & i & \\
  & i&   &-i \\
i &  & i &  
\end{pmatrix}\\
\acmm{e_{n+k}}{e_{n+j}} =& \acmm{-iZ_k+i\barZ_k}{-iZ_j+i\barZ_j} 
=-\acmm{Z_k-\barZ_k}{Z_j-\barZ_j} 
=& 
-\acmm{Z_k}{Z_j} +\acmm{\barZ_k}{Z_j} +
\acmm{Z_k}{\barZ_j} - \acmm{\barZ_k}{\barZ_j}\\
=& 2 \pi \lambda[(E_{j,n+k}+E_{k,n+j})-(E_{n+j,k} + E_{n+k,j}) +
(E_{k,j}+E_{j,k}-E_{n+k,n+j}-E_{n+j,n+k})] \\
=& 2\pi\lambda
\begin{pmatrix}
  &1 &  &1 \\
1 &  & 1 & \\
  & -1&  &-1 \\
-1&  &-1  &  
\end{pmatrix}\\
\end{align*}

\section{$\chh(U(n))$}
$\chh(K)$ means all $K$-finite elements in the represnetaion of $K\times K$ on 
$C(K)$.  Now assume $K=U(n)$. We claim that 
\[
\crr(\GL(n, \bC)) \simeq \chh(K),
\]
where $\crr(\GL(n,\bC))$ is the coordinate ring of affine variety $\GL(n,\bC)$,
or the subring of functions on $\GL(n,\bC)$ generated by entries and $\det$. 

Consider the restriction map:
\[
\pi:\crr(\GL(n,\bC)) \to C(K).
\]
It is injective.
Not so ``rigid'': since element in $\crr(\GL(n,\bC))$ are holomorphic functions
on $\GL(n,\bC)$ hence determined by restrict to $K$, a infinite compact set. 

Suppose that $f(g) = \sum p_k(g) \det^k(g) \in \crr(\GL(n,\bC))$, where
$p_k$ are polynomials on $M_n(\bC)$. Since it is a 
finite sum $f\det^k$ is a polynomial on $M_n(\bC)$ 
for sufficient large $k\in \bZ$.
Since  $\det^k(g)\neq 0$ on $\GL(n,\bC)$,
$f(g) = 0$ for all $g\in \GL(n,\bC)$ eqivalent to $f(g)\det^k(g) = 0$
for all $g\in \GL(n,\bC)$. 
Since $\GL(n,\bC)$ is (Zariski) dense in $M_n(\bC)$, above eqivalent to 
$f\det^k=0$ as polynomial on $M_n(\bC)$. 
Assume $\pi(f)=0$, i.e. $f(g) = 0$ for all $g\in U(n)$. Hence 
$f(g)\det^k(g)=0$ for all $g\in U(n)$. 
Now decomposite $f$ in to homogenous component, i.e. $f=\sum p_l$. 
Then $0=f(au)=\sum p_l(au) =\sum a^l p_l(u)$ for all $a\in U(1)$. Hence
$p_l(u) =0$. Since any $g\in \GL(n,\bC)$ can written as $\lambda u$ for 
$\lambda \in \bC$. We have $f(g) = f(\lambda u) = \sum \lambda^l p_l(u) = 0$. 
Hence $f=0$ in $\crr(GL(n,\bC)$.

It is obvious that $\Im \pi \subset \chh(K)$
 since the $K$ action on $\crr(\GL(n,\bC))$ is finite. 
For the subjectivity, we only have to check the $\Im \pi$ is dense in $C(K)$. 
Then by Peter-Weyl theorem, $\Im \pi=\chh(K)$. 
The image is dense by Stone-Weierstrass theorem. 
Fist $K$ is a compact Hausdorff space, 
We check:
\begin{enumerate}[(1)]
\item obviously, $\Im \pi$ is a subalgebra of $C(K)$.
\item obviously, it contains non-zero constant function.
\item the matrix entries $z_{ij}$ separate points.
\item Closed by conjugation is by the following observation: 
$u\in U(k)$ then $u^* = u^{-1} = {\rm Adj}(u) \det^{-1}(u)$, hence 
$\overline{z_{ij}} \in \Im\pi$. Moreover $\det(u)\in U(1)$, hence
$\overline{\det^{-1}(u)} = \det(u) \in \Im\pi$.
Combine these evidences, we see for any $f = \sum p_l \det^{-l} \in \Im \pi$,
\[
\overline{f}(u) = \sum \overline{p_l(u)}\overline{\det(u)}^{-l}
= \sum \widetilde{p}_l(\overline{u})\det(u)^l \in \Im \pi
\]  
\end{enumerate}

We can summerize the result:

\begin{lemma}
\[
\crr(\GL(n,\bC)) \simeq \chh(U(n))
\]
as representation of $K$. 
\end{lemma}

\subsection{Lie algebra actions on $\chh(K)$ }
Now take $\fkk_\bC = \agl(V)$, 

\[
r(X) = \frac{d}{dt} f(M \exp(tX)) 
= \sum_{kl} \frac{\partial}{\partial m_{kl}} f (M) (MX)_{kl}
= \sum_{kl} \frac{\partial}{\partial m_{kl}} f (M) m_{ks} x_{sl}
\]
Hence 
\[
E_{sl} \mapsto \sum_{k} m_{ks} \frac{\partial}{\partial m_{kl}}.
\]

Define 
\[
D_v(f)(x) = \frac{d}{dt}f(x+tv)
\]
\[
M_{v^*}(f) = v^* f.
\]

Then 
\[
v^*\otimes u \in \agl(V) \to  M_{v^*} D_v.
\]

Here identify $E_{sl} = $

\[
\frac{d}{dt} \det(M\exp tX) = \frac{d}{dt}\det(M)\det(tX) 
 = \det(M)  \frac{d}{dt} e^{t\tr(X)} = \tr(X) \det(M) 
\]
Same,
\[
\frac{d}{dt} \det^{-1}(M\exp tX) = \det^{-1}(M) \frac{d}{dt}e^{-t\tr(X)}
=-\tr(X) \det^{-1}(M).
\]

Hence for function $f\det^{-n} \in \chh(K)$ where $f\in \cpp(M_{n,n})$ and
$X \in \agl$, we have, 
\[
\begin{split}
r(X) f\det^{-n}(M) &= \det^{-n}(M) r(X) f(M) + f(M) r(X) \det^{-n}(M)\\
 &= \det^{-n}(M) 
(\sum_{sl} x_{sl} (\sum_{k} m_{ks} \frac{\partial}{\partial m_{kl}} f(M) 
- \delta_{sl}f(M)))
\end{split}
\]
 

\section{sp}

$\sp_\bC(n)$ has roots $\pm \epsilon_k \pm \epsilon_l$, $k\neq l$ and 
and $\pm 2\epsilon_k$, $k$.
Simple roots are 
\[
\alpha_1 = 2\epsilon_n, \alpha_2 = \epsilon_1 -\epsilon_2, \cdots, 
\alpha_n = \epsilon_{n-1}-\epsilon_{n}.
\]

Take 
\[
\begin{cases}
\eta_k = \epsilon_k & 1\leq k \leq r,\\
\eta_{r+k}= -\epsilon_{n-k+1} & 1\leq k \leq s
\end{cases}
\]
Take 
\[
H = \frac{1}{2} \sum_{k=1}^n \eta_k, = \frac{1}{2} \sum_{k=1}^r \epsilon_k 
- \frac{1}{2} \sum_{k=r+1}^n \epsilon_k
\]
\[
H' = \frac{1}{2} \sum_{k=1}^r \eta_k + \frac{1}{2} \sum_{k=r+1}^n \eta_k
= \frac{1}{2} \sum_{k=1}^n \epsilon_k
\]
Here
$\epsilon_k$ correspond to a element in $\fhh$, the usual Cartan subalgebra
-- diagnal matrix.
In fact, 
\[
\epsilon_k \leftrightarrow H_{\epsilon_k}= E_{k,k}-E_{n+k,n+k}
\] 

Here we realize $\sp_\bC$ by $2n\times 2n$ matrix
\[
\begin{pmatrix}
A & B \\
C & -A^T
\end{pmatrix}
\]
such that $B,C$ symmetric, i.e. $B=B^T$ and $C= C^T$.
Follows Howe\cite{Howe1985}, we take negative $\dagger$, where 
\[
\begin{pmatrix}
A & B\\
C & -A^T 
\end{pmatrix}^\dagger
=
\begin{pmatrix}
A^* & -C^*\\
-B^* & -\overline{A}
\end{pmatrix}
=\begin{pmatrix}
A^* & -\overline{C}\\
-\overline{B} & -\overline{A}
\end{pmatrix}
\]
(by (\ref{eq:matZ})) be the involution defining the realform of $\sp_\bC$.
The the element in realform is 
\[
\begin{pmatrix}
A & B\\
\overline{B} & \overline{A}
\end{pmatrix}
\]
since $ -A^* = A$ and $C=\overline{B}$.

Now from (\ref{eq:tZtoe})  conjugate by 
\[c = \frac{1}{2}
\begin{pmatrix}
1 & 1\\
i & i
\end{pmatrix},
\]
we have 
\[
\begin{split}
(\Ad c)
\begin{pmatrix}
A & B\\
\overline{B} & \overline{A}
\end{pmatrix}
=& \frac{1}{2}
\begin{pmatrix}
(A + \overline{A}) + (B +\overline{B}) &-i(A-\overline{A}) +i(B-\overline{B})\\
i(A-\overline{A}) + i(B-\overline{B})& -(A + \overline{A}) + (B +\overline{B})
\end{pmatrix}\\
=&\begin{pmatrix}
\real A  + \real B & \imag A - \imag B\\
-\imag A -\imag B & -(\real A+\real B)  
\end{pmatrix}
\in \sp_\bR
\end{split}
\]
since $\real A$ is antisymmetric and $\imag A, \real B, \imag B$ is symmetric. 
Hence this gives the Lie algebra isomorphism from the fix point of $\dagger$ 
to $\sp_\bR$. 

Hence 
\[
H = \frac{1}{2} \begin{pmatrix}
I_r & & & \\
 & -I_s & & \\
 & & -I_r & \\
 & & & I_s 
\end{pmatrix}
\text{ and }
H' = \frac{1}{2} 
\begin{pmatrix}
I_r & & & \\
 & I_s & & \\
 & & -I_r & \\
 & & & -I_s 
\end{pmatrix}
bb\]
Then 
\[
\theta = e^{\pi i \ad H} =\Ad \begin{pmatrix}
iI_r & & & \\
 & -iI_s & & \\
 & & -iI_r & \\
 & & & iI_s 
\end{pmatrix}
\]
and
\[
\theta' = e^{\pi i \ad H'} = \Ad \begin{pmatrix}
iI_r & & & \\
 & iI_s & & \\
 & & -iI_r & \\
 & & & -iI_s 
\end{pmatrix}
\]

For any element 
\[ X = 
\begin{pmatrix}
A & B \\
C & -A^T 
\end{pmatrix} \in \sp_\bC(n)
\]
We have 
\[
\theta'(X) = \begin{pmatrix}
A & -B\\
-C & -A^T
\end{pmatrix}
\]
Hence 
\[
\fkk_\bC' = \Set{\begin{pmatrix}
A & 0 \\
0 & -A^T 
\end{pmatrix}} \simeq \agl_{n, \bC}  
b\]
in a obvious way. 

For the fix point of $\theta$, we consider the matrix in the following form
\[
X = \begin{pmatrix}
  A & B & C & D\\
  E & F & {D}^{T} & G\\
  K & L & -{A}^{T} & -{E}^{T}\\
  {L}^{T} & M & -{B}^{T} & -{F}^{T}
\end{pmatrix}
\]
Then 
\[
\theta(X) = 
\begin{pmatrix}
  A & -B & -C & D\\
  -E & F & {D}^{T} & -G\\
  -K & L & -{A}^{T} & {E}^{T}\\
  {L}^{T} & -M & {B}^{T} & -{F}^{T}
b\end{pmatrix}
\]
Hence $\fkk_\bC$ consisting elements in the following form
\[
\begin{pmatrix}
  A & 0 & 0 & D\\
  0 & F & {D}^{T} & 0\\
  0 & L & -{A}^{T} & 0\\
 {L}^{t} & 0 & 0 & -{F}^{T}
\end{pmatrix}
\Rightarrow
\begin{pmatrix}
  A & D & 0 & 0\\
  {L}^{T} & -{F}^{T} & 0 & 0\\
  0 & 0 & -{A}^{T} & -L\\
  0 & 0 & -{D}^{T} & F
\end{pmatrix}
\]
by conjugating the permutation matrix
\[
b=\begin{pmatrix}
  1 & 0 & 0 & 0\\
  0 & 0 & 0 & 1\\
  0 & 0 & -1 & 0\\
  0 & 1 & 0 & 0
\end{pmatrix}
\]
This give a nature way to identify $\fkk_\bC$. to $\agl_n(\bC)$.
Moreover 
\[
\fpp_\bC = \Set{
\begin{pmatrix}
  0 & B & C & 0\\
 E & 0 & 0 & G\\
 K & 0 & 0 & -{E}^{T}\\
 0 & M & -{B}^{T} & 0
\end{pmatrix}
|
}
\]

In fact, 
\[
\Ad(b) 
\begin{pmatrix}
A & B & C & D\cr
E & F & {D}^{T} & G\cr
K & L & -{A}^{T} & -{E}^{T}\cr 
{L}^{T} & M & -{B}^{T} & -{F}^{T}
\end{pmatrix}
=
\begin{pmatrix}
A & D & -C & B\cr 
{L}^{T} & -{F}^{T} & {B}^{T} & M\cr
 -K & {E}^{T} & -{A}^{T} & -L\cr
 E & G & -{D}^{T} & F
\end{pmatrix}
\]
We can see that $\fmm_\bC = \fkk_\bC \cap \fkk_\bC'$, i.e. 
the fixed point of $\theta$ in $\ftt_\bC'$ is 
\[
\Set{\begin{pmatrix}
    A & & & \\
    & B & & \\
    & & -A^T &\\
    & & & -B^T 
  \end{pmatrix}|
  A\in \Mat_{r,r}(\bC), B\in \Mat_{s,s}(\bC)
}
\]

Also , $\foo = \fpp_\bC'$ consists matrix in form 
\[
\begin{pmatrix}
0 &  B \\
C & 0 
\end{pmatrix}
\]
and $B= B^T$, $C= C^T$.
And 
\[
\foo^+ = \Set{\begin{pmatrix}
0 &  B \\
0 & 0 
\end{pmatrix}
| B = B^T \in \Mat_{n,n}(\bC)
}
\]
\[
\foo^- = \Set{\begin{pmatrix}
0 &  0 \\
C & 0 
\end{pmatrix}
| C = C^T \in \Mat_{n,n}(\bC)
}
\]
\[
\foo_k^+ =  \Set{\begin{pmatrix}
0 &  0 &  0 &D \\
0 & 0  & D^T & 0 \\
0 & 0 & 0 & 0 \\
0 & 0 & 0 & 0 
\end{pmatrix}
| B = B^T \in \Mat_{n,n}(\bC)}
\]




\section{$W_{p,q}$}
Define $\theta'$-stable parabolic $\fqq$ of $\fkk_\bC$ by weight
\[
\lambda_{p,q} = -(\eta_{r-q+1} +  \cdots + \eta_r)+(\eta_{r+1} + \eta_{r+p})
 = \sum_{k=r-q+1}^{r+p} - \epsilon_{k}
\]
So the non-compact nilpotent part is 
\[
\fnn\cap \fpp = \Set{g_{\alpha}|\alpha = -(\epsilon_k+\epsilon_l), 
  k\geq r+1, l \leq r
  \text{ and } (
  k \leq r+p \text{ or } l \geq r-q+1)
}
\]

\section{notation}



Give $(\fgg_\bC, M)$-module $(\pi, V)$.

In this situation. 
Consider 
\[
\Gamma^i (V) = \Hom_{M}(\bigwedge^i(\fkk_\bC/\fmm_\bC), V\otimes \chh(K))
\]


The $\fkk$-module structure on $V\otimes \chh(K)$ given when compute the
$\Gamma^i$ is $\pi\otimes l_K$
The nature $K$ structure on $\Gamma^i$ is given by $I\otimes r_K$.

Look $V\otimes \chh(K)$ as $\chh(K, V)$ by $v\otimes f \mapsto F$, 
$F(k) = f(k)v$. 
 
Give $\fgg$ action on $\chh(K,V)$ by 
\[
(\mu(X)F)(k) = \pi(\Ad(k) X) F(k)
\]

Choose a basis $\Set{X_1, \cdots, X_n}$ of $\fgg$ and 
let $\lambda_1, \cdots,\lambda_n$ be the dual basis of $\fgg^*$.

Then (denote $\pi(X)$ $X$ here). 
\[
\mu(X) (v\otimes f)(k) = \Ad(k) X v f(k) = 
\sum_{l} \lambda_l(\Ad(k)X) X_l v f(k) 
= \sum_{l} (X_l v \otimes c_{\lambda_l,X} f)(k),
\]
where $c_{\lambda,X} = \lambda(\Ad(k)X)$.
Same as \cite{BorelWallach2000} p. 25, I.8.1,
\[
\mu(X) (v\otimes f) = \sum_l X_l v \otimes c_{\lambda_l,X} f. 
\]

This $\fgg$ structure compatible with the $K$ structure on $\Gamma^i$, i.e. 
\[
\pi(\Ad(k) X) F(k) = (r_X F)(k)\quad \forall X\in \fkk, k\in K.
\]

Let $Y=\Ad(k)X$, then $X = \Ad(k^{-1})Y$,
\[
YF(k) = \frac{d}{dt} F(k\exp(t\Ad k^{-1}Y))
= \frac{d}{dt} F(\exp(tY)k)
= l(Y)F (k). 
\]

This means the image of $\Hom_\fmm(W_{p,q},V\otimes \chh(K))$ is in 
$(V\otimes \chh(K))^\fkk$, where the $\fkk$ act by $\pi\otimes l$.

But, if restrict everyting in $(V\otimes \chh(K))^\fkk$, 
\[
\Gamma_{p,q} = \Hom_M(W_{p,q}, (V\otimes \chh(K))^\fkk)
=(W_{p,q}^*)^M \otimes (V\otimes \chh(K))^\fkk=0?
\] 


\section{Weil representation -- algebric version}
 For complex vector space $V$, define action of $V$ on $S(V)$
 by left multipication, action of $V^*$ on $S(V)$ by extend
\[
v^* \dot v = v^*(v).
\]
View $V$ as linear fuction on $\bC^n$, 
Then $V$ act by multiply $v$, 
$V^*$ act by differential operator. The image of $V\oplus V^*$
in $\End(S(V))$ give a Wely algebra $\cww$. 
Fix a basis of $V$, say $x_1, \cdots, x_n$, 
then the dual basis $V^*$ given by 
$D_{x_i} = D_i \triangleq \partial_{x_i}=  \frac{\partial}{\partial x_i}$.

Notic that $W=V\oplus V^*$ form a simplectic space 
 with symplectic form $[,]$.
In fact, 
\[
[D_{v^*},v] = v^*(v), \quad for v^*\in V^*, v\in V.
\] 
So the form is 
\[
[D_{v^*}+v,D_{u^*}+u] = v^*(u) - u^*(v)
\]
$V\oplus V^*$ is a polarization of $W$.

Consider $\fgg = S^2(V\oplus V^*) = \sspan\Set{\{w_1,w_2\}|w_1,w_2\in W}\in \cww$
generated by 
\[
x_ix_j, \frac{1}{2}(x_i \partial_{x_j} + \partial_{x_i} x_j), 
\partial_{x_i}\partial_{x_j}.
\]


$\fgg$ preserve the structure of $W$. 
\[
[\{a,b\},c] = \{[a,c],b\} + \{a,[b,c]\} = 2[a,c]b + 2[b,c]a
\]
Hence
\[
[[\{a,b\},c],d] = 2[a,c][b,d]+2[b,c][a,d] =  -[c,2[a,d]b+2[b,d]a]
=[c,[\{a,b\},d]].
\]

Use stander basis $x_{n+i} = D_{x_i}$,
From above we can see
\begin{align}
\{x_i,x_j\} &\mapsto -2 (E_{j,n+i} + E_{i,n+j})\\
\{x_i,D_j\} &\mapsto 2 (-E_{n+j,n+i}+E_{i,j})\\
\{D_i,D_j\} &\mapsto 2(E_{n+j,i}+E_{n+i,j}).
\end{align}
In another way, gives a embbeding from $\sp_\bC(W)$ to the Wely algebra $\cww$.
\begin{align}
E_{j,n+i}+E_{i,n+j} &\mapsto -x_ix_j\\
E_{i,j} - E_{n+j,n+i} &\mapsto \frac{1}{2}(x_i\partial_{x_j}+ \partial_{x_j} x_i) \\
E_{n+i,j}+ E_{n+j,i} &\mapsto \partial_{x_i}\partial_{x_j}.
\end{align}

Define involution $\dagger$ on $\cww$ so $\fgg$ by 
\begin{align*}
x_i^\dagger &= D_i,  \quad D_i^\dagger = x_i\\
s^\dagger & = \overline{s}, \quad s\in \bC \\
(uv)^\dagger &= v^\dagger u^\dagger \quad u,v\in \cww
\end{align*}


Read the action of $\dagger$ on $\sp_\bC(W)$:
\[
\begin{pmatrix}
A & B\\
C & -A^T
\end{pmatrix}
\mapsto
\begin{pmatrix}
A^* & -C^*\\
-B^* & -\overline{A}
\end{pmatrix}
\]

Let $W_0 = \Set{w\in W|w^\dagger = -w}$, is a real form of $W$.
Moreover $W_0$ has bases $\Set{e_j = x_j-D_j, f_j = i(x_j+D_j)}$. 
Clearly 
\[
[e_i,f_j] = [x_i-D_i,i(x_j+D_j)] = -2i
\frac{i}{2} [e_i,f_j] = \delta_{ij}.
\]
Hence define 
\[
\inn{w}{v} = \frac{i}{2}p[w,v], \quad \text{for } w,v\in W_0.
\] 
Let block matrix 
\[
c=\begin{pmatrix}
1 & i\\
-1 & i
\end{pmatrix}
\]
Then 
\[
\begin{pmatrix}
x_i & D_i
\end{pmatrix}
\begin{pmatrix}
1 & i\\
-1 & i
\end{pmatrix}
= 
\begin{pmatrix}
x_i-D_i & i(x_i + D_i)
\end{pmatrix}
\]

We have embeding $\sp_\bR(W_0)$ into $\sp_\bC(W)$,
For any 
\[
X = \begin{pmatrix}
A & B \\
C & -A^T
\end{pmatrix}
\in \sp_\bR,
\quad (B-B^T=0,C-C^T=0),
\]
\[
X \mapsto
\Ad(c)X
= \frac{1}{2}\begin{pmatrix}
A-A^T + i(C-B) & -(A+A^T)-i(C+B) \\
-(A+A^T) +i(C+B) & (A-A^T)-i(C-B)
\end{pmatrix}
\]


Regard $\sp_\bR$ as $S^2(X\oplus X^*)$, we can see the maps:

\subsection{embedding of $U(r,s)$ into $\Sp_\bR$}
Since $U(r,s)$ is the isometric matrix for 
\[
I_{r,s} = \begin{pmatrix}
I_r & 0 \\
0 & -I_s 
\end{pmatrix}.
\]

Then 
\[
U(r,s) = \Set{g\in \Mat_{r+s}(\bC)| g^*I_{r,s}g = I_{r,s}}
\]

Let $\Set{e_i}$ be the standard basis of $\bC_{r+s}$, let $e_{n+j} = ie_j$.
Then $\Set{e_j}$ form a $\bR$-basis of $\bC_{r+s}$, the action of 
$X = A+iB\in \Mat_{r+s}(\bC)$ is  
\[
\begin{pmatrix}
A &-B\\
B & A
\end{pmatrix}
\]
in matrix.

The Hermition form can written as 
\[
\begin{pmatrix}
I_{r,s} & 0 \\
0 & I_{r,s}
\end{pmatrix}
+ i
\begin{pmatrix}
0 & -I_{r,s}\\
I_{r,s} & 0
\end{pmatrix}
\]

Hence $U(r,s)$ preserve 
\[
\begin{pmatrix}
0 & -I_{r,s}\\
I_{r,s} & 0
\end{pmatrix}.
\]

Since
\[
\begin{pmatrix}
I_{r+s} & 0\\
0 & I_{r,s}
\end{pmatrix}
\begin{pmatrix}
0 & -I_{r,s}\\
I_{r,s} & 0
\end{pmatrix}
\begin{pmatrix}
I_{r+s} & 0\\
0 & I_{r,s}
\end{pmatrix}
= \begin{pmatrix}
0&-I\\
I & 0
\end{pmatrix}
\]
we get a embedding from $U(r,s)$ into $\Sp_\bR$ then into $\Sp_\bC$,
(also $\fuu(r,s)$ into 
$\sp_\bR(2(r+s))$ and $\sp_\bC(2(r+s))$),i.e. (as Lie algebra)
\[
\begin{split}
A+iB\mapsto &
\begin{pmatrix}
A & -B I_{r,s}\\
I_{r,s} B & I_{r,s}AI_{r,s}
\end{pmatrix}
=
\begin{pmatrix}
A & -B I_{r,s}\\
I_{r,s} B & -A^T
\end{pmatrix}\\
\mapsto &\frac{1}{2}
\begin{pmatrix}
A-A^T + i(I_{r,s}B + BI_{r,s}) & - (A+A^T) - i(I_{r,s}B-BI_{r,s})\\
-(A+A^T) + i(I_{r,s}B - BI_{r,s})& (A-A^T) - i(I_{r,s}B+BI_{r,s})
\end{pmatrix}
\end{split}
\]
by $I_{r,s} A + A^TI_{r,s} = 0$

\section{cases}
When $s=0$, then $I_{r,s} = I$, $A=-A^T$, $B=B^T$.
We have 
\[
A+iB \mapsto 
\begin{pmatrix}
A + iB & 0\\
0& -A^T +iB^T
\end{pmatrix}
\]

In general, 
we have $I_{r,s}A + A^T I_{r,s} = 0$, $I_{r,s}B -B^T I_{r,s} =0$, then we have
When ($B=0$)
\[
A = \begin{pmatrix}
E & F\\
F^T & G
\end{pmatrix}
\] 
such that $E+E^T=0$, $G+G^T=0$. 
Then it mapsto 
\[
\begin{pmatrix}
E & 0 & 0 & -F \\
0 & G & -F^T & 0 \\
0 & -F& -E^T & 0 \\
-F^T & 0& 0 & -G^T 
\end{pmatrix}
\]

When $A=0$,
\[
B =\begin{pmatrix}
K & L\\
-L^T & M  
\end{pmatrix}
\]
where $K = K^T$, $M=M^T$.
Then it maps to 
\[
\begin{pmatrix}
iK & 0 & 0 & -iL \\
0 & -iM & -iL^T & 0\\
0 & iL & -iK^T & 0\\
iL& 0 & 0 & iM^T
\end{pmatrix}
\]

So, if 
\[
X = \begin{pmatrix}
A & B\\
B^* & C
\end{pmatrix}
\] such that $A+A^*=0$, $C+C^*=0$.
Then it maps to 
\[
\begin{pmatrix}
A & 0 & 0 & -B\\
0 & -C^T & -B^T & 0\\
0 & -\overline{B} & -A^T & 0 \\
-\overline{B}^T & 0&  0& C  
\end{pmatrix}
\]

Now compute the embedding for complexified Lie algebra.
For 
\[X = 
\begin{pmatrix}
A & B\\
C & D
\end{pmatrix}
\in \agl_{r+s} = \fuu_\bC(r,s), 
\]
Then 
\[
\begin{split}
X &= \frac{1}{2}\begin{pmatrix}
A-A^* & B+C^*\\
B^*+C & D-D^*
\end{pmatrix}
- \frac{1}{2}i
\begin{pmatrix}
iA+iA^* & iB-iC^*\\
-iB^*+iC & iD+iD^*
\end{pmatrix}\\
&\mapsto
\begin{pmatrix}
A & 0 & 0 & -B \\
0 & -D^T & -B^T & 0\\
0 & -C^T & -A^T & 0 \\
-C& 0   & 0 & D
\end{pmatrix}
\end{split}
\]

If we view $\agl_{r+s}$ as $(X\oplus Y)\otimes (X^*\oplus Y^*)$, 
Then we can read the correspond map 
\begin{align*}
x_i\otimes x_j^* \mapsto&
 \frac{1}{2}(x_i \partial_{x_j} + \partial_{x_j} x_i)\\ 
x_i\otimes y_j^* \mapsto&
 x_iy_j\\ 
y_i\otimes y_j^* \mapsto&
- \frac{1}{2}(y_j \partial_{y_i} + \partial_{y_i} y_j)\\ 
y_i\otimes x_j^* \mapsto&
\partial_{y_i}\partial_{x_j}
\end{align*}

\section{$U(r,s)\times U(m)$ embedding}
Let $V = \Mat_{r+s,m}(\bC) = \otimes^m \bC_{r,s}$ with $U(r,s)$ act by left multiplication, $U(m)$ 
act by right multiplication, i.e. 
\[
(g,h)(v) = gvh^* = gvh^{-1} \quad g\in U(r,s), h\in U(m)
\]
Then $V$ has $U(r,s)\times U(m)$ invariant Hermitian form 
\[
\inn{v}{w} =  \tr(w^*I_{r,s}v) = \tr(I_{r,s} vw^*)
\]
This gives a embedding of $U(r,s)\times U(m)$ into $U(r*m, s*m)$. 
Now we caculate the Lie algebra embedding.
Suppose 
\[
v=\begin{pmatrix}
x_{1,1} & \cdots & x_{1,m}\\
\cdots & \cdots & \cdots \\
x_{r,1} & \cdots & x_{r,m}\\
y_{1,1}& \cdots & y_{1,m} \\
\cdots & \cdots & \cdots\\
y_{s,1} & \cdots & y_{s,m}
\end{pmatrix}
\]
Choose a ordered basis of $V$ by $x_{1,1}, \cdots, x_{1,m}, x_{2,1}, \cdots, x_{r,m}, y_{1,1},
\cdots, y_{s,m}$, i.e. $e_{(i-1)m+j}=x_{i,j}$, $e_{(r+i-1)m+j}=y_{i,j}$. 

\section{$\fuu_\bC(m) = \agl_m$ action}
For $X \in \agl_m$, it act on $V$ by $X.v = -vX$.
Hence ($\fuu_\bC(m) \to \fuu_\bC(m(r+s) \to \sp_\bC(2m(r+s))\to \lww$
\[
\begin{split}
X = E_{p,q}\mapsto& -(\sum_{v=1}^r x_{v,q}\otimes x_{v,p}^*
+ \sum_{v=1}^s y_{v,q}\otimes y_{v,p}^*)\\
\mapsto& \begin{cases}
-\sum_{v=1}^r x_{v,p} \partial_{x_{v,p}} +
\sum_{v=1}^s y_{v,p} \partial_{y_{v,p}}
- \frac{r-s}{2} & p=q\\
-\sum_{v=1}^r x_{v,q}\partial_{x_{v,p}}
+\sum_{v=1}^s y_{v,p}\partial_{y_{v,q}}
& p\neq q
\end{cases}
\end{split}
\]

\section{$\fuu_\bC(r,s)$ action}
For $X=E_{p,q}$, $p,q \leq r$, we have, 
\[
\begin{split}
E_{p,q} \mapsto& \sum_{v=1}^mx_{p,v}\otimes x_{q,v}^*\\
\mapsto& 
\begin{cases}
\sum_{v=1}^m x_{p,v} \partial_{x_{p,v}} + \frac{m}{2} & p=q\\
\sum_{v=1}^m x_{p,v} \partial_{x_{q,v}} & p\neq q
\end{cases}
\end{split}
\]
Similary, for $X=E_{r+p,r+q}$, $p,q\leq s$, we have
\[
\begin{split}
X=E_{r+p,r+q} \mapsto& \sum_{v=1}^{m} y_{p,v}\otimes y_{q,v}^* \\
\mapsto &
\begin{cases}
-\sum_{v=1}^m y_{p,v}\partial_{y_{p,v}} - \frac{m}{2} & p=q\\
-\sum_{v=1}^m y_{q,v}\partial_{y_{p,v}} & p\neq q
\end{cases}
\end{split}
\]
For $X= E_{p,r+q}$, $p\leq r,q\leq s$, we have
\[
\begin{split}
X = E_{p,r+q} \mapsto& \sum_{v=1}^m x_{p,v}\otimes y_{q,v}^*\\
\mapsto& \sum_{v=1}^m x_{p,v}y_{q,v}
\end{split}
\]
For $X=E_{r+p,q}$,$p\leq s, q\leq r$, we have
\[
\begin{split}
X = E_{r+p,q} \mapsto& \sum_{v=1}^m y_{p,v}\otimes x_{q,v}^*\\
\mapsto& \sum_{v=1}^m \partial_{y_{p,v}}\partial_{x_{q,v}}=\Delta_{q,p}
\end{split}
\]

\section{explicit $U(r,s) \sim U(m)$ duality}
For $\fuu_\bC(r,s)$ and $\fuu_\bC(m)$,
 we choose $E_{p,q}$, $p>q$ to span the nilpotent part,
i.e. the lower triangle matrixs. 

Consider $k \leq r, m$
\[
\Delta_{k} = \begin{vmatrix}
  x_{r-k+1,1}& \cdots& x_{r-k+1,k}\\
  \cdots & \cdots & \cdots \\
  x_{r,1} & \cdots & x_{r,k}
\end{vmatrix}
\]
\[
\tDelta_{k} = \begin{vmatrix}
  y_{1,m-k+1}& \cdots& y_{1,m}\\
  \cdots & \cdots & \cdots \\
  y_{k,m-k+1} & \cdots & y_{k,m}
\end{vmatrix}
\]
It is easy to see that $\Delta_k$ and $\tDelta_k$ both are 
highest weight with respect to the positive system we choose
in the joint harmonic space
\[
\sH = \Set{f\in \sP(V)|\Delta_{p,q}f = 0 , p\leq r, q\leq s}. 
\]

The wight of $\Delta_k$ repsct to $\fuu_\bC(r)\oplus\fuu_\bC(s)$ and $\fuu_\bC(m)$ are
\[
((0,\cdots, 0, \underbrace{1,\cdots,1}_{k})+\frac{m}{2})\otimes((0,\cdots, 0)-\frac{m}{2} 
\]
and
\[
(\underbrace{-1,\cdots, -1}_{k},0,\dots, 0)-\frac{r-s}{2}
\]

The wight of $\tDelta_k$ repsct to $\fuu_\bC(r)\oplus\fuu_\bC(s)$ and $\fuu_\bC(m)$ are
\[
((0,\cdots, 0)+\frac{m}{2})\otimes(-(\underbrace{1,\dots,1}_k,0,\cdots, 0)-\frac{m}{2}) 
\]
and
\[
(0,\dots, 0,\underbrace{1,\cdots, 1}_k)-\frac{r-s}{2}
\]

Now choose $a_1, \cdots, a_u$ and $b_1, \cdots, b_v$ s.t.$ a_1\geq a_2\geq\cdots \geq a_u>0$
and $ b_1\geq a_2\geq\cdots \geq b_v>0$. Moreover $u\leq r, v\leq s$, $u+v\leq m$. 
Consider 
\[
f = \Delta^{a_1-a_2}\cdots \Delta^{a_{u-1}-a_u}\Delta^{a_u} \tDelta^{b_1-b_2}\cdots \tDelta^{b_{u-1}-b_u}\tDelta^{b_u}
\]
$f$ is joint hieghest weight in $\sH$, and we have correspondence:
\[
\begin{split}
& ((0, \cdots, 0, a_u, \cdots, a_1)+\frac{m}{2})\otimes (-(b_1, \cdots, b_v, 0, \cdots, 0)-\frac{m}{2})\\
\leftrightarrow& (-a_1,\cdots, -a_u, 0,\cdots,0, b_v, \cdots, b_1)-\frac{r-s}{2}
\end{split}
\]


\section{Theta corresponedce for $\Sp(2n,\bR)\sim O(m)$}
Note that 
\[
O(m,\bR) = \Set{g|g^T g=1}.
\]
For $m=2k$, let 
\[
c = \frac{1}{\sqrt{2}}\begin{pmatrix}
1 & i\\
1 & -i
\end{pmatrix}
\]
\[
S = \begin{pmatrix}
0 & 1\\
1 & 0
\end{pmatrix}
\]
Then 
\[
c^TSc = 1,\quad cc^T = S
\]
So 
\[
\begin{array}{ccl}
O(m,\bR) &\to & O(m,\bC)=\Set{g| g^TSg = S}\\
g &\mapsto &cgc^T
\end{array},
\]
gives an embedding. It also gives the Lie algebra embedding.

Consider the $\bC$ vector space $W = V_1\otimes V_2$, where $V_1$ 
with symplectic form $J$, $V_2$ with symmetric form $S$.
Choose standard basis $\Set{e_1,\cdots ,e_2n}$ of $V_1$, 
$\Set{f_1, \cdots, f_m}$ of $V_2$.

Identify $\Mat_{2n,m}(\bC)$ with $W$ by $x_{ij}$ be the $e_i\otimes f_j^*$.
The symplectic form on $\Mat_{2n,m}(\bC)$ is $\inn{x}{y} = \tr(y^TJxS)$.

Choose a Lagrangian space $V$ spaned by
$\Set{e_i\otimes f_j|0\leq i\leq n, 0\leq j \leq m}$.
Then the dual basis of $e_i\otimes f_j$ is 
$e_{n+i}\otimes f_{l+j}$ or $e_{n+i} \otimes f_{j-l}$ if $j\leq l$ or $j>l$. 

Embed $\Sp(V_1)$ into $\Sp(W)$. View the $\fpp^-$ part,
we have, 
\begin{align*}
 E_{n+i,j}+ E_{n+j,i} = (e_{j}\mapsto e_{n+i}, e_{i}\mapsto e_{n+j})  
\mapsto& (e_j\otimes f_v \mapsto e_{n+i}\otimes f_v,
 e_i\otimes f_v \mapsto e_{n+j}\otimes f_v)\\
 =&( x_{j,v}\mapsto x_{n+i,l+v}\text{ or } x_{n+i,v-l}, \\
 &x_{i,v}\mapsto x_{n+j,l+v}\text{ or } x_{n+j,v-l})\\
\mapsto& -[
\sum_{v=1}^l\partial_{x_{i,l+v}}\partial_{x_{j,v}} 
+ \partial_{x_{i,v-l}}\partial_{x_{j,v}}]
\end{align*}

Let $y_{i,v} = x_{i,l+v}$ for $v\leq l$.
Then we can see 
\[
\Delta_{i,j} = \sum_{v=1}^l \partial_{y_{iv}}\partial_{x_{jv}}
+ \partial_{x_{iv}}\partial_{y_{jv}}
\] 

\[
\begin{split}
E_{i,j} =& (e_j\mapsto e_i) \mapsto (e_j\otimes f_v \mapsto e_i\otimes f_v)\\
 \mapsto& \frac{1}{2}(\sum_{v=1}^m x_{iv}\partial_{x_{jv}}+ \partial_{x_{jv}}x_{iv}) 
\end{split}
\]

\[
\aso(m,\bC) = \Set{
\begin{pmatrix}
A & B\\
C & -A^T
\end{pmatrix}| B+B^T=0, C+C^T=0}
\]

Note that $O(m)$ act on $\Mat_{2n,m}$ on right as usual,
i.e. $g\dot x = xg^T$.

\[
\begin{split}
E_{ij} - E_{l+j,l+i} \mapsto & E_{ji}-E_{l+i,l+j}
\mapsto (x_{vj}\mapsto -x_{vi}, x_{v,l+i}\mapsto x_{v,l+j})\\
\mapsto & 
+\frac{1}{2}(\sum_{v=1}^l x_{vi}\partial_{x_{vj}}+ \partial_{x_{vj}}x_{vi})
-\frac{1}{2}(\sum_{v=1}^l x_{v,l+j}\partial_{x_{v,l+i}}+ \partial_{x_{v,l+i}} x_{v,l+j})
\\
=&
+\frac{1}{2}(\sum_{v=1}^l x_{vi}\partial_{x_{vj}}+ \partial_{x_{vj}}x_{vi})
-\frac{1}{2}(\sum_{v=1}^l y_{vj}\partial_{y_{vi}}+ \partial_{y_{vi}} y_{vj})
\end{split}
\] 

\[
\begin{split}
E_{i+l,j}-E_{j+l,i}\mapsto &E_{j,i+l}- E_{i,j+l} \\
\mapsto &
+\frac{1}{2}(\sum_{v=1}^l x_{v,i+l}\partial_{x_{vj}}+\partial_{x_{vj}}x_{v,i+l})
-\frac{1}{2}(\sum_{v=1}^l x_{v,j+l}\partial_{x_{vi}}+\partial_{x_{vi}}x_{v,j+l})\\
=&
+\frac{1}{2}(\sum_{v=1}^l y_{v,i}\partial_{x_{vj}}+\partial_{x_{vj}}y_{v,i})
-\frac{1}{2}(\sum_{v=1}^l y_{v,j}\partial_{x_{vi}}+\partial_{x_{vi}}y_{v,j})
\end{split}
\]

Always choose lower trivangle as Boral subalgebra. 

Let 
\begin{align*}
\Delta_s =& \det\begin{pmatrix}
y_{n-s+1,1} & \cdots & y_{n-s+1, s} \\
\cdots & \cdots & \cdots \\
y_{n,1} & \cdots & y_{n,s}
\end{pmatrix}\\
\tDelta_s = & \det\begin{pmatrix}
x_{n+1-(2l-s), s+1} & \cdots & x_{n+1-(2l-s),l} & y_{n-s-l+1,1} & \cdots 
& y_{n+1-(2l-s),l}\\
\cdots&\cdots&\cdots&\cdots&\cdots&\cdots \\
x_{n, s+1} & \cdots & x_{n,l} & y_{n,1} & \cdots & y_{n,l}
\end{pmatrix}
\end{align*}

$\Delta_s$ has $\asp(2n,\bC)\oplus \aso(m,\bC)$ weight 
\[
(0, \cdots, 0, \underbrace{1, \cdots, 1}_{s})+\frac{m}{2}
\leftrightarrow
(\underbrace{-1,\cdots, -1}_s, 0,\cdots, 0) .
\]

$\tDelta_s$ has weight
\[
(0, \cdots, 0, \underbrace{1,\cdots, 1}_{2l-s=m-s})
\leftrightarrow (\underbrace{-1,\cdots, -1}_s, 0,\cdots, 0)
\]


In this case 
\[
\OO(m) =\SO(m) \rtimes \bZ_2.
\]

Parametrize irr $\SO(m)$ representation by highest weight
$(-n_l, \cdots, -n_1)$. If $n_l\neq 0$, 
Choose 
\[
g_0 = \begin{pmatrix}
  I_{l-1} & 0 & 0 & 0\cr
  0 & 1 & 0 & 0\cr
  0 & 0 & -I_{l-1} & 0\cr
  0 & 0 & 0 & 1
\end{pmatrix}
\in \OO(m,\bR)\setminus \SO(m,\bR).
\]
Then 
\[
\sigma = c g_0 c^T = \begin{pmatrix}
  I_{l-1} & 0 & 0 & 0\cr
  0 & 0 & 0 & 1\cr
  0 & 0 & I_{l-1} & 0\cr
  0 & 1 & 0 & 0\end{pmatrix}
\]
normalize the Borel subalgebra, hence $\sigma \lambda$ has highest weight
$(-n_1,\cdots,n_l)$.
$\Ind_{\SO(m)}^{\OO(m)}\lambda$ will be irreducible. 
If $n_l=0$, $\Ind_{\SO(m)}^{\OO(m)}\lambda$ decompose into $\lambda$ 
and $\lambda\otimes \det$. 

So parameterize irreducible $O(m)$ representation as 
$(-n_1, \cdots, -n_{l-1},|n_1|;\epsilon)$,$\epsilon$ is $\pm 1$ dependes on the action 
of $\sigma$ is trivial or not.
$(-n_1,\cdots, -n_{l-1},|n_1|;\pm 1)$ are same if $n_l\neq 0$.   


Consider 
\[
f = \Delta_1^{\alpha_1}\Delta_2^{\alpha_2} \cdots \Delta_{j-1}^{\alpha_{j-1}}
\Delta_j^{\alpha_j},
\]
it gives the correspondence for 
\[
(0,\cdots, 0, n_j, \cdots, n_1) + \frac{m}{2}
\leftrightarrow 
(-n_1, \cdots, -n_j, 0, \cdots, 0;1), \quad j\leq n, l
\]

Note that $\sigma \tDelta_s = -\tDelta_s$. Let
\[
f=\Delta_1^{\alpha_1}\Delta_2^{\alpha_2} \cdots \Delta_{j-1}^{\alpha_{j-1}}
\Delta_j^{\alpha_j-1}\tDelta_j,
\]
it gives the correspondence for 
\[
(0,\cdots, 0,\underbrace{1,\cdots, 1, n_j, \cdots, n_1}_{m-j}) + \frac{m}{2}
\leftrightarrow 
(-n_1, \cdots, -n_j, 0, \cdots, 0;-1), \quad l\leq j \leq m-n
\]
Note that, if $j=l$, on extra $1$ can appear in $\sp(2n)$ representation.

\subsection{m=2l+1}

For $m=2l+1$, let $t_v = x_{2l+1,v}$.  
we have 
the $\fpp^-$ part is 
\[
\Delta_{i,j} = 
\sum_{v=1}^l \partial_{y_{iv}}\partial_{x_{jv}}
+ \partial_{x_{iv}}\partial_{y_{jv}} + \partial_{t_v}\partial_{t_v}
\]


\[
E_{i,j} \in \asp(W)  
 \mapsto \frac{1}{2}(\sum_{v=1}^m x_{iv}\partial_{x_{jv}}+ \partial_{x_{jv}}x_{iv}) 
\]

\[
\aso(m,\bC) = \Set{
\begin{pmatrix} 
A & B & D\\
C & -A^T &F \\
-F^T& -D^T& 0
\end{pmatrix} |
B+B^T=C+C^T=0}
\]

\[
E_{i,j}-E_{j+l,i+l} \in \aso \mapsto 
 \frac{1}{2}(\sum_{v=1}^l x_{vi}\partial_{x_{vj}}+ \partial_{x_{vj}}x_{vi})
-\frac{1}{2}(\sum_{v=1}^l y_{vj}\partial_{y_{vi}}+ \partial_{y_{vi}}y_{vj})
\]

\[
E_{i+l,j}-E_{j+l,i}\mapsto 
 \frac{1}{2}(\sum_{v=1}^l y_{v,i}\partial_{x_{vj}}+\partial_{x_{vj}}y_{v,i})
-\frac{1}{2}(\sum_{v=1}^l y_{v,j}\partial_{x_{vi}}+\partial_{x_{vi}}y_{v,j})
\]

\[
\begin{split}
-E_{2l+1,j}+ E_{l+j,2l+1} \mapsto&
-E_{j,2l+1}+E_{2l+1,l+j} \\
\mapsto&
(x_{vj}\mapsto -t_v, t_v\mapsto y_{vj})\\
\mapsto&
-\frac{1}{2}\sum_{v=1}^n (t_v\partial_{x_{vj}}+\partial_{x_{vj}}t_v)
+\frac{1}{2}\sum_{v=1}^n (y_{vj}\partial_{t_v}+\partial_{t_v}y_{vj})
\end{split}
\]

Let 
\begin{align*}
\Delta_s =& \det\begin{pmatrix}
y_{n-s+1,1} & \cdots & y_{n-s+1, s} \\
\cdots & \cdots & \cdots \\
y_{n,1} & \cdots & y_{n,s}
\end{pmatrix}\\
\tDelta_s = & \det\begin{pmatrix}
x_{n-2l+s, s+1} & \cdots & x_{n-2l+s,l} & y_{n-2l+s,1} & \cdots & y_{n-2l+s,l}&t_{n-2l+s}\\
\cdots&\cdots&\cdots&\cdots&\cdots&\cdots &\cdots \\
x_{n, s+1} & \cdots & x_{n,l} & y_{n,1} & \cdots & y_{n,l}& t_n
\end{pmatrix}
\end{align*}

$\Delta_s$ has $\asp(2n,\bC)\oplus \aso(m,\bC)$ weight 
\[
(0, \cdots, 0, \underbrace{1, \cdots, 1}_s)+\frac{m}{2}
\leftrightarrow
(\underbrace{-1,\cdots, -1}_s, 0,\cdots, 0) .
\]

$\tDelta_s$ has weight
\[
(0, \cdots, 0, \underbrace{1,\cdots, 1}_{2l+1-s=m-s})+\frac{m}{2}
\leftrightarrow (\underbrace{-1,\cdots, -1}_s, 0,\cdots, 0)
\]

The difference between $\Delta_s$ and $\tDelta_s$ is the $-I_m$ action.
When $m=2l+1$, $O(m) \cong SO(m)\times \bZ_2$, where $g_0=-I_m\in O(m)$ 
is the nontrivial element in $\bZ_2$. It is easy to see that, 
\begin{align*}
g_0\cdot \Delta_s &= (-1)^s\\
g_0\cdot \tDelta_s & = (-1)^{s+1}
\end{align*}

It is clear that $\widehat{O(m)} = \widehat{SO(m)}\times \widehat{\bZ_2}$. 
So we parametrize $\widehat{O(m)}$ by $(\lambda; \epsilon)$, where $\lambda$
is $SO(m)$ hightest weight, $\epsilon=1$ if $g_0$ act trivially, $\epsilon=-1$
if $g_0$ action is non-trivial.   

Hence we got the following corresponence by letting $f$ be the joint hightest weight:
\begin{align*}
f &= \Delta_1^{\alpha_1}\Delta_2^{\alpha_2} \cdots \Delta_j^{\alpha_j}\\
\epsilon &= (-1)^{\sum_{s=1}^j s\alpha_s} = (-1)^{\sum_{s=1}^jn_s}\\
n_s &= \sum_{t=s}^j \alpha_t\\
\text{i.e. } \alpha_s &= n_s-n_{s+1}
\end{align*}

\[
(0,\cdots, 0, n_j, \cdots, n_1)+\frac{m}{2}
\leftrightarrow
((-n_1, \cdots, -n_j); \epsilon)
\]

Similary, let
\begin{align*}
f &= \Delta_1^{\alpha_1}\Delta_2^{\alpha_2} \cdots \Delta_{j-1}^{\alpha_{j-1}}\Delta_j^{\alpha_j-1}\tDelta_j\\
\epsilon &= (-1)^{\sum_{s=1}^j s\alpha_s} = (-1)^{\sum_{s=1}^j n_s+1}\\
n_s &= \sum_{t=s}^j \alpha_t\\
\text{i.e. } \alpha_s &= n_s-n_{s+1}
\end{align*}

\[
(0,\cdots, 0,\underbrace{1, \cdots, 1, n_j, \cdots, n_1}_{m-j})+\frac{m}{2}
\leftrightarrow
((-n_1, \cdots, -n_j); \epsilon)
\]


\section{Computation for the transform of the Oscillator representation}
Decomposit $\omega$ into $U(p,q)$ module, by see-saw pair.
Choose positive system by lower triangle of $U(p,q)$
We have
$\omega = \otimes_{l} L_l$, 
where $L_l$ is lowest weight module of $U(p,q)$ with lowest $U(p)\times U(q)$ 
type
\[
\begin{split}
&(\underbrace{\frac{1}{2},\cdots, \frac{1}{2}, \frac{1}{2}+l}_p, 
\underbrace{-\frac{1}{2},\cdots, -\frac{1}{2}}_q)\\
=&
(\underbrace{0, \cdots, 0, l}_p, 
\underbrace{-1,\cdots, -1}_q)+\frac{1}{2}
\quad l \geq 0
\end{split}
\]
\[
\begin{split}
&(\underbrace{\frac{1}{2},\cdots, \frac{1}{2}}_p,  
\underbrace{-\frac{1}{2}+l,-\frac{1}{2},\cdots, -\frac{1}{2}}_q)\\
=&
(\underbrace{0, \cdots,  0}_p, 
\underbrace{-1+l,\cdots, -1}_q)+\frac{1}{2}
\quad l < 0
\end{split}
\]
Call the prameter $\Lambda_l$.

Let \[s = (p,p+1,\cdots, p+q).\]
Then 
\[
s\rho-\rho = (\underbrace{0,\cdots,0}_{p-1},+1,\cdots, +1, -q)
\]

Then 
\[
s(\Lambda_l+\rho)-\rho = (0, \cdots, 0, l-q) +\frac{1}{2}
=F_{l-q}
\]
is the $K$ type of $\Gamma^q(L_l)$. 

So 
\[
\Lambda_l =  s^{-1} F_{l-q} + (s^{-1}\rho -\rho)
\]

Let 
\[
W= s^{-1}\rho -\rho
= (\underbrace{0,\cdots, 0, q}_p, \underbrace{-1, \cdots, -1}_q)
\]
 be the submodule of $\bigwedge^q \fqq$ (it is clear that $\Delta^+_M\subset s^{-1}\Delta^+$). 

Now 
\[
s^{-1} F_{l-q} = (\underbrace{0, \cdots, 0,l-q}_p,\underbrace{0,\cdots, 0}_q) 
+\frac{1}{2}
\]
is the unique weight of $F_{l-q}$, assume one weight vector is $w_{l-q}$.

The heighest weight vecter $w$ of $M$-module $W$ will map into the span of
$v_l\otimes(w_{1-q})^*$, where $v_l$ is the heighest weight of $L_l$. 

Since 
\[
\Hom_M(W\otimes F_{l-q},L_l)=\Hom_M(W, L_l\otimes F_{l-q}^*)\cong \bC
\]
is dimension $1$. 
We only need to understand the immage of heighest weight vecter in 
$L_l\otimes F_{l-q}^*$.

\subsection{direct compute}
Realize $F_{l}$ on $\sS^l(\bC^n)$. 
I.e. $F_{l}$ is the space of degree $l$ homogeneous on $\bC^n$.
$E_{i,j}\in \fuu(n)_\bC$ act by $x_i\partial_{x_j}$, for $i\neq j$.
$E_{i,i}$ act by $x_i\partial_{x_i}+\frac{1}{2}$.

Let $w^*$ denote the dual of $w\in F_{l}$ in $F_{l}^*$. 

Now,
for $X \in \asp(n,\bC)$,  
we have 
\[
w\leftrightarrow 
X(v_l\otimes \inn{w_{l-q}^*}{\cdot w})(k)
 = p(\Ad(k)X v_l\otimes \inn{w_{l-q}^*}{kw}) 
\]

Let $k = (k_{i,j})\in U(n)$
\begin{enumerate}[(i).]
\item 
\[
X = \begin{pmatrix} 
0 & E_{i,j}\\
0 & 0
\end{pmatrix}
\]
Then 
\[
\Ad(k) X = 
\begin{pmatrix}
0 & k E_{i,j} k^T\\
0 & 0
\end{pmatrix}
\]
\end{enumerate}

\subsection{intertwining operator}
As $K$-module, $\Gamma_W \omega = \bigoplus_{l\geq 0} F_l$ and 
$\omega = \bigoplus_{l\geq 0} S_l$
Choose any intertwining operator $T_i \colon F_i \to S_i$ for $i=0,1$. 

As far as we known, the $K$-type occurs in both $\omega$ and 
$\pp^+\otimes S_l$ is $S_{l+1}$. In fact, $\pp^+\cong S^2(\bC_n)$ 
as representation of $K=\widetilde{U(n)}$. 
In fact, it's easy to see from the Littlewood-Richardson rule or 
\cite{Humphreys1972}~24.4 exercise~12. Same statment is ture for $F_l$. 
 
So we have a unique spliting for the $K$-module homomphism 
$\pp^+\otimes S_l\to S_{l+2}$ defined by $X\otimes v \mapsto Xv$, 
Hence we have a unique $T_l$ make the following diagram commutes
\[
\bfig
\square[\pp^+\otimes S_l`S_{l+2}`\pp^+\otimes F_l`F_{l+2};`\id\otimes T_l`T_{l+2}`]
\efig
\]
Similar argument and Schur's lemma give following commutative diagram
\[
\bfig
\square[\pp^-\otimes S_{l+2}`S_{l}`\pp^-\otimes F_{l+2}`F_{l};`\id\otimes T_{l+2}`c_lT_{l}`]
\efig
\]
for some $c_l\in \bC$. 

Now we can define a intertwining map by $T= \otimes T_l$ if all $c_l=1$.

To show $c_1=1$, we consider the diagram
\[
\bfig
\square[\pp^-\otimes \pp^+ \otimes S_{l}`S_{l}`\pp^-\otimes \pp^+\otimes F_{l}`F_{l};`\id\otimes T_{l}`c_lT_{l}`]
\efig
\]
Assume $S_l$, $F_l$ has highest weight vector $u,v$. For some element 
$x\in \pp^-\otimes \pp^+$ if $xu=a u+ \cdots, xv=bu+\cdots$, with $a=b\neq 0$, 
then $c_1=1$. 
In fact $T_lu = dv$ for some $d\neq 0$ since $x$ is not trivial map by the 
assumption.
Now $adv +\cdots =T_l( a u +  \cdots)= T_l x u = c_l x T_l u = c_l bd v + \cdots$. Hence $c_1=1$ by $a=b\neq 0$ and $d\neq 0$. 

In fact we consider the Casimir. Let $C_\gg$ and $C_\kk$ be the Casimir
of $\gg$ and $\kk$. 
Now 
\[
C_\gg = C_\kk + \sum_{\alpha\in \Delta(\pp^+)} X_\alpha X_{-\alpha} + X_{-\alpha} X_{\alpha}
= C_\kk + \sum_{\alpha\in \Delta(\pp^+)}H_{\alpha} + 2 X_{-\alpha} X_\alpha.
\]

Take $x =  \sum_{\alpha\in \Delta(\pp^+)} X_{-\alpha} X_\alpha$.
Since $C_\gg$ act on the whole space by fixed scalar, $C_\kk$ act on $S_l$
and $F_l$ by same scalar and
$\sum_{\alpha\in \Delta(\pp^+)}H_{\alpha}$ 
act on the heighest weight vector by same scalar $\inn{\lambda_l}{\rho_{\pp^+}}$,
 we know $x$ act on $u,v$ by scalar and $a=b$. 

To varify $a\neq 0$, we just calculate $x$ action in $\omega$. 
Now, some multiple of $x$ act by $\sum_{i,j} \partial_{x_i}\partial_{x_j} x_ix_j$,
A highest weight vector in $S_l$ is $x_n^l$, so $a\geq 0$. 

\section{For other cases}
Now try to apply the method for $\theta^{p,q}(1^{\pm,\pm})$.
For other cases we also have the multiplicity free of $\pp^\pm\otimes F$, 
the point is to determine the $c$'s. 


\section{some computation in non-stable range}
Consider the computation in non-stalbe range for $(U(r,s),U(m))$
$n = r+s$, $m = p+q$.
Now 
\begin{align*}
\rho_m &= \frac{1}{2}(m-1, \cdots, -(m-1))
\rho_n &= \frac{1}{2}(n-1, \cdots, -(n-1))
\end{align*}

\[
\begin{split}
\lambda' =& (u_1, \cdots, u_r, 0,\cdots ,0, - v_s, \cdots, -v_1) + \rho_m + 
\frac{r-s}{2}
\\
=& (P, T, N)
\end{split}
\]
where
\begin{align*}
T = & \frac{1}{2}(m-1-2r, \cdots, -(m-1)+2s) + \frac{r-s}{2}\\
  = & \frac{1}{2}(m-n-1, \cdots, -m+n +1)\\ 
P = & (u_1, \cdots, u_r) + \frac{1}{2}(m-1, \cdots, m+1 -2r) + \frac{r-s}{2}\\
= & (u_1, \cdots, u_r) + \frac{1}{2}(m-1, \cdots, m+1 -2r) + \frac{r-s}{2}\\
N =& (-v_s, \cdots, -v_1) + \frac{1}{2}(-m-1 + 2s, \cdots, -m+1) + \frac{r-s}{2}
\end{align*}
Now the $T$ part is independent of real form $r,s$ and type. 
So by the correspondence of infinitesimal charicter, 
we have the infinitesimal charicter of theta lifing is 
$\lambda = (P,N)$.
Suppose that $P = (P_1, P_2)$ $u_i$'s in $P_1$ are sufficiently large, 
$Q = (Q_2,Q_1)$ $v_i$'s in $Q_1$ are sufficiently large, 
then the derive functor module will non-vanishing for certain choise 
of $P_2, Q_2$, but this module is unitarizable. 

One the other hand, the $K$ type $(a_1,\cdots, a_n)$ of 
$Sp(2n,\bR)$ module to be unitarizable 
 will have a limitation: $a_i -a_{i+1}$ should take at most one value among 
all $K$-types or infinit many value. 
If not, we can find a embedding $SL(2,\bR) \hookrightarrow Sp(2n,\bR)$
such that one $SL(2,\bR)$ sub module transfer as a module 
with this many $K$-type. By $SL(2,\bR)$ theory, this is contridict to 
the unitary condition.

From above point of view, it may be shown
that all unitarizable module is that type. 

\section{Transfer of ladder/minimal representation of $O(2,2n-2)$}
Work on double cover of all the groups. 
\subsection{Ladder representation of $O(p,q)$}
These representations only exists when $p-q$ or $p+q$ is even, denote by 
$V_{p,q}$.
It is the theta lift of trivial representation of the double cover of $SL(2,R)$.
The $K=O(p)\otimes O(q)$-type appear in this representation is 
\[
\chh_a\otimes \chh_b
\]
where $\chh_a$ is the space of harmonic of degree $a$ and $a,b \geq 0$
satisfies
\[
a+p/2 = b+q/2,
\]
or equivelently, 
\[
a = b + \frac{q-p}{2}.
\]

Hence the $K$-types of the ladder representation of $O(p,q)$ is supported on 
a line, hence called ladder representation. More over 
the non-compact part action of $O(p,q)$ will move the $K$-type up/down 
for 1-step. More precisely, let $\aso(p,q)_\bC = \fkk\oplus \fpp$, where
\[
\fkk = \aso(p,\bC) \otimes \aso(q,\bC)
\]
and
\[
\fpp = \bC^p\otimes \bC^q
\]
as $\fkk$ representation. 
By branching laws(Chapter~V, problems 15-23\cite{Knapp1996Lie},\cite{KimLee2010} ), $\fpp\otimes (\chh_a\otimes \chh_b) $ will be map into $\chh_{a+1}\otimes\chh_{b+1}\oplus \chh_{a-1}\otimes \chh_{b-1}$. 


% \subsection{decomposition of $V_{2,2n-2}$}
% Take 
% \[
% s = \begin{pmatrix}
%   I_p & 0 \\
%   0 & iI_q
% \end{pmatrix}
% \]
% Then we can have a explicit isomorphism 
% \[
% \begin{split}
% O(p,q)_\bC &\to O(p+q)\\
% X &\mapsto sXs^{-1}
% \end{split}
% \]

% By this isomorphism, we can see
% \[
% \theta_{p,q}(X)=-I_{p,q}X^T I_{p,q}
% \]
%  is an Cartan involution on $\aso(n,\bC)$ defining a real 
% Lie subagebra isomorphic to $\aso(p,q)$. 
% Moreover the complex conjugation $\sigma$ respect to $\aso(p,q)$ is 
% \[
% \sigma(X) = I_{p,q}\overline{X}I_{p,q}
% \] 

% Consider $V_{2,2n-2}$ as a $(\aso(n,\bC), O(2,2r-2)\times O(2n-2r))$-module,
% by take the drived functor, we will show that 
% \[
% \left(\Gamma_{\aso(n,\bC), O(2,2r-2)\times O(2n-2r)}^{\aso(n,\bC), O(2r)\times O(2n-2r)}\right)^j
% V_{2,2n-2} \simeq V_{2r,2n-2r}
% \]
% for certain $j\in \bN$.

% Now we will decompose $V_{2,2n-2}$ as $O(2,2r)\times O(2(n-r-1))$ module
%  by see-saw pair
% \[
% \xymatrix{
% O(2,2n-2) \ar@{-}[dr]& SL(2,\bR)\times SL(2,\bR) \\
% O(2,2r)\times O(2(n-r-1))\ar@{-}[ur] & SL(2,\bR)}
% \]

% Just as \cite{ZhuHuang1997}, suppose that $O(2,2r)\times O(2(n-r-1))$-module 
% $\mu \otimes \chh_l$ occure in $V_{2,2n-2}$.
% We will have a non-trivial map from 
% $\theta(\mu)\otimes \overline{D_{l+2(n-r-1)/2}}$ to $\bC$. 
% This means $\theta(\mu)\simeq D_{l+n-r-1}$. 

% Using the results in \cite{Howe1979Opq}, we will see that $\mu$ is a lowest 
% weight module. Then by the algebric version of Bott-Borel-Weil theorem, 
% we have $(\Gamma_{\aso(2r),O(2)\times O(2r-2)}^{O(2r})^j \mu = \chh_l$

% \begin{align*}
% &(l+n-2, 0,\cdots, 0, l, 0, \cdots,0)\\
% &(-(l+n-2), 0, \cdots, 0,l , 0, \cdots, 0)
% \end{align*}

% Temporarily, let $M = O(2)\times O(2r)$, $K = O(2r+2)$, $\fkk = \aso(2r+2)$. 
% $M_1 = K^0\cap M = SO(2)\times SO(2r)$.
% By (\cite{BorelWallach2000} p.~29), 
% $\Gamma_{\fkk,M}^{K} = Ind_{MK^0}^K\Gamma_{\fkk,M_1}^{K^0}$.


% In fact, 
% \[
% \tau+\rho(-(l+n-2),0,\cdots, 0) + (r,r-1, \cdots, 0)
% = (-l-(n-r)+2,r-1, \cdots, 0)
% \]
% Take $w$ be change sign of the first and last coordinate. 
% $w(\tau+\rho)= (l+n-2r-2, 0, \cdots, 0) + (r,\cdots, 0)$ is dominate.
% Moreover,by the algebric version of Borel-Weil-Bott, we know, 
% \[
% (\Gamma_{\fkk,M}^K)^s = (l+n-p,0, \cdots, 0),
% \]
% where 
% \[
% s = \sharp\Set{\alpha\in \Delta_\fkk|\inn{\tau+\rho}{\alpha}<0}
% = p-2.
% \]


\subsection{Lifting of discrete series}


Now we consider the lifting of discrete series, it is from Li\cite{Li1990}.
We define $D_l$ be the discrete series representation of $\SL(2,\bR)$ with 
lowest weight $l\geq 2$. The contragredient representation $\overline{D_l}$ is 
$D_{-l}$, the highest weight representation with highest weight $-l$.

Now we consider the lifting of $D_{l}$.
Asumme $p=2r, q=2s$.
Take the usual root system of $\aso(p+q,\bC)$, and usual Cartan decomposition
of $O(p,q)$. 

Consider see-saw pair 
\[
\xymatrix{
O(p,q) \ar@{-}[dr]& SL(2,\bR)\times SL(2,\bR) \\
O(p)\times O(q)\ar@{-}[ur] & SL(2,\bR)}
\] 
we get $O(p)\times O(q)$-type
 $\chh_a\otimes \chh_b$ appair in $\theta(D_{l})$ 
iff $D_{a+p/2}\otimes \overline{D_{b+q/2}}$ has a $D_{l}$ quotient.
From \cite{Howe1979Opq},
this equivelent to $D_{l}$ appair in this tensor product. 
By the results of \cite{Repka1976tensor}, we know this means
\[
a \geq b + \frac{q-p}{2} + l.
\]


Infinitesimal character is preserved by Derived functor, hence it is same as 
what in the theta-correspondence.
Infinitesimal charicater of $D_l$ is $(l-1)$. 
The infinitesimal character correspondence is 
\[
\begin{matrix}
 SL(2,\bR) & & O(2r,2s)\\
 (l-1) & \leftrightarrow & (l-1, m-2, \cdots, 1, 0)
\end{matrix}
\]
where $m=p+q$.

Now asumme $l-1>m-1$, i.e. $l>m$.

Let $x = (1, 0,\cdots, 0)\in i\ftt_0$,
\begin{align*}
\fll =& \sspan \Set{E_\alpha |\alpha(x) =0 }\\
\fuu =& \sspan \Set{E_\alpha |\alpha(x) >0 }\\
\fqq =& \fll+\fuu
\end{align*}

Now 
\[
\Delta(\fuu\cap \fpp) = \Set{e_1\pm e_j|j>r},
\]
and $\dim \fuu\cap \fpp = 2s$.
So, 
\[
\rho(\fuu\cap \fpp) = (s, 0, \cdots, 0).
\]
Take
\[
\lambda = (l+(q-p)/2-s, 0, \cdots, 0)=(l-(p+q)/2,0,\cdots, 0),
\]
which is a charicter of $\fll\cong \aso(2)\oplus \aso(2r-2,2s)$ 
such that(\cite{VoganZuckerman1984} (5.1))
\[
\inn{\alpha}{\lambda} \geq 0 \quad \forall \alpha \in \Delta(\fuu).
\]

But
\[
\tau = \lambda+2\rho(\fuu\cap\fpp) = (l+(q-p)/2, 0,\cdots, 0)
\]
is the highest weight of $K$-type occure in the both representation 
$\theta(D_l)$ and $A_\fqq(\lambda)$.
Moreover,
\[
\lambda + \rho = (l-1, m-2, \cdots, 0)
\]
is the infinitesimal charicter of $\theta(D_l)$. 

\begin{lemma}
For $(\fgg,K)$ module $V$, $\left(\Gamma_{\fgg,K}^{\fgg,G}\right)^j V = 0$ if the infinitesimal
character or $V$ is singular. Here, $G$ and $K$ are compact Lie groups.
\end{lemma}
\proof
If $(\Gamma_{\fgg,K}^{\fgg,G})^j V\neq 0$, it has a irreducible $G_0$ submodule $W$. 
The infinitesimal charicter of $W$ is $\lambda + \rho_G$, where $\lambda$ 
is the heightest weight of $W$. Hence $\inn{\lambda+\rho_G}{\alpha}>0$ for all
$\alpha\in \triangle^+$. But derived functor preserve 
the infinitesimal character. Hence should be non-singular.

Hence we get
\begin{lemma}\label{lemma:derived_D_l}
Let $\fgg = \aso(2m,\bC)$, $G = O(2m)$, $G_0 = SO(2m)$, $K = O(p)\times O(q)$, 
$K_0 = SO(p)\times SO(q)$.  $\lambda = (l-(p+q)/2,0,\cdots, 0)$.
\begin{enumerate}[a)]
\item $\Gamma^j(\theta(D_l)) = 0$ if $l<m=n/2$.  
\item When $p\neq 2$, $l\geq n/2$, $\theta(D_l)\cong A_\fqq(\lambda)$ as $G_0$ module, defined as above.
\item When $p=2$, $\fll=\fkk$ is the maximal Lie algebra of maximal compact
subgroup, $\fuu =\fpp^+=\sspan\Set{X_{e_1\pm e_j}|j\geq 2}$. Let 
$\fqq=\fll\oplus\fpp^+$ and
$\overline{\fqq}=\fll\oplus\fpp^-$, $\tau= (l+(q-p)/2,0,\cdots, 0)$ is a charicter of $\fkk$. Then
$\theta(D_l)\cong \cuu(\gg)\otimes_{\cuu(\fqq)}\bC_{-\tau}\oplus  \cuu(\gg)\otimes_{\cuu(\overline{\fqq})}\bC_{\tau} $ as $\aso(2,2n-2)$ module.

Moreover, we claim that
\[
\Gamma^q\theta(D_l) = (l-m,0,\cdot, 0; +1) 
\]
as $O(2m)$ module. 
\item $\left(\Gamma_{\fgg,K}^{\fgg,G}\right)^j\theta(D_l) = (\lambda; +1)$ for $j = q$.
\end{enumerate}
\end{lemma}
\proof
\begin{enumerate}[a)]
\item If $l<m=n/2$, $\lambda+\rho$ is singular. It follows by above lemma.
\item Clear.
\item As $SO(2)\otimes SO(2n-2)$ module $\theta(D_l)
= \bigoplus_{a\geq b+n-2+l}(a,b,0,\cdots, 0)\oplus \bigoplus_{a\geq b+n-2+l}(-a,b,0,\cdots, 0)
\triangleq V_1\oplus V_2$.
Now, $V_1$ is a lowest weight module with l.w. $\tau=(l+n-2, 0, cdots, 0)$.
In fact, $\fpp\cong \bC_p\otimes \bC_q$ as $SO(p)\otimes SO(q)$ representaton.
When $p=2$, $\bC_p$ is reducible, $\fpp^- \cong (-1, 1,0, \cdots, 0)$.
 $\fpp^-\bC_{\tau}=0$ since $\fpp^-\otimes \bC_{\tau} = (l+n-2-1,1, \cdots, 0)$ do not contain 
in $V_1$. Similary, $V_2$ is a highest weight module with h.w $-\tau$.

Now apply Borel-Weil-Bott-Kostant theorem.
Note,
\[
-\tau+\delta = (-l+1,m-2,cdots, 0),
\]
When $l\geq m$ it is regular and
\[
\#\Set{\alpha\in \Delta^+| \inn{-\tau+\delta}{\alpha}<0}
= \#\Set{e_1\pm e_j|2\leq j\leq q/2} = 1
\]
and $\dim \fpp = pq = 2q$.
So 
\[
\left(\Gamma_{\aso(2m),SO(2)\times SO(2m-2)}^{\aso(2m),SO(2m)}\right)^j V_2= (l-m,0,\cdots, 0) 
\]
as higest weight module. 

Similary
\[
\Gamma^q V_1 = (-(l-m),0,\cdots, 0) 
\]
as lowest weight module.

Let $\sigma$ be a order 2 element in $O(q)/SO(q)$, Let 
$\sigma$ also be the functor in $\ccc(\fgg,K)$(and $\ccc(\fgg,G)$
 by taking $\sigma$ inveriant. 
Consider that, 
$\theta(D_l) = \left(\Ind_{\fgg, K_0}^{\fgg,K}A_\fqq(\lambda)\right)^\sigma$

We have,
\[
\begin{split}
&\left(\Gamma_{\fgg,K}^{\fgg,G}\right)^j\circ \sigma \circ \Ind
= \left(\Gamma_{\fgg,K}^{\fgg,G} \circ \sigma\circ \Gamma_{\fgg, K_0}^{\fgg,K}\right)^j\\
=& \left(\sigma \circ \Gamma_{\fgg,K_0}^{\fgg,G}\right)^j
=  \left(\sigma \circ \Gamma_{\fgg,G_0}^{\fgg,G}\circ \Gamma_{\fgg,K_0}^{\fgg,G_0}\right)^j\\
= & \sigma\circ \Gamma_{\fgg,G_0}^{\fgg,G}\circ (\Gamma_{\fgg,K_0}^{\fgg,G_0})^j
\end{split}
\]
Apply the functor for $j=q$, we get the result. 
\end{enumerate}

For general situation, we need a theorem from Vogan-Zuckerman~(\cite{VoganZuckerman1984}):
\begin{thm}[Vogan-Zuckerman~\cite{VoganZuckerman1984}~Theorem~5.5,\cite{Wallach1988}~9.6.6 ]
Let $\fqq=\fll+\fuu$ be a $\theta$-stable parabolic subalgebra of $\fgg$,
and $\lambda\colon \fll\to \bC$ is an admissible character. 
Put $R = \dim \fuu\cap \fpp$. Suppose $F$ is a finite dimensional irreducible
represntation of $\fgg$ with highest weight $\gamma$ with respect to a 
$\theta$-stable system of positive roots compatible with $\fqq$. Then 
\[
H^j(\fgg,\fkk;A_\fqq(\lambda)\otimes F^*) \cong H^{j-R}(\fll,\fll\cap \ftt,\bC)
\cong \Hom_{\fll\cap \ftt}(\bigwedge^{j-R}(\fll\cap \fpp),\bC)
\]
if $\gamma = \lambda|_\fhh$; and
\[
H^j(\fgg,\fkk;A_\fqq(\lambda)\otimes F^*) = 0
\]
\end{thm}

Apply this theorem to our cases, 
Note that 
\begin{align*}
\fll \cong & \aso(2)\oplus\aso(p-2,q),\\
\fll\cap \ftt \cong & \aso(2)\oplus \aso(p-2)\oplus\aso(q)
\end{align*}
and 
\[
\fll \cap \fpp\cong   \bC_{p-2}\otimes \bC_{q}
\]
as representation of $\fll\cap \ftt$ and $\aso(2)$ act trivially. 


From \cite{Howe1995perspective}~4.1.1, we know as $\agl(p-2)\oplus \agl(q)$ module
\[
\bigwedge^* (\fll\cap\fpp)\cong \bigwedge^*(\bC_{p-2}\otimes \bC_{q})
\cong \bigoplus_{\substack{D\text{ has at most}\\\text{$p-2$ rows}\\\text{ and $q$ columns}}} \rho^D\otimes
\rho^{D^t}.
\]
Moreover, $\rho^D$ has $\aso(p-2)$ inveriant iff $D=2E$,
 i.e. $D$ has even columns in each row (\cite{Howe1995perspective}~3.3.2). 

So in general, the dimension of  $\Hom_{\fll\cap \ftt}(\bigwedge^{j-R}(\fll\cap \fpp),\bC)$ 
may greater than 1 for some $j>R=2s$.

So now we only consider $j=R=2s$. Then it is dimension 1.
Hence we get Lemma~\ref{lemma:derived_D_l}~(d).





\subsection{$(\OO(p,q),\SL(2,\bR))$ dual pair}
Now choose another real subsapce $W_{p,q}$ of $W$, 
which spaned by $e_1,\cdot, e_p, ie_{p+1},\cdot,ie_{n}, 
f_1, \cdots, f_p, -if_{p+1}, -if_n$. We can view $W$ as $M_{n,2}$, 
with $\OO(n,\bC)$ act on left, $SL(2,\bC)$ act on right. 
Give symplectic form on $W$ by tensoring the form 
of $\OO(n,\bC)$ and $\SL(2,\bC)$, so we have complex symplectic group act on $W$.
$\OO(n,\bC)$ and $\SL(2,\bC)$ are complex reductive dual pair 
Choose real subgroup stablize $W_{p,q}$, we have $\OO(p,q)$ and $\SL(2,\bR)$ 
as real reductive dual pair subgroup in $\Sp(2n,\bR)$. 
Note that the embedding of $\SL(2,\bR)$ in $\Sp(2n,\bC)$ is not change for
different $W_{p,q}$.



\subsection{Correspondence of $U(\fgg)^K$ }

Now consider the dual pair $(O(p,q),SL(2,\bR))=(G,G')$,
$K = O(p)\times O(q)$, $M=SL(2,\bR) \times SL(2,\bR)$.

Similary, since $G'$ act on $U(\fmm')$ reductively, 
\[
\Pi^{G'} = \pi(U(\fmm')^{G'}) = \pi(U(\fmm')^{\asl(2,\bR)})
\]

In fact, $\asl(2,\bR)$ generated by $e,f,h$, embedded diagonally into 
$\fmm'$. 
By the low dimensional accident, $\asl(2,\bR)$ can veiw as $\aso(2,1)$
by the adjoint action and Killing form(trace form). 

Notice that  by classical invarient theory (When $n>m$, $S(M_{n,m})^{O(n)}=S(M_{n,m})^{SO(n)}$ since determinate representation on occur),
 $S(\fmm)^{\asl(2,\bR)}=S(\fmm)^{O(3)}$ is generated by second degree element.
 
Using the $\asl(2,\bR)$-module isomorphism 
\[
\Gr U(\fmm) \cong S(\fmm)
\]
and standard induction on dimension, we know the $U(\fmm)^{\asl(2,\bR)}$
is generated by  the inverse image of the above generator.

Precisely, $\fgg'=\asl(2,\bR)$ has basis $\Set{e,f,h}$, 
$\fmm = \fgg'_1\oplus \fgg'_2$ (indexis denote the copy of $\fgg'$.).
Moreover $\inn{e}{f} = 1$, $\inn{h}{h}=2$.  
Hence, $S(\fmm)^{\fgg'}$ is generated by 
\begin{align*}
r_{11} = & \frac{1}{2} h_1^2 + f_1e_1 + e_1f_1 \\
r_{22} = & \frac{1}{2} h_2^2 + f_2e_2 + e_2f_2 \\
r_{12} = & \frac{1}{2} h_1h_2 + f_1e_2 + e_1f_2  
\end{align*}
Let
\[
r = \frac{1}{2} (h_1+h_2)^2 + (f_1+f_2)(e_1+e_2) + (e_1+e_2)(f_1+f_2).
\]
Then 
\[
r_{12} = \frac{r - r_{11} -r_{22}}{2}.
\]
Hence $U(\fmm)^{\asl(2,\bR)}$ is generated by $r_{11},r_{22}$ and $r$. But this 
three element just the Casimer element in $\fgg'_1, \fgg_2'$ and $\fgg'$. 

Now we know, the Casimers in othorgnal groups will corresponed to Casimers 
in $\asl(2,\bC)$ under $\omega$ 
from the realization of oscillator representation (c.f. \cite{Howe1979Opq}).
In particular, 
\[
C = C' + \frac{(p+q)(p+q-4)}{16}.
\]

So we know $\omega(\cuu(\fgg)^{O(p)\times O(q)})$
is generated by image of $C,C_1,C_2$ the Casimer element in $O(n,\bC)$,
$O(p)$ and $O(q)$. Note that Casimer is independence with real form!
In fact, Casimer is determined by the Killing form, change real form of complex
Lie algebra, just means change the basis of computation, the result will be 
the same (c.f. Prop 5.24, \cite{Knapp1996Lie}).
Hence $C, C_1, C_2$ also the Casimer of $O(r,p-r)$, $O(q)$ and $O(r,n-r)$.

Now apply the knowledge about $\cuu(\fgg)^K$ action on 
the derived functor module, in fact for the chain complex level.


Now we choose two real form $W_{r,n-r}$ and $W_{p,q}$ of $W$, we have
$G_1, G_2$ as real subgroup of $\OO(n,\bC)$ and $G'_1=G'_2$ as real subgroup 
of $SL(2,\bC)$. $\fgg$ is the complexification of both $\fgg_1$ and $\fgg_2$. 

Now we can idenfity $\cuu(\fgg)^{O(r,p-r)\times O(q)}$ and
$\cuu(\fgg)^{O(p)\times O(q)}$ as same subalgebra in $\cuu(\fgg)$.

\subsection{Lifting of trivial character}
Now we consider lifting of trivial charicter of $SL(2,\bR)$.
There is a proposition from \cite{ZhuHuang1997}
\begin{prop}[\cite{ZhuHuang1997}, Proposition~2.1]
Suppose that $(G,G')$ is in the stable range with $G'$ the small member. 
Then for any unitary character $\chi\in \widehat{\tilde{G}}$, 
The Howe quotion is already irreducible. Thus $\rho(\chi)=\theta{\chi}$
is irreducible and unitary.
\end{prop} 

First we need a result from Howe \cite{Howe1979Opq}, for any $p,q\geq 0$
\[
\omega_{p,q}\cong \int_{s} \tau_s\otimes \rho_s ds \bigoplus_m
\left( \theta_{p,q}(D_m)\otimes D_m\right).
\]
Here the integration is respect to the principle series. $D_m$ is holomprhic 
discrete series or anti-holomorphic series. 

Suppose we have $(G=O(r,s),G'=SL(2,\bR))$ dual pair. 
Now we decomposite the Howe quotion $\rho_{r,s}(1)$ into $O(r,p-r)\otimes O(q)$ representations ($p>r$, $r+s = p+q=n$ and $r,s,p,q$ are all even numbers).
Note that as $O(r,p-r)\times O(q)\times SL(2,\bR)\times SL(2,\bR)$ module
\begin{equation}\label{eq:decomega1}
\begin{split}
\omega =& \omega_{r,p-r}\otimes \omega_{0,q}\\
=& \left(\bigoplus_a \theta_{r,p-r}(D_{a+p/2})\otimes D_{a+p/2} 
\oplus \int_s\tau_s\otimes \rho_s ds \right) \otimes 
\left( \bigoplus_b \chh_b 
\otimes \overline{D_{b+q/2}}\right)\\ 
=& \left(\bigoplus_{a, b} \theta_{r,p-r}(D_{a+p/2})\otimes \chh_b \otimes D_{a+p/2}
\otimes \overline{D_{b+q/2}}\right) 
\oplus\left(\bigoplus_b \int_s\tau_s\otimes \rho_s ds \otimes \chh_b 
\otimes \overline{D_{b+q/2}}\right)
\end{split}
\end{equation}

Since $\rho_{r,s}(1)$ is the maximal quotient of $\omega$ such that $SL(2,\bR)$ act trivially. 
We have 
\[
\rho_{r,s}(1) \cong \bigoplus_{a+p/2=b+q/2} \theta_{r,p-r}(D_{a+p/2})\otimes \chh_b \otimes D_{a+p/2}
\otimes \overline{D_{b+q/2}} 
\]

Let
\[
\begin{split}
\fgg=& \aso(p+q)\\
K_1 =& O(r)\times O(s)\\
K_2 =& O(p)\times O(q)\\
G_1 = &O(r,s)\\
G_2 = &O(p,q)\\
M = & K_1\cap K_2 = O(r)\times O(p-r)\times O(q)\\
K = & G_1\cap K_2 = O(r,p-r)\times O(q) < G_1
\end{split}
\]
\[
\Gamma = \Gamma_{\aso(p+q), O(r)\times O(p-r)\times O(q)}^{\aso(p+q), O(p)\times O(q)}
\]
So as $K_2$-representation  
\[
\Gamma^{p-r}\rho_{r,s}(1) \cong \bigoplus_{\substack{a,b \geq 0\\a+p/2=b+q/2}}
\chh_a\otimes \chh_b
\]
which exactly is same as the $K_2$-type of $\rho_{p,q}(1)$.

Now pick any $K_2$-type $\chh_a\otimes \chh_b$ of $\Gamma^{p-r}\rho_{r,s}(1)$, the $U(\fgg)^{K_2}$
action on it is given by the $U(\fgg)^{K}$ 
action on $\theta_{r,p-r}(D_{a+p/2})\otimes \chh_b$. But this is given by
the action $C,C',C''$ action which is determinde by corresponding 
action on the trivial representation defined by 
$D_{a+p/2}\otimes \overline{D_{b+q/2}}\to \bC$.
So this action is same as which action gets from $\theta_{p,q}(1)$.  

Hence we get 
\begin{prop}
\[
\Gamma^{p-r}\theta_{r,s}(1)\cong \theta_{p,q}(1)
\]
\end{prop}

\subsection{Lifting of discrete series}
Now consider the transfer of the lifting of discrete series. 
From Howe\cite{Howe1979Opq}, the howe quotation 
$\rho_{r,s}(D_l)$ is already irreducible, 
hence $\theta_{r,s}(D_l) = \rho_{r,s}(D_l)$.

By see-saw pair, we have a $(O(r,p-r)\times O(q)$ map
\[
V \triangleq \theta_{r,s}(D_l) \to \theta_{r,p-r}(D_{a+p/2})\otimes \chh_b\triangleq W
\]
where
$ a\geq b+\frac{q-p}{2}+l$.

But infact from the decomposition (\ref{eq:decomega1}) $W$
 is a direct summand of $V$.
Hence 
$\Gamma^jW$ occur in $\Gamma^jV$ when it is non-zero, in particular 
$\chh_a\otimes\chh_b$ occur in $\Gamma^{p-r}\rho_{r,s}(D_l)$. 
Moreover, the action of $U(\fgg)^{K_2}$ action on this $K_2$-type is given by
$U(\fgg)^K$ action on $\theta_{r,p-r}(D_{a+p/2}) \otimes\chh_b$ determined by 
the action of $C,C',C''$. Hence we get
\begin{prop}
$\Gamma^{p-r}\theta_{r,s}(D_l)$
has a subquotation isomorphic to $\theta_{p,q}(D_l)$.
\end{prop}


\section{Transfer of the Lifting of character}
Let us consider a simple case. Suppose $(G,G')$ are reductive dual pair. 
$V = \theta_{G'\to G}(\rho)$ is the theta lifting of character $\rho$ of $G'$.

The following result is crucial, and is known by \cite{Shimura1990} 
or \cite{Zhu2003} :
\begin{thm}
Let $\rho$ be a one-dimensional representation of $K$, where $K$ is a maximal 
compact subgroup of some classical group $G$. Let $\cjj' = \Ker\rho_u|_{\cuu(\fkk)}$ where $\rho_u$ is the antihomorphism of $\cuu(\ftt)$ by extending 
$\rho_u(X) = -\rho(X)$.
Then the natural map
\[
\czz(\fgg) \to \cuu(\fgg)^K/(\cuu(\fgg)\cjj\cap\cuu(\fgg)^K)
\]
is surjective.
Hence the $\cuu(\fgg)^K$ action on the $\rho$ isotypic component $V_\rho$ 
of any $\cuu(\fgg)$ module $V$ is determind by the $\cuu(\fgg)$ action 
on $V_\rho$. In particular, if $V$ has infinitesimal character, 
$V$ is determined by its infinitesimal character.
\end{thm}


In fact, we need a silghtly different version of above theorem:
If $G$ is classical, 
\[
\czz(\fgg)\to \cuu(\fgg)^K/(\cjj\cuu(\fgg)\cap \cuu(\fgg)^K)
\]
is surjective, where $\cjj = \Ker(\rho)|_{\cuu(\fkk)}$.

We can see this easly by taking the automorphism on $\cuu(\fgg$ by 
extanding $X \mapsto -X$. It clearly perserves $\cuu(\fgg)^K$ and
$\czz(\fgg)$ and send $\cuu(\fgg)\cjj'$ to $\cjj\cuu(\fgg)$. 

Now assume $\omega$ is a $(\cuu(\fgg),K)$-modoule, $K$ not necessary compact. 
there is finite $K$ representation $\rho$ such that 
$\dim \Hom_K(\omega, \rho)=1$ (Should be consider as $(\fkk, K)$ 
module level homomorphism). Fix a intertwing map $T\in \Hom_K(\omega, \rho)$.
For any $x\in \cuu(\fgg)^K$, we have $T\circ x = c(x)T$ for some $c(x)\in \bC$. 

Hence we can difine the $\cuu(\fgg)^K$ action on $\rho$ as the appendix. 
The above theorem tell as the $\cuu(\fgg)^K$ action on $\rho$ is determind by 
the $\czz(\fgg)$ action. Since
for any $B\in \cuu(\fgg)^K$, we can find $Z\in \czz(\fgg)$ such that 
$B-Z \in \cjj\cuu(\fgg)$.
Now $Bv = Zv \mod \Ker(T)$ since $Q \in \cjj$ implies $Q v \in \Ker(T)$
for any $v\in \omega$.

Now let us consider the $\cug^K$ action after apply the derived functor. 
As above setting, $\Hom_K(V,W)$ is one-dimensional 
Let $\Gamma^i =\left(\Gamma_{(\fkk,K_0)}^{(\fkk,K)}\right)^i$ and . 
$\Gamma^iW$ is non zero irreducible. 
We ahve $\Gamma^iT \colon \Gamma^iV \to \Gamma^iW$.
$\cug^K$ act on $\Gamma^iW$ is given by the $\cug^K$ 
on $V$ by the functoriality of the derived functor:
\[
\Gamma^i T\circ \Gamma^i x = \Gamma^i T\circ x = c(x)\Gamma^i T \quad
\forall x\in \cug^K.
\]

By observing the realization of the oscillator representation, 
for any $x \in \cug^K$ we can find $x'\in \cuu(\cmm')^{G'}$. such that
$\omega(x) = \omega(x')$ and this choise is independent of the realization 
of oscillator representation. Now choose $z' \in \czz(\cmm')$ such that
$x'-z' \in \cjj\cuu(\cmm')$. Then again by the realization, we can find 
$z\in \czz(\fkk)$ independent of realization, 
such that $\omega(z) = \omega(z')$.
Hence $x$ act on $\Gamma^iW$ by $z$ 
act on $\Gamma^iW$.
Now suppose that for another dual pair $(\tG, \tG')$, character $\trho$
of $\tG'$ such that $\cjj$ annilate $\trho$ and $\theta(\trho)$ has 
$K$-type $\Gamma^i W$. Then the $x$ action on this $K$-type by $z$ actoin. 
Since the infinitesimal character determind the $K$-type, we conclude that 
$\Gamma^i\theta(\rho)$ has a composition component $\theta(\trho)$.

   

\appendix
\section{Some inveriant theory}
Let $V$ be a finit dimensional  vector space 
with non-degenerate bilinear form $\inn{}{}$, 
$G$ is the subgroup of $GL(V)$ preseve $\inn{}{}$. 

Now identify $V$ as $V^*$ as $G$-module by this form:
\[
\begin{matrix}
V &\leftarrow& V^*\\
v & \mapsto & \inn{\cdot}{v}
\end{matrix}
\] 

Let $V_k = V\oplus \cdots \oplus V$, the $k$ copy of $V$. 
The classical inveriant theory tells us 
\[
\cpp(V_k) 
\]
and it generated by $r_{ij}(v) = \inn{v}{v}_{ij}= \inn{v_i}{v_j}$,
i.e. the paring of $i,j$-th 
coordinate in $V_k$.

Now suppose $X_1, \cdots X_n$ is a basis of $V$ and 
$X^j$ are dual basis respectively by $\inn{X_i}{X^j} = \delta_{ij}$.
Let $X_{i,s}$ and $X^{j,s}$ are basis in $V_k$. 

Use following isomprhism
\[
\cpp(V_k) \cong \css(V_k^*) \cong \css(V_k),
\]
Now
\[
\begin{split}
\inn{\cdot}{\cdot}_{s,t} = \inn{\sum \inn{\cdot}{X^{i,s}}X_{i,s}}{\inn{\cdot}{X^{j,t}}X_{j,t}} = \sum_{i,j} \inn{X_{i}}{X_{j}}\inn{\cdot}{X^{i,s}}\inn{\cdot}{X^{j,t}}
=\sum_{i,j} \inn{X_i}{X_j} X^{i,s}X^{j,t}
\end{split}
\]

\section{Representations of $U(n)$}
In the Fork module of oscillator representatoin,  
$U(n)$ act on the ploynomail algebra. 


Give the norm on $P(\bC^n)$ by
\[
\inn{f}{g} = C_n\int_\bC^nf(z)\overline{g(z)}e^{-\abs{z}^2}dzd\overline{z}
\]

\def\dd{\mathrm{d}}

In fact, 
\[
\begin{split}
 &\int_\bC z^a \overline{z^b} e^{-\abs{z}^2}dz\overline{dz} \\
=& -2i \int_\bC z^a \overline{z^b} 2i\,\dd x\,\dd y\\
=& -2i\int_{0}^\infty r^{a+b+1}e^{-r^2}\dd r \int_0^{2\pi} e^{i\theta(a-b)}d\theta\\
=& \begin{cases}
0 & a\neq b\\
\begin{split}
 &-4\pi i\int_0^\infty r^{2a+1} e^{-r^2} \dd r \\
=& -2\pi i \int_0^\infty s^a e^{-s}\dd s \\
=& -2\pi i a! 
\end{split}
& a=b 
\end{cases}
\end{split}
\]

Hence take $C_n = (i/2\pi)^n$, we have orthonrmal basis 
\[
\set{e_\alpha = \frac{x^\alpha}{\sqrt{\alpha!}}=\frac{\prod_i x^{\alpha_i}}{\sqrt{\prod_i \alpha_i!}}}
\]
under this inner product.

The representation of $U(n)$ on $P(\bC^n)$ is given by  $k f(z)  =f(kz)$. 
Fixing a $u\in V$, the map 
\begin{align*}
\phi\colon V &\to  \sH(K)\\
v & \mapsto (F(k)=\inn{kv}{u})
\end{align*}
gives a imbedding of $V$ in to $\sH(K)$, 
where $\sH(K)$ view as $K$ representation under right translation.

We'd like to reconstruct $v$ by $F(k)$ by using Schur's orthogonality relation. 
suppose $v = \sum_{\alpha} \inn{v}{e_\alpha} e_\alpha$, 
\[
\begin{split}
\int_K F(k) \overline{\inn{ke_\alpha}{u}} \dd k \\
= &\int_K  \inn{kv}{u}\overline{\inn{ke_\alpha}{u}}\dd k\\
= &\inn{v}{e_\alpha}\inn{u}{u}/\dim V.
\end{split}
\]

Hence we have 
\[
v = \int_K F(k) K(k) \dd k
\]
where 
\[
K(k) = \sum_{\alpha} \frac{\overline{\inn{ke_\alpha}{u}}\dim V}{\inn{u}{u}} e_\alpha 
\]
is a kernel function. 

Back to previous section. $u_l = c_l z_p^l/\sqrt{l!}$, $V_l$ is the space of homogeneous polynomial of degree $l$.
$\dim V_l  = \binom{n+l-1}{l}$.
   
So 
\[
f(z) = \int_K F(k) K(k,z) \dd k 
\]
where 
\[
\begin{split}
K(k,z) =& \sum_{\alpha} \binom{n+l-1}{l} 
\overline{\inn{k z^\alpha/\sqrt{\alpha!}}{c_lz_p^l}}\frac{1}{c_l^2l!}
\frac{z^\alpha}{\sqrt{\alpha!}}\\
= & \sum_\alpha \binom{n+l-1}{l}\frac{1}{c_l \alpha!}
\prod_j k_{pj}^{\alpha_j} z^\alpha
\end{split}
\]
for $k = (k_{ij}) \in U(n)$. 

\section{See-saw pair and reciprocity law}
Suppose we have see-saw pair
\[
\xymatrix{G \ar@{-}[dr] & H'\\
H\ar@{-}[ur] & G'
}
\]
where $H < G$, $H'> G'$. 

For representation $\sigma$ of $G'$ which can be realized as a quotient of
the Weil representation, by Howe we have
\[
 \omega/\cnn_\sigma= \rho(\sigma)\otimes \sigma
\]
for some representation $\rho(\sigma)$ of $G$ as $G\times G'$ representation.

\[
\cnn_\sigma = \bigcap_{\substack{\cnn\subset \omega\\ \omega/\cnn\simeq \sigma}}\cnn
\]

For any irreducible representation $\tau$ of $H$.
\[
\begin{split}
&\Hom_{H}(\rho(\sigma), \tau)\\
= & \Hom_{H\times G'}(\rho(\sigma)\otimes \sigma, \tau\otimes \sigma)
=  \Hom_{H\times G'}(\omega/\cnn_\sigma, \tau\otimes \sigma)\\
= & \Hom_{H\times G'}(\omega, \tau\otimes \sigma)\\
= & \Hom_{H\times G'}(\omega/\cnn_\tau, \tau\otimes \sigma)
= \Hom_{H\times G'}(\tau\otimes \rho(\tau), \tau\otimes \sigma)\\
= & \Hom_{G'}(\rho(\tau), \sigma).
\end{split}
\]
The third equality above valid by the definition of $\cnn_\sigma$. 


Now suppose that $\dim \Hom_{H}(\rho(\sigma),\tau)=1$,
Hence $T\circ x = c(x) T$ for any $x\in \Hom_{H}(\rho(\sigma),\rho(\sigma))$, where $c(x)\in \bC$.
Since $\rho(\sigma)$ is $\fgg$ module, define a $\cuu(\fgg)^H$ action 
on $\tau$ by $\rho(\sigma)$ as following.
Choose any $0 \neq T\in \Hom_{H}(\rho(\sigma),\tau)$,
\[
x v = T(x w) \quad  \forall x\in \cuu(\fgg)^H, v=T(w) 
\text{ where } w\in \rho(\sigma).
\] 

Since $\rho$ is irreducible, and $T\circ x = c(x) T$, 
the action is well defined, independent of $T$. 
Similarly, define $\cuu(\fhh')^{G'}$ action on $\sigma$.

Consider following commutative digram:
\[
\xymatrix{
\omega \ar[r]^{S_2} \ar[d]_{S_1}\ar[dr]^{Q} & \tau\otimes \rho(\tau)\ar[d]^{T_2}\\
\rho(\sigma)\otimes \sigma\ar[r]_{T_1} & \tau\otimes \sigma 
}
\]
Here $T_i = T\otimes \id$.

Note that $\omega(\cuu(\fgg)^H) = \omega(\cuu(\fhh')^{G'})$,
for $x\in \cuu(\fgg)^H$, 
there is a $y\in \cuu(\fhh')^{G'}$.
such that $\omega(x) = \omega(y)$.


Now for any $v\otimes w\in \tau\otimes \sigma$, there is $a\in \omega$ such
that $Q(a) = v\otimes w$.
Hence, 
\[
(x \cdot v)\otimes w 
= T_1 (x S_1 (a))
= T_1 \circ S_1 (\omega(x) a)
= T_2 \circ S_2 (\omega(y) a)
= v\otimes (y\cdot w) 
\]

\subsection{Characters of $\cuu(\fgg)^K$}
Assume $K$ is a maximal compact subgroup of $G$.

Suppose $\chi\colon \cuu(\fgg)^K \to \bC$ is a character of $\cuu(\fgg)^K$. 
Note that $\cuu(\fgg)^K$ has a natural grading. 


\begin{lemma}
Suppose $M$ is a graded $K$ module, with $K$ act reductively, 
Then $\Gr M^K = (\Gr M)^k$
\end{lemma}
\proof Note that, $M^i = M_i \oplus M^{i-1}$ since $K$ act reductively.
Hence, $\Gr M^K = \bigoplus M_i^K  = \bigoplus (M^i/M^{i-1})^K= (\Gr M)^K$. \qed

\section{skew duality}
This is from \cite{Howe1995perspective} Section~4.1. 
I give the detils as doing an  exercise.

\subsection{Skew $(\GL_n,\GL_m)$- duality}
\begin{thm}[\cite{Howe1995perspective} Theorem~4.1.1]
We have the $\GL_n\times \GL_m$-module decomposition
\[
\bigwedge ( \bC^n\otimes \bC^m)\cong \sum_{D}\rho^D_n\otimes \rho^{D^t}_m.
\]
In this sum, $D$ ranges over all diagrams with at most $n$ rows 
and with rows lengths at most $m$. The diagram $D^t$ is the transpose 
of $D$: the rows of $D$ are the columns of $D^t$, and vice versa.
\end{thm} 
\proof
We use the Schur Duality. Let $S_k$ be the symmetric group of degree $n$ and
$m$ respectively.
Let $U= \bC^n$,$V=\bC^m$.
Note that 
\[
\begin{split}
\bigwedge^k \bC^n\otimes \bC^m 
\cong&  \left(\left(U\otimes V\right)^{\otimes^k}\right)^{sign,S_k}\\
\cong&  \left(U^{\otimes^k} \otimes V^{\otimes^k}\right)^{\sign,\triangle S_k}\\
\cong& \left((\sum_{D}\rho_U^D\otimes \sigma^D)
  \otimes (\sum_{E}\rho_V^E\otimes\sigma^E)\right)^{\sign,\triangle S_k}\\
\cong& \sum_{D,E} \rho_U^D\otimes \rho_V^E 
\otimes (\sigma^D\otimes \sigma^E)^{\sign,\triangle S_k}\\
\cong& \sum_{D} \rho_U^D\otimes \rho_V^{D^t}
\end{split}
\]
The summation of $D,E$ is over young diagram of size $k$ at most depth $m$, 
The last equation valide since 
\[
\Hom_{S_k}(\sigma^E\otimes \sigma^D, \sign) 
\cong \Hom_{S_k}(\sigma^E, \sign\otimes (\sigma^D)^*)
\cong \Hom_{S_k}(\sigma^E,\sign\otimes \sigma^D)
\cong \Hom_{S_k}(\sigma^E,\sigma^{D^t})
\]
for $\sigma^D\otimes \sign \cong \sigma^{D^t}$, see \cite{Fulton1991}~Exercise~4.4 or \cite{James1981}~2.1.8.  

\section{Fock model}
Assume $W_0$ is a symplectic space over $\bR$.
Fix a complete polarization $W_0 = \cxx_0\oplus \cyy_0$.  
$e_1,\cdots, e_n$ is basis of $\cxx_0$, and $f_1,\cdots, f_n$ is dual basis 
of $e_j$'s in $\cyy_0$, i.e. $\inn{e_\alpha}{e_\beta}=\delta_{\alpha\beta}$.

Now choose a chartor of $\bR$, i.e. $\phi(x) = \lambda x$. 
For $\phi$ be unitary, choose $\lambda \in i\bR$. Usually, we set $\lambda=i$. 
Let
\[
\Omega(W) = T(W) / \left\langle v\otimes w - w\otimes v - \phi(\inn{v}{w})\right\rangle.
\]
Now $\Omega(W)$ has a filtraction from $T(W)$, i.e. 
$\Omega^k(W)$ is the image of $\bigoplus_{j\leq k}T^j(W)$.
It is easy to see, 
\[
[\Omega^i(W),\Omega^j(W)] \subset \Omega^{i+j-2}(W).
\]
We have short exact sequence of Lie Algebras:
\begin{align}
0\to \Omega^0(W) \to &\Omega^1(W) \to W \to 0\\
0\to \Omega^1(W) \to &\Omega^2(W) \to \sp(W)\to 0 
\end{align}
More over $\Omega^2(W) \cong \sp(W) \oplus \fhh$, 
where $\fhh\cong \Omega^1(W)$ is a hisenberg Lie algebra.
The identify cation of $\Omega^2(W)/\Omega^1(W)$ is given by
consider the $\Omega^2(W)$ action of $\Omega^1(W)/\Omega^0(W)\cong W$. 

Let $W=W_0\otimes_\bR\bC$ be the complexification of $W$. 
Define $J$ as multiply by ``$i$'' on $W_0$. And define unitary structure
\[
U_J(x,y) = \inn{Jx}{y} + i\inn{x}{y}
\]
Consdier $W = X\oplus Y$, where $X$ and $Y$ are $\pm i$ eigen-space of $J$. 
Now it's easy to see that $U_J$ structure is 
invarient under $X\otimes Y$ action($v\in X, w\in Y, x,y\in W$):
\[
\begin{split}
U_J(vw\cdot x,y) =& \inn{J(vw\cdot x)}{y} + i\inn{vw\cdot x}{y}\\
=&\inn{-i\lambda\inn{v}{x} w}{y}
+\inn{i\lambda\inn{w}{x} v}{y}
 - i \inn{x}{vw\cdot y}\\
=& \inn{-i\lambda\inn{w}{y}v}{x}
+ \inn{i\lambda \inn{v}{y}w}{x}
-i\inn{x}{vw\cdot y}\\
=& \inn{x}{i\lambda{w}{y}v}
+ \inn{x}{-i\lambda{v}{y}w}
-i\inn{x}{vw\cdot y}\\
=& \inn{x}{J vw\cdot y} -i\inn{x}{vw\cdot y}\\
=& -\inn{Jx}{vw\cdot y}  -i\inn{x}{vw\cdot y}= -U_J(x,vw\cdot y).
\end{split}
\]

Note that, $\sp(n,\bC)\cong S^2(W)$ and the real form $\sp(n,\bR)$ is the 
Lie subalgebra preserve the real symplectic space $W_0$. 



Algebra $\Omega(W)$ act on $\Omega(W)/\Omega(W)X \cong \bC[z_1, \cdots,z_n]$ by differential operator:
\begin{align*}
e_i &\leftrightarrow \lambda \frac{\partial}{\partial z_i}\\
f_i &\leftrightarrow z_i\\
\end{align*}


\subsection{relation of differential operators between Schordinger and Fock model}
Choose a polaization of real symplectic space $W_0$, $\Set{e,f}$ be bases

Now let 
\begin{align*}
\omega(e) &=  x\\
\omega(f) &= i\partial_ x = i\frac{\partial}{\partial x}
\end{align*}
Then it satisfy $\omega(\inn{e,f})=\omega(1)=i$, 
Let 
\begin{align*}
z =& e + if\\
\bz =& e -if 
\end{align*}
Then $\inn{z}{\bz}=2i$, hence $\omega(\inn{z}{\bz}) = -2$. 
So let
\begin{align*}
\omega(z) &= \sqrt{2} z \\
\omega(\bz) &=  \sqrt{2} \partial_z
\end{align*}
Hence 
\begin{align*}
x = \omega(e) 
&= \omega(\frac{1}{2}(z+\bz)) = \frac{1}{\sqrt{2}}\left(\partial_z + z\right)\\
\partial_x = \omega(-if) &= \omega(\frac{1}{2}(-z + \bz)) = 
\frac{1}{2}\left(\partial_z -z \right)
\end{align*}

\section{$\cuu(\fgg)^K$}
\subsubsection{Case $\fgg=\aso(n,\bC)$} 
Consider $O(n,\bC)$ act on $V=\bC^n$ with orthorgnal basis $e_1, \cdots, e_n$, 
Choese real subspace $V_{p,q} = \sspan{e_1, \cdots, e_p, ie_{p+1},\cdots, ie_n}$.
Let $O(p,q)$ be the subgroup of $O(n,\bC)$ preserve $V_{p,q}$. We also have corresponding Lie algebra embedding. 
Choose $K = O(p,\bC)\otimes O(q,\bC)$. Now consider a real form of 
$K_0 = K\cap O(r,s) \cong O(r,p-r)\times O(q)$ (assume $r<p$).

\begin{lemma} Under above notation.
\[\cuu(\fgg)^{K} = \cuu(\fgg)^{K_0},\]
hence, $\cuu(\fgg)^{K_0}$ is independent with the real form $O(r,s)$. 
\end{lemma}
\proof Note that we can have $K=K'Z$, where $K' = SO(n,\bC)$ is the identity component of
 $K$, $Z$ is some $2$ element subgroup of $K_0$ 
for example $\Set{\diag{\pm 1,1, \cdots, 1}}$. Since $K_0\subset K$, 
it is clear that $\cuu(\fgg)^K\subset \cuu(\fgg)^{K_0}$.
On the other hand, $K$ act on $\cuu(\fgg)$ reductively
and all irreducible subrepresentation of $K$ in $\cuu(\fgg)$ 
are finite dimensional by the unitary trick. Hence, 
$\cuu(\fgg)^K$ is the set of $\aso(n,\bC)$ and $Z$ invarint.
Since $\aso(n,\bC)$ action is the competicification of $\aso(p,q)$ action.
We have, $\cuu(\fgg)^{K_0} \subset (\cuu(\fgg)^{\aso(p,q)})^Z = \cuu(\fgg)^K$.
\qed

\subsection{$\cuu(\fgg)^K$ action}
We do not assume $K$ is connected here.  
Asseme there is a $(\fgg,K)$ module $(\pi,V)$.
We would like to study the $U(\fgg)^K$ action on $(\pi,V)$. 

It is obvious that, 
$\pi(x)$ commute with $\pi(K)$ for all $x \in U(\fgg)^K$, i.e. 
\[
\pi(U(\fgg)^K) \subset \pi(U(\fgg))^{\pi(K)}
\]

%A theorem of Harish-Chandra or Lepowsky calims $U(\fgg)^K$
%module structure on a paticuler $K$-type in $V$, determine $V$. 

Clearly $U(\fgg)^K$ fixed the $K$-isotypic component for any module.
But 
\[
\caa =\Set{x\in U(\fgg)}
\]

Now consider things for dual pairs.
Consider the following see-saw pair, 
\[
\seesawpair{G}{M}{K}{G'}
\]
where $K$ is a maximal compact subgroup of $G$. 

Suppose the image of universal enveloping algebra of $G,G',M$
in Howe's setting \cite{Howe1989Rem} is $\Gamma,\Gamma', \Pi$.

Then
\[
\pi(U(\fgg)^K) = \Gamma^K = \Gamma \cap \Pi = \Pi^{G'} =\Pi^{\Gamma'}
\]
where first equality valid because $K$ act reductively on $U(\fgg)$,
last equality valid only if $G'$ is connected.



\section{Borel-Weil-Bott-Kostant Theorem}
Reference \cite{KnappVogan1995} Chapter~IV, Section~11 and
\cite{EnrightWallach1980}.

\begin{thm}[\cite{KnappVogan1995} Corollary~4.160]
Let $G$ be a compact connected Lie group, $\ftt_0$ be the maximal abelian 
subalgebra, let $\fbb=\ftt\oplus \fnn$ be the Borel subalgebra corresponding 
to a positive system $\Delta^+(\fgg,\ftt)$, let $\fqq=\fll\oplus \fuu$ 
be a parabolic subalgebra with $\ftt\subset \fbb \subset \fqq$, and let
$L$ be the corresponding Levi subgroup. Let $Z$ be a irreducible $(\fll,L)$
module with $\Delta^+(\fll,\ftt)$ highest weight $\lambda$. 
\begin{enumerate}[(a)]
\item If $\lambda+\delta$ is orthogonal to some root, 
then $\Pi_j(P_{\fqq,L}^{\fgg,L}(\cff_{\fll,L}^{\fqq,L}(Z)))$ and 
$\Gamma^j(I_{\overline{\fqq},L}^{\fgg,L}(\cff_{\fll,L}^{\overline{\fqq},L}(Z)))$
are both $0$ for all $j\geq 0$.
\item If $\lambda+\sigma$ is orthogonal to no root, put
\[
q=\#\Set{\alpha\in \Delta^+(\fgg,\ftt)|\inn{\lambda+\delta}{\alpha}<0}.
\]
There exists a unique element $w\in W$ such that 
$w^{-1}(\lambda+\delta)$ is $\Delta^+(\fgg,\ftt)$ dominant;
this $w$ has lenght $q$ and is in  
$W^1=\Set{w\in W|w^{-1} \Delta^+(\fll,\ftt)\subset \Delta^+(\fgg,\ftt)}$.
\[
\begin{split}
&\Pi_j(P_{\fqq,L}^{\fgg,L}(\cff_{\fll,L}^{\fqq,L}(Z)))=
\Gamma^j(I_{\overline{\fqq},L}^{\fgg,L}(\cff_{\fll,L}^{\overline{\fqq},L}(Z)))\\
=&\text{irreducible representation of $G$ have highest weight }
w^{-1}(\lambda+\delta)-\delta.
\end{split}
\]
\end{enumerate}
\end{thm}

Note that $P_{\fqq,L}^{\fgg,L}(\cff_{\fll,L}^{\fqq,L}(Z) 
= \cuu(\fgg)\otimes_{\cuu{\fqq}}Z$.
Use duality theorem below
\begin{thm}[Zuckerman duality \cite{KnappVogan1995}~Corollary~3.7]
Let $(\fgg,L)\hookrightarrow (\fgg,K)$ be an inclusion of pairs, 
and let $m=\dim(\fkk/\fll)$. For $0\leq j\leq m$ and $V$ in 
$\ccc(\fgg,L)$, there are $\ccc(\fgg,K)$ isomorphisms
\begin{align*}
\Pi_j(V\otimes_\bC (\bigwedge^m(\fkk/\fll))^*)&\cong \Gamma^{m-j}(V)\\
(\Pi_j(V\otimes_\bC \bigwedge^m(\fkk/\fll)))^c &\cong \Pi_{m-j}(V^c)\\
(\Gamma^j(V\otimes_\bC\bigwedge^m(\fkk/\fll)))^c& \cong \Gamma^{m-j}(V)
\end{align*} 
nautural in $V$.
\end{thm}
 
we get
\begin{thm}[\cite{Enright1985}~Proposition~6.2]
If $\fgg$ act on $(\bigwedge^m(\fgg/\fll))^*$ trivially,
$Z$ is highest weight $\lambda$ respect to $\Delta^+(\fll,\ftt)$. 
\[
\begin{split}
 &\Gamma^{m-i}(\cuu(\fgg)\otimes_{\cuu(\fqq)}Z)\\
\cong& \Pi_i(\cuu(\fgg)\otimes_{\cuu(\fqq)}Z \otimes (\bigwedge^m(\fgg/\fll))^*)\\
\cong& \Pi_i(\cuu(\fgg)\otimes_{\cuu(\fqq)}Z)\\
\cong& \begin{cases}
\text{highest weight } w^{-1}(\lambda+\delta)-\delta & \text{if } i=\text{length of $w$, $\lambda+\delta$ non-singular}\\
0 & \text{otherwise}  
\end{cases}
\end{split}
\]
\end{thm}
When $\fgg$ is semisimple Lie algebra, automatically be a $(\bigwedge^m(\fgg/\fll))^*$ is trivial representation of $\fgg$,
since it is $1$-dimension, and the action vanish on $[\fgg,\fgg]=\fgg$.

\section{Unitary Lowest Weight Module}
Reference \cite{Enright1983}.
\subsection{$\aso(2,2n-2)$}
Let 
\[
\lambda_z = (-2n+3, 0,\cdots, 0) + z(1, 0, \cdots, 0)
= (-2n+3+z,0, \cdots, 0)
\]
Now
\[
N(\lambda_z)=\cuu(\fgg)\otimes_{\cuu(\fbb)}\bC_{\lambda_z}
\]
has a unique irreducible quotation $L(\lambda_z)$ which is unitarizable iff $z\leq n-1$ or $z=2n-3$ 
\cite{Enright1983}~Theorem~10.4. 

In fact, when $z=2n-3$ it just the trivial representation. 
If $z= n-1$, 
\[
\lambda_{z} = (-(n-2), 0, \cdots, 0)
\]
is reducible, with the irreducible quotation be the Ladder representation. 
When $z < n-1$ and is integer,
we find it is a $\aso(2,2n-2)$ subrepresentation of $\theta(D_l)$ for some $l\geq 2$.  

\section{Invariant differential operator}

Let $V$ be a finite dimensional representation of $K$ where $K$ is a maximal
compact subgroup of Lie group $G$. 
Define
\[
C^\infty(G,V)  = V \otimes C^\infty(G)
\]
be the set of $C^\infty$ function on $G$ with value in $V$. 
Consider $C^\infty(\rho)$ be the subset of $C^\infty(G,V)$
such that $f(kx) = \rho(k)f(x)$ for all $k\in K$ 
if and only if $f\in C^\infty(\rho)$.
Then $G$ act on $C^\infty(\rho)$ by right translation.

Let $\cuu(\fgg)$ act on right of $C^\infty(G,V)$ as right invariant differential
operators. That is, for $B\in \fgg$,
\[
(f\cdot X)(g) = \left.\frac{d}{dt}\right|_{t=0} f(\exp(tX)g).
\]
One check that this induce a $\cuu(\fgg)$ action on $C^\infty(G,V)$.
In paticular for $X\in \fkk$, 
\[
(f\cdot X)(g) = \ddt(\exp(tX)g) = \ddt \rho(\exp (tX))f(g) = \rho(X) f(g)
\quad
\forall f\in C^\infty(\rho)
\]
Now $\cuu(\fkk)$ act on $C^\infty(\rho)$ by the usual action on $V$, 
since (by induction)
\[
\begin{split}
((f\cdot u)\cdot Y)(g) =& \ddt(f\cdot u)(\exp(tY)g)
 = \ddt \rho(u)f(\exp(tY)g) \\
=& \rho(u)\ddt f(\exp(tY)g)
= \rho(u)\rho(Y) f(g) \\
=& \rho(uY)f(g) 
\quad 
\forall u\in \cuu^r(\fkk), Y\in \fkk, f\in C^\infty(\rho)
\end{split}
\]

If $B= X_1\cdots, X_n\in \fgg$,
\[
\begin{split}
(f \Ad(k)B)(g) = & (f\Ad(k)(X_1\cdots X_n))(g)\\
=& \left.\frac{d}{dt_n}\right|_{t_n=0} (f\Ad(k)(X_1\cdots X_{n-1}))(\exp(t_n\Ad(k)X_n)g)\\
=& \left.\frac{d}{dt_n}\right|_{t_n=0}
\left.\frac{d}{dt_{n-1}}\right|_{t_{n-1}=0}
 (f\Ad(k)(X_1\cdots X_{n-2}))(\exp(t_{n-1}\Ad(k)X_{n-1})\exp(t_{n}\Ad(k)X_{n})g)\\
=&\left.\frac{d}{dt_n}\right|_{t_n=0} \cdots \left.\frac{d}{dt_{1}}\right|_{t_{1}=0}
f(\exp(t_1\Ad(k)X_1)\cdots \exp(t_n\Ad(k)X_n) g)\\
=&\rho(k)
\left.\frac{d}{dt_n}\right|_{t_n=0} \cdots \left.\frac{d}{dt_{1}}\right|_{t_{1}=0}
f(\exp(t_1X_1)\cdots \exp(t_nX_n) k^{-1}g)\\
=& \rho(k)(fB)(k^{-1}g)
\end{split}
\]

So we have
\[
[f(\Ad(k)B)](g)= \rho(k)(fB)(k^{-1}g) \quad \forall B\in \cuu{\fgg}, k\in K, 
g\in G. 
\]
If $B\in \cug^K$, we have $(fB)(g)  = \rho(k)f(B)(k^{-1}g)$.
Hence $\cug^K\in D(\rho)$, where $D(\rho)$ is the subset of $\cuu(G)$ preserve 
space $C^\infty(\rho)$. Let the image of $D(\rho)$ in the
set of differential opeartors  be $\cdd(\rho)$.

Now we have
\begin{prop}[Harish-Chandra, Shimura~\cite{Shimura1990} Prop~2.1]
\begin{enumerate}[(1)] 
\item An element $T$ of $\cuu(\fgg)$ annihilates 
$C^{\infty}(\rho)$ if and only if $T \in \cjj\cuu(\fgg)$, where
$\cjj = \Ker(d\rho)$.
\item $D(\rho) = \cjj\cug + \cug^K$.
\item The natural map of $D(\rho)$ onto $\cdd(\rho)$ gives an 
isomorphism of $\cug^K/(\cug^K\cap \cjj\cug)$ onto $\cdd(\rho)$. 
\end{enumerate}
\end{prop}
\proof
\begin{enumerate}[(1)]
Clearly $\cjj\cug$ annihilates $C^\infty(\rho)$.
Since $G = K \exp(\fpp_0)$, 
fix a basis $\Set{X_1, \cdots, X_n}$ of $\fpp_0$
we can identify $C^\infty(\rho)$ 
with $C^\infty(\bR^n,V)$ by 
$f_h(k\exp(\sum_{i=1}^n t_i X_i) = \rho(k)h(t_1,\cdots, t_n)$.


\end{enumerate}


\section{Derived functor for disconnected groups}



\bibliography{bib/reppapers}{}
\bibliographystyle{alpha}

\end{document}

