\documentclass{amsart}
\usepackage{pdfsync}

\usepackage[left=2cm,top=2cm,right=3cm]{geometry}

\usepackage{hyperref}
\usepackage{amssymb}
\usepackage{amsmath, amsthm}
\usepackage{braket}
\usepackage{paralist}
\usepackage{eufrak}
\usepackage[all,cmtip]{xy}
\usepackage{diagxy}

\def\sign{{\rm sign}}
\def\Ker{{\rm Ker}}
\def\Im{{\rm Im}}
\def\Hom{{\rm Hom}}
\def\End{{\rm End}}
\def\Mat{{\rm Mat}}
\def\Ind{{\rm Ind}}
\def\bR{{\mathbb{R}}}
\def\bN{{\mathbb{N}}}
\def\bZ{{\mathbb{Z}}}
\def\bC{{\mathbb{C}}}
\def\bQ{{\mathbb{Q}}}
\def\bB{{\mathbb{B}}}
\def\bA{{\mathbb{A}}}
\def\bs{{\mathbf{s}}}
\def\bd{{\mathbf{d}}}
\def\bT{{\mathbb{T}}}
\def\bt{{\mathbf{t}}}
\def\br{{\mathbf{r}}}
\def\vv{{\vec{v}}}
\def\vw{{\vec{w}}}
\def\vx{{\vec{x}}}
\def\vy{{\vec{y}}}
\def\v0{{\vec{0}}}
\def\ol{\overline}
\def\sspan{\rm{span}}
\def\sl2{{\mathfrak{sl}(2)}}
\def\slc{{\mathfrak{sl}(2,\bC)}}
\def\sp{{\mathfrak{sp}}}
\def\gg{{\mathfrak{g}}}
\def\kk{{\mathfrak{k}}}
\def\pp{{\mathfrak{p}}}
\def\qq{{\mathfrak{q}}}
\def\sP{\mathcal{P}}
\def\sH{\mathcal{H}}
\def\sU{\mathcal{U}}
\def\sC{\mathcal{C}}
\def\sS{\mathcal{S}}
\def\Stab{{\rm Stab}}
\def\ad{{\rm ad}}
\def\Ad{{\rm Ad}}
\def\id{{\rm id}}
\def\sgn{{\rm sgn}}
\def\gcd{{\rm gcd}}
\def\inn#1#2{\left\langle{#1},{#2}\right\rangle}
\def\abs#1{\left|{#1}\right|}
\def\norm#1{{\left\|{#1}\right\|}}
\def\Sp{{\rm Sp}}
\def\SL{{\rm SL}}
\def\O{{\rm O}}
\def\SO{{\rm SO}}
\def\det{{\rm det}}

\def\tG{{\widetilde{G}}}
\def\tK{{\widetilde{K}}}
\def\tM{{\widetilde{M}}}
\def\tJ{{\widetilde{J}}}
\def\trho{{\widetilde{\rho}}}
\def\tsigma{{\widetilde{\sigma}}}
\def\tDelta{\widetilde{\Delta}}
\def\barZ{{\overline{Z}}}
\def\bz{{\overline{z}}}

\def\Ad{\mathrm{Ad}}
\def\fgl{\mathfrak{gl}}
\def\fsl{\mathfrak{sl}}
\def\fso{\mathfrak{so}}
\def\diag#1{\mathrm{diag}(#1)}
\def\lww{\mathcal{W}}
\def\lxx{\mathcal{X}}
\def\lyy{\mathcal{Y}}
\def\fbb{\mathfrak{b}}
\def\fhh{\mathfrak{h}}
\def\fnn{\mathfrak{n}}
\def\fuu{\mathfrak{u}}
\def\fll{\mathfrak{l}}
\def\foo{\mathfrak{o}}
\def\fpp{\mathfrak{p}}
\def\fqq{\mathfrak{q}}
\def\ftt{\mathfrak{t}}
\def\fgg{\mathfrak{g}}
\def\fkk{\mathfrak{k}}
\def\caa{\mathcal{A}}
\def\ccc{\mathcal{C}}
\def\cdd{\mathcal{D}}
\def\chh{\mathcal{H}}
\def\cjj{\mathcal{J}}
\def\crr{\mathcal{R}}
\def\css{\mathcal{S}}
\def\cpp{\mathcal{P}}
\def\cww{\mathcal{W}}
\def\cnn{\mathcal{N}}
\def\cuu{\mathcal{U}}
\def\cxx{\mathcal{X}}
\def\cyy{\mathcal{Y}}
\def\cff{\mathcal{F}}
\def\czz{\mathcal{Z}}
\def\cug{\cuu(\fgg)}
\def\fmm{\mathfrak{m}}
\def\GL{\mathrm{GL}}
\def\SO{\mathrm{SO}}
\def\OO{\mathrm{O}}
\def\rad{\mathrm{rad}}
\def\SL{\mathrm{SL}}
\def\tr{\mathrm{tr}}
\def\Mat{\mathrm{Mat}}
\def\U{\mathrm{U}}
\def\Gr{\mathrm{Gr\,}}
\def\tU{{\widetilde{U}}}
\def\pz#1{\partial z_{#1}}
\def\ddt{\left.\frac{d}{dt}\right|_{t=0}}
\def\csigma\
\def\Ind{{\rm Ind}}


\def\cmm#1#2{\left[{#1},{#2}\right]}
\def\acmm#1#2{\left\{{#1},{#2}\right\}}

\def\seesawpair#1#2#3#4{
\xymatrix{
{#1} \ar@{-}[dr]& {#2} \\
{#3} \ar@{-}[ur] & {#4}}
}

\def\real{{\rm Re\,}}
\def\imag{{\rm Im\,}}

\def\rmk{{\bf Remark:}}


\begin{document}
\title{research plan}
\author{Ma Jia jun}
\maketitle

\section{Introduction}
In this note, I will present two topics in the representation theory of
classical groups which I am currently  working on. I will discuses
some results I obtained and then  give some 
directions I will consider in the next several months. Before that I
will also  provide some
backgrounds at the beginning of  each section..


\section{Transfer of $K$-type, $\cuu(\fgg)^K$ action and  theta correspondence}
The explicit constructions of representations are important questions
in the representation theory of real reductive groups. By
Harish-Chandra's theory, we can forces on admissible $(\fgg,K)$-module
where $\fgg$ is the Lie algebra of a real reductive group $G$ and $K$ is
a maximal compact subgroup of $G$. 

Now I introduce a sophistic
construction of $(\fgg,K)$-module by derived functor as following.  Let
$\sC(\fgg,K)$ be the category of admissible $(\fgg, K)$-modules.  Let
$M$ be a subgroup of $K$.  Zuckerman functor
$\Gamma_{\fgg,M}^{\fgg,K}$ is the functor from $\sC(\fgg,M)$ to $
\sC(\fgg,K)$ by taking the subspace of $K$ finite vectors in a $(\fgg,
M)$-module.  This functor is only left exact in general.  So we can
consider the derived functor of Zuckerman functor and it can be shown
that there is a natural $(\fgg,K)$-module structure on the $i$-th
derived functor module 
$\left(\Gamma_{\fgg,M}^{\fgg,K}\right)^iW$ for any $W\in \sC(\fgg,K)$.


For classical groups, there is an alternative way to construct
representations which is 
via (local) theta correspondence.  
Fix a symplectic space $W$, define
\emph{Heisenberg group} $\mathrm{H} = W\oplus \bR$. For any fixed nontrivial
unitary character $\psi$ of $\bR$, there is a unqiue (up to
isomorphism) irreducible unitary represntation of $\mathrm{H}$ with
central character $\psi$.  This representation can be extended to a
representation $\omega$ of $\widetilde{\Sp(W)}\ltimes \mathrm{H}$,
where $\widetilde{\Sp(W)}$ is a nontrivial double covering of the
symplectic group $\Sp(W)$.
A pair of subgroups of $\Sp(W)$ is called  a reductive dual pair in
$\Sp(W)$ if $G$and $G'$  are subgroups of $\Sp(W)$  acting on $W$ reductively
and they are mutual centralizer in $\Sp(W)$. 
Denote $\tG$ for the preimage of $G$ in the covering group
$\widetilde{\Sp(W)}$ for a subgroup $G$ of $\Sp$.  Let
$\sR(\tG,\omega)$ be the set of infinitesimally isomorphism classes of irreducible
smooth representations of $\tG$ which can be realized as a quotation
of $\omega$.  Howe~\cite{Howe1989Tran} proved an one-one correspondence
$\theta$ between $\sR(\tG,\omega)$ and $\sR(\tG',\omega)$.  This give
a construction of representation $\theta_{G',G}(\sigma)$ of $\tG$ if one have a representation $\sigma$
of $\tG'$. 

Over last two decades, fruitful results about above two constructions has obtained. However there still a lot of open questions.
My studies is initialed by following example.
\begin{eg}\label{eg:wz2004}
Let $p\leq s$, $q\leq r$ and
$p+q=m\leq n=r+s$. 
$\theta^{p,q}$ is the theta lifting of  from $\O(p,q)$ to $\Sp(2n,\bR)$.
Wallach and Zhu \cite{WallachZhu2004} have shown that as $U(n)$-representation
\[
\Gamma_{p,q}(\theta^{0,m}(1)) \cong \theta^{p,q}(1^{\xi,\eta}),
\]
where $\xi \equiv r-q, \eta\equiv s-p \pmod{2}$, $1$ is the trivial representation of $\O(p,q)$, $1^{\xi,\eta}$ are certain characters of $\O(p,q)$
determined by $(\xi, \eta)$ and
$\Gamma_{p,q}\theta^{0,m}(1)$ is certain submodule of 
\[
\left(\Gamma_{\sp(2n),U(r)\times U(s)}^{\sp(2n), U(r+s)}\right)^{rs-(r-q)(s-p)}\theta^{0,m}(1)
\]
They guessed in their paper that the above isomorphism is also a $(\sp(2n), U(n))$-module isomorphism.

The techniques  I described later will confirm above guess.
\end{eg}

Let me state the theorem I obtained for this question.
\begin{thm}\label{thm:scaler}
  Suppose that 
  $G_1$ and $G_2$ are two real forms of a reductive Lie algebra $\fgg$.
  There are Cartan involutions of $G_i$ commute to each other(c.f. \cite{WallachZhu2004}).
  $G_1'$ and $G_2'$ are two real forms of a reductive Lie algebra $\fgg'$.
  $(G_1,G_1')$ and $(G_2,G_2')$ form reductive dual pairs.
  Let
  $\fgg=\fkk_i \oplus \fpp_i$ under above Cartan involution of $G_i$.
  $K_i$ be the maximal compact subgroup of $G_i$ with Lie algebre $\fkk_i$.
 
  Let $\Gamma=\Gamma_{\fgg,K_1\cap K_2}^{\fgg, K_2}$ be the Zuckerman functor
  and $\Theta_{G'_i,G_i}(\sigma_i)$ be corresponding the Howe-quotient.
  Suppose 
  \begin{itemize}
    \item $\sigma_i$ are characters of $G_i'$, where $i=1,2$. 
  \item  $\Ker(\sigma_1|_{\cuu(\fgg')}) = \Ker(\sigma_2|_{\cuu(\fgg')}) \subset \cuu(\fgg')$.
  \item $\tau_1$ is a $(\fkk_2,K_1\cap K_2)$-module such that
    $\dim \Hom_{(\fkk_2,K_1\cap K_2)}(\Theta(\sigma_1),\tau_1) = 1$.
  \item $\tau_2$ is a $K_2$-module occure in the
    image of map $\Gamma^i \Theta(\sigma_1) \to \Gamma^i \tau$ 
    such that 
    $\dim \Hom_{K_2}(\Theta(\sigma_2),\tau_2) =1$.
  \end{itemize}
  Then $\Gamma^i \Theta(\sigma_1)$ and 
  $\Theta(\sigma_2)$ has a isomrophic irreducible subquotient 
  with common $K_2$-type $\tau_2$.
\end{thm}

\begin{rmk}
It is clear that we can apply above theorem to give positive answer of Example~\ref{eg:wz2004}.
One can find more examples in Enright et al.'s paper  \cite{Enright1985}
though they are not presented in that way. 
\end{rmk}

The key observation is following lemma:
\begin{lemma}\label{lem:ugk}
  Let $(G,G')$ and $(H,M)$ are two dual pairs in $\Sp(W)$ If $H<G$ and
  $G'<M$, we say they form a see-saw pair.  Let $\omega$ be a
  oscillator representation of $\widetilde{\Sp(W)}$, then
  \[
  \omega(\cuu(\fgg)^H) = \omega(\cuu(\fhh')^{G'}).
  \]
  Hence for any $x\in \cuu(\fgg)^H$ we can choose $y\in
  \cuu(\fhh')^{G'}$ such that $\omega(x) = \omega(y)$.  Moreover this
  choice is independent of real form.
\end{lemma}
Combine this lemma and following well-know result of
Helgason~\cite{Shimura1990} one can prove Theorem~\ref{thm:scaler} 
by comparing the $\cuu(\fgg)^{K_2}$ action on $\tau_2$ isotypic component of 
$\Gamma^i\Theta(\sigma_1)$ and $\Theta(\sigma_2)$.
\begin{lemma}\label{lemma:scalerk}
  Suppose that $\fgg$ be the complexification of 
  a classical Lie algebra.
  Consder a pair of groups $(G,H)$ where
  $H$ is a subgroup $G$ meet all the component of $G$.
  Moreover there is a involution, such that $\fgg=Lie(G)_\bC$ decomposite into 
  $\fgg = \fkk+\fpp$ where 
  $\fkk=Lie(H)_\bC$.
  
  Let $\sigma$ be a one-dimensional representation of $H$.
  Let $\cjj = \Ker(\sigma|_{\cuu(\fkk)})$.
  Then the following map is surjective:
  \[
  \czz(\fgg) \to \cuu(\fgg)^H/(\cjj\cuu(\fgg)\cap\cuu(\fgg)^H)
  \]
\end{lemma}

By same argument. I also can obtain following ``meta-theorem'':
\begin{thm}[meta-theorem]\label{thm:meta}
  Suppose that 
  $G_1$ and $G_2$ are two real forms of a reductive Lie algebra $\fgg$.
  There are Cartan involutions $\tau_i$ of $G_i$ commute to each other(c.f. \cite{WallachZhu2004}).
   $(G_1,G')$ and $(G_2,G')$ form reductive dual pairs.
  Let
  $\fgg=\fkk_i \oplus \fpp_i$ under involution $\tau_i$.
  $K_i$ be the maximal compact subgroup of $G_i$ with Lie algebre $\fkk_i$.
  $H$ be the maximal subgroup of $G$ with Lie algebra $\fkk_2$.
  Suppose $(G_2,G', K_2,M)$ and $(G_1,G',H,M)$ form see-saw pairs. 
  Let $\mu$ be a irreducible representation of $K_2$, $L(\mu)$ be the lifting of $\mu$ to $M$ which is a lowest weight module~\cite{Howe1989Rem}.
  Let $\sigma$ be a irreducible representation of $G'$.
  Assume:
\begin{enumerate}[(i)]
\item $\sigma$ occur in  the theta correspondence of both dual pair $(G_i,G')$;
\item $\Gamma^r\Theta_{M,H}(L(\mu))\cong \mu$ where $\Gamma=\Gamma_{\fkk_2,K_1\cap K_2}^{\fkk,K_2}$ and $\Theta_{M,H}$ is the Howe quotient lifting from $M$ to $H$ (we use similar notation later). 
\end{enumerate}
Then
$\Gamma^r\Theta_{G', G_1}(\sigma)$ has a subquotient isomorphic to a subquotient of 
$\Theta_{G', G_2}(\sigma)$ with same $K_2$ type $\mu$.
\end{thm}

One can apply above ``meta-theorem'' to prove following theorem given in~\cite{LokeMaTang2011}. 
Note the prove of that theorem I claimed here is different with that in above paper.
\begin{eg}[Theorem~1.2~\cite{LokeMaTang2011}] \label{eg:loke} Suppose
  $(p,n, r + r')$ such that $n,r+r'\geq2p$ and $\max{n,r+r'}> 2p$.
  Let $\tau$ be a representation of $\O(r')$ and $\theta^{n,r}(\tau) =
  \theta_{\Sp(2p,\bR),\O(n,r)}(\theta_{O(r'),\Sp(2p,\bR)}(\tau))$ be
  the successive theta lifting of $\tau$. Let $0 < t< r$.

  If $2p \leq r+r'-t$, then
\[
\Gamma^{pt}(\theta_p^{n,r}(\tau)) =
  \theta_p^{n+t,r-t}(\tau)
\]
as $(\fso(n+r,\bC), \O(n+r)\times\O(r-t))$-modules where $\Gamma = \Gamma_{\fso(n+r,\bC),\O(n)\times \O(r)\times \O(r-t)}^{\fso(n+r,\bC),\O(n+r)\times \O(r-t)}$.
\end{eg}

{\bf Further direction:}
As far as I know, Lemma~\ref{lem:ugk} is never applied to study the representation of classical group before. So it deserve further study.  

In the point of view of Lemma~\ref{lem:ugk} and
Theorem~\ref{thm:meta}, there should be more examples other
than~Example~\ref{eg:wz2004} and Example~\ref{eg:loke}, at least, this
should give the answer of the transfer of $K$-type of the theta
lifting of lowest weight module. One main direction for my further
study is to find more interesting examples can apply
Theorem~\ref{thm:meta}.
% The main difficult my be the occurrence of representations in theta
% correspondence.

Another direction is using Lemma~\ref{lem:ugk} to study the relation between 
cohomological induction and local theta correspondence. 


\section{Associative cycle of theta lifted module}
The following results I presented is a joint work with Loke and Tan~\cite{lokematan2011b}.

Let $V$ be admissible $(\fgg,K)$-module, $\fgg = \fkk\oplus \fgg$
under the Cartan decomposition.  One can assign a element in the group
$\bZ$-coefficient combination of $K$-orbit $\fpp$ called the
associative cycle of $V$. Associative cycle is a important invariant
of representation.  We refer to \cite{Vogan1989Var} for a more
discussion of the associative cycle.

Some calculation of associative cycle of a particular $(\fgg,K)$-module
has been down by calculating the corresponding Hilbert polynomial.

We find a ``geometric'' way to compute the associative cycle for
$\theta^{n,r}(\tau)$(c.f. Example~\ref{eg:loke}).

We retain the notion of Example~\ref{eg:loke} to stat the following theorem:
\begin{thm}[Corrollary~6.5~\cite{lokematan2011b}] \label{thm:lokema2011}
Suppose $r\geq p$, then the associated cycle of $\theta^{n,r}(\tau)$ is 
$(\dim T)[\overline{\sO}]$ where $\sO$ is the nilpotent orbit $O(n)\times O(r)$ orbit lifting from the associative
orbit of $\theta_{\O(r'),\Sp(2p,\bR)}(\tau)$~\cite{NishyamaZhu2004} 
and $T= \tau$ if $p>r'$, $T=\tau^{\O(n-p)}$ if $r'\geq p$.  
\end{thm}
Note that the above theorem recovers the Case I in Theorem~4.7  \cite{NishyamaZhu2004}.

The proof of above theorem based on a canonical realization of the
graded module of $V$ as a vector bundle on the orbit. This realization
also suggests for some  $n,r,r'$,  $\theta^{n,r}(\mathbf{1})$ is
actually ``unipotent''  in the sense of Conjecture~12.1 in
\cite{Vogan1989Var}. The main requirement of Conjecture~12.1 is the
following $K$-module isomorphism:
 \[
\theta^{n,r}(\mathbf{1})\cong \Ind_{K(\lambda)}^K\chi
\]
where $\lambda$ is an element in the orbit $\sO$, $K(\lambda)$ is its
stabilizer and 
$\chi$ is an  admissible representations of $K(\lambda)$.
It is not easy to check in general.
These will be examples of ``unipotent'' representation attached to
nilpotent orbit of height~3. 



{\bf Further directions:} It should be no difficulty to generalize Theorem~\ref{thm:lokema2011}
to other cases of lifting of lowest weight module. One direction I would like to consider
is to give a canonical construction of vector bundle associated to
the graded module of $\theta(\sigma)$ from known admissible data for $\sigma$.

Another direction is to relate these results with the space of
Whittiker functional of $\theta(\sigma)$. A reasonable consideration
is to extend the result of irreducible highest weight modules in~\cite{Yamashita2001cayley}.

\section{Conclusion}
Form above discussion, I may get some interesting result following the plan.

\bibliography{bib/reppapers}{}
\bibliographystyle{alpha}



\end{document}
