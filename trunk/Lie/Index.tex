\documentclass[11pt]{amsart}
\usepackage[margin=2cm]{geometry}
\usepackage{amssymb}
\usepackage{graphicx}
\usepackage{braket}
\usepackage{paralist}

\newtheorem{Thm}{Theorem}
\newtheorem{Prop}[Thm]{Proposition}
\newtheorem{Cor}[Thm]{Corollary}
\newtheorem{Lem}[Thm]{Lemma}
\newtheorem{Def}[Thm]{Definition}
\newtheorem{Rmk}[Thm]{Remark}


\def\cA{{\mathcal{A}}}
\def\cAg{{\mathcal{A}_G}}
\def\cL{{\mathcal{L}}}
\def\cN{{\mathcal{N}}}
\def\cLE{\cL^\cE}
\def\cE{{\mathcal{E}}}
\def\cS{{\mathcal{S}}}
\def\bC{{\mathbb{C}}}
\def\bR{{\mathbb{R}}}
\def\bZ{{\mathbb{Z}}}
\def\bA{{\mathbb{A}}}
\def\bAg{{\mathbb{A}_\mathfrak{g}}}
\def\Fg{{F_{\fgg}}}
\def\fgg{{\mathfrak{g}}}
\def\ftt{{\mathfrak{t}}}
\def\frr{{\mathfrak{r}}}
\def\fnn{{\mathfrak{n}}}
\def\fso{{\mathfrak{so}}}
\def\bZ{{\mathbb{Z}}}
\def\dg{{d_{\fgg}}}
\def\End{{\mathrm{End}}}
\def\Str{{\mathrm{Str}}}
\def\kw#1{{\bf \em #1}}
\def\ch{\mathrm{ch}}
\def\Ah{{\hat{A}}}
\def\chg{\ch_\fgg}
\def\Ahg{\Ah_\fgg}
\def\ddt{\left.\frac{d}{dt}\right|_{t=0}}
\def\sproof{{\bf sketch of the proof:}}
\def\inn#1#2{\left<{#1},{#2}\right>}
\def\brk#1{\left<{#1}\right>}
\def\sign{\mathrm{sign\,}}
\def\sspan{\mathrm{span}}
\def\ind{\mathrm{ind}}
\def\Tr{\mathrm{Tr}}
\def\Mr{M^{\gamma}}

\title{Equivariant Index Theorem}
\author{Ma Jia Jun, HT071232M}


\begin{document}
\maketitle

This almost a notes on the book Heat Kernels and Dirac Operators\cite{berline2004heat}. The example can be found in \cite{vergne1982representations}.

\section{Notation and useful facts}
Let $G$ be a Lie group and $\fgg$ is the Lie algebra of $G$.
\begin{Def}
We say that $\pi:\cE\to M$ is a  $G$-equivariant fiber bundle if $G$ act smoothly on left both on $\cE$ and $M$ such that 
\[
\gamma \circ \pi = \pi \circ \gamma \quad \forall \gamma\in G.
\]
When $\cE$ is a vector bundle, we require in addition that the $G$
action on fiber, $\gamma^\cE:\cE_x \to \cE_{\gamma\cdot x}$,  is linear.

Now $G$ act on the space of sections $\Gamma(M,\cE)$ by 
\[
(\gamma\cdot s)(x) = \gamma^\cE\cdot s(\gamma^{-1}\cdot x) \quad
\forall x\in M, s\in \Gamma(M,\cE). 
\]
For $X \in \fgg$, we denote $\cL^\cE(X)$ the infinitesimal action of
$\Gamma(M,\cE)$, 
\[
\cL^\cE(X) s =\left.\frac{d}{dt}\right|_{t=0} exp(tX)\cdot s,
\]
which is called the Lie derivative.
\end{Def}
For any vector field $X$ on $M$, it defines a famliy of diffeomorphisms $\phi_t$
on $M$ for sufficient small $t$. This diffeomorphism will induce the
Lie derivative on tensor bundles, denote by $\cL(X)$. 

Now consider the bundle of exterior differentials $\bigwedge
T^*M$, denote $\cA(M)=\Gamma(M,\bigwedge T^*M)$  the space of
differential forms, which is a graded algebra by nature grading
inherited from $\bigwedge T^*M$.
Denote $\cA(M,\cE)=\Gamma(M,\bigwedge T^*M\otimes \cE)$  the space
of $\cE$ valued differential forms. We can take $\cE$ be the trivial
line bundle on $M$, then $\cA(M) \cong \cA(M,\cE)$ as $\cA(M)$ module.  

For any differential form $\alpha$, denote the
left exterior multiplication  by $\alpha$ operator on $\cA(M, \cE)$ by
$\varepsilon(\alpha)$.
For any vector field $X$ on $M$, denote the contraction map induced
from the natural paring between $TM$ and $T^*M$ by 
$\tau(X):\cA^\bullet(M,\cE)\to
\cA^{\bullet-1}(M,\cE)$. Also denote the exterior differential
operator by $d:\cA^\bullet(M,\cE) \to \cA^{\bullet+1}(M,\cE)$. The
  following E.Cartan's homotopy formula is useful:
\[
\cL(X) = d\circ \tau(X) + \tau(X)\circ d.
\]
For any $X\in \fgg$ it induce a vector feild $X_M$ on $M$, we will
abuse the notation still call this vector feild $X$.

We will assume the group $G$ is a compact Lie group and $\cE$ is a
equivariant vector bundle with base space $M$ in the following part
of this notes. Let $n=\dim M$, $l = \dim M/2$.
For any differentail form $\alpha$ let $\alpha_{[n]}$ be the degree
$n$ part of this form. For Clifford algebra $C(V)$ over quadratic vector space,
let $\sigma_V\colon C(V) \to \bigwedge V$ be the symbol maps and 
$c \colon V \to C(V)$ be the natural embedding.

\section{Equivariant Index Theorem}
\label{Sec:fix}
Here we assume $G$ action preserve the orientation and is isometric.
Hence induce a $G$ action on Clifford bundle $C(M)$, which $g$ act as algebra
automorphism. 

\begin{Def}
An \kw{equivariant Clifford module} $\cE$ over $M$ 
is an $G$-equivariant bundle, with a $\bZ_2$-grading, 
a Clifford module structure 
and Hermitian inner product preserved by group action.

The Dirac operator associated to a Clifford connection $\nabla^\cE$ on $\cE$
is given by ($c\colon \cA(M) \to C(M)$ is the qantization map)
\[
D = \sum_i c(dx^i)\nabla^\cE_{\partial_i}\colon \Gamma(M,\cE) \to \Gamma(M,\cE),
\]
which it is odd parity first order diffrential operator.
When $\Delta^\cE$ is $G$-invariant, $G$-action commutes with the Dirac operator.
So $\ker(D)$ becomes a $\bZ_2$-graded representation of $G$.
Define the \kw{equivariant index} to be the virtual charactor of this representation, that is 
\[
\ind_G(\gamma, D) = \Str(\gamma, \ker(D)) 
= \Tr(\gamma,\ker(D^+)) - \Tr(\gamma,\ker(D^-))\quad \forall \gamma\in G.
\]
\end{Def}

Now we have a generalization of the McKean-Singer formula
\begin{Prop}
Let $k_t(\gamma, x)|dx|=\brk{x|\gamma e^{-tD^2}|x}$ be the restriction of the
kernel of the operator $\gamma e^{-tD^2}$ to the 
diagonal. Then for $t>0$, 
\[
\ind_G(\gamma,D) = \Str(\gamma e^{-tD^2} = \int_M \Str(k_t(\gamma,x))|dx|.
\]
\end{Prop}

Observe that $k_t(\gamma,x) = \gamma^\cE_{\gamma^{-1}x}\brk{\gamma^{-1}x|e^{-tD^2}|x}$,
so when $t\to 0$, $k_t$ decay very fast if $\gamma^{-1}x \neq x$ and 
tends to the dirac measure on $x$ if $\gamma^{-1}x=x$.
Therefor we can expect the above integration will defend to integration
over the set fixed point. We will state the equivariant index theorem.

Denote the fixed point set of $\gamma$ by $M^\gamma$. It is a submainforld 
may have several components of different dimension. We have a orthonormal
decomposition
of tangent bundle $TM$ along $M^\gamma$ respect to the Riemannian metric as
\[
TM|_{M^\gamma} \cong T\Mr \oplus \cN
\]
where $\cN$ is the normal bundle along $\Mr$. 
Noticing that Levi-Civita connection preserves the decompositon, so 
induces connections of $T\Mr$ and $\cN$. Let $R^0$ and $R^1$ be the curvature
associated to these connections respectively. $R^0$ is the Riemannian curvature
of $M^\gamma$, 
hence $\Ah$-genus of $M^\gamma$ 
equals to $\det^{1/2}\left(\frac{R^0/2}{\sinh(R^0/2}\right)$.

Since the action of $\gamma$ on $\cE$ along $M^\gamma$ commuts with 
$c(\alpha)$ for $\alpha\in \Gamma(\Mr,T^*\Mr)$. $\gamma$ can view as a 
section $\gamma^\cE$ of $C(\cN^*)\otimes \End_{C(M)}(\cE)$.

Let $F_0^{\cE/\cS}\in \cA(M^\gamma,\End_{C(M)}(\cE))$ be the restriction of the twisteing curvature of
connection $\nabla^\cE$. 
\begin{Def}
Define the \kw{localized ralative Chern character} form
\[
\ch_G(\gamma, \cE/\cS) = \frac{2^{l_1}}{\det^{1/2}(1-\gamma_1)}
\Str_{\cE/\cS}(\sigma_{\cN}(\gamma^\cE)\exp(-F_0^{\cE/\cS})) \in \cA(M^\gamma,\det(\cN)).
\]
Here $\Str_{\cE/\cS}$ is the extension of usual supertrace to 
\[
\Str_{\cE/\cS}\colon \cA(\Mr, \End_{C(M)}(\cE)\otimes \det(\cN))
\to \cA(\Mr,\det(\cN)).
\]
$l_1 = \dim(\cN)$ is a locally constant function on $\Mr$. 
\end{Def}

Let $T_M\colon \Gamma(\Mr, \bigwedge T^*M) \to C^{\infty}(\Mr)$ be the 
Berezin integral. Denote the $\gamma$ action on $\cN$ be $\gamma_1$.

We have following form of equivariant index theorem.
\begin{Thm}[Atiyah-Segal-Singer]
The equivariant index $\ind_G(\gamma,D)$ of an equivariant Dirac operator $D$
associated to an ordinary connection satisfy the following formula:
\begin{equation}\label{eq:fixed}
\int_G(\gamma, D) = i^{-\dim M/2}\int_{\Mr}(2\pi)^{-\dim \Mr/2} T_M
\left(\frac{\Ah(\Mr)\ch_G(\gamma,\cE/\cS)}{\det^{1/2}
(1-\gamma_1\exp(-R^1)}\right).
\end{equation}
\end{Thm}

We are interested in the special case of $\Mr$ is a set of isolated points.

\section{Equivariant Cohomology}

Consdier the space $\bC[\fgg]\otimes \cA(M,\cE)$, which is the space of polynomial maps from $\fgg$ to $\cA(M,\cE)$. View this space as a $G$-space by action
\[
(g\cdot \alpha)(X) = g\cdot (\alpha (g^{-1}\cdot X)) \quad \forall g\in G,
X \in \fgg, \alpha \in \bC[\fgg]\otimes \cA(M,\cE).
\]
Give a $\bZ$-grading of this space by 
\[
\deg(P\otimes \alpha) = 2\deg(P)+\deg(\alpha), 
\quad \forall P\in \bC[\fgg], \alpha \in \cA(M,\cE)
\]

\begin{Def}
Let 
\[
\cA_G(M,\cE) = \left(\bC[\fgg]\otimes \cA(M,\cE)\right)^G
\]
be the $G$-invariant subspace of $\bC[\fgg]\otimes \cA(M,\cE)$, vectors in this space called equivariant differential forms with value in $\cE$.

Define operator $\dg$ on $\bC[\fgg]\otimes \cA(M,\cE)$ by
\[
\dg \alpha(X) = d(\alpha(X)) - \tau(X)(\alpha(X)).
\]
It can be show that $\dg$ increase the degree by one and restrict on 
$\cAg(M)$ we have $\dg^2= 0$. Hence we can define
the equivariant cohomology $H_G^*(M)$ to be the cohomology of complex $\left(\cAg(M),\dg\right)$.
\end{Def}

\begin{Rmk}
By homotopy formula,
\[
(\dg^2 \alpha)(X) = - \cL(X)\alpha(X)\quad \forall \alpha
\in \bC[\fgg]\otimes \cA(M)
\]
The infinitesimal condition to define a $G$-equivariant form is 
\[
\cL(Y)\alpha(X) - \alpha([Y,X]) = 0 \quad \forall X,Y \in \fgg.
\]
Hence $\dg^2 = 0$ on $\cAg(M)$.
\end{Rmk}

\begin{Def}
Suppose a superbundle $\cE = \cE^+\oplus \cE^-$ 
with both even and odd part are $G$-equvariant.
A superconnection on $\cE$ called $G$-invariant if 
\[
[\bA,\cL^\cE(X)]=0.
\]
The equivariant superconnection $\bAg$ is the operator given by 
\[
(\bAg \alpha)(X) = (\bA-\tau(X))(\alpha(X)) \quad \forall  \bC[\fgg]\otimes \cA(M,\cE).
\]
It satisfies \[
\bAg(\alpha\wedge \theta) = \dg \alpha \wedge \theta +
(-1)^{|\alpha|}\alpha\wedge \bAg\theta \quad \forall\alpha \in \cA(M),
\theta \in \cA(M,\cE).
\]
Define the \kw{equivariant curvature} $\Fg$ by
\[
\varepsilon(\Fg(X))\alpha(X) = \bAg^2\alpha(X) + \cL^\cE(X)\alpha(X),
\] 
which can be show is in $\cAg^+(M,\End(\cE))$.
Moreover the \kw{equivariant Bianchi formula} holds:
\[
\bAg\Fg \triangleq [\bAg,\varepsilon(\Fg)] = 0.
\]
Define the \kw{moment} of $X\in \fgg$ by 
\[
\mu(X) = \cLE(X) - [\tau(X),\bA] = \Fg(X) - F \in \cAg^+(M,\End(\cE)),
\]
where $F = \bA^2$ is the usual curvature for superconnection $\bA$.
One example is the Riemennian moment for the tangent bundule with Levi-Civita connection $\nabla$.
That is (by $\nabla$ is torsion free)
\begin{equation}\label{eq:Rm}
\mu^M(X)Y = [X,Y] - \nabla_X Y = -\nabla_Y X
\end{equation}

If $f(z)$ is a polynomial in the indeterminate $z$, we define an
\kw{equivariant characteristic form} by taking supertrace,
i.e. $\Str(f(\Fg)) \in \cAg^+(M)$. In paticular, we have
\kw{equivariant Chern character} 
\[\chg(\bA) = \Str(\exp(-\Fg))
\]
 and
\kw{equivariant $\Ah$-genus} 
\[
\Ahg(\nabla)(X) =
{\det}^{1/2}\left(\frac{\Fg(X)/2}{\sinh(\Fg(X)/2)}\right)
\]
for $G$-equvariant connection $\nabla$.
\end{Def}

\begin{Rmk}
We extend $\cL^\cE(X)$ to act on $\cA(M,\cE)$ by 
\[
\cLE(X)(\alpha\wedge \theta) =\cL(X) \alpha \wedge \theta +\alpha
\wedge \cLE(X)\theta \quad \forall \alpha\in \cA(M), \theta\in \cA(M,\cE).
\]

As in the non-equivariant case, we have $\Str(f(\Fg))$ is
equivariantly closed, i.e. $\dg \Str(f(\Fg)) = 0$ and its equivariant
cohomology class is independent of the choice of the $G$-invariant
superconnection $\bA$.
\end{Rmk}


\subsection{symplectic manifold}
A symplectic manifold is a manifold $M$ with symplectic two form
$\Omega$, where $\Omega\in \cA^2(M)$ such that $d\Omega=0$ and
the bilinear form $\Omega_x(X,Y)$ on $T_xM$ is non-degenerate.
For any $f\in C^\infty(M)$ one can define the \kw{Hamiltonian vector
  field} generated by $f$ by 
\[
df = \tau(H_f)\Omega.
\] 
$G$ acts on $M$ Hamiltonian means that there is a $G$-equivariant linear
map $\mu:\fgg\to C^\infty(M)$ such that 
\[
d\mu(X) = \tau(X) \Omega \quad \forall X\in \fgg.
\]
One define the \kw{symplectic moment map} of the action to be the
$C^\infty$ map $\mu\colon M\to \fgg^*$ given by $(X,\mu(m)) =
\mu(X)(m)$.

Call the volume form 
\[
d\beta=(e^{\Omega/2\pi})_[n] = \frac{\Omega^l}{(2\pi)^ll!}
\] 
on $M$ \kw{Liouville form}.

\subsection{Coadjoint Orbit}\label{Sec:co1}
The most importent example of symplectic manifold with Hamiltonian $G$
action is the Coadjoint Orbit. 
Let $\fgg^*$ be the dual vector space to the vector space of Lie
algebra $\fgg$ of $G$. $G$ act on $\fgg$ by adjoint action and on
$\fgg^*$ by corresponding dual action.

Let $M=M_\lambda$ be an orbit of coadjoint represention of $G$ on
$\fgg^*$ contains $\lambda \in \fgg^*$, i.e $M = G\cdot \lambda\cong
G/G_\lambda$, where $G_\lambda$ is the isotropy subgroup at $\lambda$.

Define the moment map $\mu(X)\colon M \to \bC$ by the evaluation map
$\mu(X)(f) = f(X)$ for $f\in M \subset \fgg^*$.
Define a $2$-form by $\Omega(X_M,Y_M)_f =-
f([X,Y])$. Note that the vector feild generated by $\fgg$ span the
tangent space at every point in $M$, so it is sufficient to define and
check the properties only for vector feilds generated by $\fgg$. It can be show this form is well defined non-degenerate on
$M$.
Moreover 
\[
\begin{split}
d\mu(X)(Y_M)_f  =& \ddt \mu(X)(\exp(-tY)f) =  \ddt(\exp(-tY)f)(X) \\
=&
\ddt f(\exp(tY)\cdot X) = f([Y,X]) = \Omega(X_M,Y_M).
\end{split}
\]
By 
\[
0 = d^2\mu(X) = d\tau(X_M)\Omega = -\tau(X_M) d\Omega
\] ($\cL(X_M)\Omega=0$), $d\Omega=0$.
Hence, $\Omega$ is the symplectic form one $M$ and $G$ act on $M$
Hamiltonian with symplectic moment map $\mu$.



\subsection{The Localization Formula}
By using a equivariant Riemannian metric $(\,,\,)$ on $M$ (which is always exist
in our case since $G$ is compact). 
Let $M_0(X)$ be the set of zero of vector field $X_M$ for $X\in \fgg$.
One can show that 
\begin{Prop}
Let $\alpha$ be an equivariantly closed differential form on $M$. 
Then $\alpha(X)_{[n]}$ $M_0(X)$ is exact outside $M_0(X)$.
\end{Prop}
\sproof 
Define one form $\theta$ by $\theta(\xi) = (X_M, \xi)$. 
Then $d_X^2 \theta  = -\cL(X)\theta  = 0$ and $d_X\theta  = |X|^2 + d\theta$ is
invertable outside $M_0(M)$. 
Let $\beta$ defined by $ d_X\theta \wedge \beta = \theta \wedge \alpha(X)$ outside $M_0(M)$.
Note that $d_X^2 \theta=0$ and $d_X\alpha(X) = 0$, we have
\[
d_X\theta \wedge \alpha(X) = 
d_X(d_X\theta \wedge \beta) = d_X\theta \wedge d_X\beta.
\]
So 
\[
\alpha(X) = d_X(\beta) = d_X\left(\frac{\theta\wedge \alpha(X)}{d_X\theta}\right).
\]
Read the above result of highest degree piece of each side, we have
\[
\alpha(X)_{[n]} = d\left(\frac{\theta\wedge \alpha(X)}{d_X\theta}\right)_{[n-1]}.
\] \qed

From the above result, it is tempting to reduce the integration
$\int_M\alpha(X)$ to a integration over $M_0(X)$, the set of zeroes $X_M$.
%%In general, this will be subtle since $M_0(X)$ may not even be a manifold. 

Let us consider the simplest but important case, $M_0(X)$ is a set of
isolated points.
For any point $p\in M_0(X)$,
$\cL(X)\xi = [X_M,\xi]$ will gives rise to a linear
transformation $L_p$ of $T_pM$ since 
\[\cL(X)(f\xi) = (X(f))\xi + f\cL(X)\xi = f\cL(X)\xi \quad \forall \xi
\in \Gamma(M,TM), f\in C^\infty(M).
\]. This linear transformation is invertible, because if $\xi\in T_pM$ is
annihilated by $L_p$ all the  points on the curve $\exp_p(s\xi)$ will
be fiexed by $\exp(tX)$ which contridict to  $p$ is an isolated point
in $M_0(X)$.
Since $G$ is compact, $L_p\in SO(n)$ by choosing a $G$-invariant
metric on $T_pM$. Now $\det^{1/2} (L_p)$ make sense.

Now we have the Localization formula for isolated zeroes.
  
\begin{Thm}
Let $\alpha$ be an equivariantly closed differential form
  on $M$. Assume $X_M$ has only isolated zeros. Then
\begin{equation}
\label{eq:localization}
\int_M \alpha(X) = (-2\pi)^{\dim(M)/2}\sum_{p\in M_0(X)}\frac{\alpha(X)(p)}{\det^{1/2}(L_p)},
\end{equation}
where $\alpha(X)(p)$ means the value of the function $\alpha(X)_{[0]}$
at point $p\in M$.
\end{Thm}

We also develop the general localization formula by introduce the 
actions and curvature on normal bundle $\cN$ over the zero set $M_0$ 
of $X$ similar to Section~\ref{Sec:fix}. But let me omit it here.

\subsection{The Fourier Transform of Coadjoint Orbits}
We follow the notation in Section~\ref{Sec:co1}.
For any coadjoint orbit $M$, we define the Fourier transform of an orbit
by 
\[
F_M(X) = \int_M e^{i f(X)}d\beta = (2\pi i )^{-l}\int_Me^{i\Omega_\fgg(X)}, 
\]
where $\Omega_\fgg(X) = \mu(X) + \Omega$ is a equivariantly closed form.
%Above may not make sence when $M$ is not sufficiently well-behaved.
%%can be realized as a equivariant curvi 

Now assume $G$ is connected compact Lie group.
Fix a Cartan subalgebra $\ftt$ in $\fgg$ so that $\fgg=\ftt\oplus \frr$ (where $\frr = [\ftt,\fgg]$ is the span of roots of $\ftt$). Let $T = Z_G(\ftt)$ be the corresponding Cartan subgroup. Denote the root system by $\Delta \subset \ftt^* \subset \ftt_\bC^*$ (In fact, all the roots take purly imaginary value on $\ftt$).
Let $W = N_G(T)/T$ be the Weyl group which can be identified with the algebric Weyl group $W(\fgg_\bC,\ftt_\bC)$ generated by simple reflection $s_\alpha$ on $\ftt_\bC$. 
For any root $\alpha\in \Delta$ pick $H_\alpha \in i\ftt$ such that $\inn{\alpha}{H_\alpha}=2$.
%$X_\alpha \in \fgg_\alpha, {X_{\bC}}_{\alpha}\in {\fgg_{\bC}_{-\alpha}$ and $ H%_\alpha\in \ftt_\bC $ such that they form a $\mathfrak{sl}(2,\bC)$ system, i.e.% 
%$[H_\alpha, X_\alpha] = 2X_\alpha, [H_\alpha,X_{-\alpha}] = -2 X_\alpha$ and $[X_{}]$
For given $\lambda\in\ftt^*$,
let 
\begin{equation}\label{eq:positive}
P_\lambda = \Set{\alpha\in \Delta|\inn{\lambda}{iH_\alpha}>0}.
\end{equation}
In particularly, when$\lambda$ is regular, i.e. 
the isotropy subgroup $G(\lambda)$ of $\lambda$ is $T$ 
or equivalently the orbit $M_\lambda$ has maximal possible dimension (we define $X\in \ftt$ ragular in the similar way), $P_\lambda$ will give a positive system of the root system $\Delta$.

\begin{Thm}[Harish-Chandra]
For given $\lambda\in \ftt^*$, let $W_\lambda = \Set{w\in W|w\lambda = \lambda}$
be the subgroup of $W$ stabilize $\lambda$. For regular element $X\in \ftt$, 
\begin{equation}\label{eq:fourier}
F_{M_\lambda}(X) = \sum_{w\in W/W_\lambda}\frac{e^{i\inn{w\lambda}{X}}}{
\prod_{\alpha\in P_\lambda}\inn{w\alpha}{X}}. 
\end{equation}
Especially, when $\lambda$ is regular, the denominator in above formula are 
all same up to sign $\sign(w) = (-1)^{l(w)}$( $l(w)$ is the length of $w$ in the sence to decompose $w$ into product of simple reflection)
\begin{equation}
\label{eq:harish1}
F_{M_\lambda}(X) = \frac{\sum_{w\in W} \sign(w)e^{i\inn{w\lambda}{X}}}{\prod_{\alpha\in P_\lambda}\inn{w\alpha}{X}}
\end{equation}
\end{Thm}
\sproof When $X$ is regular, the zero set of $Z_M$
(or equivlantly the fixed point of $\exp(tX)$ on $M_\lambda$ is exactly
the finite set 
\[
M_0(X) = \Set{w\lambda|w\in W/W_\lambda}.
\]
Let $\frr_{[i\alpha]}$ be the subspace of $\frr$ 
such that $X$ act on it by infinitesimal rotation with angle $i\alpha(X)$.
One can identify $\frr_\lambda = \sspan\Set{\frr_{[i\alpha]}|alpha \in P_\lambda}$ 
with $T_\lambda M$. So $\det^{1/2}(\cL(X)) = \prod_{\alpha\in P_\lambda}(i\alpha(X))$.
Now apply the localization Formula~(\ref{eq:localization}) 
to (\ref{eq:fourier}) one get the result.

\section{Kirillov Formula}
The right hand side of the Equation~(\ref{eq:harish1}) is very similar to 
Weyl Character Formula for irreducible representation of Compact Lie groups.
We will see the relations in soon. 

Let me indroduce necessary notations to state the Kirillove formula first.
Then we will see the Kirillov Character Formula will be a consequence of 
compute the index of a line bundle in two different way.  

Let $R_\fgg = \mu^M+ R \in \cA^2_G(M,\fso(M))$ be the equivariant Riemannian 
curvature where $\mu^M$ is the Riemannian moment map.
The equivariant $\Ah$-genus of the tengent budle is 
\[
\Ah_\fgg(X,M) = {\det}^{1/2}\left(\frac{R_\fgg(X)/2}{\sinh(R_\fgg(X)/2}\right).
\]
The \kw{twisting moment} is 
\[
\mu^{\cE/\cS}(X) = \mu^\cE(X) - \frac{1}{2}\sum_{i<j}(\mu^M(X)e_i,e_j)c(e^i)c(e^j)
\]
where $\Set{e^i|1\leq i\leq n}$ is an orthonormal frame of $T^*M$.

Now we can define the \kw{equivariant twisting curvature} to be
\[
F_\fgg^{\cE/\cS}(X) = \mu^{\cE/\cS}(X) + F^{\cE/\cS}\in \cA^2_G(M,\End_{C(M)}(\cE))
\]
and the \kw{equivariant ralative Chern character} form to be
\[
\ch_\fgg(X, \cE/\cS) = \Str_{\cE/\cS}(\exp(-F_\fgg^{\cE/\cS}(X))).
\]

Now we have 
\begin{Thm}[Kirillov formula]
For $X\in \fgg$ sufficiently close to zero, 
\[
\ind_G(e^{-X},D)=(2\pi i)^{-n/2}\int_M(\Ah_\fgg(X,M)\ch_\fgg(X,\cE/\cS).
\]
\end{Thm}
This theorem can be proved by using the general localization formula.
The approach is to show that the forms in this integral is equal to the 
forms appeared in the fixed point formula~\ref{eq:fixed} of equivariant index theorem.   

\subsection{Character formulas }
Let $T$ be a maximal torus of connected compact Lie group. 
Then $T\cong \ftt/L_T$ as abelian Lie group 
where $L_T = \Set{X\in \ftt|e^X = 1}$ is a lattic in $\ftt$.

Define the dual lattic 
\[
L_T^* = \Set{l\in \ftt^*|l(X) \in 2\pi \bZ \,\,\, \forall X \in L_T}
\subset \ftt^*.
\]
Choose a positive system $P\subset \Delta(\fgg_\bC,\ftt_\bC)$ of roots.
Let $\rho_P = \frac{1}{2}\sum_{\alpha\in P} \alpha$, 
Now let $X_G = i\rho_P + L_T^*$ be the shifted dual lattice.

The representation theory of connected compac Lie group tells us:
\begin{Thm}
The set of irreducible representation of $G$ is one-one correspond to 
the Weyl group $W$ orbit of the regular element of $X_G$.
In particular, 
and dominate respect to $P$ there exits a unique 
finite-dimensional irreducible representation $T_\lambda$ of $G$
(with infinitesimal character $\lambda$).
If $\lambda$ is dominate with respect to $P$ (this can always be down by let
$P=P_\lambda$ see Equation~(\ref{eq:positive})), we have 
\[
\Tr(T_\lambda(e^X)) = \frac{\sum_{w\in W}\sign(w)e^{i\inn{w\lambda}{X}}}{
\prod_{\alpha\in P}(e^{\inn{\alpha}{X}/2} - e^{-\inn{\alpha}{X}/2} }.
\] 
\end{Thm}

Define the analytic $G$-invariant function $j_\fgg^{1/2}$ on $\fgg$, 
which restrict on $\ftt$ is equal to 
\[
j_\fgg^{1/2} (X) 
= \prod_{\alpha\in P}\frac{e^{\inn{\alpha}{X}/2} - e^{-\inn{\alpha}{X}/2}}{\inn{\alpha}{X}}
\]

By comapring the value restrict on $\ftt$ of the bothside of following 
equation(for right handside, which is the Fourier transform of Coadjoint Orbit
$M_\lambda$, c.f. Equation~(\ref{eq:harish1}))
we get the identity by Chevalley's theorem 
for $G$-invariant functions.
\begin{Thm}[Kirillov Character Formula]
Let $\lambda\in X_G$. We have
\[
j_\fgg^{1/2}(X) \Tr(T_\lambda(e^X)) = \int_{M_\lambda}e^{if(x)} d\beta.
\] 
\end{Thm} 

Now we construct a line bundle $\cL_\lambda = G\times_T\bC_\lambda$ over
$M_\lambda \cong G/T$ for any $\lambda\in X_G$.

Kostant shows that $\cL_\lambda$ has a canonical $G$-invariant canonical 
Hermitian connection given by $\nabla_X = \cL(X) - i\mu^\lambda(X)$.
The corresponding equivariant curvature is $i(\mu^\lambda(X)+\Omega_\lambda)$.

Now assume $G$ is simple connected, $i\rho_P\in L^*_T$, so that $X_G = L_T^*$.
We can construct a a Spinor bundle $\cS = G\times_T\bigwedge \fnn$, where 
$\fnn = \bigoplus_{\alpha\in P} \fgg_\alpha$. %%check.

Consider the twisted $G$-invariant Dirac operator 
$D_\lambda$ on $\Gamma(G/T,\cS\otimes \cL_\lambda)$.
Then
\begin{Thm}
The fixed point formula for the equivariant index of $D_\lambda$ is given by 
\[
\ind_G(e^X, D_\lambda) =  \frac{\sum_{w\in W}\sign(w)e^{i\inn{w\lambda}{X}}}{
\prod_{\alpha\in P}(e^{\inn{\alpha}{X}/2} - e^{-\inn{\alpha}{X}/2} }.
\]
The Kirillove formula for $X\in \fgg$ small gives 
\[
\ind_G(e^X, D_\lambda) = \epsilon(P,P_\lambda)\int_{M_\lambda} j_\fgg^{-1/2}(X)
e^{if(x)} d\beta.
\] 
Here $\epsilon(P,P_\lambda)$ is the some sign function which can given by 
$\sign(w)$ where $w$ is the unique element in $W$ such that $wP = P_\lambda$.
\end{Thm}
 We can see the Kirillov Character Formula is a special case of above Theorem.

\section{A example: $G = SU(2)$}
Now let us look at a simple example: $G=SU(2)$.
The Lie algebra 
\[
\fgg  = \mathfrak{su}(2) = 
\Set{\begin{pmatrix}
ix_3 & -x_1+ix_2\\
x_1+ix_2& -ix_3
\end{pmatrix}
}\cong \bR^3.
\] 
Let
\begin{align*}
H =& \begin{pmatrix}
1 & 0\\
0 & -1
\end{pmatrix} &
E =& \begin{pmatrix}
0 & 1\\
0 & 0
\end{pmatrix} &
F =& \begin{pmatrix}
0 & 0\\
1 & 0
\end{pmatrix}
\end{align*}
\begin{align*}
h =& \begin{pmatrix}
i & 0\\
0 & -i
\end{pmatrix} &
e =& \begin{pmatrix}
0 & 1\\
-1 & 0
\end{pmatrix} &
f =& \begin{pmatrix}
0 & i\\
i & 0
\end{pmatrix}
\end{align*}
One check that 
\begin{align*}
[H,E] &= 2E& [H,F] &= -2F& [E,F] &= H\\ 
[h,e] &= 2 f & [h,f] &= -2e &[e,f] &= 2h\\
e &= E -F& f &= iE+iF & h& = iH.
\end{align*}
Now the Killing form $B_K(X,Y) = -\frac{1}{4}\Tr(XY)$.

We identify $\fgg^*$ as $\fgg$ by the from 
$B(X,Y) = -\frac{1}{2}\Tr(XY)$.
Choose $\ftt = \bR h$, so $\ftt_\bC = \bC H$. 
Now the root system $\Delta(\fgg,\ftt) = \Set{\pm 2 H} \subset \fgg_\bC^* $
and the Weyl group act on $\ftt$ by change sign, ie.e $wh = -h$.
The lattic $L_T = \Set{2\pi h} \subset \ftt$, $L_T^* = \bZ h$.
The $G$ act on $\fgg\cong \bR^3$ by rotation, so the $G$ orbits in $\fgg^*$ 
are spheres $x_1^2+x_2^2+x_3^2=r^2$. For $\lambda \in L_T^*$, $M_\lambda$ is the
spheres with integer radius. 




\bibliographystyle{alpha}
\bibliography{reppapers}

\end{document}
