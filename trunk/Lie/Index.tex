\documentclass[11pt]{amsart}
\usepackage[margin=2cm]{geometry}
\usepackage{amssymb}
\usepackage{graphicx}
\usepackage{braket}
\usepackage{paralist}

\newtheorem{Thm}{Theorem}
\newtheorem{Cor}[Thm]{Corollary}
\newtheorem{Lem}[Thm]{Lemma}
\newtheorem{Def}[Thm]{Definition}
\newtheorem{Rmk}[Thm]{Remark}


\def\cA{{\mathcal{A}}}
\def\cAg{{\mathcal{A}_G}}
\def\cL{{\mathcal{L}}}
\def\cE{{\mathcal{E}}}
\def\bC{{\mathbb{C}}}
\def\bA{{\mathbb{A}}}
\def\bAg{{\mathbb{A}_\mathfrak{g}}}
\def\fgg{{\mathfrak{g}}}
\def\bZ{{\mathbb{Z}}}
\def\dg{{d_{\fgg}}}

\title{Index}
\author{Ma Jia Jun, HT071232M}


\begin{document}
\maketitle

\section{Equivariant vector bundle}

$H$ is a Lie group, 

\section{Equivariant Cohomology}
Let $G$ be a Lie group. 
$\cE$ be a equivariant vector bundle. 

Consdier the space $\bC[\fgg]\otimes \cA(M,\cE)$, which is the space of polynomial maps from $\fgg$ to $\cA(M,\cE)$. View this space as a $G$-space by action
\[
(g\cdot \alpha)(X) = g\cdot (\alpha (g^{-1}\cdot X)) \quad \forall g\in G,
X\ in \fgg, \alpha \in \bC[\fgg]\otimes \cA(M,\cE).
\]
Give a $\bZ$-grading of this space by 
\[
\deg(P\otimes \alpha) = 2\deg(P)+\deg(\alpha), 
\quad \forall P\in \bC[\fgg], \alpha \in \cA(M,\cE)
\]

\begin{Def}
Let 
\[
\cA_G(M,\cE) = \left(\bC[\fgg]\otimes \cA(M,\cE)\right)^G
\]
be the $G$-invariant subspace of $\bC[\fgg]\otimes \cA(M,\cE)$, vectors in this space called equivariant differential forms with value in $\cE$.

Define operator $\dg$ on $\bC[\fgg]\otimes \cA(M,\cE)$ by
\[
\dg \alpha(X) = d(\alpha(X)) - \tau(X)(\alpha(X)).
\]
It can be show that $\dg$ increase the degree by one and restrict on 
$\cAg(M,\cE)$ we have $\dg^2= 0$. Hence we can define
$H_G^*(M,\cE)$ be the cohomology of complex $\left(\cAg(M,\cE),\dg\right)$, 
called equivariant cohomology with coffecient in $\cE$. 
\end{Def}

\begin{Rmk}
Some useful formulars, 
\[
(\dg^2 \alpha)(X) = - \cL^\cE(X)\alpha(X)\quad \forall \alpha
\in \bC[\fgg]\otimes \cA(M,\cE)
\]
The infinitesimal condition to define a $G$-equivariant form is 
\[
\cL^\cE(Y)\alpha(X) - \alpha([Y,X]) = 0.
\]
If $\cE$ is the trivial line bundle on $M$ with $G$ act trivially on the fiber, it will give the usual equivariant cohomology.
\end{Rmk}


\begin{Def}
Suppose a superbundle $\cE = \cE^+\oplus \cE^-$ 
with both even and odd part are $G$-equvariant.
A superconnection on $\cE$ called $G$-invariant if 
\[
[\bA,\cL(X)]=0.
\]
The equivariant superconnection $\bAg$ is the operator given by 
\[
(\cAg \alpha)(X) = (\cA-\tau(X))(\alpha(X)) \quad \forall  \bC[\fgg]\otimes \cA(M,\cE).
\]
\end{Def}

\end{document}
