\documentclass[12pt]{article}
\usepackage[margin=2cm]{geometry}
\usepackage{pdfsync}

\usepackage{hyperref}

\usepackage{amssymb}
\usepackage{amsmath, amsthm}
\usepackage{graphicx}
%\usepackage{amscd}
\usepackage{braket}
\usepackage{paralist}
\usepackage{eufrak}
%\usepackage{calrsfs}
%\usepackage[small,nohug,heads=littlevee]{diagrams}
%\diagramstyle[labelstyle=\scriptstyle]
%\usepackage{diagrams}
\usepackage[all,cmtip]{xy}
\usepackage{diagxy}
%\usepackage{pictexwd,dcpic}
%\usepackage{mathrsfs}


\newtheorem{lemma}{Lemma}
\newtheorem{thm}{Theorem}
\newtheorem{prop}{Proposition}
\newtheorem{cor}{Corollary}
\newenvironment{expl}{\it}{\color{black}\normalsize}
\DeclareMathAlphabet{\mathpzc}{OT1}{pzc}{m}{it}

\def\sign{{\rm sign}}
\def\Ker{{\rm Ker}}
\def\Im{{\rm Im}}
\def\Hom{{\rm Hom}}
\def\End{{\rm End}}
\def\Mat{{\rm Mat}}
\def\Ind{{\rm Ind}}
\def\bR{{\mathbb{R}}}
\def\bN{{\mathbb{N}}}
\def\bZ{{\mathbb{Z}}}
\def\bC{{\mathbb{C}}}
\def\bQ{{\mathbb{Q}}}
\def\bB{{\mathbb{B}}}
\def\bA{{\mathbb{A}}}
\def\bs{{\mathbf{s}}}
\def\bd{{\mathbf{d}}}
\def\bT{{\mathbb{T}}}
\def\bt{{\mathbf{t}}}
\def\br{{\mathbf{r}}}
\def\vv{{\vec{v}}}
\def\vw{{\vec{w}}}
\def\vx{{\vec{x}}}
\def\vy{{\vec{y}}}
\def\v0{{\vec{0}}}
\def\ol{\overline}
\def\sspan{\rm{span}}
\def\sl2{{\mathfrak{sl}(2)}}
\def\slc{{\mathfrak{sl}(2,\bC)}}
\def\sp{{\mathfrak{sp}}}
\def\gg{{\mathfrak{g}}}
\def\kk{{\mathfrak{k}}}
\def\pp{{\mathfrak{p}}}
\def\qq{{\mathfrak{q}}}
\def\sP{\mathcal{P}}
\def\sH{\mathcal{H}}
\def\sU{\mathcal{U}}
\def\sC{\mathcal{C}}
\def\sS{\mathcal{S}}
\def\Stab{{\rm Stab}}
\def\ad{{\rm ad}}
\def\Ad{{\rm Ad}}
\def\id{{\rm id}}
\def\sgn{{\rm sgn}}
\def\gcd{{\rm gcd}}
\def\inn#1#2{\left\langle{#1},{#2}\right\rangle}
\def\abs#1{\left|{#1}\right|}
\def\norm#1{{\left\|{#1}\right\|}}
\def\Sp{{\rm Sp}}
\def\SL{{\rm SL}}
\def\O{{\rm O}}
\def\SO{{\rm SO}}
\def\det{{\rm det}}

\def\tG{{\widetilde{G}}}
\def\tK{{\widetilde{K}}}
\def\tM{{\widetilde{M}}}
\def\tJ{{\widetilde{J}}}
\def\trho{{\widetilde{\rho}}}
\def\tDelta{\widetilde{\Delta}}
\def\barZ{{\overline{Z}}}
\def\bz{{\overline{z}}}

\def\Ad{\mathrm{Ad}}
\def\agl{\mathfrak{gl}}
\def\asl{\mathfrak{sl}}
\def\aso{\mathfrak{so}}
\def\asp{\mathfrak{sp}}
\def\au{\mathfrak{u}}
\def\diag#1{\mathrm{diag}(#1)}
\def\lww{\mathcal{W}}
\def\lxx{\mathcal{X}}
\def\lyy{\mathcal{Y}}
\def\fbb{\mathfrak{b}}
\def\fhh{\mathfrak{h}}
\def\fnn{\mathfrak{n}}
\def\fuu{\mathfrak{u}}
\def\fll{\mathfrak{l}}
\def\foo{\mathfrak{o}}
\def\fpp{\mathfrak{p}}
\def\fqq{\mathfrak{q}}
\def\ftt{\mathfrak{t}}
\def\fgg{\mathfrak{g}}
\def\fkk{\mathfrak{k}}
\def\caa{\mathcal{A}}
\def\ccc{\mathcal{C}}
\def\cdd{\mathcal{D}}
\def\chh{\mathcal{H}}
\def\cjj{\mathcal{J}}
\def\crr{\mathcal{R}}
\def\css{\mathcal{S}}
\def\cpp{\mathcal{P}}
\def\cww{\mathcal{W}}
\def\cnn{\mathcal{N}}
\def\cuu{\mathcal{U}}
\def\cxx{\mathcal{X}}
\def\cyy{\mathcal{Y}}
\def\cff{\mathcal{F}}
\def\czz{\mathcal{Z}}
\def\cug{\cuu(\fgg)}
\def\fmm{\mathfrak{m}}
\def\GL{\mathrm{GL}}
\def\SO{\mathrm{SO}}
\def\OO{\mathrm{O}}
\def\rad{\mathrm{rad}}
\def\SL{\mathrm{SL}}
\def\tr{\mathrm{tr}}
\def\Mat{\mathrm{Mat}}
\def\U{\mathrm{U}}
\def\Gr{\mathrm{Gr\,}}
\def\tU{{\widetilde{U}}}
\def\pz#1{\partial z_{#1}}
\def\ddt{\left.\frac{d}{dt}\right|_{t=0}}

\def\Ind{{\rm Ind}}


\def\cmm#1#2{\left[{#1},{#2}\right]}
\def\acmm#1#2{\left\{{#1},{#2}\right\}}

\def\seesawpair#1#2#3#4{
\xymatrix{
{#1} \ar@{-}[dr]& {#2} \\
{#3} \ar@{-}[ur] & {#4}}
}

\def\real{{\rm Re\,}}
\def\imag{{\rm Im\,}}

\hypersetup{
    bookmarks=false,         % show bookmarks bar?
    unicode=false,          % non-Latin characters in Acrobat��s bookmarks
    pdftoolbar=true,        % show Acrobat��s toolbar?
    pdfmenubar=false,        % show Acrobat��s menu?
    pdffitwindow=false,      % page fit to window when opened
    %pdftitle={},    % title
    pdfauthor={Ma Jia Jun},     % author
    %pdfsubject={Subject},   % subject of the document
    pdfcreator={Hoxide},   % creator of the document
    %pdfproducer={Producer}, % producer of the document
    %pdfkeywords={keywords}, % list of keywords
    pdfnewwindow=true,      % links in new window
    colorlinks=true,       % false: boxed links; true: colored links
    linkcolor=blue,          % color of internal links
    citecolor=green,        % color of links to bibliography
    filecolor=magenta,      % color of file links
    urlcolor=cyan           % color of external links
}


\title{Lifting of character}

\begin{document}
\maketitle




Now assume $H$ is a subgroup of $G$,
Suppose that $V$ is a $G$ module, $W$ is a $K$ module, such that
$\Hom_{H}(V,W)\neq 0$.
Define \[
\Omega_V(W) = V / \cnn_{V,W}
\], 
where 
\[
\cnn_{V,W} = \bigcap_{T\in \Hom_H(V,W)}\Ker(T)
\]
which is a generization 
of Howe quotion. It is clear that, any $S\in \Hom_H(V,V)$ act on 
$\Omega_V(W)$ via the quotient map, 
the action is well defined by the construction
(For any $v\in \cnn_{V,W}$, $T\circ S(v) = 0$ since $T\circ S\in \Hom_H(V,W)$ 
for any $T\in \Hom_H(V,W)$. Hence $S(v)\in \cnn_{V,W}$.).
One crucial example is the $\cuu(\fgg)^K$ action on the 
$K$ isotypic component $V(\tau)$ of 
$V$ where $W\in \tau$.

For simplicity,
suppose that $\dim \Hom_{H}(V,W)=1$,
Hence $T\circ S = c(S) T$ for any $S\in \Hom_{H}(V,V)$ by Schur's lemma,
where $c_{V,X}(x)\in \bC$. 
Hence $S$ act on $\Omega_{V,W}$ by this scaler $c_{V,W}(S)$.
In fact, let $\phi$ be 
the projection map, if $v\in V$ such 
that $\phi(v) \neq 0$, $S \phi(v) = \phi \circ S (v) = c_{V,W}(S) \phi(v)$. 


Consider the situation of see-saw pair:
Suppose we have see-saw pair
\[
\xymatrix{G \ar@{-}[dr] & H'\\
H\ar@{-}[ur] & G'
}
\]
where $H < G$, $H'> G'$. 

For representation $\sigma$ of $G'$ which can be realized as a quotient of
the Weil representation, by Howe we have
\[
 \Omega_\omega(\sigma) = \rho(\sigma)\otimes \sigma
\]
for some representation $\rho(\sigma)$ of $G$ as $G\times G'$ representation.

For any irreducible representation $\tau$ of $H$.
\[
\begin{split}
&\Hom_{H}(\rho(\sigma), \tau)\\
= & \Hom_{H\times G'}(\rho(\sigma)\otimes \sigma, \tau\otimes \sigma)
=  \Hom_{H\times G'}(\omega/\cnn_\sigma, \tau\otimes \sigma)\\
= & \Hom_{H\times G'}(\omega, \tau\otimes \sigma)\\
= & \Hom_{H\times G'}(\omega/\cnn_\tau, \tau\otimes \sigma)
= \Hom_{H\times G'}(\tau\otimes \rho(\tau), \tau\otimes \sigma)\\
= & \Hom_{G'}(\rho(\tau), \sigma).
\end{split}
\]
The third equality above valid by the definition of $\cnn_\sigma$. 

Again assume $\dim \Hom_{H}(\rho(\sigma),\tau) = 1$.
Since $\rho(\sigma)$ is $\fgg$ module,  $\cuu(\fgg)^H$ act 
on $\Omega_{\rho(\sigma),\tau} \cong \tau$ via character $c_{\rho(\sigma),\tau}$.
Similary we have $\dim \Hom_G'(\rho(\tau),\sigma) = 1$,
and $\cuu(\fhh')^{G'}$ action on $\Omega_{\rho(\tau),\sigma}\cong \sigma$
via $c_{\rho(\tau),\sigma}$.

\begin{lemma}\label{lemma:ugkcorr}
For see-saw pair as above, 
\[
\omega(\cuu(\fgg)^H) = \omega(\cuu(\fhh')^{G'}).
\]
Hence for any $x\in \cuu(\fgg)^H$ we can choose $y\in \cuu(\fhh')^{G'}$ 
such that $\omega(x) = \omega(y)$ and this choose is independent of 
realization of $\omega$.
\end{lemma}
\proof Clear from the realization of oscillate representation and Howe's paper.
\qed

\begin{lemma}
The map $c_{\rho(\tau),\sigma}$ and $c_{\rho(\sigma),\tau)}$ determin each other 
in a canonical way.
\end{lemma}
\proof 
Consider following commutative digram:
\[
\xymatrix{
\omega \ar[r]^{S_2} \ar[d]_{S_1}\ar[dr]^{Q} & \tau\otimes \rho(\tau)\ar[d]^{T_2}\\
\rho(\sigma)\otimes \sigma\ar[r]_{T_1} & \tau\otimes \sigma 
}
\]
Here $T_i = T\otimes \id$.

Note that $\omega(\cuu(\fgg)^H) = \omega(\cuu(\fhh')^{G'})$,
for $x\in \cuu(\fgg)^H$, 
there is a $y\in \cuu(\fhh')^{G'}$.
such that $\omega(x) = \omega(y)$.


Now for any $v\otimes w\in \tau\otimes \sigma$, there is $a\in \omega$ such
that $Q(a) = v\otimes w$.
Hence, 
\[
(x \cdot v)\otimes w 
= T_1 (x S_1 (a))
= T_1 \circ S_1 (\omega(x) a)
= T_2 \circ S_2 (\omega(y) a)
= v\otimes (y\cdot w).
\]
\qed


Let us consider a simple case. Suppose $(G,G')$ are reductive dual pair. 
$V = \theta_{G', G}(\rho)$ is the theta lifting of character $\rho$ of $\tG'$.

The following result is crucial, and is known by \cite{Shimura1990}, 
\cite{Zhu2003} or \cite{Wallach1992real}:
\begin{lemma}
  $K$ is a maximal compact subgroup of some classical group $G$ 
  of some classical group.
  Let $\sigma$ be a one-dimensional representation of $K$.
  Let $\cjj = \Ker(\sigma)|_{\cuu(\fkk)}$.
  Then the natural map
  \[
  \czz(\fgg) \to \cuu(\fgg)^K/(\cjj\cuu(\fgg)\cap\cuu(\fgg)^K)
  \]
  is surjective.
\end{lemma}

We need a slightly generized version. 
\begin{lemma}\label{lemma:scalerk}
  Consder pair $(G,K)$ where 
  $K$ is a subgroup $G$ meet all the component of $G$.
  Suppose that $\fgg$ be the complexification of 
  a complex classical Lie algebra.
  More over there is a involution, such that $\fgg$ decomposite into 
  $\fgg = \fkk+\fpp$ where 
  $\fkk$ is the complexification of the Lie algebra of $K$.
  
  Let $\sigma$ be a one-dimensional representation of $K$.
  Let $\cjj = \Ker(\sigma)|_{\cuu(\fkk)}$.
  Then the natural map
  \[
  \czz(\fgg) \to \cuu(\fgg)^K/(\cjj\cuu(\fgg)\cap\cuu(\fgg)^K)
  \]
  is surjective.
\end{lemma}
\proof TBA \qed

\begin{lemma}
In above setting, let $V$ be a $G$ module and $W$ be a character of $K$
such that $\dim \Hom_K(V,W)=1$. 
Then the $\cuu(\fgg)^K$ action on $\Omega_{V,W}$ is given by the
$Z(\fgg)$ action on $V$.
\end{lemma}
\proof
Consider the short exact sequence:
\[
0\to \cnn_{V,W}\to V\to W\to 0.
\]
For any $x\in \cuu(\fgg)^K$, choose $z\in Z(\fgg)$ such that 
$x-z \in \cjj\cuu(\fgg)$.
\qed

\begin{thm}
Let $\sigma \in R(G', \omega)$ be a character of $G'$. 
Suppose that $\tau \in R(K, \omega)$ such that 
$\dim \Hom_{K}(\rho(\sigma),\tau) = 1$.
Then the $\cuu(\fgg)^K$ action on $\Omega_{\rho(\sigma),\tau}$ 
is determined by the infinitesimal character of $\tau$ and annilater ideal 
$\cjj = \Ker(\sigma)|_{\cuu(\fhh')}$  
\end{thm}
\proof
By Lemma~\ref{lemma:ugkcorr}, 
for any $x \in \cug^K$ we can find $x'\in \cuu(\fmm')^{G'}$. Such that
$\omega(x) = \omega(x')$ and this choise is independent of the realization 
of oscillator representation. Now choose $z' \in \czz(\fmm')$ such that
$x'-z' \in \cjj\cuu(\fmm')$ by Lemma~\ref{lemma:scalerk}.

Then again by Lemma~\ref{lemma:ugkcorr}, we can find 
$z\in \czz(\fkk)$ independent of realization,
such that $\omega(z) = \omega(z')$.
Hence $x$ act on $\Omega_{\rho(\sigma),\tau}$ by $z$.
 
\qed


Now suppose that for another dual pair $(\tG, \tG')$, character $\trho$
of $\tG'$ such that $\cjj$ annilate $\trho$ and $\theta(\trho)$ has 
$K$-type $\Gamma^i W$. Then the $x$ action on this $K$-type by $z$ actoin. 
Since the infinitesimal character determind the $K$-type, we conclude that 
$\Gamma^i\theta(\rho)$ has a composition component $\theta(\trho)$.
\qed

\begin{cor}
Suppose that:
\begin{enumerate}[(1)]
\item $\rho$ is a character of $G'$, $\pi = \theta_{G,G'}(\rho)$ is the theta
lifting of $\rho$.
\item   
\end{enumerate}
\end{cor}



\begin{lemma}
Suppose $G$ is a classical group, $K$ is a maximal compact subgroup of $G$, 

\end{lemma}



\bibliography{bib/reppapers}{}
\bibliographystyle{alpha}

\end{document}
