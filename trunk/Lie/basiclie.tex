\documentclass[12pt]{amsart}
\usepackage[margin=3cm]{geometry}

\usepackage{hyperref}

\usepackage{amssymb}
\usepackage{graphicx}
%\usepackage{amscd}
\usepackage{braket}
\usepackage{paralist}
\usepackage{eufrak}
%\usepackage{calrsfs}
%\usepackage[small,nohug,heads=littlevee]{diagrams}
%\diagramstyle[labelstyle=\scriptstyle]
%\usepackage{diagrams}
\usepackage{amscd}
%\usepackage{pictexwd,dcpic}
%\usepackage{mathrsfs}


\newtheorem{lemma}{Lemma}
\newtheorem{thm}{Theorem}
\newtheorem{prop}{Proposition}
\newtheorem{cor}{Corollary}
\newenvironment{expl}{\it}{\color{black}\normalsize}
\DeclareMathAlphabet{\mathpzc}{OT1}{pzc}{m}{it}



\def\Ker{\rm{Ker}}
\def\Im{\rm{Im}}
\def\Hom{\rm{Hom}}
\def\Mat{\rm{Mat}}
\def\bR{{\mathbb{R}}}
\def\bN{{\mathbb{N}}}
\def\bZ{{\mathbb{Z}}}
\def\bC{{\mathbb{C}}}
\def\bQ{{\mathbb{Q}}}
\def\bB{{\mathbb{B}}}
\def\bA{{\mathbb{A}}}
\def\bs{{\mathbf{s}}}
\def\bd{{\mathbf{d}}}
\def\bT{{\mathbb{T}}}
\def\bt{{\mathbf{t}}}
\def\br{{\mathbf{r}}}
\def\vv{{\vec{v}}}
\def\vw{{\vec{w}}}
\def\vx{{\vec{x}}}
\def\vy{{\vec{y}}}
\def\v0{{\vec{0}}}
\def\ol{\overline}
\def\sspan{\rm{span}}
\def\sl2{{\mathfrak{sl}(2)}}
\def\slc{{\mathfrak{sl}(2,\bC)}}
\def\sP{\mathcal{P}}
\def\sH{\mathcal{H}}
\def\sU{\mathcal{U}}
\def\sC{\mathcal{C}}
\def\ad{{\rm ad}}
\def\Ad{{\rm Ad}}
\def\id{{\rm id}}
\def\sgn{{\rm sgn}}
\def\gcd{{\rm gcd}}
\def\inn#1#2{\left\langle{#1},{#2}\right\rangle}
\def\abs#1{\left|{#1}\right|}
\def\norm#1{{\left\|{#1}\right\|}}
\def\Sp{{\rm Sp}}


\hypersetup{
    bookmarks=false,         % show bookmarks bar?
    unicode=false,          % non-Latin characters in Acrobat’s bookmarks
    pdftoolbar=true,        % show Acrobat’s toolbar?
    pdfmenubar=false,        % show Acrobat’s menu?
    pdffitwindow=false,      % page fit to window when opened
    pdftitle={Qualify Examination Answers - Algebra},    % title
    pdfauthor={Ma Jia Jun},     % author
    pdfsubject={Subject},   % subject of the document
    pdfcreator={Creator},   % creator of the document
    pdfproducer={Producer}, % producer of the document
    pdfkeywords={keywords}, % list of keywords
    pdfnewwindow=true,      % links in new window
    colorlinks=true,       % false: boxed links; true: colored links
    linkcolor=blue,          % color of internal links
    citecolor=green,        % color of links to bibliography
    filecolor=magenta,      % color of file links
    urlcolor=cyan           % color of external links
}


\newcounter{ssection}
\setcounter{ssection}{0}
\renewcommand{\subsection}{
  \addtocounter{ssection}{1}{\bf  \arabic{ssection}.\  }}

\title{Basic Facts about Lie groups and its representation.}

\def\Ad{\mathrm{Ad}}
\def\agl{\mathrm{gl}}
\def\asl{\mathrm{sl}}
\def\au{\mathrm{u}}
\def\diag#1{\mathrm{diag}(#1)}
\def\fh{\mathfrak{h}}
\def\ft{\mathfrak{t}}
\def\fg{\mathfrak{g}}
\def\GL{\mathrm{GL}}
\def\SO{\mathrm{SO}}
\def\rad{\mathrm{rad}}
\def\SL{\mathrm{SL}}
\def\tr{\mathrm{tr}}
\def\Mat{\mathrm{Mat}}
\def\U{\mathrm{U}}
\begin{document}
\maketitle

\section{notation}
\begin{description}
\item[$B_K$] Killing form, i.e. 
  \[
  B_K(X,Y) = \tr(\ad X \ad Y)
  \]
\item[$B_T$] Trace form 
  \[
  B_T(X,Y) = \Re\tr(XY^*)
  \]
\item[$\Mat_n(F)$] set of matrix over field $F$;   
\item[$I_n$] identity matrix in $\Mat_n(\bC)$;
\item[$E_{k,l}$] matrix with only $(k,j)$-th entry $1$ and others are
  zero. 
\item[$e_l$] can be consider as an element in $\ft^*$ the dual space
  of cartan algebra.
\item[$H_\alpha$] coroot corresponding to $\alpha$, i.e. 
  \[
  B_K(H,H_\alpha) = \alpha(H).
  \]
\item[$\inn{\alpha}{\beta}$] 
  $\inn{\alpha}{\beta} = B_K(H_\alpha,H_\beta) = \alpha(H_\beta) = \beta(H_\alpha)$
\item[$\theta$] cartan involution of lie algebras
\item[$\Theta$] cartan involution of lie groups.
\end{description}

\section{Structures}
\subsection{basic facts}
Note that 
\[
E_{k,l}E_{s,t} = \delta_{l,s}E_{k,t}
\]
Then 
\[
[E_{s,s},E_{k,l}] = E_{s,s}E_{k,l} - E_{k,l}E_{s,s} 
= \delta_{s,k}E_{k,l} - \delta_{l,s}E_{k,l}
\]

\subsection{simple, semisimple, reductive groups}

\subsubsection{simple lie groups}

\subsubsection{semisimple lie groups}

\subsubsection{reductive group}
$\GL(n,\bC)$

\subsection{$\GL(n,\bC)$}
\[
\GL(n,\bC) = \Set{g\in \Mat_{n\times n}(\bC)|\det g \neq 0}
\]

Hence the Lie algebra is 
\[
\agl(n,\bC) = \Mat_{n\times n}(\bC)
\]

Note that 
\[
\agl(n,\bC) = \asl(n,\bC) \oplus Z_{\agl(n,\bC)}
\]
where 
\[
Z_{\agl(n,\bC)} = \bC I_n
\]

$\GL(n,\bC)$ is a complex lie group, with complex dimension $n^2$ and real
dimension $2n^2$.

\subsection{$\SL(n,\bC)$}
\[
\SL(n,\bC) = \Set{g\in \Mat_{n\times n}(\bC)|\deg g = 1}
\]

By $\det(\exp X ) = \exp(\tr X)$ 
\[
\asl(n,\bC) = \Set{g\in \Mat_{n\times n}(\bC)|\tr X = 0}
\]
it is a simple complex lie algebra of type $A_{n-1}$

$\SL(n,\bC)$ is a complex lie group, with complex dimension $n^2-1$,
real dimension $2(n^2-1)$.

The catan algebra is 
\[
\ft = \Set{X \in \asl(n,\bC) | X = \diag(a_1, \cdots, a_n)}
\]
correspond $T$ is all diagonal matrix wih determinat $1$. 

Now, choose basis of $\ft$ be $H_l = iE_{l,l}$, $l = 1,\cdots, n$.
\[
[H_l, E_{l,s}] = i E_{l,s}, \quad l\neq s
\]
\[
[H_s, E_{l,s}] = -iE_{l,s}, \quad l\neq s
\]
Hence 
The nontrival roots are $e_l - e_s$ with root sapces
$E_{l,s}$ ($l\neq s$).

Choose a \index{positive root system} be $e_l-e_s$ positive iff $l<s$.
Corresponding standard\index{simple roots} are $e_l-e_{l+1}$ for $l=1, \cdots, n-1$. Call it $A_{n-1}$.

Since $\asl(n,\bC)$ is a simple complex lie algebra. 
The invariant bilinear form is 
%%need an reference to sec:bilinear_form_simple
unique up to scale.

The relation between killing form over $\asl(n,\bC)$ and trace form is 
\[
2n\,B_T(X,Y) = B_K(X,Y).
\]
This can be calculated by let $X=Y=E_{1,1}-E_{2,2}$.

Let $H_{st} = E_{s,s} - E_{t,t}$.

Then 
\[
H_{e_l-e_k} = \frac{1}{2n} (E_{l,l}-E_{k,k}).
\]

In fact,
\[
(e_l-e_k)(H_{st}) =\delta_{ls} - \delta_{lt} - \delta_{ks} +\delta_{kt}=
B_T(H_{lk}H_{st}).
\]

Then 
\[
\inn{e_l-e_k}{e_s-e_t} 
= \frac{1}{4n^2}(\delta_{ls} - \delta_{lt} - \delta_{ks} + \delta_{kt}).
\]

\[
\frac{2\inn{e_l-e_{l+1}}{e_k-e_{k+1}}}{\inn{e_k-e_{k+1}}{e_k-e_{k+1}}} =
\begin{cases}
  2 & l=k \\
  -1 & l = k\pm 1
\end{cases} 
\]
 
Cartan matrix respect to the standard simple root is 
\[
\begin{pmatrix}
  2  & -1 &  0 & 0 & \cdots & 0 \\
  -1 &  2 & -1 & 0 & \cdots & 0 \\
  0  & -1 &  2 & -1& \cdots & 0 \\
  \cdots & \cdots & \cdots & \cdots & \cdots &\cdots \\
  0 & 0 & 0 & 0 & \cdots & 2 
\end{pmatrix}
\] 


Borel subalgebra

\subsection{$\SO(n,\bC$}
\[
\SO(n,\bC) = \Set{g\in \GL(n,\bC)| g^Tg = 1}
\]
Correspond lie algebra
\[
\aso(n,\bC) = \Set{X\in \agl(n,\bC)| X^T + X = 0}
\]


\subsection{$\GL(n,\bR)$}
\[
\GL(n,\bR) = \GL(n,\bC) \cap \Mat_{n\times n}(\bR)
\]
Consider the complex conjugate $\sigma$ of $\agl(n,\bC)$, 
i.e $\sigma(X) = \overline{X}$. It is a involution on $\agl(n,\bC)$.
$\agl(n,\bR)$ is the set of fix points of $\sigma$.

Then one cartan involution of $\agl(n,\bR)$ is $\theta(X) = -X^*$,
which commutes with $\sigma$.

Now 
\[ 
\begin{split}
B_\theta(X,Y) &= -B_K(X, \theta(Y)) 
\end{split}
\]

\subsection{$\SL(n,\bR)$}
\[
\SL(n,\bR) = \SL(n,\bC) \cap \GL(n,\bR)
\]



\subsection{$U(n)$}
Consider the involution $\theta$ of $\agl(n,\bR)$:
\[
\theta(X) = -X^*
\]
$\au(n)$ is the set of fixed point of $\theta$.

\[
\U(n) = \Set{g \in \GL(n,\bC)|g^*g = I}
\]

Lie algebra: 

\[
u(n,\bC) = \Set{X\in \GL(n,\bC)|X^* + X = 0}
\]

Then the real dimension of $U(n)$ is $2\frac{(n-1)n}{2}+ n = n^2+1$. 
It is simpliy connected
when $n>1$. When $n=1$ it is torus, hence connected with fundermantal
group $\bZ$.

The typical cartan subalgebra is 
\[
\ft = \Set{X\in \Mat_{n\times n}(\bC)| X = \diag{a_1, \cdots, a_n},
  a_l\in i\bR}
\] 
corresponding maximal torus $T$ of $U(n,\bC)$ is the diagnal matrics with
entries on the torus $S^1$. Clearly $T\cong S^n$.



The roots correspond to $T$ spaned by  
\[
\Set{e_l}
\]
Where 

Then the nontrivial roots are $e_l-e_s$ and $-(e_l-e_s)$for any $l\neq s$, 
with root vectors $ d$ and $(1-i)E_{ls}$


\section{bilinear form}
\subsection{invariant form over lie algebra}
Invariant bilinear form $B$ over simple lie algebra $\fg$ is a bilinear 
form satifies 
\[
B(\ad(Z)X,Y) =  -B(X,\ad(Z)Y), \quad X,Y\in \fg
\]

If $\fg$ is a lie algebra of some lie group $G$, then 
\[
B(\Ad(g)X, \Ad(g)Y) = B(X,Y), \quad g\in G, X,Y\in \fg
\]

\subsection{killing form}
Define \index{killing form} by 
\[
B_K(X,Y) = \tr(\ad X \ad Y), \quad X,Y\in \fg
\]

\subsection{trace form}
For real linear lie algebra,
define \index{trace form}
\[
B_T(X,Y) = \tr(XY).
\]
clearly, it is a symetric bilnear form (by compare the explicit expressions of
both sides), i.e. 
\[
B_T(X,Y) = B_T(Y,X)
\]
moreover 
\[
\begin{split}
B_T(\ad(Z)X,Y) &= \tr([Z,X]Y) = \tr(ZXY-XZY) \\
&=  \tr(XYZ - XZY) =
\tr(X[Y,Z]) = - B_T(X, \ad(Z)Y)
\end{split}
\]

\subsection{bilinear form over simple lie algebra} 
\label{sec:bilinear_form_simple}
The radical $\rad(B) = \Set{Y\in \fg|B(X,Y) = 0\, \forall X\in \fg}$ 
of the invariant bilinear form $B$ is a ideal of lie algebra
$\fg$. 
In fact, if $Y\in \rad(B)$, 
\[
B(X,\ad(Z) Y) = -B(\ad(Z)X,Y) = 0.
\]
Hence $\ad(Z) Y \in \rad(B)$. 

So, if the lie algebra is simple, nontrival invariant bilinear form
is nondegenerate. 

In fact, the bilinear form give a hermitian structure on the adjoint
representation of $\fg$. $\fg$ is simple lie algebra means this
representation is irreducible. Thus the bilinear form is unique up to scaler.

\section{Index}

\end{document}

