\documentclass[12pt]{amsart}
\usepackage[margin=3cm]{geometry}

\usepackage{hyperref}

\usepackage{amssymb}
\usepackage{graphicx}
%\usepackage{amscd}
\usepackage{braket}
\usepackage{paralist}
\usepackage{eufrak}
%\usepackage{calrsfs}
%\usepackage[small,nohug,heads=littlevee]{diagrams}
%\diagramstyle[labelstyle=\scriptstyle]
%\usepackage{diagrams}
\usepackage{amscd}
%\usepackage{pictexwd,dcpic}
%\usepackage{mathrsfs}


\newtheorem{lemma}{Lemma}
\newtheorem{thm}{Theorem}
\newtheorem{prop}{Proposition}
\newtheorem{cor}{Corollary}
\newenvironment{expl}{\it}{\color{black}\normalsize}
\DeclareMathAlphabet{\mathpzc}{OT1}{pzc}{m}{it}



\def\Ker{\rm{Ker}}
\def\Im{\rm{Im}}
\def\Hom{\rm{Hom}}
\def\Mat{\rm{Mat}}
\def\bR{{\mathbb{R}}}
\def\bN{{\mathbb{N}}}
\def\bZ{{\mathbb{Z}}}
\def\bC{{\mathbb{C}}}
\def\bQ{{\mathbb{Q}}}
\def\bB{{\mathbb{B}}}
\def\bA{{\mathbb{A}}}
\def\bs{{\mathbf{s}}}
\def\bd{{\mathbf{d}}}
\def\bT{{\mathbb{T}}}
\def\bt{{\mathbf{t}}}
\def\br{{\mathbf{r}}}
\def\vv{{\vec{v}}}
\def\vw{{\vec{w}}}
\def\vx{{\vec{x}}}
\def\vy{{\vec{y}}}
\def\v0{{\vec{0}}}
\def\ol{\overline}
\def\sspan{\rm{span}}
\def\sl2{{\mathfrak{sl}(2)}}
\def\slc{{\mathfrak{sl}(2,\bC)}}
\def\sP{\mathcal{P}}
\def\sH{\mathcal{H}}
\def\sU{\mathcal{U}}
\def\sC{\mathcal{C}}
\def\ad{{\rm ad}}
\def\Ad{{\rm Ad}}
\def\id{{\rm id}}
\def\sgn{{\rm sgn}}
\def\gcd{{\rm gcd}}
\def\inn#1#2{\left\langle{#1},{#2}\right\rangle}
\def\abs#1{\left|{#1}\right|}
\def\norm#1{{\left\|{#1}\right\|}}
\def\Sp{{\rm Sp}}


\hypersetup{
    bookmarks=false,         % show bookmarks bar?
    unicode=false,          % non-Latin characters in Acrobat’s bookmarks
    pdftoolbar=true,        % show Acrobat’s toolbar?
    pdfmenubar=false,        % show Acrobat’s menu?
    pdffitwindow=false,      % page fit to window when opened
    pdftitle={Qualify Examination Answers - Algebra},    % title
    pdfauthor={Ma Jia Jun},     % author
    pdfsubject={Subject},   % subject of the document
    pdfcreator={Creator},   % creator of the document
    pdfproducer={Producer}, % producer of the document
    pdfkeywords={keywords}, % list of keywords
    pdfnewwindow=true,      % links in new window
    colorlinks=true,       % false: boxed links; true: colored links
    linkcolor=blue,          % color of internal links
    citecolor=green,        % color of links to bibliography
    filecolor=magenta,      % color of file links
    urlcolor=cyan           % color of external links
}


\newcounter{ssection}
\setcounter{ssection}{0}
\renewcommand{\subsection}{
  \addtocounter{ssection}{1}{\bf  \arabic{ssection}.\  }}

\title{ON CERTAIN GROUPS of unitary operators}

\begin{document}
\maketitle

\section{locally compact abelian goups}

\subsection{}	
In this chapter, it is mainly about locally compact abelian groups ,
 on which we will most often no restrictive assumptions, 
but some our results is irrelevant unless $G$ is isomorphic to its dual.
All subsequent applications relating to one of the following cases:
\begin{enumerate}[(a)]
\item $G$ is a vector space
$X$ finite dimension $k$ over a locally compact non discrete
\item $G$ has the form $X_A=X_K\otimes A_k$, where $A_k$ is the ring
  of a body adeles $k$ which can be either a body of algebraic
  numbers,
  or a body of algebraic functions of dimension $1$ over a finished, 
  and oh k X is a finite dimensional vector on It. 
\end{enumerate}
We refer to these cases
saying that $G$ is ``type in the local case'' in case(a), ``type
ade1ique'' in case (b);
 if $G$ is a local or adelique, 
it is isomorphic to its dual. 
Locally compact groups  locally will often notes additives.

$T$ mean the multiplicative group of complex numbers such as
$t\bar{t}=1$,
 a character of $G$ is a morphism of $G$ to $T$. If $G$ and $H$ 
are locally compact abelian groups abe1iens, a bicharacter of $G\times
H$ is the continuous fuction $f$ of  $G\times H$ to T such that,
 for all $y\in H$, $x \to f(x, y)$ is a character of $G$, for all 
$x\in G$, $y \to f(x,y)$ is a character of $H$.

	
The continuous function $f$ of $G$ to $T$ will be will be called a second
degree character of $G$ if the fuction 
\[
(x, y)\to \frac{f( x + y)}{f(x)f (y)}
\]
is a bicharacter of $G\times G$, or 
, which is the same, if $f$ satisfies the relationship
\[
f (x + y + z) f (x) f(y) f (z) = f(x + y) f (y + z) f (z + x)
\]
whenever $x, y, z$ in $G$.

	
We always denote $G^*$  the dual of $G$ (written additive too), 
and we write
$(x, x^*)$, for $X\in G$, $x^* \in G^*$, as the value on $x$
corresponding the character  $x^*$ of $G$.
We always identify $G$ with the double-dual $(G^*)^*$ of  $G$ such that
\[
\inn{x}{x^*} = \inn{x^*}{x}
\]
(it can made this identification such that
$\inn{x}{x^*} ={-x^*}{x}$ as well, and it would be even more
convenient 
in some respects, 
but this shock habits received too)
If $x\to x\alpha$ is a morphism from $G$ to $H$, its dual $\alpha^*$
is the morphism from $H^*$ to $G^*$ such that 
\[
\inn{x\alpha}{y^*}=\inn{x}{y^*\alpha^*}
\]
whenever $x\in G$,$y^*\in H^*$.

All bicharacter of $G\times H$ can be written uniquely in the form 
\[
f(x,y)=\inn{x}{y\alpha}=\inn{y}{x\alpha^*}
\]
where $\alpha$ is a morphism from $H$ to $G^*$, 
$\alpha^*\colon G\to H^*$ is the dual of $\alpha^*$.
If $G=H$, it is both necessary and sufficient that $f$ is symmetric 
in $x, y$, then we have $\alpha=\alpha^*$.
We say that the morphism $\alpha$ from $G$ to $G^*$ is {\it symmetric}. 
 
If $f$ is a character of second degree of $G$, we have 
\begin{equation}\label{eq:1}
\frac{f(x+y)}{f(x)f(y)} = \inn{x}{y\rho},
\end{equation}
or $\rho=\rho(f)$ is a morphism, obviously symmetric, from $G$ to
$G^*$;
In expression $\ref{eq:1}$, we say that $f$ and $\rho$ are associated
to each other. If we denote $X_2(G)$  the multiplication group of 
characters of second degree on $G$, function $f\to \rho(f)$ is a
homomorphism from $X_2(G)$ to the additive group of symetric
morphisms from $G$ to $G^*$;
the kernal of this homomorphism is the multiplicative group $X_1(G)$
of characher on $G$.
We can say more in the case that $x\to 2x$ is an automorphism of $G$
 (which occurs for example if $G$ is local or on a body adelie k
 characteristics other than 2);
then, we denote $x\to 2^{-1}x$  the inverse of the automorphism $x\to
2x$ of $G$. In this case, if $\rho$ is a symetric morphism from $G$ to
$G^*$,
it is associated with the second degree character
 $f_\rho(x)=\inn{x}{2^{-1}x\rho}$; if we denote $X^0_2(G)$ the
 subgroup of $X_2(G)$ formed by $f_\rho$, then $X_2(G) =
 X^0_2(G)\times X_1(G)$, and $X^0_2(G)$ is isomorphic to the additive
 group of symetric morphisms from $G$ to $G^*$.

We say that the character of second degree $f$ is non {\it degenerate} 
if the symetric morphism associated with $f$ is a isomrophism from $G$ 
to $G^*$;the existance of such character is necessary (but not
sufficient) for $G$ isomorphic to $G^*$.

\subsection{}
Section 2

Given a Haar measure $dx$ on G, a Fourier transform $\mathfrak{T}$
is that we associate with each function $\Phi$ on $G$, a function
$\Phi^{*}=\mathfrak{T}(\Phi)$ on $G^{*}$ defined as \[
\Phi^{*}(x^{*})=\int\Phi(x)\cdot<x,x^{*}>\cdot dx\]


\noindent whenever this integral, or a suitable extension of it, makes
sense. Then there is a Haar measure $dx^{*}$ on $G^{*}$, called
the $dual$ of $dx$, such that the inverse transformation $\mathfrak{T}^{-1}$
of $\mathfrak{T}$ is given by the formula

\[
\Phi(x)=\int\Phi^{*}(x^{*})\cdot<x,-x^{*}>\cdot dx^{*};\]


For this measure, we have the Plancherel formula

\[
\int|\Phi(x)|^{2}dx=\int|\Phi^{*}(x^{*})|^{2}dx^{*}\]


It is clear that, for every $c>0$, the Haar measure on $G^{*}$,
dual of $c\cdot dx$, is $c^{-1}dx^{*}$. This remark can be expressed
as follows,

\begin{lemma}\label{l:1}
Let $G$, $H$ be two locally compact abelian groups,
with Haar measure $dx$, $dy$; let $G^*$, $H^*$ be their duals,
equipped with Haar measure $dx^*$, $dy^*$ respectively. Then,
if $\alpha$ is an isomorphism of $G$ on $H$, $\alpha^{*}$ is an
isomorphism of $H^{*}$ on $G^{*}$, and one has $|\alpha^{*}|=|\alpha|$.
\end{lemma}

Recall that, if $G$ and $H$ are locally compact groups (commutative
or not) with Haar measures, the module of an isomorphism $\alpha$
of $G$ on $H$ is the number $|\alpha|=d(x\alpha)/dx$ defined by
the formula

\[
\int F(y)dy=|\alpha|\cdot\int F(x\alpha)dx\]


where $F\in L^{1}(H)$; if $G=H$, it is generally understood that
we take $dx=dy$, and then $|\alpha|$ is independent of the choice
of $dx$. To prove the lemma, let $m=|\alpha|$; by transport of structure,
$\alpha$ transforms $dx$ into a Haar measure $d'y$ on $H$, and
as soon as we see that $d'y=m^{-1}dy$; it follows that $\alpha^{*}$transforms
the dual measure of $d'y$, which is, as noted above, $m\cdot dy^{*}$
into $dx^{*}$; thus $\alpha*$ transforms $dy^{*}$ into $m^{-1}dx^{*}$,
which shows the lemma.

\subsection{}
Let $x\to z\sigma$ is an automorphism of $G\times G^*$; if we set
$z=(x,x^*)$, we can also written in a matrix form:
\[
(x,x^*)\to (x,x^*)\cdot 
\begin{pmatrix} 
\alpha & \beta \\
\gamma & \delta
\end{pmatrix}
\]
which means, of course:
\[
(x,x^*)\to (x\alpha + x^*\gamma,x\beta + x^*\delta)
\]
where $\alpha$, $\beta$, $\gamma$, $\delta$ are morphisms from $G$ to
$G$,
from $G$ to $G^*$, from $G^*$ to $G$ and $G^*$ to $G^*$ respectively. 
Note that the dual $\sigma^*$ of the $G\times G^*$ automorphism
$\sigma$ 
is the automorphism defined by these formulas:
\[
\sigma^* \to \begin{pmatrix}
\alpha^* & \gamma^* \\
\beta^* & \delta^*
\end{pmatrix}
\]
on $G^*\times G$.
Let $\eta$ be the isomorphism $\begin{pmatrix}0 & 1\\ -1 & 0\end{pmatrix}$ 
from $G\times G^*$ to $G^*\times G$, or, it's the same, 
the isomorphism $(x,x^*)\to (-x^*, x)$ (we refer them as the same automorphism  for any groups.)  The forumla
\begin{equation}\label{eq:2}
\sigma^I = \eta \sigma^* \eta^I = 
\begin{pmatrix} \delta^* & - \beta^*\\-\gamma^* & \alpha^*\end{pmatrix}
\end{equation}
defines an automorphism of $G\times G^*$, and Lemma~\ref{l:1} of section~2
shows that $\abs{\sigma^I}=\abs{\sigma}$. 
Note that $\sigma\to \sigma^I$ is an anti-automorphism involution of the
group of automorphisms of $G\times G^*$.
For the convenience of writing,  we will denote $F$ the bicharacter of 
$(G\times G^*)\times (G\times G^*)$ defined by 
\begin{equation}\label{eq:3}
F(z_1,z_2) = \inn{x_1}{x_2^*}\quad (z_1=(x_1,x_1^*), z_2=(x_2,x_2^*)
\end{equation}
An automorphism $\sigma$ of $G\times G^*$ is called {\it symplectic} if
it lead invariant of the bicharacter $F(z_1,z_2)F(z_2,z_1)^{-1}$, i.e. if
you have 
\[
\frac{F(z_1\sigma, z_2\sigma)}{F(z_2\sigma, z_1\sigma}= 
\frac{F(z_1, z_2)}{F(z_2,z_1)}.
\]
whenever $z_1$, $z_2$ in $G\times G^*$; we write $\Sp(G)$ as the group 
formed by these automorphisms. For $\sigma$ to be symplectic, 
it is both necessary and sufficiently (as shown by a calculation immediately) 
that $\sigma \sigma^I = 1$, $\sigma^I$ is defined by (\ref{eq:2});
$\abs{\sigma^I}=\abs{\sigma}$, it follows that the module of any symplectic 
automorphism is $1$. The relationship $\sigma \sigma^I=1$ gives 
$\alpha\beta^* = \beta\alpha^*$ and $\gamma \delta^*=\delta \gamma^*$ 
in particular, which means that $\alpha\beta^*$ and $\gamma \delta^*$
 are symmetric morphisms from $G$ to $G^*$ and from $G^*$ to $G$ respectively;
through the relationship $\sigma^I \sigma = 1$, we see that $\beta^*\delta$ and
$\gamma^*\alpha$ in the same way.

\subsection{}
For any element $w=(u,u^{*})$ of $G\times G^{*}$, we pick an operator
$U(w)$, which, for any function $\Phi$ on $G$, gives a function
$\Phi'=U(w)\Phi$ such that

\[
\Phi'(x)=(U(w)\Phi)(x)=\Phi(x+u)\cdot<x,u^{*}>\]


For short, we write $U(w)\Phi(x)$ instead of $(U(w)\Phi)(x)$. Applied
to functins $\Phi\in L^{2}(G)$, the $U(w)$ is obviously the unitary
operator, and it was, for any $w_{1},w_{2}$ in $G\times G^{*}$:

\[
U(w_{1})U(w_{2})=F(w_{1},w_{2})\cdot U(w_{1}+w_{2})\]


wherein $F$ is the function defined above by (3). It concludes that
the operators $t\cdot U(w)$, for $w\in G\times G^{*}$ and $t\in T$,
form a group, whose composition law is given by

\begin{equation}\label{eq:4}
(w_{1},t_{1})\cdot(w_{2},t_{2})=(w_{1}+w_{2},F(w_{1},w_{2})t_{1}t_{2})
\end{equation}

In other words, the formula (\ref{eq:4}) defines a group law
 on the set $G\times G^{*}\times T$;
and if we denote by $A(G)$ the group thus defined (which, with the
obvious topology on $G\times G^{*}\times T$, is a locally compact
group), the mapping $(w,t)\rightarrow t\cdot U(w)$ defines a unitary
representation $A(G)$. We denote by $\mathbf{A}(G)$ the group formed
by the operators $t\cdot U(w)$; if we choose the topology induced
by the strong topology  in the
group of automorphisms of $L^{2}(G)$ (cf. later in Sectin 35), it
is easy to verify that $(w,t)\rightarrow t\cdot U(w)$ is even an
isomorphism of topological groups.

The center of the group $A(G)$ is obviously formed by the elements
$(0,t)$; it is isomorphic to $T$, and denoted $T$ for short. It
is clear that $(w,t)\rightarrow w$ is a homomorphism of $A(G)$ on
$G\times G^{*}$, with kernel $T$; it allows to identify $A(G)/T$
with $G\times G^{*}$.
\subsection{}
Let $B(G)$ be the automorphism group of $A(G)$. 
An automorphism $s$ in  $B(G)$ induce an automorphism of the center $T$ of 
 $A(G)$, which may not be t-> t or t-> i; 
and induced by passing to the quotient, an automorphism a $A(G)/T$, i.e. 
$G\times G^*$. We denoted $B_0(G)$ the automorphism group of $A(G)$, 
which induce the identity map on the center $T$ of $A(G)$.
We  consider $B_0(G)$ now, although the following results can partially 
coverte to $B(G)$. Let $s$ be an element of $B_0(G)$, which induced a 
automorphism $\sigma$ of $G\times G^*$. it is immediatly written $s$ as
\begin{equation}\label{eq:5}
(w,t)s = (w\sigma, f(w)t),
\end{equation}
where $f$ is a continous function from $G\times G^*$ to $T$. 
This formula defines an automorphism of $A(G)$, it is necessary and sufficient 
that
\begin{equation}\label{eq:6}
\frac{f(w_1+w_2)}{f(w_1)f(w_2)}=\frac{F(w_1\sigma, w_2\sigma)}{F(w_1,w_2)}
\end{equation}
where $w_1,w_2$ are in $G\times G^*$.
This shows in particular that $f$ 
is a character in the second degree of $G\times G^*$. 
Moreover,the second is symmetric in $w_1$ and $w_2$,
 we see that $\sigma$ must be symplectic.

Write $s = (\sigma, f)$ where $s$ is the automorphism of $A(G)$ defined by 
(\ref{eq:5}), $f$ and $\sigma$ satisfy the relation ($\ref{eq:6}$).
The group law is given by 
\[
(\sigma, f)\cdot(\sigma',f') = (\sigma\sigma', f'')
\]
for any $w\in G\times G^*$, $f$ is defined by the formula
\begin{equation}\label{eq:7}
f''(w)=f(w)f'(w\sigma)
\end{equation}
The map $s\to \sigma$ is a homomorphism from $B_0(G)$ to $\Sp(G)$;
its kernel is formed by  the elements $(1,f)$, where $f$, by (\ref{eq:6}),
is a character of $G\times G^*$ of the form 
\[
f(u,u^*) = \inn{u}{a^*}\inn{a}{u^*}
\]
with $a\in G$, $a^*\in G^*$. 	 	
But it immediately satisfies that $(1,f)$ is the inner
 automorphism of $A(G)$ determind by the element $(-a, a^*, 1)$. 
The kernel of $s\to \sigma$ formed by the inner automorphism of $A(G)$,
is isomorphic to $A(G)/T$, therefore, $G\times G^*$.

 	
We can go one step further by explaining the second term of (\ref{eq:6}).
Let $\sigma$ be placed in matrix form as in Section~$3$. Let
\[
f'(u,u^*)=f(u,u^*)\inn{u^*\gamma}{-u\beta}
\]
Easy calculation gives (\ref{eq:6}) in the form
\[
f'(u_1+u_2, u^*_1+u^*_2) = f'(u_1,u^*_1)f'(u_2,u^*_2)\inn{u_1}{u_2\alpha\beta^*}
\inn{u^*_1\gamma\delta^*}{u^*_2}.
\]
Let $g(u) = f'(u,0)$, $h(u^*)=f'(0,u^*)$; by $u_2=0$, $u_1^*=0$ in the relation
about, we see that $f'(u,u^*)$ is no other than $g(u)h(u^*)$, then $g$ and $h$
satisfy the relations
\begin{align*}
g(u_1+u_2)&=g(u_1)g(u_2)\inn{u_1}{u_2\alpha\beta^*}\\
h(u^*_1+u^*_2)&=h(u^*_1)g(u^*_2)\inn{u^*_1\gamma\delta^*}{u^*_2},
\end{align*}
in other words they are the characters of second degree of $G$ and $G^*$,
respectively associated with symmetric morphisms $\alpha\beta^*$, 
$\gamma\delta^*$ from $G$ to $G^*$ and $G^*$ to $G$. Then:
\[
f(u,u^*)=g(u)h(u^*)\inn{u^*\gamma}{u\beta}.
\]
	
It of course has more accurate results when $x-> 2x$ is an automorphism of $G$.
In view of Subsection~1, above formulas show that,
 for all symplecitc automorphism $\sigma$, it correspond to an element 
$(\sigma, f)$ of $B_0(G)$, obtained by
\[
g(u)=\inn{u}{2^{-1}u\alpha\beta^*}, h(u^*)=\inn{2^{-1}u^*\gamma\delta^*}{u^*}.
\]
	
Furthermore, these formulas define a monomorphisme from $\Sp(G)$ to $B_0(G)$, 
and $B_0(G)$ is the semidirect product of the image of $\Sp(G)$ by this map 
and the inner automorphisms of $A(G)$; consequently, 
$B_0(G)$ is isomorphic to a semidirect product of $Sp (G)$ and  $G\times G^*$.

\subsection{}
Returning to the general case, let $s(\sigma,f)$ be an element from
$B_{0}(G)$, and let's write $\sigma$ in the matrix form introduced
in Section 3. Consider first the case wherein $\beta=0$ and $\gamma=0$;
the symplectic considition, $\sigma\sigma^{I}=1$, then gives $\delta=\alpha^{*-1}$;
it follows that the second member in (6) has the value $1$; it agrees
with (6) when taking $f=1$, and $s$ differs from $(\sigma,1)$ by
only an inner automorphism. For any automorphism $\alpha$ of $G$,
we let

\[
d_{0}(\alpha)=\left(\left(\begin{array}{cc}
\alpha & 0\\
0 & \alpha^{*-1}\end{array}\right),1\right);\]
$\alpha\rightarrow d_{0}(\alpha)$ is then a monomorphism of the group
of automorphisms of $G$ into the group $B_{0}(G)$.

Now let $\alpha=0$, $\delta=0$; as $\sigma$ is an automorphism
of $G\times G^{*}$, it implies that $\beta$, $\gamma$ are the isomorphisms
of $G$ on $G^{*}$ and of $G^{*}$ on $G$, respectively. Then $\sigma\sigma^{I}=1$
gives $\beta=-\gamma^{*-1}$, and we verify immediately that (6) is
satisfied when $f(u,u^{*})=<u,-u^{*}>$. If $\gamma$ is an isomorphism
of $G^{*}$ on $G$, we let

\[
d_{0}^{'}(\gamma)=\left(\left(\begin{array}{cc}
0 & -\gamma^{*-1}\\
\gamma & 0\end{array}\right),<u,-u^{*}>\right).\]


Again let $\alpha=1$, $\delta=1$, $\gamma=0$; $\sigma\sigma^{I}=1$
reduces to $\beta=\beta^{*}$, and the formulas of Section 5 show
that $f$ is of the form $g(u)h(u^{*})$, where $h$ is a character
of $G^{*}$ and $g$ a character of the second degree of $G$ associated
to $\beta$. Whenever $f$ is a character of the second degree of
$G$, and $\rho$ is the symmetric morphism of $G$ to $G^{*}$associated
to $f$, this leads to

\[
t_{0}(f)=\left(\left(\begin{array}{cc}
1 & \rho\\
0 & 1\end{array}\right),f\right);\]


$f\rightarrow t_{0}(f)$ is then a monomorphism of the group $X_{2}(G)$
of characters of the second degree of $G$ into the group $B_{0}(G)$.
Similarly, if $f'$ is a character of the seconde degree of $G^{*}$,
associated with symmetric morphism $\rho'$ of $G^{*}$ to $G$, we
write

\[
t_{0}^{'}(f')=\left(\left(\begin{array}{cc}
1 & 0\\
\rho' & 1\end{array}\right),f\right),\]
 which defines a monomorphism of $X_{2}(G)$ into $B_{0}(G)$.

If $f$ is a character of the seconde degree of $G$, and $\alpha$
an automorphism of $G$, we write

\[
f^{\alpha}(x)=f(x\alpha^{-1})\]
(however, as this notation would lead to confusion when $\alpha=-1$,
we write $f^{-}(x)=f(-x)$); along with this notation, we have

\[
d_{0}(\alpha)^{-1}t_{0}(f)d_{0}(\alpha)=t_{0}(f^{\alpha}),\]
\[
d_{0}(\alpha)t_{0}^{'}(f')d_{0}(\alpha)^{-1}=t_{0}^{'}(f^{'\alpha^{*}}).\]


If $\alpha$ is as above, and if $\gamma$ is an isomorphism of $G^{*}$
on $G$, we have

\[
d_{0}^{'}(\gamma\alpha)=d_{0}^{'}(\gamma)d_{0}(\alpha),\]
\[
d_{0}^{'}(\alpha^{*-1}\gamma)=d_{0}(\alpha)d_{0}^{'}(\gamma);\]


the first relation shows in particular that the set of elements of
$B_{0}(G)$ of the form $d_{0}^{'}(\gamma)$, if it is nonempty, is
a right coset relative to the subgroup of $B_{0}(G)$ formed by the
elements of the form $d_{0}(\alpha)$. More generally, we observe
that, according to (6), if an element $s$ of $B_{0}(G)$ is of the
form $(\sigma,1)$, the bi-character $F$ must be invariant under
$\sigma$; as $G^{*}$ is the set of $z_{1}\in G\times G^{*}$ such
that $F(z_{1,}z_{2})=1$ for every $z_{2}$, and that $G$ is the
set of $z_{2}\in G\times G^{*}$ such that $F(z_{1},z_{2})=1$ for
every $z_{1}$, it follows that $\sigma$ is then of the form $\left(\begin{array}{cc}
\alpha & 0\\
0 & \delta\end{array}\right)$, and therefore, as been seen above, that we have $s=d_{0}(\alpha)$;
the formula (7) then shows that, for any two elements $s=(\sigma,f)$
and $s^{"}=(\sigma",f")$ of $B_{0}(G)$ belonging to the same right
coset relative to the subgroup of elements of the form $d_{0}(\alpha)$,
it is necessary and sufficent that $f=f^{"}$.

\subsection{}
Now agree, for $s=(\sigma, f)$ and 
$\sigma=\begin{pmatrix}\alpha &\beta \\ \gamma &\delta
\end{pmatrix}$, 
put $\gamma = \gamma(s)$; and denote $\Omega_0(G)$  all the $s\in B_0(G)$
such that $\gamma(s)$ is an isomorphism from $G^*$ to $G$
(this set can be empty).
It has the following result:
\begin{prop}\label{p:1}
These set $\Omega_0(G)$ of $s\in B_0(G)$ such that $\gamma(s)$ is an 
isomorphism from $G^*$ to $G$ is the set of elements of $B_0(G)$  of the forme 
\begin{equation}\label{eq:8}
s = t_0(f_1)d'_0(\gamma)t_0(f_2)
\end{equation}
where $\gamma$ is a isomorphism from $G^*$ to $G$ and $f_1$, $f_2$ are 
characters of second degree for $G$;
and any element of $\Omega_0(G)$ has unique decomposition in that form.  
\end{prop}
if $s$ is given by (\ref{eq:8}), $\gamma(s)=\gamma$, then 
$s$ is in $\Omega_0(G)$.
Conversely, let $s =(\sigma, f) \in  \Omega_0(G)$; if (\ref{eq:8}) holds, 
we must take $\gamma=\gamma(s)$. Then let 
$\sigma=\begin{pmatrix}\alpha &\beta \\ \gamma &\delta \end{pmatrix}$; easy 
calculation shows that (\ref{eq:8}) is satisfied, 
it is necessary and sufficient that 
\[
f_1(u)=f(u,-u\alpha\gamma^{-1}), \quad f_2(u)=f(0,u\gamma^{-1}).
\]
These demonstrated by  the proposition. 
Note that applying (\ref{eq:8}) the homomorphism
$s\to \sigma$, by the relation:
\[
\begin{pmatrix}\alpha &\beta \\ \gamma &\delta \end{pmatrix}
= \begin{pmatrix}
1 & \alpha\gamma^{-1}\\
0 & 1
\end{pmatrix}
\begin{pmatrix}
0 & -\gamma^{*-1}\\
\gamma & 0
\end{pmatrix}
\begin{pmatrix}
1 & \gamma^{-1}\delta\\
0 & 1
\end{pmatrix};
\]
because of the symplectic of $\sigma$, $\alpha\gamma^{-1}$ and $\gamma^{-1}\delta$
are symmetric morphisms from $G$ to $G^*$; they are associated with
 $f_1$ and $f_2$ respectively.

 	
This gives a important relationship by considering a second degree character $f$
which is non degenerated of $G$, 	
this means that the morphism $\rho$ associated with $f$ is an isomorphism
from $G$ to $G^*$; then the function $f'$ to $G^*$, defined by 
\[
f'(x^*) = f(-x^*\rho^{-1}),
\]
is a character of second degree of $G^*$, with the associated symmetric 
morphism $\rho^{-1}$ from $G^*$ to $G$. 
By the proposition~\ref{p:1}, applied to $t'_0(f')$, gives
\[
t'(f')=t_0(f)d'_0(\rho^{-1})t_0(f^-)
\]
where $f^-$ is defined by $f^-(x)=f(-x)$ 	
as mentioned above.
A simple calculation gives the other:
\[
t'_0(f') = d'_0(\rho^{-1})t_0(f^{-1})d'_0(-\rho^{-1}).
\]
At the same time we have $d'_0(\rho^{-1})^2=d_0(-1)$, we derived the following:
\begin{equation}\label{eq:9}
d'_0(-\rho^{-1})t_0(f)d'_0(\rho^{-1})t_0(f^-)= t_0(f^{-1})d'_0(-\rho^{-1}).
\end{equation}
Taking into account the relationships obtained in subsection~6, 
we have (\ref{eq:9}) in a simpler form
\[
(t_0(f)d'_0(-\rho_0^{-1}))^3=e,
\] 
where $e$ is the identity element of $B_0(G)$; in that form, 
it is well known in the theory classic modular group.
But it is the relationship (9), as written above, we have to use later.

\subsection{}

The automorphisms of the group $\mathbf{A}(G)$, (which is )isomorphic
to $A(G)$, which was introduced in Section 4, are of course the same
as those of $A(G)$. We now propose to show that any automorphism
$s\in B_{0}(G)$ of $\mathbf{A}(G)$ is induced on $\mathbf{A}(G)$
by an inner automorphism of the group of all the unitary operators.
This theorem is due to I.Segal{[}6{]} in the case where $x\rightarrow2x$
is an automorphism of $G$, and we borrow his method of demonstration,
which includes the introduction of an algebra of operators naturally
associated to the group $\mathbf{A}(G)$. For this, we propose, in
a sense which will be precise in a moment,

\[
U(\varphi)=\int U(w)\varphi(w)dw,\]


where $\varphi$ denotes a function on $G\times G^{*},$ and where
$w=(u,u^{*})$ and $dw=du\cdot du^{*}$ (measure which does not depend
on the choice of the measure $du$ on $G$). In other words, if $\Phi$
is a function on $G$, $U(\varphi)\Phi$ is the functin defined by

\begin{equation}\label{eq:10}
U(\phi)\Phi(x)=\int U(w)\Phi(x)\cdot \phi(w)dw =
\int \Phi(x+u)\cdot \inn{x}{u^*} \cdot \phi(u,u^*) du du^* 
\end{equation}

where we assume provisionally, to fix the ideas, that $\varphi$ and
$\Phi$ are two continuous function with compact support. This can
also be written as

\begin{equation}\label{eq:11}
U(\phi)\Phi(x)=\int K(x,y)\Phi(y)dy
\end{equation}


where $K$ is given by

\[
K(x,y)=\int\varphi(y-x,u^{*})\cdot<x,u^{*}>\cdot du^{*}\]



or, which is the same as

\[
K(x,x+u)=\int\varphi(u,u^{*})\cdot<x,u^{*}>\cdot du^{*}\]


one obtains then $K(-x,-x+u)$ from $\varphi(u,u^{*})$ while applying,
for any value of $u$, the Fourier transform where $\varphi(u,u^{*})$
considered as function of $u^{*}$. Under the conditions when the
validity of Fourier inversion formula holds, we then have

\[
\varphi(u,u^{*})=\int K(x,x+u)\cdot<x,-u^{*}>\cdot dx;\]


furthermore, in virtue of Plancherel theorem, we have

\[
\int|K(x,y)|^{2}dxdy=\int|\varphi(u,u^{*})|^{2}dudu^{*},\]


this shows that the correspondence between the functions $\varphi$
on $G\times G^{*}$ and the functions $K$ on $G\times G$, defined
by the above formulas, is extended continuouly to an isomorphism $W$
of $L^{2}(G\times G^{*})$ onto $L^{2}(G\times G)$.

When $K$ is the function defined by $K(x,y)=P(x)Q(y)$, we write
$K=P\otimes Q$; and, if $P$ and $Q$ are in $L^{2}(G)$, we write
$(P,Q)=\int P(x)\overline{Q(x)}dx$; along with these notations, the
formulas above are given in particular

\begin{equation}\label{eq:12}
W^{-1}(P\otimes \overline{Q})(w)=(P,U(w)Q)
\end{equation}

\subsection{}
Now let $\phi_1$, $\phi_2$ are two functions on $G\times G^*$, 	
temporarily assume that continues a compact support; after (\ref{eq:10}),
we have
\[
U(\phi_1)U(\phi_2) = U(\phi_3)
\]
where $\phi_3$ is given by the formula 
\begin{equation}\label{eq:13}
\phi_3(w) = \int \phi_1(w-w_1)\phi_2(w_1)F(w-w_1,w_1) dw_1;
\end{equation}
as before, $F$ denot the function defined here by (\ref{eq:3}) subsection~3.
If we let $K_i=W(\phi_i)$ for $i = 1, 2, 3,$ (\ref{eq:11}) shows that $K_3$
 is given by
\begin{equation}\label{eq:14}
K_3(x,y)=\int K_1(x,z)K_2(z,y)dz
\end{equation}
we write $K_3 = K_1\times K 2$.
Moreover, the above formulas are extended by continuity 
to the space $L^2(G\times G^*)$, $L^2 (G\times G)$.
We will need the following lemma:
\begin{lemma}\label{l:2}
Let $K\in L^2(G\times G)$; K is in the form $P\otimes Q$, 
with $P$ and $Q$ in $L^2(G)$, 
it is both necessary and sufficient that for any $K'\in L^2(G\times G)$,
$K x K' x K$ and $K$ differs by a scalar factor.
Let $K = P\otimes Q$, $K'= P'\otimes Q'$, with $P, Q, P', Q '$in $L^2(G)$;
for P and P '(resp. Q and Q ') 	
no different from another by a scalar factor, 
it is both necessary and sufficient that for any
$K''=P''\otimes Q''$ with $P''$ and $Q''$ in $L^2(G)$, $K\times K''$
 and $K'\times K''$ (resp. $K''\times K$ and $K''\times K'$) 
are different by a scalar factor.
\end{lemma}
	 	
The second part is obvious, and it is evident that in the first part, 
the required condition is necessary;
to see that it is sufficient, simply apply to the case $K'= P'\otimes Q'$. 
The only consequence of this lemma we need is the following:
\begin{lemma}\label{l:3}
Let $K\to K^s$ an an automorphism of Hilbert space $L^2(G\times G)$
with composition $(K_1,K_2) \to K_1\times K_2$ defined by (\ref{eq:14}).
So there is an automorphism $t$ of  $L^2(G)$ such that, for all $P$ and $Q$ in
$L^2(G)$, $(P\otimes Q)^s = P^t \otimes Q^{\overline{t}}$, 
and $\overline{t}$ is the imaginary conjugation of  $t$, defined by 
$\overline{Q}^{\overline{t}}= (\overline{Q^t})$.
\end{lemma}
Indeed, after Lemma 2, any element $(Q\otimes P)^s$ of $L^2(G\times G)$
 is of the form $P'\otimes Q'$. Choose $P_0$ as $\norm{P_0}=1$; 
then $s$ conserve the norm, we can put 
$(P_0\otimes \overline{P}_0)^s$ in the form $P'_0\otimes Q'_0$ 
with $\norm{P'_0}=\norm{Q'_0}=1$. 	
The second part of lemma 2 shows, whenever $P,Q$ in $L^2(G)$
$(P\otimes \overline{P}_0)^s$ and $(P+0\otimes Q)^s$ are, respectively,
has unique form $P'\otimes Q'_0$ and $P'_0\otimes Q'$. 
If describes $P'= P^t$, $Q'= Q^u$, it is clear that $t, u$ are linear maps
from $L^2(G)$ to $L^2(G)$ and $p^t_0 = P'_0$, $\overline{P}^u_0=Q'_0$;
$s$ preserve the norm of $L^2(G\times G)$, 	
it is the same as $t$ and $u$ in $L^2(G)$.
As we have $P\otimes Q = (P\otimes \overline{P}_0)\times (P_0\otimes Q)$,
it follows that $(P\otimes Q)^s=c\cdot P^t\otimes Q^u$, 
and $c=(P'_0, \overline{Q}'_0)$; for $P=P_0$, $Q=Q_0$, it gives $c=1$.
As 
\[
(P\otimes Q)\times (P\otimes Q) = (P,\overline{Q})P\otimes Q,
\]
we see that for $P'= P^t$, $Q'=Q^u$, $(P',\overline{Q}')=(P,\overline{Q})$;
it follows that $u=\overline{t}$. 
Finally, since
$s^{-1}$ has the same properties as $s$, $t$ and $u$ are invertible and
 are therefore automorphism of $L^2(G)$.  

\subsection{}
Suppose that $s=(\sigma,f)$ is an automorphism of $A(G)$ belonging
to $B_{0}(G)$; the center of the isomorphism between $A(G)$ and
$\mathbf{A}(G)$, as defined in Section 4, operates on $\mathbf{A}(G)$
in the obvious manner. In particular, the transformation of $U(w)$
by $s$ will be $U(w)^{s}=f(w)\cdot U(w\sigma)$. We thereof deduce
immediately an automorphism of the algebras of the operators $U(\varphi)$
introduced in Section 8:

\[
U(\varphi)^{s}=\int U(w\sigma)f(w)\varphi(w)dw,\]


then we can write $U(\varphi)^{s}=U(\varphi^{s})$, where $\varphi^{s}$
is given by

\[
\varphi^{s}(w)=f(w\sigma^{-1})\varphi(w\sigma^{-1}).\]


It follows that $\varphi\rightarrow\varphi^{s}$, which is evidently
a unitary operator in $L^{2}(G\times G^{*})$, leaves the composition
law (13) invariant; it is easy, of course, to verify directly. Therefore,
if we write, in these conditions, $K=W(\varphi)$ and $K^{s}=W(\varphi^{s})$,
in other words, if we define a mapping $K\rightarrow K^{s}$ of $L^{2}(G\times G)$
to $L^{2}(G\times G)$ by the formula

\[
W^{-1}(K^{s})=(W^{-1}(K))^{s},\]


this mapping will satisfy the hypothesis of Lemma 3. After this lemma,
there is therefore an automorphism $t$ of $L^{2}(G)$ such that it
has, for any $P$, $Q$ in $L^{2}(G)$, $(P\otimes Q)^{s}=P^{t}\otimes Q^{\overline{t}}$.
Now we change the notations, writing $P\rightarrow s^{-1}P$ instead
of $P\rightarrow P^{t}$; substitute $Q$ by $\overline{Q}$, and
apply (12); this gives:

\[
(P,U(w)Q)^{s}=(s^{-1}P,U(w)s^{-1}Q).\]


By the definition of $\varphi^{s}$, the first member has the value

\[
f(w\sigma^{-1})\cdot(P,U(w\sigma^{-1})Q)=(P,f(w\sigma^{-1})^{-1}U(w\sigma^{-1})Q)\]


since $(P,Q)$ is antilinear in $Q$ and that $f$ takes its values
in $T$; and the second member is equal to $(P,sU(w)s^{-1}Q)$ since
$s$ is unitary. As the relation obtained is valid for any $P$, $Q$,
we thus have

\[
f(w\sigma^{-1})^{-1}U(w\sigma^{-1})=sU(w)s^{-1}\]


whence, replacing $w$ by $w\sigma$:

\begin{equation}\label{eq:15}
s^{-1}U(w)s=f(w)\cdot U(w\sigma)=U(w)^{s}.
\end{equation}

This amount to say that the inner automorphism determined by $s$
in the unitary group induces the automorphism $s$ on $\mathbf{A}(G)$.
Reciprocally, given $s$, this relation determines an element near
the centralizer of $\mathbf{A}(G)$. Yet, if a unitary operator commutes
with all $U(w)$, it is so with all $U(\varphi)$, therefore with
the operators of the form (11) for any $K\in L^{2}(G\times G)$. For
$K=P\otimes\overline{Q}$, (11) defines the operator $\Phi\rightarrow(\Phi,Q)\cdot P$;
if $\Phi\rightarrow\Phi^{t}$ commute with this one, then we have

\[
(\Phi,Q)\cdot P^{t}=(\Phi^{t},Q)\cdot P\]


for any $P$, $Q$, $\Phi$ in $L^{2}(G)$; then $\Phi\rightarrow\Phi^{t}$
is of the form $\Phi\rightarrow t\cdot\Phi$, where $t$ is a scaler;
if this operator is unitary, we have $t\in T$. We denote by $\mathbf{T}$
the group formed by the operatore of this form; it is the center of
$\mathbf{A}(G)$, and it is also the center of the group of all the
automorphisms of $L^{2}(G)$. We have then shown the following theorem:

\begin{thm}\label{th:1}
The centralizer of $\mathbf{A}(G)$ in the group of the
automorphisms of $L^{2}(G)$ is the center $\mathbf{T}$of these two
groups; furthermore, if $\bB_{0}(G)$ is the normalizer of $\mathbf{A}(G)$
in the same group, any automorphism of $\mathbf{A}(G)$ inducing the
identity on $\mathbf{T}$ is induced on $\mathbf{A}(G)$ by the inner
automorphism determined by an element of $\mathbf{B}_{0}(G)$; and
$\bB_0(G)/T$ is isomorphic to $B_0(G)$, 	
i.e. the group of automorphisms of $A(G)$ inducing identity on $T$.
\end{thm}

\subsection{}
We know that the Fourier transform induces 
an automorphism on a certain space of continuous functions 
(say, quite wrongly, ``indefinitely differentiable a rapid decay'');
 the space $S(G)$ has been introduced primarily for this reason, 
by L. Schwartz ([4], Chap. VII) in the case of $\bR^n$ 
and F. Bruhat [1] in the general case.
We will see that the operators of $\bB_0(G)$ have the same property.

Recall the definition of $S(G)$ for a locally compact abelian group $G$.
First, consider a ``elementary'' group, i.e. of the form 
$G=\bR^n\times \bZ^p\times T^q\times F$, where $F$ is a finite group. 
A polynomial function on $G$ will be, by definition, 
a function that can be written as  a polynomial relative to 
$\bR$ and $\bZ$ coordinate in the product of $G$;
$S(G)$ is the set of all founctions $\Phi$, indefinitely differentiable 
on $G$; such that $P\cdot D\Phi$ is bounded on $G$ 
whenever differential operator invariant under 
translation $D$ and the polynomial $P$; 	
topology of $S(G)$ is induced from all seminorms $\sup\abs{P\cdot D\Phi}$.
In the general case, we will introduce
all couples $(H, H')$ of subgroups of G with the following properties:
\begin{enumerate}[(i)]
\item $H$ is generated by a compact neighborhood of $0$ 
in G (it is open and closed in $G$);
\item $H'$is a compact subgroup of $H$ and $H/H'$ is isomorphic to
 a elementary group.  
\end{enumerate}
For such a couple, we corresponds a family  $S(H, H')$
of continuous functions on $G$, which supported in $H$, and constant on cosets 
of $H'$, 	
by restriction to $H$ and passing to the quotient $H/H'$, 
it  belongs to $S(H/H')$. 	
Then $S(G)$ is the union of $S(H,H')$ 
and  the topology is given by ``inductive limit'' of $S(H/H')$, 
ie a convex set X is a neighborhood of $0$ in $S(G)$ if, 
for all couple $(H,H')$, the image of $X\cap S(H,H')$ in $S(H/H')$ 
is a neighborhood of $0$ in $S(H/H')$.
We intend to see that any $\bs\in \bB_0(G)$ induce an automorphism of $S(G)$;
We only have to show that $\bs$ induce a continuous map 
from $S(G)$ to itself, we will follow step by step of the proof of 
the Theorem~\ref{th:1}.  	
We will write again $t$ instead of $\bs^{-1}$, $P^t$ 	
instead of $\bs^{-1}P$ for $P\in L^2(G)$, 	
and we will prove for the operator $P\to  P^t$.
After the foregoing, if we are given $Q\neq 0$ in $S(G)$,
the map $P\to P^t$ is the following composition:
\[
(a) P\to K=P\otimes Q;\quad (b)K\to \phi=W^{-1}(K);\quad (c) \phi\to \phi^s
\]
\[
(d) \phi^s\to K^s=W(\phi^s);\quad (e) K^s= P^t\otimes Q^{\overline{t}}\to P^t;
\]
it is sufficient to show that maps  from $S(G)$ to  $S(G\times G)$, 
then to $S(G\times G^*)$,
to $S(G\times G^*)$, to $S(G\times G)$ and to $S(G)$ are all continuous.
For $(a)$ is immediately; (e) is also immediately if function 
$K\in S(G\times G)$ is in the form $P\otimes Q$ in the sense of $L^2(G\times G)$
where $P$ and $Q$ are in $S(G)$, 
and then, for $Q\neq 0$ in $S(G)$, the map $P\to  P\otimes Q$ 
is an isomorphism from 
$S(G)$ to a closed sub-space of $S(G\times G)$.
\begin{expl}
\def\tH{{\tilde{H}}}
We have to prove that $E=\Set{P\otimes Q\in S(G\times G)}$ is closed
and $i:P\to P\times P\otimes Q$ is isomorphism.

By the definition of inductive limit, $i$ is countinous iff
$i|_{S(H,H')}$ is countinous.
Suppose $Q\in S(\tH,\tH')$ then 
$i_{S(H,H')}:S(H,H') \to S(H\times \tH, H'\times \tH')\hookrightarrow S(G\times G)$
$P_n$ converges in $S(H,H')\cong S(H/H')$
\[
\abs{p(x,y)D^{\alpha_1}D^{\alpha_2}P_n\otimes Q}\leq \abs{(p(x)q(y)+C)D^{\alpha_1}P_n\otimes D^{\alpha_2}Q}
\to 0
\]
$i$ continous.

\[
\abs{p D^{\alpha} P} < \abs{p(x)D^{\alpha} P\otimes Q} 
\]
\end{expl}

For (b) and (d), it is intended to show that $W$
determines an isomorphism from $S(G\times G^*)$ to $S(G\times G)$, 
but $W$ is composed of the operator
$F(x, y) \to F(y-x,-x)$, which obviously an automorphism of $S(G\times G)$,
and the partial Fourier transform on the second factor of 
the product $G\times G^*$.
It is therefore requires to verify that if $A$ and $B$ are
locally compact abelian groups 
and if $B^*$ is the dual of $B$, the partial Fourier transform
\[
f(a,b)\to f'(a,b^*) = \int f(a,b)\inn{b}{b^*} db
\]
determines an isomorphism from $S(A\times B)$ to  $S(A\times B^*)$.
This is an easy generalization of the theorem similar to
ordinary Fourier Transform.
\begin{expl}
Should be $F(x,y)\to F(y-x,x)$?
\[
\begin{matrix}
W \colon & S(G\times G^*) & \to & S(G\times G) & \to & S(G\times G)&\\
&\phi &\mapsto & \int \phi(x,u^*)\inn{u^*}{y}du^* 
& \mapsto & \int \phi(y-x, u^*)\inn{x}{u^*}du^* &= W(\phi)
\end{matrix}
\]

$f\in S(H,K)$, $H$ can be choosen bigger s.t. $H=H_1\times H_2$,
$K$ can be choosen smaller s.t. $K = K_1\times K_2$.
Partial Fourier Transform maps $f$ to 
$S(H_1\times K_2^\perp, K_1\times H_2^\perp)$.
\[
0\to H_2^\perp \to K_2^\perp \to (H_2/K_2)^\perp\to 0
\]
$\int_G f(x,y)\inn{y}{\alpha^*}dy 
= \int_H f(x,y)\inn{y}{\alpha^*}dy$
is constant if $\alpha^* \in H_2^\perp$.
$\int_G f(x,y)\inn{y}{\alpha^*}dy 
= \int_{H_2/K_2} \int_{K_2} f(x,y+k)\inn{y+k}{\alpha^*} dk\,dy
= \int_{H_2/K_2}  f(x,y)\inn{y}{\alpha^*}
\int_{K_2} \inn{k}{\alpha^*} dk\,dy
=0$
if $\alpha^*\notin K_2^\perp$.

This transform can be consieder as Partial Fourier transform 
from $S(H_1/K_1 \times H_2/K_2)$ 
to $S(H_1/K_1 \times (H_2/K_2)^*)$.
Then is automorphism by elementary case. 
\end{expl}

\subsection{}
I1 we still consider a (c). Since an automorphism $\sigma$ of a $G\times G^*$ 
obviously determine an automorphism of $S(G\times G^*)$,
we (after replacing $G$ to $G\times G^*$) reduces to prove that: 
\begin{expl}
\[
\phi^s = f(w\sigma^{-1})\phi(w\sigma^{-1})
\]
\end{expl}
\begin{prop}\label{p:2}
Let $f$ be a character of second degree of $G$.
Then $\Phi\to \Phi f$ is an automorphism of $S(G)$.
\end{prop}
First, let $G=\bR^n\times \bZ^p \times T^q \times F$, 
with $F$ a finite group; it is easy to see, all remains to show that 
 any differential operator $D$ is invariant by translation on $G$,
there is a polynomial function $P$ on $G$ such that $\abs{Df}<\abs{P}$.
\begin{expl}
\[
\abs{p D^\alpha \Phi f} = \abs{\sum_{\beta\leq \alpha} p C_\beta D^{\alpha-\beta} \Phi D^{\alpha-\beta}f}
\]
this gives $\Phi \to \Phi f$ continuous.
\end{expl}
This is not difficulty to varify the  expressing $f$ on the  coset of 
$\bR^n\times T^q$ in $G$ by using the formula (\ref{eq:1}) in subsection~1,
and noting that, for $\bR^n\times T^q$, $f$ is necessarily of the form 
$e^{iF(x)}\chi(x,y)$ where $x\in \bR^n$, $y\in T^q$, 
$F$ is a quadratic form on $\bR^n$ and 
$\chi$ is a character of $\bR^n\times T^q$. 	
\begin{expl}
\[
\frac{f((x_1,y_1)+(x_2,y_2))}{f((x_1,y_1))f((x_2,y_2)} = \inn{(x_1,y_1)}{(x_2,y_2)\rho}
\]
$\rho\colon G\to G^*$ is a morphism.
$G^* = \bR^{n*}\times \bT^{q*} \cong \bR^n\times \bZ^q$ 
So $\rho$ maps into $\bR^{n*}\times 0$ by connectness. 
The image of $0\times \bT^{q}$ is trivial by campactness and connectness. 
Then $\rho$ only depends on $x$.
$2$ is auto. So $f(x,y) = \inn{x}{2^{-1}x\rho}\chi(x,y)$
(by section~1, ker of $X_2(G)=X_1(G)\times X^0_2(G)$)
\end{expl}
Moving to the general case, let $\rho$ be a symmetric morphism of from 
$G$ to $G^*$ associated to $f$, and give a subgroup $H$ of $G$ 
generated by a compact neighborhood of $0$.
For a subgroup $H'$ of $H$ satisfies the condition $(ii)$ 
of the definition of $S(H, H')$, 	
it is both necessary and sufficient,
 as we know (see [1], $n^o$~9, p. 60), that
the group $H'_*$ which is associated by duality in $G^*$ 
(the ``orthogonal'' of $H'$) is generated by a compact neighborhood of $0$, 
\begin{expl}
\begin{enumerate}[(1)]
\item
(c.f. On the Schwartz-Bruhat Space and the Peley-Winer Theorem 
for Locally Compact Abelian Groups Scott Osborne.)
$K$ is ``good'' iff $G/K$ is a Lie group.

$K$ is ``good'' iff $K^\perp$ is open and compactly generated
(i.e. $K^\perp$ is generated by a compact neighborhood of $0$).

This gives $H'_*$ comapct generated
\item
$0\to H'_* \to H^* \to H'^* \to 0$ 
(c.f. A First Course in Harmonic Analysis 109 Ex~7.8)
$H'_* \cong (H/H')^*$
$H'$ compact, then $H'^*$ discrete, So $H'_*$ is open.
\item (c.f. Scott Osborne)
$G$ is compactly generated iff $G^*$ is a Lie group.
\proof
Let $C$ be a compact subset of $G$, $U$ a neighborhood of $1$ in $\bT$. 
Let $W(C,U) = \Set{x^*\in G^*| x^*(C)\subset U}$ 
be a neighberhood of $0$ of $G^*$.
Let $<C>$ is the group $C$ generated, then $F^\perp \subset W(C,U)$. 
Suppose that $U$ contains no nonidentity subgroups of $\bT$
(we always can choose such $U$).
Let $E$ be a subgroup in $W(C,U)$, 
clearly $\Set{\inn{c}{e}|e\in E}$ is a subgroup of $\bT$ for any fixed $c\in C$.
Then $\inn{c}{e}=1$ for all $c\in C, e\in E$, $E \subset <C>^\perp$.
Thus $G$ is compactly generated(for example by $C$) iff $G^*$ has
no small subgroups, i.e. iff $G^*$ is a Lie group. 

(In 1953, Hidehiko Yamabe obtained the final answer to Hilbert’s Fifth Problem: 
A connected locally compact group G is 
a projective limit of a sequence of Lie groups,
and if G has ``no small subgroups'', then it is a Lie group.
``no small subgroups''  if there is a neighbourhood N of $e$
 containing no subgroup bigger than the trivial one.

The locally compact abelian group case was solved in 1934 by Lev Pontryagin. )
\end{enumerate}
end.
\end{expl}
then the group $H'_* +H\rho$ has the same property, 
this shows that replacing $H'_*$ with $H'$  a smaller group satisfing (ii),
 we can ensure that we have $H\rho \subset H'_*$. 
\begin{expl}
Note that if $K<G$, $K^{\perp\perp} = K$.
Clearly $K \leq K^{\perp\perp}$, If $K\lneq K^{\perp\perp}$,
 $K^{\perp\perp}/K$ nontrivial, then exists nontrivial 
character $\alpha^*$ on $K^{\perp\perp}/K$. 
Extend it to $G$, call it $\alpha^*$ also.
Then $\alpha^* \in K^\perp$ but not in $K^{\perp\perp\perp}$.
It's clearly that $K^\perp = K^{\perp\perp\perp}$($\perp$ is a Galois relation).
This leads a contradiction.

Now let  $\tilde{H}'_*=H'_* + H\rho$.
Then $\tilde{H}' = \tilde{H}_*^{'\perp} < H_*^{'\perp} = H'$.

By above $H/\tilde{H}'$ is a Lie group. But 
compactly generated abelian Lie group is elementary!
\end{expl}
By (\ref{eq:1}) subsection~1, this gives $f(h + h')=f(h)f(h')$ 
for each $h\in H$, $h' \in H'$.  
\begin{expl}
\[
\frac{f(h+h')}{f(h)f(h')} = \inn{h'}{h\rho} = 1
\]
\end{expl}
In particular, $f$  induces a character of $H'$, we can extend to a
character of $G$  of the form $\inn{h'}{a^*}$ with $a^*\in G^*$. 	
\begin{expl}
\[
0\to H'\to G\to G/H' \to 0
\]
induced 
\[
0 \to (G/H')^* \to G^* \to H'^* \to 0
\]
So we can extend it. 
\end{expl}
Replacing $H'_*$ by the group generated by $H'_*$ and $a^*$,
we can make that $a^*$ in $H'_*$;
\begin{expl}
as above
\end{expl}
since, $f$ induces the constant $1$ on $H'$ and 
and is constant in each coset of $H'$.
\begin{expl}
\[
f(h+h') = f(h)f(h')\inn{h'}{h\rho} = f(h)
\]
\end{expl} 
After what has been proved in the case $G$ is a  elementary group,
it follows, by passing to the quotient,
 that $\Phi\to \Phi f$ determine an automorphism of $S(H, H')$.
As is the case, for any $H$ satisfying (i), provided that $H'$ 
has been made small enough and satisfies (ii) it completed the
prove.
(given the definition of the topology of $S(G)$ by inductive limit).

\subsection{}
The homomorphism $\bs \to s = (\sigma, f)$ from $\bB_0(G)$ to $B_0(G)$
 determined by (\ref{eq:15}) will be denote $\pi_0$
  and is called the {\it canonical projection} 
from the first group to the second.
In general (as shown below the example of the type of local groups), 
there is no section of $\bB_0(G)$ over $B_0(G)$ 
is also a subgroup of $\bB_0 (G)$. 
But it is very useful to know that we can at least define
 sections over the subgroups and
subsets of $B_0(G)$ have been introduced in $n^0~6$ and $7$. 

Let $\Phi\in L^2(G)$. For any automorphism $\alpha$ of $G$ 
we set
\[
\bd_0(\alpha)\Phi(x)=\abs{\alpha}^{\frac{1}{2}}\Phi(x\alpha).
\]
\begin{expl}
$\bd_0(\alpha)$ is unitary.
$\int \abs{\alpha}\abs{\Phi(x\alpha)}^2 = \int \abs{\Phi(x)}^2$
\end{expl}
For any seconde degree characher $f$ of $G$, we set
\[
\bt_0(f)\Phi(x) = \Phi(x)f(x).
\]
For any isomorphism $\gamma$ from $G^*$ to $G$, we set
\[
\bd'_0(\gamma)\Phi(x) = |\gamma|^{-\frac{1}{2}} \Phi^*(-x\gamma^{*-1}),
\]
\begin{expl}
Fourier transform is unitary.
\end{expl}
where, $\Phi^*$ denotes the Fourier transform of $\Phi$.
We can easily verify that $\bd_0$, $\bt_0$, $\bd'_0$ lifting of  
$d_0, t_0, d'_0$ defined in $n^o 6$ to  $\bB_0(G)$, 
i.e. $d_0 = \pi_0 \circ \bd_0$, $t_0=\pi_0\circ \bt_0$, 
$d'_0 = \pi_0 \circ \bd'_0$. 
\begin{expl}
Let $(w,t) = ((u,u^*),t)$,
\begin{enumerate}[(1)]
\item
$\bd_0(\alpha^{-1})(w,t)\bd_0(\alpha)$
maps $\Phi(x)$ to $\abs{\alpha}^{\frac{1}{2}}\Phi(x\alpha)$
to $t \abs{\alpha}^{\frac{1}{2}}\Phi((x+u)\alpha)\inn{x}{u^*}$
to $t\Phi((x\alpha^{-1}+u)\alpha)\inn{x\alpha^{-1}}{u^*}
=t\Phi(x+u\alpha)\inn{x}{u^*\alpha^{*-1}}$.
But 
\[
((w,t)d_0(\alpha)) = ((u\alpha, u^*\alpha^{*-1}),1)
\]
\item
$\bt_0(f^{-1})(w,t)\bt_0(f)$
maps $\Phi(x) \mapsto \Phi(x)f(x) \mapsto
t\Phi(x+u)f(x+u)\inn{x}{u^*} \to t \Phi(x+u) f(x+u) f^{-1}(u)\inn{x}{u^*}
 = \Phi(x+u) f(x)\inn{x}{u\rho+u^*} $.\\
$(w,t)\left(\begin{pmatrix}1& \rho \\ 0 & 1\end{pmatrix},f\right) 
= ((u,u\rho +u^*),f(u)t)$ act on $\Phi$ gives $t\Phi(x+u)f(u)\inn{x}{u\rho+u^*}$
\item 
$\bd_0(-\gamma^*) \bd_0(\gamma) = \id$
$\bd_0(-\gamma^*)(w,t)\bd_0(\gamma)$ maps $\Phi(x)$ 
to $\abs{\gamma}^{-1/2}\Phi^*(-x\gamma^{*-1})$
to  $\abs{\gamma}^{-1/2}t\Phi^*(-(x+u)\gamma^{*-1})\inn{x}{u^*}$
to  
\[
\begin{split}
&\abs{-\gamma^*}^{-1/2}\abs{\gamma}^{-1/2}t
\int \Phi^*(-(y+u)\gamma^{*-1})\inn{y}{u^*}\inn{y}{x\gamma^{-1}}dy\\
=& \abs{\gamma}^{-1} t
\int \Phi^*(-y\gamma^{*-1})\inn{y-u}{u^*+x\gamma^{-1}}dy\\
=& \abs{\gamma}^{-1} t
\int \Phi^*(-y\gamma^{*-1})\inn{y-u}{u^*+x\gamma^{-1}}dy\\
=& \abs{\gamma}^{-1} t \inn{-u}{u^*+x\gamma^{-1}}
\int \Phi^*(-y\gamma^{*-1})\inn{y}{u^*+x\gamma^{-1}}dy\\
=&  t \inn{-u}{u^*+x\gamma^{-1}}
\int \Phi^*(y)\inn{-y\gamma^*}{u^*+x\gamma^{-1}}dy\\
=&  t \inn{-u}{u^*+x\gamma^{-1}}
\int \Phi^*(y)\inn{-y}{u^*\gamma+x}dy\\
=& t \inn{-u}{u^*+x\gamma^{-1}}\Phi(u^*\gamma+x)
\end{split}
\]
$(w,t)d_0(\gamma) = ((u^*\gamma, -u\gamma^{*-1}),t\inn{u}{-u^*})$,
maps $\Phi$ to $t\Phi(x+u^*\gamma)\inn{u}{-u^*}\inn{x}{-u\gamma^{*-1}}
=t\Phi(x+u^*\gamma)\inn{-u}{u^*}\inn{-u}{x\gamma^{-1}}$
\end{enumerate}
\end{expl}
Moreover, $\bd_0$ and $\bt_0$ are monomrophisms
from $\bB_0(G)$ to the group of the automorphisms of $G$, and the group
$X_2(G)$ respectively; and when $\alpha$, $f$, $\gamma$ are like above,
we have:
\[
\bd_0(\alpha)^{-1}\bt_0(f)\bd_0(\alpha) = \bt_0(f^\alpha),\quad
\bd'_0(\gamma \alpha) = \bd'_0(\gamma)\bd_0(\alpha),\quad
\bd'_0(\alpha^{*-1}\gamma) = \bd_0(\alpha)\bd'_0(\gamma).
\]
\begin{expl}
\[
\begin{split}
\bd'_0(\gamma\alpha)\Phi(x)
&= \abs{\gamma \alpha}^{-\frac{1}{2}} \int \Phi(y) \inn{y}{-x{\gamma\alpha}^{*-1}}dy \\
&= \abs{\gamma}^{-\frac{1}{2}} \abs{\alpha}^{-\frac{1}{2}} \int \Phi(y) 
\inn{y}{-x\gamma^{*-1}\alpha^{*-1}}dy \\
&= \abs{\gamma}^{-\frac{1}{2}} \abs{\alpha}^{-\frac{1}{2}} \int \Phi(y) 
\inn{y\alpha^{-1}}{-x\gamma^{*-1}}dy \\
&= \abs{\gamma}^{-\frac{1}{2}} \abs{\alpha}^{\frac{1}{2}} \int \Phi(y\alpha) 
\inn{y}{-x\gamma^{*-1}}dy
\end{split}
\]
\[
\bd'(\alpha^{*-1}\gamma) = \abs{\gamma}^{-\frac{1}{2}} \abs{\alpha^{*-1}}^{-1/2} 
\Phi^*(-x\alpha \gamma^{*-1})
\]
\end{expl}
The propersiton~\ref{p:1} of $n^o~7$ 
allows us raise $B_0 (G)$ for all the element of the set
$\Omega_0(G)$ defined in this propersition.
 Indeed, according to this, any $s\in \Omega_0(G)$
 is has a unique form as (\ref{eq:8}); 
 $s$ is given by (\ref{eq:8}), we set
\[
\br_0(s)=\bt_0(f_1)\bd'_0(\gamma)\bt_0(f_2).
\]
An easy calculation shows this formula;  
by writing, as usual, $s = (\sigma ,f)$,  
$\sigma = \begin{pmatrix}\alpha &\beta \\ \gamma & \delta\end{pmatrix}$,
 we obtains
\[
\br_0(s)\Phi(x) = 
\abs{\gamma}^{\frac{1}{2}}\int \Phi(x\alpha + x^*\gamma) f(x,x^*) dx^*.
\]

\begin{expl}
$\br_0(s)$ maps $\Phi(x)$ to $\Phi(x)f_2(x)$ 
to $\abs{\gamma}^{-1/2}\int \Phi(y) f_2(y)\inn{y}{-x\gamma^{*-1}}dy$
to $\abs{\gamma}^{-1/2}f_1(x)\int \Phi(y) f_2(y)\inn{y}{-x\gamma^{*-1}}dy $.
\[
\begin{split}
=& \abs{\gamma}^{1/2}f_1(x)\int \Phi(y^*\gamma) f_2(y^*\gamma)
\inn{y^*\gamma}{-x}dy \\
=& \abs{\gamma}^{1/2}f_1(x)\int \Phi(y^*\gamma+x\alpha) f_2(y^*\gamma+x\alpha)
\inn{y^*+x\alpha\gamma^{-1}}{-x}dy 
\end{split}
\]

By $n^o~5$,
\begin{align*}
f(u,u^*) &= g(u)h(u^*)\inn{u^*\gamma}{u\beta}\\
g(u_1+u_2) &=g(u_1)g(u_2)\inn{u_1}{u_2\alpha\beta^*}\\
h(u^*_1+u^*_2) &= h(u^*_1)h(u^*_2)\inn{u_1^*\gamma \delta^*}{u^*}
\end{align*}
Then
\[
\begin{split}
& f_2(y^*\gamma + x\alpha)\inn{-x}{y^*+x\alpha\gamma^{-1}}f_1(x)\\
=&f(0,y^*+x\alpha\gamma^{-1})f(x,-x\alpha\gamma^{-1})
\inn{-x}{y^*+x\alpha\gamma^{-1}}\\
=&g(0)h(y^*+x\alpha\gamma^{-1}g(x)h(-x\alpha\gamma^{-1})\inn{-x\alpha}{x\beta}
\inn{-x}{y^*+x\alpha\gamma^{-1}}\\
=&g(x)h(y^*)\inn{-x\alpha\delta^*}{y^*+x\alpha\gamma^{-1}}^{-1}
\inn{-x\alpha}{x\beta}\inn{-x}{y^*+x\alpha\gamma^{-1}}\\
=&g(x)h(y^*)\inn{x\alpha\delta^*-x}{y^*+x\alpha\gamma^{-1}}
\inn{-x\alpha}{x\beta}\\
=&g(x)h(y^*)\inn{x(\alpha\delta^*\gamma^{*-1}-\gamma^{*-1})}{y^*\gamma+x\alpha}
\inn{-x\alpha}{x\beta}\\
=&g(x)h(y^*)\inn{x(\alpha\gamma{-1}\delta-\gamma^{*-1})}{y^*\gamma+x\alpha}
\inn{-x\alpha}{x\beta}\\
=&g(x)h(y^*)\inn{x\beta}{y^*\gamma+x\alpha}
\inn{-x\alpha}{x\beta}\\
=&g(x)h(y^*)\inn{x\beta}{y^*\gamma}\\
=&f(x,y^*)
\end{split}
\]
\end{expl}

The conditions of this formula valid are obviously same as 
those of the formula of  definition transformation of Fourier, 
who was useful has to clarify $\bd'_0$; 
for example, 
it is valid almost everywhere if $\Phi\in L^2(G)\cap L^1(G)$;
it is valid for all x if $\Phi\in S(G)$,
the two members then defining a same function of $S(G)$.

\subsection{}
One obtains the important result on the {}``splitting'' of $\mathbf{B}_{0}(G)$,
the kernel of the mapping in Section 13, the relation between the
elements of $B_{0}(G)$; that is what we are going to do for the relation
(9) of Section 7. As in Section 7, we considered then a nondegenerate
character of the second degree $f$ of $G$, associated with the symmetric
isomorphism $\rho$ of $G$ on $G^{*}$. For a moment, denote by $\mathbf{s}$
and $\mathbf{s}'$ respectively the operators which give the first
and the second member of (9) while replacing $\mathbf{d}_{0}^{'}$,
$\mathbf{t}_{0}$ with $d_{0}^{'}$, $t_{0}$. In addition, with $\Phi$
being assumed to be continuous with compact support at this moment,
let $\Phi_{1}=\Phi\star f$, where the notation denotes naturally
the usual product of composition

\[
\Phi_{1}(x)=\int\Phi(u)f(x-u)du;\]
 then an easy calculation shows that $\mathbf{s}\Phi$, $\mathbf{s}'\Phi$
are defined by

\[
\mathbf{s}\Phi(x)=|\rho|\Phi_{1}^{*}(x\rho),\]
 \[
\mathbf{s}'\Phi(x)=|\rho|^{\frac{1}{2}}\Phi^{*}(x\rho)\cdot f(x)^{-1}.\]


The operators introduced here are all unitary, so it follows that
the operator $\Phi\rightarrow\Phi\star f$ is continuous with respect
to $L^{2}(G)$. Moerover, the relation (9), which if written now as
$\pi_{0}(\mathbf{s})=\pi_{0}(\mathbf{s}')$, implies that $\mathbf{s}$
and $\mathbf{s}'$ can differ only by a scalar factor with absolute
value 1. We write $\mathbf{s}=\gamma(f)\mathbf{s}'$, which gives,
while substitute $x^{*}\rho^{-1}$ with $x$:

\[
\mathfrak{I}(\Phi\star f)=\gamma(f)|\rho|^{-\frac{1}{2}}\mathfrak{I}(\Phi)\cdot g,\]
 (17)where $g$ is the character of the second degree of $G^{*}$,
associated with $-\rho^{-1}$, and defined by

\[
g(x^{*})=(x^{*}\rho^{-1})^{-1}.\]
 In accordance with the usual conventions in the theory of Fourier
transform, we express (17) by saying that $\gamma(f)|\rho|^{-\frac{1}{2}}g$
is the Fourier transform of $f$. We have thus shown the following: 
\begin{thm}
Let $f$ be a nondegenerate character of the seconde degree of $G$,
associated with the symmetric isomorphism $\rho$ of $G$ onto $G^{*}$.
Then $f$ admits a Fourier transform $\mathfrak{I}(f)$, defined by
the formula \[
\mathfrak{I}(f)(x^{*})=\gamma(f)|\rho|^{-\frac{1}{2}}f(x^{*}\rho^{-1})^{-1},\]
 where $\gamma(f)$ is a scalar factor with absolute value 1.
\end{thm}
Let us repeat that this assertion must be understood in the following
sense: the mapping $\Phi\to\Phi\star f$ extends by the continuity
with $L^{2}(G)$, and, for any $\Phi\in L^{2}(G)$, one has $\mathfrak{I}(\Phi\star f)=\Phi^{*}\cdot\mathfrak{I}(f)$.
Using transport of structure by means of the isomorphisms $\rho$
of $G$ onto $G^{*}$, one concludes from it that then one has also
$\mathfrak{I}(\Phi f)=\Phi^{*}\star\mathfrak{I}(f)$. By means of
the Proposition 2 in Section 12, one sees moreover that, for $\Phi\in\mathbf{S}(G)$,
both members of the last relation are continuous functions, thus that
equality holds, not only in the sense of $L^{2}(G^{*})$, but also
for every point, then, by transport of structure, that it is the same
for the preceding relation. We state this result explicitly in the
form of corollary: 
\begin{cor}
The hypothesis and notations being those of theorem 2, in addition
let $\Phi\in\mathbf{S}(G)$ .
\end{cor}

\setcounter{ssection}{16}
\subsection{}
\begin{equation}\label{eq:22}
\Theta(z+\zeta)=\Theta(z)F(\zeta,z)^{-1} 
\quad (z\in G\times G^*, \zeta \in \Gamma\times \Gamma_*)
\end{equation}

\subsection{}
We will define $H(G,F)$ the Hilbert space of solutions $\Theta$ of (\ref{eq:22}),
 locally intgerable on $G\times G^*$ 
and such as $\|\Theta\|_Q< +\infty$, this space being provided with the norm $\|\Theta\|_Q$.

\subsection{}
We now define  $B_0(G, \Gamma)$ be the subgroup of $B_0(G)$ 
forms by elements $s=(\sigma, f)$ in $B_0(G)$ such that
$f$ take value $1$ on $\Gamma\times \Gamma_*$ and $\sigma$ 
induce a automorphism of $\Gamma\times \Gamma_*$. For all 
$s=(\sigma, f)$ of $B_0(G,\Gamma)$, we define an operator 
$\br_\Gamma(s)$ of $H(G,\Gamma)$ by the formula
\begin{equation}\label{eq:25}
\br_\Gamma(s)\Theta(z) = \Theta(z\sigma)f(z),
\end{equation} 
we checks that immediately it transforms 
any solution $\Theta$ of (\ref{eq:22}) 
to a solution of the same equation.
It is then obvious that $\br_\Gamma$ is a unitary representation of $B_0(G, \Gamma)$
.

\subsection{}
\begin{thm}[Theorem 5]
If $f$ is a charctor of seconde degree of $G$, 
taking value $1$ on a closed subgroup $\Gamma$ of $G$;
let $G^*$ be the dual of $G$, $\Gamma_*$ be the
subgroup of $G^*$ correspond to $\Gamma$,and 
suppose that symmetric homomorphism  $\sigma$ from G to G* associate to $f$
is an isomorphism from $(G, \Gamma)$ to $(G^*, \Gamma_*)$.  
Then $\gamma(f)=1$.
\end{thm}

\setcounter{ssection}{22}

\section{Application  with Quadratic reciprocity}
\subsection{}
To be consistent as much as possible with our algebraic notations in Chapter I,
we assume, that $X$ is a vector space (always finite dimension) on a field $k$,
denotes its dual by $X^*$, and let  $[x, x^*]$, for $x\in X, x^* \in X^*$,
the value on $x$ of the linear form $x*$ of $X$;
we will identify $X$ with its bidual $(X^*)^*$ 
by  formula 
\[
[x, x^*] = [x^*, x];
\]
and, when $\alpha$ is a linear map (called `` morphism'') from a space $X$ 
to a space $Y$, we will denote $\alpha^*$ as its transport, 
it is a linear map from $Y^*$ to $X^*$, for $X \in X$, $y^* \in Y^*$, 
defined by the formula 
\[
[x\alpha, y^*] = [x, y^*\alpha^*].
\]
Any bilinear form on $X\times Y$ can be written as $[X, y\alpha]$,
where $\alpha$ is a morphism form $Y$ to $X^*$;
For $X=Y$, we said $\alpha$ is symmetric if $[x, y\alpha]$
 is symmetric in $x$ and $y$, therefore if $\alpha=\alpha^*$.

If $f$ is a quadratic form on X, we have, whenever $x \in X, y \in X$: 
\[
f (x + y) -f(x) - f (y) = [x, y\rho], 
\]
where $\rho$ is a symmetric morphism form $X$ to $X^*$; we 
say $f$ and $\rho$ are associated with each other;
$f$ is called non-degenerate if $\rho$ is an isomorphism from $X$ to  $X^*$,
 and additive if $\rho=0$.
There is no additive quadratic form other that $0$, if $k$ has characteristic  $2$. 
If $k$ is not characteristic 2,  all symmetric morphism $\rho$ form $X$ $X^*$
is associated with a unique quadratic form $f$,
as we know that which is given by $f(x) = [x, 2^{-1}x\rho]$.
In all cases,  $2f(x)= [x, x\rho]$. We let $Q(X)$ 
be the vector space of quadratic forms on $X$,
 and $Q_a(X)$ be the subspace of $Q(X)$ consisting the additive forms. 

\subsection{}
Firstly let $k$ be a \emph{local field}; by which we mean a locally
compact non-discrete commutative (topological) field; it is then,
either isomorphic to $\mathbf{R}$ or $\mathbf{C}$, or with a discrete
valuation%
\footnote{which is function $v:k\to\mathbb{Z}\cup\{\infty\}$ such that 
$v(x\cdot y)=v(x)+v(y)$;
$v(x+y)\geq min\{v(x),v(y)\}$; and $v(x)=0\Leftrightarrow x=\infty$%
}; and, in the latter case, it is a finite extension, either of a field
$\mathbf{Q}_{p}$ ($p$-adic completion of the field $\mathbf{Q}$
of rationals) if it is of characteristic $0$, or of the field of
formal series with one parameter over the prime field $\mathbf{F}_{p}$
if it is of characterist $p$. If $k$ has discrete valuation, one
denote by $\mathfrak{o}$ the ring of integers of $k$ and $\pi$
a prime element of $\mathfrak{o}$, i.e. a generator of the prime
ideal $\mathfrak{p}$ of $\mathfrak{o}$; one denotes by $q$ the
number of elements of the finite field $\mathfrak{o}/\mathfrak{p}$.
One chooses once and for all a character $\chi$ of the additive group
of $k$, where we assume only that it is nontrivial (i.e. that it
is not of constant value 1); it is possible, as one knows, to choose
$\chi$ in a canonical way, but that is irrelevant to us. One knows
that $\chi(xy)$ is then a bicharacter of $k\times k$ which puts
$k$ self-dual in the sense of the thoery of locally compact abelian
groups. More generally, suppose $X$ is a vector space (of finite
dimension, as always) over $k$; suppose $X^{*}$ is the dual; $X$
and $X^{*}$ being provided with the obvious topology, one is then
able to identify $X^{*}$ with the dual of $X$ in the sense of the
theory of locally compact abelian groups so as to have, for any $x\in X$
and any $x^{*}\in X^{*}$:\[
<x,x^{*}>=\chi([x,x^{*}]).\]
This identification (which depends on the choice of $\chi$) will
be made from now on once and for all, so that one will not have to
distinguish between the algebraic dual and the dual in the sense of
Chapter I.

If $f$ is a quadratic form on the space $X$ over $k$, $\chi\circ f$
is a character of the second degree on $X$ in the sense of Chapter
I, Section 1; the morphism of $X$ into $X^{*}$ associated with $\chi\circ f$
is the same one which is associated to $f$; in particular, whenever
$\chi\circ f$ is nondegenerate, it is necessary and sufficient that
$f$ is so, and, when this is the case, one can apply to $\chi\circ f$
the theorem $2$ of Chapter I, Section $14$, which defines a number
$\gamma(\chi\circ f)$ of absolute value $1$. For short, one writes
$\gamma(f)$ instead of $\gamma(\chi\circ f)$, but it should not
be forgotten that the symbol $\gamma(f)$ depends on the choice of
$\chi$. On the other hand, as $|\gamma(f)|=1$, it does not depend
on the choice of Haar measure which occurs in its definition, since,
when the measures are changed, it only modifies the formulas in theorem
$2$ of Chapter I, Section $14$, and its corollaries, by a positive
real factor; this remark allows even, in the calculation of $\gamma(f)$
by means of these corollaries, to abandon the convention, followed
in Chapter I, to always take, in the dual $G^{*}$ of a group $G$,
the dual measure of the one we choose in $G$.



\subsection{}
The remark above shows in particular that,
 if $f$ and $X$ are as above and if $f'=f\circ \alpha$, 
where $\alpha$ is an isomorphism from a space $X'$ to $X$,
we have $\gamma(f')=\gamma(f)$;
in other words, $gamma$ has same value on the two ``equivalent'' forms
$f,f'$.
In additional, since $\chi\circ(-f)$ is the imaginary conjugation of 
$\chi\circ f$, we have $\gamma(-f)=\gamma(f)^{-1}$. 
If $-1$ is a square in $k$ (n particular if $k$ has characteristic $2$),
$-f$ is equivalent to $f$; 
In this case, we have $\gamma(f) = \pm 1$ for all $f$.

\begin{prop}\label{p:3}
The map $f\to \gamma(f)$ determin a character of the Witt group of local field
$k$.
\end{prop} 
As one knows, a non-degnenerate form $f$ corresponds to the element $0$
of the Witt group(we will say that it is {\it trivial})
if it is equivalent to the form $\sum_1^n x_i x_{n+i}$ on $k^{2n}$ 
for some $n$; that is also means 
that $f$ is equivalent to the form $[x, x^*]$ on $X\times X^*$ 
for a certain choice of $X$.
But then the corollary of Theorem~$5$ of Chapter~I, $n^o$~20,
shows that $\gamma(f)=1$.In addition, $f_1,f_2$
are  non-degenerate forms on spaces $X_1$, $X_2$;
that means the form $f$ given by  $f(x_1, x_2) = f_l (x_l) +f_2 (x_2)$
 on the direct sum of $X_1$ and $X_2$; it is non-degenerate, obviously, 
according to the definition of $\gamma$ in theorem~2 of Chapter I, $n^o$~14,
 $\gamma(f) =\gamma(f1)\gamma(f_2)$. 
That proves the proposition.  

\def\fo{\mathfrak{o}}
\setcounter{ssection}{28}
\subsection{}
Suppose that $k$ is a algebric number field or a function field of 1 dimension 
on a finit field. We will define by $k_v$ be the completion of $k$,
 by $\fo$ be the integer ring of $k_v$ for each $k_v$ has discrete valuation,
and $A_k$ be the adele ring of $k$. Let $X_k$ be a vector spaces 
(finite dimension) of $k$; we set $X_A= X_k\otimes A_k$, and for any $v$, 
$X_v=X_k\otimes k_v$. If $X^\circ$ is a basis of $X_k$ for $k$, we denote
$X_v^\circ$, when $k_v$ has discrete valuation, the set of points of $X_v$
whose coordinate respect to base $X^\circ$ in $\fo_v$;
is a lattice in $X_v$. We denote $S$ be the set of all 
 finite complement of $k$, containing all which are isomorphic 
to $\bR$ or $\bC$. Under these conditons, $X_A$ is the union ( and same, 
as a topological space, inductive limit) of product
\begin{equation}\label{eq:29}
X_S^\circ = \prod_{v\in S} X_v \times \prod_{v\in S} X_v^\circ.
\end{equation}

Every compact set of $X_A$ is contained in a set of form $\prod C_v$, 
where $C_v$ is for all $v$, a compact set of $X_v$ and for almost all $v$
(it is to say that all except finite number of $v$), is equal to $X_v^\circ$.
We conclude from this any subgroup of $X_A$, 
generate by a compact neighborhood of $0$, contained in a subgroup of form
$H=\prod H_v$, where $H_v$ is equal to $X_v$ when $k_v$ is isomorphic to $\bR$
or $\bC$, a lattice in $X_v$ when $k_v$ has discrete valuation, and is $X_v$
for almost all $v$. Suppose $H$ is choose as this, then any compact subgroup 
$H'$ of $H$ such that $H/H'$ is a elementary group(cf. Chapter I, $n^o$~11)
contain a similer subgroup of form $H'=\prod H'_v$,  where $H'_v$ is 
$\Set{0}$ when $k_V$ is isomorphic to $\bR$ or $\bC$, is a lattic of
$X_v$ containd in $H_v$ for $k_v$ has discrete valuation, is $X_v^\circ$
for almost all $v$. We easily deduce form the sturcture of $S(X_A)$;
$S(X_A)$ contain, in any case, all the function of the 
form $(x_v) \to \prod \Phi_v(x_v)$, where $\Phi_v \in S(X_v)$ for all 
$v$, and for almost all $v$, $\Phi_v$ is the characteristic function 
of $X_v^\circ$; for $k=\bQ$, or for $k$ of characteristic $p\neq 0$, 
$S(X_A)$ is the set of all finite linear combination, has constant coefficients,
of function obtained that (for the corresponding assertion when 
$k$ is an algebraic field of numbers other that $\bQ$, 
see further, with $n^o$~39).

\section{The Metaplectic group(local and adelie cases)}
\setcounter{ssection}{30}
\subsection{}
 As in $n^o$~23, Let $X$ be a vector space (finit dimension, as usual) 
on a field $k$;keeping the notations introduced in $n^o$~23, we 
will consider automorphisms $z\to z\sigma$ on $X\times  X^*$,
as in Chapter I (cf $n^o$~3), in matrix form
\[
\sigma = \begin{pmatrix} \alpha & \beta\\ \gamma & \delta
\end{pmatrix}
\]
As in $n^o$~3, we set 
\[
\sigma^I = \begin{pmatrix} \delta^* & -\beta^* \\ -\gamma^* & \alpha^*
\end{pmatrix}
\]
For $(X\times X^*)\times (X\times X^*)$, we consider the bilinear form
\[
B(z_1,z_2) = [x_1,x_2^*] \quad (z_1 = (x_1,x_1^*), z_2=(x_2,x_2^*)),
\]
and we call an automorphism $\sigma$ of $X\times X^*$ is symplectic 
if it leaves the bilinear $B(z_l,z_2) - B(z_2,Z_1)$ form invariant; 
for it is true, it is necessary and sufficient that $\sigma \sigma^I=l$. 
These automorphisms form a group called the symplectic group of $X$ 
and which one denote $Sp(X)$.

On $X\times X^* \times k$, we put a group structure by the following law 
(analog (4) of Chapter I):
\begin{equation}\label{eq:30}
(z_1,t_1)(z_2,t_2)=(z_1+z_2,B(z_1,z_2)+t_1+t_2);
\end{equation}
\def\fA{{\mathfrak{A}}}
$\fA(x)$ denote the group defined above.
For an automorphism $\sigma$ of $X \times X^*$ and a quadratic form $f$ 
on $X \times X^*$; so that the formula
\begin{equation}\label{eq:31}
(z,t) \to (z\delta, f(z)+t) \quad (z\in X\times X^*, t\in k)
\end{equation}
define a automorphism of $\fA(X)$, it is necessary and sufficient that $\sigma$ and $f$ 
satisfy the relation
\begin{equation}\label{eq:32}
f(z_1+z_2)-f(z_1)-f(z_2) = B(z_1\sigma, z_2\sigma)-B(z_1,z_2),
\end{equation}
analog (6) of Chapter~I.
Then, we will indicate the automorphism 
of $\fA(X)$ defined by (\ref{eq:31});   
the group forms by these automorphisms called the  
pseudosymplectic group of $X$ and will be denote $Ps(X)$.  
The group law in $Ps(X)$ is given by 
\[
(\sigma,f)\cdot (\sigma',f') = (\sigma\sigma',f''), 
\]
where $f''$ is the quadratic form defined by 
\[
f''(z) =f(z)+f'(z\sigma).
\]

If (\ref{eq:32}) is satisfied, 
the second member must be symmetrical in $z_1$ and $z_2$;
so it is necessary and sufficient that $\sigma$ is symplectic;
thus $(\sigma, f)\to \sigma$ is a homomorphism from $Ps(X)$ to $Sp(X)$.
If $k$ is not characteristic $2$, (\ref{eq:32}) corresponded all $\sigma \in Sp(X)$
a unique quadratic form $f$ on $X\times X^*$; 
consequently, in this case, $(\sigma,f)\to \sigma$ is a
isomorphism from $Ps(X)$ to  $Sp(X)$,
means identify these groups with the other.
On the contrary, when $k$ is of characteristic $2$,
 by taking $z_1 =z_2$ in (\ref{eq:32}), that the nondegenerate quadratic form $B(z,z)$ 
of $X\times X^*$  invariant under $\sigma$;  consequently, $(\sigma,f)\to \sigma$ is a homomorphism from $Ps(X)$
into the orthogonal group $O(B)$ of this form on$X\times X^*$;it is easy to checke that it is 
surjective; its kernel is formd by the elements $(1,f)\in Ps (X)$, according to (\ref{eq:32}),
for all the additive quadratic forms $f$ on $X\times X^*$.

Now we starts from a vector space $X_k$ on $k$, 
and that we denotes $X$ the extension of $X_k$ with a``universal domain'' on $k$,
we can apply that precedes, if $X_k$ has $k$ and
$X$ with the universal domain. The groups $Sp(X)$, $\fA(X)$, $Ps(X)$ are then algebraic groups defined on $k$, 
and $Sp(X_k)$, $\fA(X_k)$, $Ps(X_k)$ are no other than the groups 
$Sp(X)_k$, $\fA(X)_k$, $Ps(X)_k$  formed by the elements of $Sp(X)$, $\fA(X)$, 
$Ps(X)$ which are rational on $k$. 
Note $Ps(X)$ is of dimension $m(2m + 1)$ if $m =dim(X)$, for any characteristic of $k$;
We can say (in the language of the scheme may precisely) that $Ps(X)$ in 
caractdristic $2$ is a  is a degeneration of the
 symplectic group with caractdristic $0$. 
It is well-known that the symplectic group, in any characteristic, 
is connected, 
simply connected and semisimple;
it is the same of $Ps(X)$ when $k$ is not characteristic $2$, 
but it is no more true in characteristic $2$.

\subsection{}

\subsection{}

\setcounter{ssection}{35}
\subsection{}
 In the case $k$ is a discrete valuation field,
we also can, instead useing the lifting $\br_0$, 
use the lifting $\br_\Gamma$ which was defined in $n^\circ$~19 of Chapter I;
we will choose a lattice $L$ of $X$ for $\Gamma$. In the notation of Chapter I, 
we replace $G, G^*$, 
$\Gamma, \Gamma_*$, by $X, X^*$, $L$, $L_*$, where $L$ is a lattice in $X$ and $L_*$
is the lattic which corresponds to $L$ by duality in $X^*$
(say, all $x^*\in X^*$ such that $\chi([x, x^*]) =1$ for all $x\in L$).
Just as we have substituted,  consider $Ps(X)$ and $B_0(X)$,
we will substituted here, consider the group $B_0(X, L)$ 
which obtained by applying the definitions in $n^o$~19 Chapter I and 
subgroup $Ps(X, L)$ of $Ps(X)$ forms elements $s=(\sigma, f)$ in $Ps(X)$
such that $\chi\circ f$ constant $1$ on $L\times L_*$ and that $\sigma$
induces an automorphism of $L\times L_*$. 
We immediately varified that it is an open subgroup of $Ps(X)$,
compact if $k$ is not characteristic $2$,
and the homomorphism $\mu$ from $P_s(X)$ to $B_0(X)$
 gives homomorphism $P_s(X, L)$ to $B_0(X, L)$. 

With above notation, space $H(G,\Gamma)$ of Chapter I, $N^o$~18, 
becomes Hilbert space $H(X,L)$, and formulate (\ref{eq:25})
in $n^o$~19  defines a representation $\br_L$ from $B_0(X,L)$ to 
the automorphism group of $H(X,L)$, then a representation 
$\br_L\circ \mu$ from $Ps(X, L)$ to this group;
by transport of structure by isomorphism 
$Z^{-1}$ from $H(X,L)$ to $L^2(X)$ defined in $n^o$~18 of the same chapter,
From the representations of $B(X, L)$ and of $Ps (X, L)$, denote $\br_\Gamma$
and $\br_L\circ \mu$ by abuse of notation, we give representation of 
automorphism group of $L^2(X)$.
Moreover, one immediately deduces from (\ref{eq:25}) that 
representation of $Ps(X, L)$ in the group of the automorphisms of $H(X, L)$, 
the same, of $L^2(X)$, 
is continuous when this last group is provided with strong topology. 
Then let us denote $\br'_L$ the representation
\[
s\to (s,\br_L(\mu(s)))
\]
from $Ps(X,L)$ to $Mp(X)$;
from that $\br'_L$ is an isomorphism from $Ps(X, L)$ 
to its image in $Mp(X)$, $(s,\tau)\to \tau \br'_L(s)$ is a isomorphism from 
$Ps(X,L)\times T$ to an open subgroup of $Mp(X)$. 
Moreover, immediately from the end of  $n^o$~21  
Chapter I, for all $\Phi\in S(X)$, the map $s\to \br'_L(s)\Phi$ from $Ps(X,L)$
to $S(X)$ is locally constant.  

\subsection{}
We proceed now to extend the preceding results to the adele case.
We again assume naturally here the hypothesis and the notations of
Sections 29 and 30 of the Chapter II, which extends in an obvious
manner with all the algebraic groups defined on the field of base
$k$; in particular, one writes $Ps(X)_{k}$, $Ps(X)_{v}$ for the
groups formed by the elements of the algebraic group $Ps(X)$ which
is rational respectively over $k$ and over $k_{v}$, and one writes
$Ps(X)_{A}$ for the adelic group attached with $Ps(X)$ in the usual
manner. As in Section 29, we denote by $X^\circ$ a base of $X$, and
by $(X^*)^\circ$ a base of $X^*$, of which one can assume, to fix the
notions, that it is the dual base of $X^\circ$. For any completion
$k_{v}$ of $k$ with discrete valuation, one denotes by $Ps(X)_{v}^\circ$
the group formed by the elements $(\sigma,f)$ of $Ps(X)_{v}$ such
that $\sigma$ induces on $X_{v}^\circ\times(X^*)_{v}^\circ$ an automorphism
of this lattice, and that $f$ induces on this same lattice a function
with integer values (that is to say belonging to the ring $\fo_{v}$
of the integers of $k_{v}$). Then $Ps(X)_{A}$ is the collection,
and same as the inductive limit, of the group
\[
Ps(X)_{S}^\circ=\prod_{v\in S}Ps(X)_{v}\times\prod_{v\notin S}Ps(X)_{v}^\circ
\]
when one takes for $S$, as of the convention, all the finite sets
of the completions of $k$ containing the set $S_{\infty}$ which
is isomorphic with $\mathbb{R}$ or with $\mathbb{C}$.

Just as in the local case (cf. Section 33), one has a homomorphism
$(w,t)\to(w,\chi(t))$ of $\fA(X)_{A}$ into the group $A(X_{A})$
attached to locally compact group $X_{A}$ in the sense of the Chapter
I, Section 4. In the same way the formula\[
\mu_{A}((\sigma,f))=(\sigma,\chi\circ f)\]
defines a homomorphism $\mu_{A}$ of $Ps(X)_{A}$ into the group $B_{0}(X_{A})$
attached with $X_{A}$ in the sense of the Chapter I, $n^o$~5;
as in the local case, $\mu_{A}$ is injective when $k$ is not of
characteristic $2$.

As in the local case, we then define the \emph{metaplectic group}
$Mp(X)_{A}$ as being the subgroup of $Ps(X)_{A}\times\mathbf{B}_{0}(X_{A})$
formed by the elements $(s,\mathbf{s})$ of this product such that
$\mu_{A}(s)=\pi_{0}(\mathbf{s})$; this is given the topology induced
by the one of the ambient group when one gives with $\mathbf{B}_{0}(X_{A})$
the strong topology and with $Ps(X)_{A}$ the usual adelic topology.
We will denote frome now on by $\pi$ the projection of this group
onto $Ps(X)_{A}$; it is surjective and has as kernel the group ${e}\times\mathbf{T}$,
which we will write as, more simply, $\mathbf{T}$.

\subsection{}
We now define a continuous lifting in $Mp(X)_A$ from an open set of 
$Ps(X)_A$,
what, as in the local case, will make it possible to conclude that $Mp(X)_A$ 
is locally compact and has same locally homeomorphic of $Ps(X)_A \times T$,
and that $\pi$ is 
an open map from $Mp(X)_A$ to  $Ps(X)_A$.
We let $\Omega_v = \Omega(X)_v$ for all $v$. 
it is a nonempty open set of $Ps(X)_v$. Then let, for any finite set $S$ of
the completions of $k$, contain $S_\infty$:
\[
\Omega_S=\prod_{v\in S} \Omega_v \times \prod_{v\notin S} Ps(X)_v^\circ;
\] 
it is an open set of $Ps(X)_S^\circ$, then of $Ps(X)_A$. For each $\Omega_v$, 
according to $n^o$~34, a lefting from $\Omega_v$ to $Mp(X)_v$, will denote by 
$\br_v$; in addition, according to $n^o$~36, for all $v$, $k_v$ is discrete valuation, and for any lattice L in $X_v$, a lifting $\br'_L$ from $Ps(X_v,L)$ to
$Mp(X_v)$; moreover, according to $n^o$~21 of Chapter I, 
this map $Ps(X_v, L)$ to a subgroup of $Mp(X_v)$ which leaves 
the characteristic function of the network $L$ invariant.
For almost all $v$, $Ps(X)_v^\circ$ is a subgroup of $Ps(X_v,L)$
 when $L =X_v^\circ$; We will denote the set $S_0$ 
(finite, and containing $S_\infty$) of complement 
of $k$ for which $v\notin S_0$, we will show $\br'_v$ induce
 the lefting by $\br'_L$ of
$Ps(X)_v^\circ$ when $L=X_v^\circ$.

In addition, for all $v$, $\Phi_v$ is a function in  $L^2(X_v)$,
and assume that, for almost all $v$, $\Phi_v$ 
is the characteristic function of $X_v^\circ$; let $\Phi$ is a function on $X_A$,
for $x=(x_v)\in X_A$, define by the formula
\begin{equation}\label{eq:36}
\Phi(x) = \prod_v \Phi_v(x_v).
\end{equation}
The mesure on $X_A$ is defined in in conformity with $n^o$~30 of Chapter II,
$\Phi$ is in $L^2(X_A)$, and linear combinations of  functions in this 
form are dense in $L^2(X_A)$.
For any $S\supset S_0$, and any $s=(s_v)\in \Omega_S$, we set
\[
\br_S(s)\Phi(x) = \prod_{v\in S} \br_v(s_v)\Phi_v(x_v) 
\times \prod_{v\notin S} \br'_v(s_v)\Phi_v(x_v).
\]
According to what we have above,
almost all the factors of the second product are
respectively equal to the caractdristic functions of the lattice $X_v^\circ$,
so that the function defined has same form of $\Phi$.
It then results from $n^o$~22 of Chapter I that 
the map $\Phi\to \br_S(s)\Phi$ defineded above for the functions 
with form (\ref{eq:36}) extend to be an automorphisme of $L^2(X_A)$, and 
then define a continuous lifting $\br_S$ from $\Omega_S$ to $Mp(X)_A$.
From that we draw the above conclusions.  

\subsection{}
As in the local case, we now show that $(S, \Phi)\to S\Phi$ given 
a continuous map from $Mp(X_A)\times S(X_A)$ to $S(X_A)$;
as in $n^o$~35, it is sufficient to show that $(s,\Phi) \to \br_S(s) \Phi$ 
is a continous map from $\Phi_S\times S(X_A)$ to $S(X_A)$. 

By the definition in $n^o$~11 of Chapter I, and the remarks of $n^o$~29 of 
Chapter II, $S(X_A)$ are composed of finite linear combinations,
 constant coefficients, function of form
\begin{equation}\label{eq:37}
\Phi_\infty(x_\infty) \prod_{v\notin S_\infty}\Phi_v(x_v)
\end{equation} 
where $\Phi_\infty$ in the space $S(X_\infty)$ of the product $X_\infty =\prod X_v$
where $v\in S_\infty$ (so that $X_\infty$ is a finite dimensional 
vector space on $\bR$), and  $\Phi_v$ belongs to  $S(X_v)$ for all $v$,
and, for almost all $v$, is the characteristic function of $X_v^\circ$.

Let $\Phi$ be the function defined by (\ref{eq:37}), and let $s\in \Omega_S$. 
As above, $\br'_v$ leave $\Phi_v$ invariant for almost all $v$. 
For all $v\in S-S_\infty$, $s_v\to \br_v(s_v)\Phi_v$ is a continous map, 
and locally constant, from $Ps(X)_v^\circ$ to $S(X_v)$. 
Fininally, from $n^o$~22 Chapter I, and $n^o$~35, 
apply to the product $X_\infty$, then $s_v$(resp, $\br_v(s_v)$) 
for $v\in S_\infty$ determine an element $s_\infty$ of $Ps(X_\infty)$
(resp, an element $\br_\infty(s_\infty)$ of $Mp(X_\infty)$),
 their ``tensor product'', so that $s_\infty \to \br_\infty(s_\infty)$ is a 
continuous map; it follows, by $n^o$~35, 
that $s_\infty \to r_\infty(s_\infty)\Phi_\infty$ is a continous map from $S(X_\infty)$
to the product $\prod \Omega_v$ for $v\in S_\infty$. 
Combine these results, we conclude that $s\to \br_S(s)\Phi$ is a continous map 
from $\Omega_S$ to $S(X_A)$, then $S\to S\Phi$ is an continous map from
$Mp(X)_A$ to $S(X_A)$, where $\Phi \in S(X_A)$.
	
To prove the continuity of $(s, \Phi)\to \br_S(s)\Phi$, it is sufficient 
to verify  that: $K$ a compact set of $\Omega_S$, and $U$ 
a convex neihgborhood of $0$ of $S(X_A)$; $U'$ is the set 
of $\Phi\in S(X_A)$ such that $\br_S(s)\Phi\in U$ for all $s\in K$ is a
 neighborhood of $0$ of $S(X_A)$. Since $U'$ is convex, 
by the definition of the topology of $S(X_A)$ as inductive limit of 
$S(H,H')$(cf. $n^o$~11), we have to show $U'\cap S(H,H')$ is a neighborhood of 
$0$ of $S(H,H')$ for all $H$ and $H'$. 
However, for $K$, $H$ and $H'$ give, there is a finite set $S'$ of complement 
of $k$ having the following properties: (a) for all $s=(s_v)\in K$, and all 
$v\notin S'$, we have $s_v\in Ps(X)_v^o$; (b) for all function in $S(H,H')$
is linear combination of function in form (\ref{eq:37}) where $\Phi_v$ 
is the characteristic function of $X_v^\circ$ for all $v\notin S'$. 
Under these conditions, the assertion the assertion which it is a
immediate consdquence of the properties of continuity proof in $n^o$~35 
for the local case.

\end{document}

