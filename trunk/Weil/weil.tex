\documentclass[12pt]{amsart}
\usepackage[margin=3cm]{geometry}

\usepackage{hyperref}

\usepackage{amssymb}
\usepackage{graphicx}
%\usepackage{amscd}
\usepackage{braket}
\usepackage{paralist}
\usepackage{eufrak}
%\usepackage{calrsfs}
%\usepackage[small,nohug,heads=littlevee]{diagrams}
%\diagramstyle[labelstyle=\scriptstyle]
%\usepackage{diagrams}
\usepackage{amscd}
%\usepackage{pictexwd,dcpic}
%\usepackage{mathrsfs}


\newtheorem{lemma}{Lemma}
\newenvironment{subproof}{\newline\footnotesize\it}{\normalsize}
\DeclareMathAlphabet{\mathpzc}{OT1}{pzc}{m}{it}


\def\Ker{\rm{Ker}}
\def\Im{\rm{Im}}
\def\Hom{\rm{Hom}}
\def\Mat{\rm{Mat}}
\def\bR{{\mathbb{R}}}
\def\bN{{\mathbb{N}}}
\def\bZ{{\mathbb{Z}}}
\def\bC{{\mathbb{C}}}
\def\bQ{{\mathbb{Q}}}
\def\vv{{\vec{v}}}
\def\vw{{\vec{w}}}
\def\vx{{\vec{x}}}
\def\vy{{\vec{y}}}
\def\v0{{\vec{0}}}
\def\ol{\overline}
\def\sspan{\rm{span}}
\def\sl2{{\mathfrak{sl}(2)}}
\def\slc{{\mathfrak{sl}(2,\bC)}}
\def\sP{\mathcal{P}}
\def\sH{\mathcal{H}}
\def\sU{\mathcal{U}}
\def\sC{\mathcal{C}}
\def\ad{{\rm ad}}
\def\Ad{{\rm Ad}}
\def\id{{\rm id}}
\def\sgn{{\rm sgn}}
\def\gcd{{\rm gcd}}
\def\inn#1#2{\left\langle{#1},{#2}\right\rangle}
\def\abs#1{\left|{#1}\right|}
\def\Sp{{\rm Sp}}

\hypersetup{
    bookmarks=false,         % show bookmarks bar?
    unicode=false,          % non-Latin characters in Acrobat��s bookmarks
    pdftoolbar=true,        % show Acrobat��s toolbar?
    pdfmenubar=false,        % show Acrobat��s menu?
    pdffitwindow=false,      % page fit to window when opened
    pdftitle={Qualify Examination Answers - Algebra},    % title
    pdfauthor={Ma Jia Jun},     % author
    pdfsubject={Subject},   % subject of the document
    pdfcreator={Creator},   % creator of the document
    pdfproducer={Producer}, % producer of the document
    pdfkeywords={keywords}, % list of keywords
    pdfnewwindow=true,      % links in new window
    colorlinks=true,       % false: boxed links; true: colored links
    linkcolor=blue,          % color of internal links
    citecolor=green,        % color of links to bibliography
    filecolor=magenta,      % color of file links
    urlcolor=cyan           % color of external links
}


\title{ON CERTAIN GROUPS of unitary OPIRATEURS}

\begin{document}
\maketitle

\section{locally compact abelian goups}

\subsection{}	
In this chapter, it is mainly about locally compact abelian groups ,
 on which we will most often no restrictive assumptions, 
but some our results is irrelevant unless $G$ is isomorphic to its dual.
All subsequent applications relating to one of the following cases:
\begin{enumerate}[(a)]
\item $G$ is a vector space
$X$ finite dimension $k$ over a locally compact non discrete
\item $G$ has the form $X_A=X_K\otimes A_k$, where $A_k$ is the ring
  of a body adeles $k$ which can be either a body of algebraic
  numbers,
  or a body of algebraic functions of dimension $1$ over a finished, 
  and oh k X is a finite dimensional vector on It. 
\end{enumerate}
We refer to these cases
saying that $G$ is ``type in the local case'' in case(a), ``type
ade1ique'' in case (b);
 if $G$ is a local or adelique, 
it is isomorphic to its dual. 
Locally compact groups  locally will often notes additives.

$T$ mean the multiplicative group of complex numbers such as
$t\bar{t}=1$,
 a character of $G$ is a morphism of $G$ to $T$. If $G$ and $H$ 
are locally compact abelian groups abe1iens, a bicharacter of $G\times
H$ is the continuous fuction $f$ of  $G\times H$ to T such that,
 for all $y\in H$, $x \to f(x, y)$ is a character of $G$, for all 
$x\in G$, $y \to f(x,y)$ is a character of $H$.

	
The continuous function $f$ of $G$ to $T$ will be will be called a second
degree character of $G$ if the fuction 
\[
(x, y)\to \frac{f( x + y)}{f(x)f (y)}
\]
is a bicharacter of $G\times G$, or 
, which is the same, if $f$ satisfies the relationship
\[
f (x + y + z) f (x) f(y) f (z) = f(x + y) f (y + z) f (z + x)
\]
whenever $x, y, z$ in $G$.

	
We always denote $G^*$  the dual of $G$ (written additive too), 
and we write
$(x, x^*)$, for $X\in G$, $x^* \in G^*$, as the value on $x$
corresponding the character  $x^*$ of $G$.
We always identify $G$ with the double-dual $(G^*)^*$ of  $G$ such that
\[
\inn{x}{x^*} = \inn{x^*}{x}
\]
(it can made this identification such that
$\inn{x}{x^*} ={-x^*}{x}$ as well, and it would be even more
convenient 
in some respects, 
but this shock habits received too)
If $x\to x\alpha$ is a morphism from $G$ to $H$, its dual $\alpha^*$
is the morphism from $H^*$ to $G^*$ such that 
\[
\inn{x\alpha}{y^*}=\inn{x}{y^*\alpha^*}
\]
whenever $x\in G$,$y^*\in H^*$.

All bicharacter of $G\times H$ can be written uniquely in the form 
\[
f(x,y)=\inn{x}{y\alpha}=\inn{y}{x\alpha^*}
\]
where $\alpha$ is a morphism from $H$ to $G^*$, 
$\alpha^*\colon G\to H^*$ is the dual of $\alpha^*$.
If $G=H$, it is both necessary and sufficient that $f$ is symmetric 
in $x, y$, then we have $\alpha=\alpha^*$.
We say that the morphism $\alpha$ from $G$ to $G^*$ is {\it symmetric}. 
 
If $f$ is a character of second degree of $G$, we have 
\begin{equation}\label{eq:1}
\frac{f(x+y)}{f(x)f(y)} = \inn{x}{y\rho},
\end{equation}
or $\rho=\rho(f)$ is a morphism, obviously symmetric, from $G$ to
$G^*$;
In expression $\ref{eq:1}$, we say that $f$ and $\rho$ are associated
to each other. If we denote $X_2(G)$  the multiplication group of 
characters of second degree on $G$, function $f\to \rho(f)$ is a
homomorphism from $X_2(G)$ to the additive group of symetric
morphisms from $G$ to $G^*$;
the kernal of this homomorphism is the multiplicative group $X_1(G)$
of characher on $G$.
We can say more in the case that $x\to 2x$ is an automorphism of $G$
 (which occurs for example if $G$ is local or on a body adelie k
 characteristics other than 2);
then, we denote $x\to 2^{-1}x$  the inverse of the automorphism $x\to
2x$ of $G$. In this case, if $\rho$ is a symetric morphism from $G$ to
$G^*$,
it is associated with the second degree character
 $f_\rho(x)=\inn{x}{2^{-1}x\rho}$; if we denote $X^0_2(G)$ the
 subgroup of $X_2(G)$ formed by $f_\rho$, then $X_2(G) =
 X^0_2(G)\times X_1(G)$, and $X^0_2(G)$ is isomorphic to the additive
 group of symetric morphisms from $G$ to $G^*$.

We say that the character of second degree $f$ is non {\it degenerate} 
if the symetric morphism associated with $f$ is a isomrophism from $G$ 
to $G^*$;the existance of such character is necessary (but not
sufficient) for $G$ isomorphic to $G^*$.

\subsection{}
Section 2

Given a Haar measure $dx$ on G, a Fourier transform $\mathfrak{T}$
is that we associate with each function $\Phi$ on $G$, a function
$\Phi^{*}=\mathfrak{T}(\Phi)$ on $G^{*}$ defined as \[
\Phi^{*}(x^{*})=\int\Phi(x)\cdot<x,x^{*}>\cdot dx\]


\noindent whenever this integral, or a suitable extension of it, makes
sense. Then there is a Haar measure $dx^{*}$ on $G^{*}$, called
the $dual$ of $dx$, such that the inverse transformation $\mathfrak{T}^{-1}$
of $\mathfrak{T}$ is given by the formula

\[
\Phi(x)=\int\Phi^{*}(x^{*})\cdot<x,-x^{*}>\cdot dx^{*};\]


For this measure, we have the Plancherel formula

\[
\int|\Phi(x)|^{2}dx=\int|\Phi^{*}(x^{*})|^{2}dx^{*}\]


It is clear that, for every $c>0$, the Haar measure on $G^{*}$,
dual of $c\cdot dx$, is $c^{-1}dx^{*}$. This remark can be expressed
as follows,

\begin{lemma}\label{l:1}
Let $G$, $H$ be two locally compact abelian groups,
with Haar measure $dx$, $dy$; let $G^*$, $H^*$ be their duals,
equipped with Haar measure $dx^*$, $dy^*$ respectively. Then,
if $\alpha$ is an isomorphism of $G$ on $H$, $\alpha^{*}$ is an
isomorphism of $H^{*}$ on $G^{*}$, and one has $|\alpha^{*}|=|\alpha|$.
\end{lemma}

Recall that, if $G$ and $H$ are locally compact groups (commutative
or not) with Haar measures, the module of an isomorphism $\alpha$
of $G$ on $H$ is the number $|\alpha|=d(x\alpha)/dx$ defined by
the formula

\[
\int F(y)dy=|\alpha|\cdot\int F(x\alpha)dx\]


where $F\in L^{1}(H)$; if $G=H$, it is generally understood that
we take $dx=dy$, and then $|\alpha|$ is independent of the choice
of $dx$. To prove the lemma, let $m=|\alpha|$; by transport of structure,
$\alpha$ transforms $dx$ into a Haar measure $d'y$ on $H$, and
as soon as we see that $d'y=m^{-1}dy$; it follows that $\alpha^{*}$transforms
the dual measure of $d'y$, which is, as noted above, $m\cdot dy^{*}$
into $dx^{*}$; thus $\alpha*$ transforms $dy^{*}$ into $m^{-1}dx^{*}$,
which shows the lemma.

\subsection{}
Let $x\to z\sigma$ is an automorphism of $G\times G^*$; if we set
$z=(x,x^*)$, we can also written in a matrix form:
\[
(x,x^*)\to (x,x^*)\cdot 
\begin{pmatrix} 
\alpha & \beta \\
\gamma & \delta
\end{pmatrix}
\]
which means, of course:
\[
(x,x^*)\to (x\alpha + x^*\gamma,x\beta + x^*\delta)
\]
where $\alpha$, $\beta$, $\gamma$, $\delta$ are morphisms from $G$ to
$G$,
from $G$ to $G^*$, from $G^*$ to $G$ and $G^*$ to $G^*$ respectively. 
Note that the dual $\sigma^*$ of the $G\times G^*$ automorphism
$\sigma$ 
is the automorphism defined by these formulas:
\[
\sigma^* \to \begin{pmatrix}
\alpha^* & \gamma^* \\
\beta^* & \delta^*
\end{pmatrix}
\]
on $G^*\times G$.
Let $\eta$ be the isomorphism $\begin{pmatrix}0 & 1\\ -1 & 0\end{pmatrix}$ 
from $G\times G^*$ to $G^*\times G$, or, it's the same, 
the isomorphism $(x,x^*)\to (-x^*, x)$ (we refer them as the same automorphism  for any groups.)  The forumla
\begin{equation}\label{eq:2}
\sigma^I = \eta \sigma^* \eta^I = 
\begin{pmatrix} \delta^* & - \beta^*\\-\gamma^* & \alpha^*\end{pmatrix}
\end{equation}
defines an automorphism of $G\times G^*$, and Lemma~\ref{l:1} of section~2
shows that $\abs{\sigma^I}=\abs{\sigma}$. 
Note that $\sigma\to \sigma^I$ is an anti-automorphism involution of the
group of automorphisms of $G\times G^*$.
For the convenience of writing,  we will denote $F$ the bicharacter of 
$(G\times G^*)\times (G\times G^*)$ defined by 
\[
F(z_1,z_2) = \inn{x_1}{x_2^*}\quad (z_1=(x_1,x_1^*), z_2=(x_2,x_2^*)
\] 
An automorphism $\sigma$ of $G\times G^*$ is called {\it symplectic} if
it lead invariant of the bicharacter $F(z_1,z_2)F(z_2,z_1)^{-1}$, i.e. if
you have 
\begin{equation}
\frac{F(z_1\sigma, z_2\sigma)}{F(z_2\sigma, z_1\sigma}= 
\frac{F(z_1, z_2)}{F(z_2,z_1)}.
\end{equation}
whenever $z_1$, $z_2$ in $G\times G^*$; we write $\Sp(G)$ as the group 
formed by these automorphisms. For $\sigma$ to be symplectic, 
it is both necessary and sufficiently (as shown by a calculation immediately) 
that $\sigma \sigma^I = 1$, $\sigma^I$ is defined by (\ref{eq:2});
$\abs{\sigma^I}=\abs{\sigma}$, it follows that the module of any symplectic 
automorphism is $1$. The relationship $\sigma \sigma^I=1$ gives 
$\alpha\beta^* = \beta\alpha^*$ and $\gamma \delta^*=\delta \gamma^*$ 
in particular, which means that $\alpha\beta^*$ and $\gamma \delta^*$
 are symmetric morphisms from $G$ to $G^*$ and from $G^*$ to $G$ respectively;
through the relationship $\sigma^I \sigma = 1$, we see that $\beta^*\delta$ and
$\gamma^*\alpha$ in the same way.

\subsection{}
Let $B(G)$ be the automorphism group of $A(G)$. $s$ be an auto
\end{document}
