\documentclass[11pt]{beamer}
\usepackage{cite}

%\beamertemplatenumberedballsectiontoc
%\beamertemplateboldpartpage

%\usepackage{diagrams} 
%\diagramstyle[labelstyle=\scriptstyle] 
\usepackage[all,cmtip]{xy} 
%\usepackage{diagxy} 
\usepackage{amsfonts}

\usetheme{umbc4}

\usepackage{hyperref}

\usepackage{amsmath,amssymb}
\usepackage{graphics}
\usepackage{multimedia}

\usepackage{color}
\usepackage{braket}


\author[JJ, Ma]{Ma Jia Jun}
\title{Theta correspondence and cohomological induction}
\subtitle[Unitarity]{with the unitarity the constructions}
\institute[Dept. of Math.]{Departement of Mathematics, \\
  National University of Singapore\\[1ex]
  \texttt{g0701232@nus.edu.sg}
}

\def\bZ{\mathbb{Z}}
\def\bR{\mathbb{R}}

\newtheorem{thm}{Theorem}

\begin{document}

\begin{frame}[plain]
  \titlepage
\end{frame}

\def\fk{{\mathfrak{k}}}
\def\fh{{\mathfrak{h}}}
\def\fg{{\mathfrak{g}}}
\def\ft{{\mathfrak{t}}}
\def\fu{{\mathfrak{u}}}
\def\fm{{\mathfrak{m}}}
\def\fl{{\mathfrak{l}}}
\def\fq{{\mathfrak{q}}}
\def\Ad{{\mathrm{Ad}}}
\def\Hom{{\mathrm{Hom}}}
\def\H{{\mathrm{H}}}
\def\bC{{\mathbb{C}}}


\section{Cohomological induction}
\subsection{Notations}
\frame{
 \frametitle{\subsecname}
 We following Wallach's approach~\cite{Wallach1984}~\cite{Wallach1988}
 \begin{description}
 \item[$G$] a real reductive Lie group of inner type 
 \item[$\theta$] Cartan involution of $G$, with $\fk$ compact subalgebra. 
 \item[$\fh $] fundamental Cartan subalgebra of $\fg$ 
 \item[$\ft$] $\fk\cap \fh$ is maximal abelian in $\fk$
 \item[$H$] a element in $i\ft$.
 \item[$\fl$] $\Set{X\in \fg|[H,X]=0}$
 \item[$\fu$] $\Set{X\in \fg|[H,X]=\lambda X, \lambda >0}$
 \item[$\fq$] $=\fl_\bC+ \fu$, $\theta$-stable parabolic subalgebra.
 %%\item[$L$] $\Set{g\in G|\Ad(g)H=H}$
 \item[$\fm$] $=\fk\cap \fl$.
 \item[$\fq_k$] $\triangleq \fq\cap \fk_\bC = \fm_\bC + \fu_k$, 
   $\theta$-satable parabolic of $\fk$.
 \item[$M$] $=\Set{g\in K|\Ad(g)H=H}=K\cap L$.
 \end{description}
}

\subsection{Cohomology induction}
%\frame[<+->]{
\frame{
  \frametitle{\subsecname}
  Fix a $(\fl,M)$-module $W$.
  \begin{enumerate}[(I)]
  \item Regard $W$ as a $\fq$-module with $\fu$ act trivially on it. 
  \item Define $M(\fq,W) = U(\fg)\otimes_{U(\fl)}W$, 
    which is a $(\fk,M)$-module.
  \item Applying the Zuckerman functor, $\Gamma^j(M(\fl,W))$.
  \item $\Gamma^j(M(\fl,W))$ is
    nontrivial for all $j<n$. ($n=\dim\fu_k$)
  \end{enumerate}
}

\subsection{Zuckerman functor}
\frame{
  \frametitle{\subsecname}
  The Zuckerman functor is defined by 
  ($H(K)$:left $K$-finite smooth
  functions on $K$)
  \[
  \Gamma^j(V) = \H^j(\fk,M;V\otimes H(K)).
  \]
  $\H^j(V)$ is the cohomology of complex 
  \[
  C^j(\fk,M;V)= \Hom_M(\bigwedge^j(\fk/\fm), V)
  \]

  $\Gamma^j$ is a functor from $C(\fg,M)$ to $C(\fg,K)$, where $K$
  action given by right transform of $H(K)$ and $\fg$ action given by
  the commutative relation 
  \[
  \xymatrix{
    \Gamma^j(U(\fg)\otimes V) \ar[d]_{\Gamma^j(V)} 
    \ar[r]^{T_{U(\fg)}(V)}
    &  U(\fg)\otimes \Gamma^j(V) \ar[d]^{m}  \\ 
    \Gamma^j(V) \ar[r]^{\mathrm{Identity}} & \Gamma^j(V)
  }
  \]
}

\def\inn#1#2{\left<#1,#2\right>}

\subsection{Hermition Form}
\frame{
  \frametitle{\subsecname}
  Now assume $W$ be an irreducible $(\fl,M)$-module admits a positive definite 
  Hermition form $\inn{\cdot}{\cdot}$. 
  We can construct a hermition form on $\Gamma^n(M(\fq,W))$.
  \begin{enumerate}[(I)]
  \item Shapovalov Form,
    $(x\otimes w, y\otimes v) = \inn{p(y^*x)w}{v}$, \\
    $p$ is the projection from $U(\fg$ to $U(\fl_\bC)$ by
    decompostion 
    \[
    U(\fg_\bC)=U(\fl_\bC)\oplus(\overline{\fu}U(\fg_\bC)+U(\fg_\bC)\fu).
    \]
  \item The Hermition form on $\Gamma^n(M(\fq,W))$ comes from 
    the nature pairing between complex 
    $C^j(M(\fq,W))\subset \bigwedge^j(\ft/fm)_\bC^*\otimes M(\fq,W)$ and
    $C^{(2n-j)}(M(\fq,W))$:
    \[
    \inn{\alpha\otimes w}{\beta\otimes v} = (\alpha,\beta)(w,v)
    \]
  \end{enumerate}
  If $M(\fq,W)$ is also irreducible, $(\cdot,\cdot)$ non-degenerate,
  and only $\Gamma^n(M(\fq,W)$ nontrivial.
}

\subsection{Signature charactor}
\frame{
  \frametitle{\subsecname}
  
}

\subsection{Unitarizablity}
\frame{
  \frametitle{\subsecname}
  
}


\section{Theta correspondence}
\subsection{The duality relation}

\frame{
 \frametitle{\subsecname}

}

\subsection{Theta correspondence}
\frame{
 \frametitle{\subsecname}

}

\subsection{Unitarity version}
\frame{
 \frametitle{\subsecname}

}

\section{Unitary dual}

\frame{
 \frametitle{\subsecname}

 We can construct principle series

}


\frame{
  \frametitle{References}
  \bibliographystyle{alpha}
  \bibliography{bib/reppapers}	
}

\end{document}
