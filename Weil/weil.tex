\documentclass[12pt]{amsart}
\usepackage[margin=3cm]{geometry}

\usepackage{hyperref}

\usepackage{amssymb}
\usepackage{graphicx}
%\usepackage{amscd}
\usepackage{braket}
\usepackage{paralist}
\usepackage{eufrak}
%\usepackage{calrsfs}
%\usepackage[small,nohug,heads=littlevee]{diagrams}
%\diagramstyle[labelstyle=\scriptstyle]
%\usepackage{diagrams}
\usepackage{amscd}
%\usepackage{pictexwd,dcpic}
%\usepackage{mathrsfs}


\newtheorem{lemma}{Lemma}
\newtheorem{thm}{Theorem}
\newtheorem{prop}{Proposition}
\newenvironment{expl}{\small\it\color{red}}{\color{black}\normalsize}
\DeclareMathAlphabet{\mathpzc}{OT1}{pzc}{m}{it}



\def\Ker{\rm{Ker}}
\def\Im{\rm{Im}}
\def\Hom{\rm{Hom}}
\def\Mat{\rm{Mat}}
\def\bR{{\mathbb{R}}}
\def\bN{{\mathbb{N}}}
\def\bZ{{\mathbb{Z}}}
\def\bC{{\mathbb{C}}}
\def\bQ{{\mathbb{Q}}}
\def\bB{{\mathbb{B}}}
\def\bA{{\mathbb{A}}}
\def\bs{{\mathbf{s}}}
\def\bd{{\mathbf{d}}}
\def\bT{{\mathbb{T}}}
\def\bt{{\mathbf{t}}}
\def\br{{\mathbf{r}}}
\def\vv{{\vec{v}}}
\def\vw{{\vec{w}}}
\def\vx{{\vec{x}}}
\def\vy{{\vec{y}}}
\def\v0{{\vec{0}}}
\def\ol{\overline}
\def\sspan{\rm{span}}
\def\sl2{{\mathfrak{sl}(2)}}
\def\slc{{\mathfrak{sl}(2,\bC)}}
\def\sP{\mathcal{P}}
\def\sH{\mathcal{H}}
\def\sU{\mathcal{U}}
\def\sC{\mathcal{C}}
\def\ad{{\rm ad}}
\def\Ad{{\rm Ad}}
\def\id{{\rm id}}
\def\sgn{{\rm sgn}}
\def\gcd{{\rm gcd}}
\def\inn#1#2{\left\langle{#1},{#2}\right\rangle}
\def\abs#1{\left|{#1}\right|}
\def\norm#1{{\left\|{#1}\right\|}}
\def\Sp{{\rm Sp}}


\hypersetup{
    bookmarks=false,         % show bookmarks bar?
    unicode=false,          % non-Latin characters in Acrobat��s bookmarks
    pdftoolbar=true,        % show Acrobat��s toolbar?
    pdfmenubar=false,        % show Acrobat��s menu?
    pdffitwindow=false,      % page fit to window when opened
    pdftitle={Qualify Examination Answers - Algebra},    % title
    pdfauthor={Ma Jia Jun},     % author
    pdfsubject={Subject},   % subject of the document
    pdfcreator={Creator},   % creator of the document
    pdfproducer={Producer}, % producer of the document
    pdfkeywords={keywords}, % list of keywords
    pdfnewwindow=true,      % links in new window
    colorlinks=true,       % false: boxed links; true: colored links
    linkcolor=blue,          % color of internal links
    citecolor=green,        % color of links to bibliography
    filecolor=magenta,      % color of file links
    urlcolor=cyan           % color of external links
}


\title{ON CERTAIN GROUPS of unitary OPIRATEURS}

\begin{document}
\maketitle

\section{locally compact abelian goups}

\subsection{}	
In this chapter, it is mainly about locally compact abelian groups ,
 on which we will most often no restrictive assumptions, 
but some our results is irrelevant unless $G$ is isomorphic to its dual.
All subsequent applications relating to one of the following cases:
\begin{enumerate}[(a)]
\item $G$ is a vector space
$X$ finite dimension $k$ over a locally compact non discrete
\item $G$ has the form $X_A=X_K\otimes A_k$, where $A_k$ is the ring
  of a body adeles $k$ which can be either a body of algebraic
  numbers,
  or a body of algebraic functions of dimension $1$ over a finished, 
  and oh k X is a finite dimensional vector on It. 
\end{enumerate}
We refer to these cases
saying that $G$ is ``type in the local case'' in case(a), ``type
ade1ique'' in case (b);
 if $G$ is a local or adelique, 
it is isomorphic to its dual. 
Locally compact groups  locally will often notes additives.

$T$ mean the multiplicative group of complex numbers such as
$t\bar{t}=1$,
 a character of $G$ is a morphism of $G$ to $T$. If $G$ and $H$ 
are locally compact abelian groups abe1iens, a bicharacter of $G\times
H$ is the continuous fuction $f$ of  $G\times H$ to T such that,
 for all $y\in H$, $x \to f(x, y)$ is a character of $G$, for all 
$x\in G$, $y \to f(x,y)$ is a character of $H$.

	
The continuous function $f$ of $G$ to $T$ will be will be called a second
degree character of $G$ if the fuction 
\[
(x, y)\to \frac{f( x + y)}{f(x)f (y)}
\]
is a bicharacter of $G\times G$, or 
, which is the same, if $f$ satisfies the relationship
\[
f (x + y + z) f (x) f(y) f (z) = f(x + y) f (y + z) f (z + x)
\]
whenever $x, y, z$ in $G$.

	
We always denote $G^*$  the dual of $G$ (written additive too), 
and we write
$(x, x^*)$, for $X\in G$, $x^* \in G^*$, as the value on $x$
corresponding the character  $x^*$ of $G$.
We always identify $G$ with the double-dual $(G^*)^*$ of  $G$ such that
\[
\inn{x}{x^*} = \inn{x^*}{x}
\]
(it can made this identification such that
$\inn{x}{x^*} ={-x^*}{x}$ as well, and it would be even more
convenient 
in some respects, 
but this shock habits received too)
If $x\to x\alpha$ is a morphism from $G$ to $H$, its dual $\alpha^*$
is the morphism from $H^*$ to $G^*$ such that 
\[
\inn{x\alpha}{y^*}=\inn{x}{y^*\alpha^*}
\]
whenever $x\in G$,$y^*\in H^*$.

All bicharacter of $G\times H$ can be written uniquely in the form 
\[
f(x,y)=\inn{x}{y\alpha}=\inn{y}{x\alpha^*}
\]
where $\alpha$ is a morphism from $H$ to $G^*$, 
$\alpha^*\colon G\to H^*$ is the dual of $\alpha^*$.
If $G=H$, it is both necessary and sufficient that $f$ is symmetric 
in $x, y$, then we have $\alpha=\alpha^*$.
We say that the morphism $\alpha$ from $G$ to $G^*$ is {\it symmetric}. 
 
If $f$ is a character of second degree of $G$, we have 
\begin{equation}\label{eq:1}
\frac{f(x+y)}{f(x)f(y)} = \inn{x}{y\rho},
\end{equation}
or $\rho=\rho(f)$ is a morphism, obviously symmetric, from $G$ to
$G^*$;
In expression $\ref{eq:1}$, we say that $f$ and $\rho$ are associated
to each other. If we denote $X_2(G)$  the multiplication group of 
characters of second degree on $G$, function $f\to \rho(f)$ is a
homomorphism from $X_2(G)$ to the additive group of symetric
morphisms from $G$ to $G^*$;
the kernal of this homomorphism is the multiplicative group $X_1(G)$
of characher on $G$.
We can say more in the case that $x\to 2x$ is an automorphism of $G$
 (which occurs for example if $G$ is local or on a body adelie k
 characteristics other than 2);
then, we denote $x\to 2^{-1}x$  the inverse of the automorphism $x\to
2x$ of $G$. In this case, if $\rho$ is a symetric morphism from $G$ to
$G^*$,
it is associated with the second degree character
 $f_\rho(x)=\inn{x}{2^{-1}x\rho}$; if we denote $X^0_2(G)$ the
 subgroup of $X_2(G)$ formed by $f_\rho$, then $X_2(G) =
 X^0_2(G)\times X_1(G)$, and $X^0_2(G)$ is isomorphic to the additive
 group of symetric morphisms from $G$ to $G^*$.

We say that the character of second degree $f$ is non {\it degenerate} 
if the symetric morphism associated with $f$ is a isomrophism from $G$ 
to $G^*$;the existance of such character is necessary (but not
sufficient) for $G$ isomorphic to $G^*$.

\subsection{}
Section 2

Given a Haar measure $dx$ on G, a Fourier transform $\mathfrak{T}$
is that we associate with each function $\Phi$ on $G$, a function
$\Phi^{*}=\mathfrak{T}(\Phi)$ on $G^{*}$ defined as \[
\Phi^{*}(x^{*})=\int\Phi(x)\cdot<x,x^{*}>\cdot dx\]


\noindent whenever this integral, or a suitable extension of it, makes
sense. Then there is a Haar measure $dx^{*}$ on $G^{*}$, called
the $dual$ of $dx$, such that the inverse transformation $\mathfrak{T}^{-1}$
of $\mathfrak{T}$ is given by the formula

\[
\Phi(x)=\int\Phi^{*}(x^{*})\cdot<x,-x^{*}>\cdot dx^{*};\]


For this measure, we have the Plancherel formula

\[
\int|\Phi(x)|^{2}dx=\int|\Phi^{*}(x^{*})|^{2}dx^{*}\]


It is clear that, for every $c>0$, the Haar measure on $G^{*}$,
dual of $c\cdot dx$, is $c^{-1}dx^{*}$. This remark can be expressed
as follows,

\begin{lemma}\label{l:1}
Let $G$, $H$ be two locally compact abelian groups,
with Haar measure $dx$, $dy$; let $G^*$, $H^*$ be their duals,
equipped with Haar measure $dx^*$, $dy^*$ respectively. Then,
if $\alpha$ is an isomorphism of $G$ on $H$, $\alpha^{*}$ is an
isomorphism of $H^{*}$ on $G^{*}$, and one has $|\alpha^{*}|=|\alpha|$.
\end{lemma}

Recall that, if $G$ and $H$ are locally compact groups (commutative
or not) with Haar measures, the module of an isomorphism $\alpha$
of $G$ on $H$ is the number $|\alpha|=d(x\alpha)/dx$ defined by
the formula

\[
\int F(y)dy=|\alpha|\cdot\int F(x\alpha)dx\]


where $F\in L^{1}(H)$; if $G=H$, it is generally understood that
we take $dx=dy$, and then $|\alpha|$ is independent of the choice
of $dx$. To prove the lemma, let $m=|\alpha|$; by transport of structure,
$\alpha$ transforms $dx$ into a Haar measure $d'y$ on $H$, and
as soon as we see that $d'y=m^{-1}dy$; it follows that $\alpha^{*}$transforms
the dual measure of $d'y$, which is, as noted above, $m\cdot dy^{*}$
into $dx^{*}$; thus $\alpha*$ transforms $dy^{*}$ into $m^{-1}dx^{*}$,
which shows the lemma.

\subsection{}
Let $x\to z\sigma$ is an automorphism of $G\times G^*$; if we set
$z=(x,x^*)$, we can also written in a matrix form:
\[
(x,x^*)\to (x,x^*)\cdot 
\begin{pmatrix} 
\alpha & \beta \\
\gamma & \delta
\end{pmatrix}
\]
which means, of course:
\[
(x,x^*)\to (x\alpha + x^*\gamma,x\beta + x^*\delta)
\]
where $\alpha$, $\beta$, $\gamma$, $\delta$ are morphisms from $G$ to
$G$,
from $G$ to $G^*$, from $G^*$ to $G$ and $G^*$ to $G^*$ respectively. 
Note that the dual $\sigma^*$ of the $G\times G^*$ automorphism
$\sigma$ 
is the automorphism defined by these formulas:
\[
\sigma^* \to \begin{pmatrix}
\alpha^* & \gamma^* \\
\beta^* & \delta^*
\end{pmatrix}
\]
on $G^*\times G$.
Let $\eta$ be the isomorphism $\begin{pmatrix}0 & 1\\ -1 & 0\end{pmatrix}$ 
from $G\times G^*$ to $G^*\times G$, or, it's the same, 
the isomorphism $(x,x^*)\to (-x^*, x)$ (we refer them as the same automorphism  for any groups.)  The forumla
\begin{equation}\label{eq:2}
\sigma^I = \eta \sigma^* \eta^I = 
\begin{pmatrix} \delta^* & - \beta^*\\-\gamma^* & \alpha^*\end{pmatrix}
\end{equation}
defines an automorphism of $G\times G^*$, and Lemma~\ref{l:1} of section~2
shows that $\abs{\sigma^I}=\abs{\sigma}$. 
Note that $\sigma\to \sigma^I$ is an anti-automorphism involution of the
group of automorphisms of $G\times G^*$.
For the convenience of writing,  we will denote $F$ the bicharacter of 
$(G\times G^*)\times (G\times G^*)$ defined by 
\begin{equation}\label{eq:3}
F(z_1,z_2) = \inn{x_1}{x_2^*}\quad (z_1=(x_1,x_1^*), z_2=(x_2,x_2^*)
\end{equation}
An automorphism $\sigma$ of $G\times G^*$ is called {\it symplectic} if
it lead invariant of the bicharacter $F(z_1,z_2)F(z_2,z_1)^{-1}$, i.e. if
you have 
\[
\frac{F(z_1\sigma, z_2\sigma)}{F(z_2\sigma, z_1\sigma}= 
\frac{F(z_1, z_2)}{F(z_2,z_1)}.
\]
whenever $z_1$, $z_2$ in $G\times G^*$; we write $\Sp(G)$ as the group 
formed by these automorphisms. For $\sigma$ to be symplectic, 
it is both necessary and sufficiently (as shown by a calculation immediately) 
that $\sigma \sigma^I = 1$, $\sigma^I$ is defined by (\ref{eq:2});
$\abs{\sigma^I}=\abs{\sigma}$, it follows that the module of any symplectic 
automorphism is $1$. The relationship $\sigma \sigma^I=1$ gives 
$\alpha\beta^* = \beta\alpha^*$ and $\gamma \delta^*=\delta \gamma^*$ 
in particular, which means that $\alpha\beta^*$ and $\gamma \delta^*$
 are symmetric morphisms from $G$ to $G^*$ and from $G^*$ to $G$ respectively;
through the relationship $\sigma^I \sigma = 1$, we see that $\beta^*\delta$ and
$\gamma^*\alpha$ in the same way.

\subsection{}
For any element $w=(u,u^{*})$ of $G\times G^{*}$, we pick an operator
$U(w)$, which, for any function $\Phi$ on $G$, gives a function
$\Phi'=U(w)\Phi$ such that

\[
\Phi'(x)=(U(w)\Phi)(x)=\Phi(x+u)\cdot<x,u^{*}>\]


For short, we write $U(w)\Phi(x)$ instead of $(U(w)\Phi)(x)$. Applied
to functins $\Phi\in L^{2}(G)$, the $U(w)$ is obviously the unitary
operator, and it was, for any $w_{1},w_{2}$ in $G\times G^{*}$:

\[
U(w_{1})U(w_{2})=F(w_{1},w_{2})\cdot U(w_{1}+w_{2})\]


wherein $F$ is the function defined above by (3). It concludes that
the operators $t\cdot U(w)$, for $w\in G\times G^{*}$ and $t\in T$,
form a group, whose composition law is given by

\begin{equation}\label{eq:4}
(w_{1},t_{1})\cdot(w_{2},t_{2})=(w_{1}+w_{2},F(w_{1},w_{2})t_{1}t_{2})
\end{equation}

In other words, the formula (\ref{eq:4}) defines a group law
 on the set $G\times G^{*}\times T$;
and if we denote by $A(G)$ the group thus defined (which, with the
obvious topology on $G\times G^{*}\times T$, is a locally compact
group), the mapping $(w,t)\rightarrow t\cdot U(w)$ defines a unitary
representation $A(G)$. We denote by $\mathbf{A}(G)$ the group formed
by the operators $t\cdot U(w)$; if we choose the topology induced
by the strong topology  in the
group of automorphisms of $L^{2}(G)$ (cf. later in Sectin 35), it
is easy to verify that $(w,t)\rightarrow t\cdot U(w)$ is even an
isomorphism of topological groups.

The center of the group $A(G)$ is obviously formed by the elements
$(0,t)$; it is isomorphic to $T$, and denoted $T$ for short. It
is clear that $(w,t)\rightarrow w$ is a homomorphism of $A(G)$ on
$G\times G^{*}$, with kernel $T$; it allows to identify $A(G)/T$
with $G\times G^{*}$.
\subsection{}
Let $B(G)$ be the automorphism group of $A(G)$. 
An automorphism $s$ in  $B(G)$ induce an automorphism of the center $T$ of 
 $A(G)$, which may not be t-> t or t-> i; 
and induced by passing to the quotient, an automorphism a $A(G)/T$, i.e. 
$G\times G^*$. We denoted $B_0(G)$ the automorphism group of $A(G)$, 
which induce the identity map on the center $T$ of $A(G)$.
We  consider $B_0(G)$ now, although the following results can partially 
coverte to $B(G)$. Let $s$ be an element of $B_0(G)$, which induced a 
automorphism $\sigma$ of $G\times G^*$. it is immediatly written $s$ as
\begin{equation}\label{eq:5}
(w,t)s = (w\sigma, f(w)t),
\end{equation}
where $f$ is a continous function from $G\times G^*$ to $T$. 
This formula defines an automorphism of $A(G)$, it is necessary and sufficient 
that
\begin{equation}\label{eq:6}
\frac{f(w_1+w_2)}{f(w_1)f(w_2)}=\frac{F(w_1\sigma, w_2\sigma)}{F(w_1,w_2)}
\end{equation}
where $w_1,w_2$ are in $G\times G^*$.
This shows in particular that $f$ 
is a character in the second degree of $G\times G^*$. 
Moreover,the second is symmetric in $w_1$ and $w_2$,
 we see that $\sigma$ must be symplectic.

Write $s = (\sigma, f)$ where $s$ is the automorphism of $A(G)$ defined by 
(\ref{eq:5}), $f$ and $\sigma$ satisfy the relation ($\ref{eq:6}$).
The group law is given by 
\[
(\sigma, f)\cdot(\sigma',f') = (\sigma\sigma', f'')
\]
for any $w\in G\times G^*$, $f$ is defined by the formula
\begin{equation}\label{eq:7}
f''(w)=f(w)f'(w\sigma)
\end{equation}
The map $s\to \sigma$ is a homomorphism from $B_0(G)$ to $\Sp(G)$;
its kernel is formed by  the elements $(1,f)$, where $f$, by (\ref{eq:6}),
is a character of $G\times G^*$ of the form 
\[
f(u,u^*) = \inn{u}{a^*}\inn{a}{u^*}
\]
with $a\in G$, $a^*\in G^*$. 	 	
But it immediately satisfies that $(1,f)$ is the inner
 automorphism of $A(G)$ determind by the element $(-a, a^*, 1)$. 
The kernel of $s\to \sigma$ formed by the inner automorphism of $A(G)$,
is isomorphic to $A(G)/T$, therefore, $G\times G^*$.

 	
We can go one step further by explaining the second term of (\ref{eq:6}).
Let $\sigma$ be placed in matrix form as in Section~$3$. Let
\[
f'(u,u^*)=f(u,u^*)\inn{u^*\gamma}{-u\beta}
\]
Easy calculation gives (\ref{eq:6}) in the form
\[
f'(u_1+u_2, u^*_1+u^*_2) = f'(u_1,u^*_1)f'(u_2,u^*_2)\inn{u_1}{u_2\alpha\beta^*}
\inn{u^*_1\gamma\delta^*}{u^*_2}.
\]
Let $g(u) = f'(u,0)$, $h(u^*)=f'(0,u^*)$; by $u_2=0$, $u_1^*=0$ in the relation
about, we see that $f'(u,u^*)$ is no other than $g(u)h(u^*)$, then $g$ and $h$
satisfy the relations
\begin{align*}
g(u_1+u_2)&=g(u_1)g(u_2)\inn{u_1}{u_2\alpha\beta^*}\\
h(u^*_1+u^*_2)&=h(u^*_1)g(u^*_2)\inn{u^*_1\gamma\delta^*}{u^*_2},
\end{align*}
in other words they are the characters of second degree of $G$ and $G^*$,
respectively associated with symmetric morphisms $\alpha\beta^*$, 
$\gamma\delta^*$ from $G$ to $G^*$ and $G^*$ to $G$. Then:
\[
f(u,u^*)=g(u)h(u^*)\inn{u^*\gamma}{u\beta}.
\]
	
It of course has more accurate results when $x-> 2x$ is an automorphism of $G$.
In view of Subsection~1, above formulas show that,
 for all symplecitc automorphism $\sigma$, it correspond to an element 
$(\sigma, f)$ of $B_0(G)$, obtained by
\[
g(u)=\inn{u}{2^{-1}u\alpha\beta^*}, h(u^*)=\inn{2^{-1}u^*\gamma\delta^*}{u^*}.
\]
	
Furthermore, these formulas define a monomorphisme from $\Sp(G)$ to $B_0(G)$, 
and $B_0(G)$ is the semidirect product of the image of $\Sp(G)$ by this map 
and the inner automorphisms of $A(G)$; consequently, 
$B_0(G)$ is isomorphic to a semidirect product of $Sp (G)$ and  $G\times G^*$.

\subsection{}

\subsection{}
Now agree, for $s=(\sigma, f)$ and 
$\sigma=\begin{pmatrix}\alpha &\beta \\ \gamma &\delta
\end{pmatrix}$, 
put $\gamma = \gamma(s)$; and denote $\Omega_0(G)$  all the $s\in B_0(G)$
such that $\gamma(s)$ is an isomorphism from $G^*$ to $G$
(this set can be empty).
It has the following result:
\begin{prop}\label{p:1}
These set $\Omega_0(G)$ of $s\in B_0(G)$ such that $\gamma(s)$ is an 
isomorphism from $G^*$ to $G$ is the set of elements of $B_0(G)$  of the forme 
\begin{equation}\label{eq:8}
s = t_0(f_1)d'_0(\gamma)t_0(f_2)
\end{equation}
where $\gamma$ is a isomorphism from $G^*$ to $G$ and $f_1$, $f_2$ are 
characters of second degree for $G$;
and any element of $\Omega_0(G)$ has unique decomposition in that form.  
\end{prop}
if $s$ is given by (\ref{eq:8}), $\gamma(s)=\gamma$, then 
$s$ is in $\Omega_0(G)$.
Conversely, let $s =(\sigma, f) \in  \Omega_0(G)$; if (\ref{eq:8}) holds, 
we must take $\gamma=\gamma(s)$. Then let 
$\sigma=\begin{pmatrix}\alpha &\beta \\ \gamma &\delta \end{pmatrix}$; easy 
calculation shows that (\ref{eq:8}) is satisfied, 
it is necessary and sufficient that 
\[
f_1(u)=f(u,-u\alpha\gamma^{-1}), \quad f_2(u)=f(0,u\gamma^{-1}).
\]
These demonstrated by  the proposition. 
Note that applying (\ref{eq:8}) the homomorphism
$s\to \sigma$, by the relation:
\[
\begin{pmatrix}\alpha &\beta \\ \gamma &\delta \end{pmatrix}
= \begin{pmatrix}
1 & \alpha\gamma^{-1}\\
0 & 1
\end{pmatrix}
\begin{pmatrix}
0 & -\gamma^{*-1}\\
\gamma & 0
\end{pmatrix}
\begin{pmatrix}
1 & \gamma^{-1}\delta\\
0 & 1
\end{pmatrix};
\]
because of the symplectic of $\sigma$, $\alpha\gamma^{-1}$ and $\gamma^{-1}\delta$
are symmetric morphisms from $G$ to $G^*$; they are associated with
 $f_1$ and $f_2$ respectively.

 	
This gives a important relationship by considering a second degree character $f$
which is non degenerated of $G$, 	
this means that the morphism $\rho$ associated with $f$ is an isomorphism
from $G$ to $G^*$; then the function $f'$ to $G^*$, defined by 
\[
f'(x^*) = f(-x^*\rho^{-1}),
\]
is a character of second degree of $G^*$, with the associated symmetric 
morphism $\rho^{-1}$ from $G^*$ to $G$. 
By the proposition~\ref{p:1}, applied to $t'_0(f')$, gives
\[
t'(f')=t_0(f)d'_0(\rho^{-1})t_0(f^-)
\]
where $f^-$ is defined by $f^-(x)=f(-x)$ 	
as mentioned above.
A simple calculation gives the other:
\[
t'_0(f') = d'_0(\rho^{-1})t_0(f^{-1})d'_0(-\rho^{-1}).
\]
At the same time we have $d'_0(\rho^{-1})^2=d_0(-1)$, we derived the following:
\begin{equation}\label{eq:9}
d'_0(-\rho^{-1})t_0(f)d'_0(\rho^{-1})t_0(f^-)= t_0(f^{-1})d'_0(-\rho^{-1}).
\end{equation}
Taking into account the relationships obtained in subsection~6, 
we have (\ref{eq:9}) in a simpler form
\[
(t_0(f)d'_0(-\rho_0^{-1}))^3=e,
\] 
where $e$ is the identity element of $B_0(G)$; in that form, 
it is well known in the theory classic modular group.
But it is the relationship (9), as written above, we have to use later.

\subsection{}

The automorphisms of the group $\mathbf{A}(G)$, (which is )isomorphic
to $A(G)$, which was introduced in Section 4, are of course the same
as those of $A(G)$. We now propose to show that any automorphism
$s\in B_{0}(G)$ of $\mathbf{A}(G)$ is induced on $\mathbf{A}(G)$
by an inner automorphism of the group of all the unitary operators.
This theorem is due to I.Segal{[}6{]} in the case where $x\rightarrow2x$
is an automorphism of $G$, and we borrow his method of demonstration,
which includes the introduction of an algebra of operators naturally
associated to the group $\mathbf{A}(G)$. For this, we propose, in
a sense which will be precise in a moment,

\[
U(\varphi)=\int U(w)\varphi(w)dw,\]


where $\varphi$ denotes a function on $G\times G^{*},$ and where
$w=(u,u^{*})$ and $dw=du\cdot du^{*}$ (measure which does not depend
on the choice of the measure $du$ on $G$). In other words, if $\Phi$
is a function on $G$, $U(\varphi)\Phi$ is the functin defined by

\begin{equation}\label{eq:10}
U(\phi)\Phi(x)=\int U(w)\Phi(x)\cdot \phi(w)dw =
\int \Phi(x+u)\cdot \inn{x}{u^*} \cdot \phi(u,u^*) du du^* 
\end{equation}

where we assume provisionally, to fix the ideas, that $\varphi$ and
$\Phi$ are two continuous function with compact support. This can
also be written as

\begin{equation}\label{eq:11}
U(\phi)\Phi(x)=\int K(x,y)\Phi(y)dy
\end{equation}


where $K$ is given by

\[
K(x,y)=\int\varphi(y-x,u^{*})\cdot<x,u^{*}>\cdot du^{*}\]



or, which is the same as

\[
K(x,x+u)=\int\varphi(u,u^{*})\cdot<x,u^{*}>\cdot du^{*}\]


one obtains then $K(-x,-x+u)$ from $\varphi(u,u^{*})$ while applying,
for any value of $u$, the Fourier transform where $\varphi(u,u^{*})$
considered as function of $u^{*}$. Under the conditions when the
validity of Fourier inversion formula holds, we then have

\[
\varphi(u,u^{*})=\int K(x,x+u)\cdot<x,-u^{*}>\cdot dx;\]


furthermore, in virtue of Plancherel theorem, we have

\[
\int|K(x,y)|^{2}dxdy=\int|\varphi(u,u^{*})|^{2}dudu^{*},\]


this shows that the correspondence between the functions $\varphi$
on $G\times G^{*}$ and the functions $K$ on $G\times G$, defined
by the above formulas, is extended continuouly to an isomorphism $W$
of $L^{2}(G\times G^{*})$ onto $L^{2}(G\times G)$.

When $K$ is the function defined by $K(x,y)=P(x)Q(y)$, we write
$K=P\otimes Q$; and, if $P$ and $Q$ are in $L^{2}(G)$, we write
$(P,Q)=\int P(x)\overline{Q(x)}dx$; along with these notations, the
formulas above are given in particular

\begin{equation}\label{eq:12}
W^{-1}(P\otimes \overline{Q})(w)=(P,U(w)Q)
\end{equation}

\subsection{}
Now let $\phi_1$, $\phi_2$ are two functions on $G\times G^*$, 	
temporarily assume that continues a compact support; after (\ref{eq:10}),
we have
\[
U(\phi_1)U(\phi_2) = U(\phi_3)
\]
where $\phi_3$ is given by the formula 
\begin{equation}\label{eq:13}
\phi_3(w) = \int \phi_1(w-w_1)\phi_2(w_1)F(w-w_1,w_1) dw_1;
\end{equation}
as before, $F$ denot the function defined here by (\ref{eq:3}) subsection~3.
If we let $K_i=W(\phi_i)$ for $i = 1, 2, 3,$ (\ref{eq:11}) shows that $K_3$
 is given by
\begin{equation}\label{eq:14}
K_3(x,y)=\int K_1(x,z)K_2(z,y)dz
\end{equation}
we write $K_3 = K_1\times K 2$.
Moreover, the above formulas are extended by continuity 
to the space $L^2(G\times G^*)$, $L^2 (G\times G)$.
We will need the following lemma:
\begin{lemma}\label{l:2}
Let $K\in L^2(G\times G)$; K is in the form $P\otimes Q$, 
with $P$ and $Q$ in $L^2(G)$, 
it is both necessary and sufficient that for any $K'\in L^2(G\times G)$,
$K x K' x K$ and $K$ differs by a scalar factor.
Let $K = P\otimes Q$, $K'= P'\otimes Q'$, with $P, Q, P', Q '$in $L^2(G)$;
for P and P '(resp. Q and Q ') 	
no different from another by a scalar factor, 
it is both necessary and sufficient that for any
$K''=P''\otimes Q''$ with $P''$ and $Q''$ in $L^2(G)$, $K\times K''$
 and $K'\times K''$ (resp. $K''\times K$ and $K''\times K'$) 
are different by a scalar factor.
\end{lemma}
	 	
The second part is obvious, and it is evident that in the first part, 
the required condition is necessary;
to see that it is sufficient, simply apply to the case $K'= P'\otimes Q'$. 
The only consequence of this lemma we need is the following:
\begin{lemma}\label{l:3}
Let $K\to K^s$ an an automorphism of Hilbert space $L^2(G\times G)$
with composition $(K_1,K_2) \to K_1\times K_2$ defined by (\ref{eq:14}).
So there is an automorphism $t$ of  $L^2(G)$ such that, for all $P$ and $Q$ in
$L^2(G)$, $(P\otimes Q)^s = P^t \otimes Q^{\overline{t}}$, 
and $\overline{t}$ is the imaginary conjugation of  $t$, defined by 
$\overline{Q}^{\overline{t}}= (\overline{Q^t})$.
\end{lemma}
Indeed, after Lemma 2, any element $(Q\otimes P)^s$ of $L^2(G\times G)$
 is of the form $P'\otimes Q'$. Choose $P_0$ as $\norm{P_0}=1$; 
then $s$ conserve the norm, we can put 
$(P_0\otimes \overline{P}_0)^s$ in the form $P'_0\otimes Q'_0$ 
with $\norm{P'_0}=\norm{Q'_0}=1$. 	
The second part of lemma 2 shows, whenever $P,Q$ in $L^2(G)$
$(P\otimes \overline{P}_0)^s$ and $(P+0\otimes Q)^s$ are, respectively,
has unique form $P'\otimes Q'_0$ and $P'_0\otimes Q'$. 
If describes $P'= P^t$, $Q'= Q^u$, it is clear that $t, u$ are linear maps
from $L^2(G)$ to $L^2(G)$ and $p^t_0 = P'_0$, $\overline{P}^u_0=Q'_0$;
$s$ preserve the norm of $L^2(G\times G)$, 	
it is the same as $t$ and $u$ in $L^2(G)$.
As we have $P\otimes Q = (P\otimes \overline{P}_0)\times (P_0\otimes Q)$,
it follows that $(P\otimes Q)^s=c\cdot P^t\otimes Q^u$, 
and $c=(P'_0, \overline{Q}'_0)$; for $P=P_0$, $Q=Q_0$, it gives $c=1$.
As 
\[
(P\otimes Q)\times (P\otimes Q) = (P,\overline{Q})P\otimes Q,
\]
we see that for $P'= P^t$, $Q'=Q^u$, $(P',\overline{Q}')=(P,\overline{Q})$;
it follows that $u=\overline{t}$. 
Finally, since
$s^{-1}$ has the same properties as $s$, $t$ and $u$ are invertible and
 are therefore automorphism of $L^2(G)$.  

\subsection{}
Suppose that $s=(\sigma,f)$ is an automorphism of $A(G)$ belonging
to $B_{0}(G)$; the center of the isomorphism between $A(G)$ and
$\mathbf{A}(G)$, as defined in Section 4, operates on $\mathbf{A}(G)$
in the obvious manner. In particular, the transformation of $U(w)$
by $s$ will be $U(w)^{s}=f(w)\cdot U(w\sigma)$. We thereof deduce
immediately an automorphism of the algebras of the operators $U(\varphi)$
introduced in Section 8:

\[
U(\varphi)^{s}=\int U(w\sigma)f(w)\varphi(w)dw,\]


then we can write $U(\varphi)^{s}=U(\varphi^{s})$, where $\varphi^{s}$
is given by

\[
\varphi^{s}(w)=f(w\sigma^{-1})\varphi(w\sigma^{-1}).\]


It follows that $\varphi\rightarrow\varphi^{s}$, which is evidently
a unitary operator in $L^{2}(G\times G^{*})$, leaves the composition
law (13) invariant; it is easy, of course, to verify directly. Therefore,
if we write, in these conditions, $K=W(\varphi)$ and $K^{s}=W(\varphi^{s})$,
in other words, if we define a mapping $K\rightarrow K^{s}$ of $L^{2}(G\times G)$
to $L^{2}(G\times G)$ by the formula

\[
W^{-1}(K^{s})=(W^{-1}(K))^{s},\]


this mapping will satisfy the hypothesis of Lemma 3. After this lemma,
there is therefore an automorphism $t$ of $L^{2}(G)$ such that it
has, for any $P$, $Q$ in $L^{2}(G)$, $(P\otimes Q)^{s}=P^{t}\otimes Q^{\overline{t}}$.
Now we change the notations, writing $P\rightarrow s^{-1}P$ instead
of $P\rightarrow P^{t}$; substitute $Q$ by $\overline{Q}$, and
apply (12); this gives:

\[
(P,U(w)Q)^{s}=(s^{-1}P,U(w)s^{-1}Q).\]


By the definition of $\varphi^{s}$, the first member has the value

\[
f(w\sigma^{-1})\cdot(P,U(w\sigma^{-1})Q)=(P,f(w\sigma^{-1})^{-1}U(w\sigma^{-1})Q)\]


since $(P,Q)$ is antilinear in $Q$ and that $f$ takes its values
in $T$; and the second member is equal to $(P,sU(w)s^{-1}Q)$ since
$s$ is unitary. As the relation obtained is valid for any $P$, $Q$,
we thus have

\[
f(w\sigma^{-1})^{-1}U(w\sigma^{-1})=sU(w)s^{-1}\]


whence, replacing $w$ by $w\sigma$:

\begin{equation}\label{eq:15}
s^{-1}U(w)s=f(w)\cdot U(w\sigma)=U(w)^{s}.
\end{equation}

This amount to say that the inner automorphism determined by $s$
in the unitary group induces the automorphism $s$ on $\mathbf{A}(G)$.
Reciprocally, given $s$, this relation determines an element near
the centralizer of $\mathbf{A}(G)$. Yet, if a unitary operator commutes
with all $U(w)$, it is so with all $U(\varphi)$, therefore with
the operators of the form (11) for any $K\in L^{2}(G\times G)$. For
$K=P\otimes\overline{Q}$, (11) defines the operator $\Phi\rightarrow(\Phi,Q)\cdot P$;
if $\Phi\rightarrow\Phi^{t}$ commute with this one, then we have

\[
(\Phi,Q)\cdot P^{t}=(\Phi^{t},Q)\cdot P\]


for any $P$, $Q$, $\Phi$ in $L^{2}(G)$; then $\Phi\rightarrow\Phi^{t}$
is of the form $\Phi\rightarrow t\cdot\Phi$, where $t$ is a scaler;
if this operator is unitary, we have $t\in T$. We denote by $\mathbf{T}$
the group formed by the operatore of this form; it is the center of
$\mathbf{A}(G)$, and it is also the center of the group of all the
automorphisms of $L^{2}(G)$. We have then shown the following theorem:

\begin{thm}\label{th:1}
The centralizer of $\mathbf{A}(G)$ in the group of the
automorphisms of $L^{2}(G)$ is the center $\mathbf{T}$of these two
groups; furthermore, if $\bB_{0}(G)$ is the normalizer of $\mathbf{A}(G)$
in the same group, any automorphism of $\mathbf{A}(G)$ inducing the
identity on $\mathbf{T}$ is induced on $\mathbf{A}(G)$ by the inner
automorphism determined by an element of $\mathbf{B}_{0}(G)$; and
$\bB_0(G)/T$ is isomorphic to $B_0(G)$, 	
i.e. the group of automorphisms of $A(G)$ inducing identity on $T$.
\end{thm}

\subsection{}
We know that the Fourier transform induces 
an automorphism on a certain space of continuous functions 
(say, quite wrongly, ``indefinitely differentiable a rapid decay'');
 the space $S(G)$ has been introduced primarily for this reason, 
by L. Schwartz ([4], Chap. VII) in the case of $\bR^n$ 
and F. Bruhat [1] in the general case.
We will see that the operators of $\bB_0(G)$ have the same property.

Recall the definition of $S(G)$ for a locally compact abelian group $G$.
First, consider a ``elementary'' group, i.e. of the form 
$G=\bR^n\times \bZ^p\times T^q\times F$, where $F$ is a finite group. 
A polynomial function on $G$ will be, by definition, 
a function that can be written as  a polynomial relative to 
$\bR$ and $\bZ$ coordinate in the product of $G$;
$S(G)$ is the set of all founctions $\Phi$, indefinitely differentiable 
on $G$; such that $P\cdot D\Phi$ is bounded on $G$ 
whenever differential operator invariant under 
translation $D$ and the polynomial $P$; 	
topology of $S(G)$ is induced from all seminorms $\sup\abs{P\cdot D\Phi}$.
In the general case, we will introduce
all couples $(H, H')$ of subgroups of G with the following properties:
\begin{enumerate}[(i)]
\item $H$ is generated by a compact neighborhood of $0$ 
in G (it is open and closed in $G$);
\item $H'$is a compact subgroup of $H$ and $H/H'$ is isomorphic to
 a elementary group.  
\end{enumerate}
For such a couple, we corresponds a family  $S(H, H')$
of continuous functions on $G$, which supported in $H$, and constant on cosets 
of $H'$, 	
by restriction to $H$ and passing to the quotient $H/H'$, 
it  belongs to $S(H/H')$. 	
Then $S(G)$ is the union of $S(H,H')$ 
and  the topology is given by ``inductive limit'' of $S(H/H')$, 
ie a convex set X is a neighborhood of $0$ in $S(G)$ if, 
for all couple $(H,H')$, the image of $X\cap S(H,H')$ in $S(H/H')$ 
is a neighborhood of $0$ in $S(H/H')$.
We intend to see that any $\bs\in \bB_0(G)$ induce an automorphism of $S(G)$;
We only have to show that $\bs$ induce a continuous map 
from $S(G)$ to itself, we will follow step by step of the proof of 
the Theorem~\ref{th:1}.  	
We will write again $t$ instead of $\bs^{-1}$, $P^t$ 	
instead of $\bs^{-1}P$ for $P\in L^2(G)$, 	
and we will prove for the operator $P\to  P^t$.
After the foregoing, if we are given $Q\neq 0$ in $S(G)$,
the map $P\to P^t$ is the following composition:
\[
(a) P\to K=P\otimes Q;\quad (b)K\to \phi=W^{-1}(K);\quad (c) \phi\to \phi^s
\]
\[
(d) \phi^s\to K^s=W(\phi^s);\quad (e) K^s= P^t\otimes Q^{\overline{t}}\to P^t;
\]
it is sufficient to show that maps  from $S(G)$ to  $S(G\times G)$, 
then to $S(G\times G^*)$,
to $S(G\times G^*)$, to $S(G\times G)$ and to $S(G)$ are all continuous.
For $(a)$ is immediately; (e) is also immediately if function 
$K\in S(G\times G)$ is in the form $P\otimes Q$ in the sense of $L^2(G\times G)$
where $P$ and $Q$ are in $S(G)$, 
and then, for $Q\neq 0$ in $S(G)$, the map $P\to  P\otimes Q$ 
is an isomorphism from 
$S(G)$ to a closed sub-space of $S(G\times G)$.
\begin{expl}
\def\tH{{\tilde{H}}}
We have to prove that $E=\Set{P\otimes Q\in S(G\times G)}$ is closed
and $i:P\to P\times P\otimes Q$ is isomorphism.

By the definition of inductive limit, $i$ is countinous iff
$i|_{S(H,H')}$ is countinous.
Suppose $Q\in S(\tH,\tH')$ then 
$i_{S(H,H')}:S(H,H') \to S(H\times \tH, H'\times \tH')\hookrightarrow S(G\times G)$
$P_n$ converges in $S(H,H')\cong S(H/H')$
\[
\abs{p(x,y)D^{\alpha_1}D^{\alpha_2}P_n\otimes Q}\leq \abs{(p(x)q(y)+C)D^{\alpha_1}P_n\otimes D^{\alpha_2}Q}
\to 0
\]
$i$ continous.

\[
\abs{p D^{\alpha} P} < \abs{p(x)D^{\alpha} P\otimes Q} 
\]
\end{expl}

For (b) and (d), it is intended to show that $W$
determines an isomorphism from $S(G\times G^*)$ to $S(G\times G)$, 
but $W$ is composed of the operator
$F(x, y) \to F(y-x,-x)$, which obviously an automorphism of $S(G\times G)$,
and the partial Fourier transform on the second factor of 
the product $G\times G^*$.
It is therefore requires to verify that if $A$ and $B$ are
locally compact abelian groups 
and if $B^*$ is the dual of $B$, the partial Fourier transform
\[
f(a,b)\to f'(a,b^*) = \int f(a,b)\inn{b}{b^*} db
\]
determines an isomorphism from $S(A\times B)$ to  $S(A\times B^*)$.
This is an easy generalization of the theorem similar to
ordinary Fourier Transform.
\begin{expl}
Should be $F(x,y)\to F(y-x,x)$?
\[
\begin{matrix}
W \colon & S(G\times G^*) & \to & S(G\times G) & \to & S(G\times G)&\\
&\phi &\mapsto & \int \phi(x,u^*)\inn{u^*}{y}du^* 
& \mapsto & \int \phi(y-x, u^*)\inn{x}{u^*}du^* &= W(\phi)
\end{matrix}
\]

$f\in S(H,K)$, $H$ can be choosen bigger s.t. $H=H_1\times H_2$,
$K$ can be choosen smaller s.t. $K = K_1\times K_2$.
Partial Fourier Transform maps $f$ to 
$S(H_1\times K_2^\perp, K_1\times H_2^\perp)$.
\[
0\to H_2^\perp \to K_2^\perp \to (H_2/K_2)^\perp\to 0
\]
$\int_G f(x,y)\inn{y}{\alpha^*}dy 
= \int_H f(x,y)\inn{y}{\alpha^*}dy$
is constant if $\alpha^* \in H_2^\perp$.
$\int_G f(x,y)\inn{y}{\alpha^*}dy 
= \int_{H_2/K_2} \int_{K_2} f(x,y+k)\inn{y+k}{\alpha^*} dk\,dy
= \int_{H_2/K_2}  f(x,y)\inn{y}{\alpha^*}
\int_{K_2} \inn{k}{\alpha^*} dk\,dy
=0$
if $\alpha^*\notin K_2^\perp$.

This transform can be consieder as Partial Fourier transform 
from $S(H_1/K_1 \times H_2/K_2)$ 
to $S(H_1/K_1 \times (H_2/K_2)^*)$.
Then is automorphism by elementary case. 
\end{expl}

\subsection{}
I1 we still consider a (c). Since an automorphism $\sigma$ of a $G\times G^*$ 
obviously determine an automorphism of $S(G\times G^*)$,
we (after replacing $G$ to $G\times G^*$) reduces to prove that: 
\begin{expl}
\[
\phi^s = f(w\sigma^{-1})\phi(w\sigma^{-1})
\]
\end{expl}
\begin{prop}\label{p:2}
Let $f$ be a character of second degree of $G$.
Then $\Phi\to \Phi f$ is an automorphism of $S(G)$.
\end{prop}
First, let $G=\bR^n\times \bZ^p \times T^q \times F$, 
with $F$ a finite group; it is easy to see, all remains to show that 
 any differential operator $D$ is invariant by translation on $G$,
there is a polynomial function $P$ on $G$ such that $\abs{Df}<\abs{P}$.
\begin{expl}
\[
\abs{p D^\alpha \Phi f} = \abs{\sum_{\beta\leq \alpha} p C_\beta D^{\alpha-\beta} \Phi D^{\alpha-\beta}f}
\]
this gives $\Phi \to \Phi f$ continuous.
\end{expl}
This is not difficulty to varify the  expressing $f$ on the  coset of 
$\bR^n\times T^q$ in $G$ by using the formula (\ref{eq:1}) in subsection~1,
and noting that, for $\bR^n\times T^q$, $f$ is necessarily of the form 
$e^{iF(x)}\chi(x,y)$ where $x\in \bR^n$, $y\in T^q$, 
$F$ is a quadratic form on $\bR^n$ and 
$\chi$ is a character of $\bR^n\times T^q$. 	
\begin{expl}
\[
\frac{f((x_1,y_1)+(x_2,y_2))}{f((x_1,y_1))f((x_2,y_2)} = \inn{(x_1,y_1)}{(x_2,y_2)\rho}
\]
$\rho\colon G\to G^*$ is a morphism.
$G^* = \bR^{n*}\times \bT^{q*} \cong \bR^n\times \bZ^q$ 
So $\rho$ maps into $\bR^{n*}\times 0$ by connectness. 
The image of $0\times \bT^{q}$ is trivial by campactness and connectness. 
Then $\rho$ only depends on $x$.
$2$ is auto. So $f(x,y) = \inn{x}{2^{-1}x\rho}\chi(x,y)$
(by section~1, ker of $X_2(G)=X_1(G)\times X^0_2(G)$)
\end{expl}
Moving to the general case, let $\rho$ be a symmetric morphism of from 
$G$ to $G^*$ associated to $f$, and give a subgroup $H$ of $G$ 
generated by a compact neighborhood of $0$.
For a subgroup $H'$ of $H$ satisfies the condition $(ii)$ 
of the definition of $S(H, H')$, 	
it is both necessary and sufficient,
 as we know (see [1], $n^o$~9, p. 60), that
the group $H'_*$ which is associated by duality in $G^*$ 
(the ``orthogonal'' of $H'$) is generated by a compact neighborhood of $0$, 
\begin{expl}
\begin{enumerate}[(1)]
\item
(c.f. On the Schwartz-Bruhat Space and the Peley-Winer Theorem 
for Locally Compact Abelian Groups Scott Osborne.)
$K$ is ``good'' iff $G/K$ is a Lie group.

$K$ is ``good'' iff $K^\perp$ is open and compactly generated
(i.e. $K^\perp$ is generated by a compact neighborhood of $0$).

This gives $H'_*$ comapct generated
\item
$0\to H'_* \to H^* \to H'^* \to 0$ 
(c.f. A First Course in Harmonic Analysis 109 Ex~7.8)
$H'_* \cong (H/H')^*$
$H'$ compact, then $H'^*$ discrete, So $H'_*$ is open.
\item (c.f. Scott Osborne)
$G$ is compactly generated iff $G^*$ is a Lie group.
\proof
Let $C$ be a compact subset of $G$, $U$ a neighborhood of $1$ in $\bT$. 
Let $W(C,U) = \Set{x^*\in G^*| x^*(C)\subset U}$ 
be a neighberhood of $0$ of $G^*$.
Let $<C>$ is the group $C$ generated, then $F^\perp \subset W(C,U)$. 
Suppose that $U$ contains no nonidentity subgroups of $\bT$
(we always can choose such $U$).
Let $E$ be a subgroup in $W(C,U)$, 
clearly $\Set{\inn{c}{e}|e\in E}$ is a subgroup of $\bT$ for any fixed $c\in C$.
Then $\inn{c}{e}=1$ for all $c\in C, e\in E$, $E \subset <C>^\perp$.
Thus $G$ is compactly generated(for example by $C$) iff $G^*$ has
no small subgroups, i.e. iff $G^*$ is a Lie group. 

(In 1953, Hidehiko Yamabe obtained the final answer to Hilbert��s Fifth Problem: 
A connected locally compact group G is 
a projective limit of a sequence of Lie groups,
and if G has ``no small subgroups'', then it is a Lie group.
``no small subgroups''  if there is a neighbourhood N of $e$
 containing no subgroup bigger than the trivial one.

The locally compact abelian group case was solved in 1934 by Lev Pontryagin. )
\end{enumerate}
end.
\end{expl}
then the group $H'_* +H\rho$ has the same property, 
this shows that replacing $H'_*$ with $H'$  a smaller group satisfing (ii),
 we can ensure that we have $H\rho \subset H'_*$. 
\begin{expl}
Note that if $K<G$, $K^{\perp\perp} = K$.
Clearly $K \leq K^{\perp\perp}$, If $K\lneq K^{\perp\perp}$,
 $K^{\perp\perp}/K$ nontrivial, then exists nontrivial 
character $\alpha^*$ on $K^{\perp\perp}/K$. 
Extend it to $G$, call it $\alpha^*$ also.
Then $\alpha^* \in K^\perp$ but not in $K^{\perp\perp\perp}$.
It's clearly that $K^\perp = K^{\perp\perp\perp}$($\perp$ is a Galois relation).
This leads a contradiction.

Now let  $\tilde{H}'_*=H'_* + H\rho$.
Then $\tilde{H}' = \tilde{H}_*^{'\perp} < H_*^{'\perp} = H'$.

By above $H/\tilde{H}'$ is a Lie group. But 
compactly generated abelian Lie group is elementary!
\end{expl}
By (\ref{eq:1}) subsection~1, this gives $f(h + h')=f(h)f(h')$ 
for each $h\in H$, $h' \in H'$.  
\begin{expl}
\[
\frac{f(h+h')}{f(h)f(h')} = \inn{h'}{h\rho} = 1
\]
\end{expl}
In particular, $f$  induces a character of $H'$, we can extend to a
character of $G$  of the form $\inn{h'}{a^*}$ with $a^*\in G^*$. 	
\begin{expl}
\[
0\to H'\to G\to G/H' \to 0
\]
induced 
\[
0 \to (G/H')^* \to G^* \to H'^* \to 0
\]
So we can extend it. 
\end{expl}
Replacing $H'_*$ by the group generated by $H'_*$ and $a^*$,
we can make that $a^*$ in $H'_*$;
\begin{expl}
as above
\end{expl}
since, $f$ induces the constant $1$ on $H'$ and 
and is constant in each coset of $H'$.
\begin{expl}
\[
f(h+h') = f(h)f(h')\inn{h'}{h\rho} = f(h)
\]
\end{expl} 
After what has been proved in the case $G$ is a  elementary group,
it follows, by passing to the quotient,
 that $\Phi\to \Phi f$ determine an automorphism of $S(H, H')$.
As is the case, for any $H$ satisfying (i), provided that $H'$ 
has been made small enough and satisfies (ii) it completed the
prove.
(given the definition of the topology of $S(G)$ by inductive limit).

\subsection{}
The homomorphism $\bs \to s = (\sigma, f)$ from $\bB_0(G)$ to $B_0(G)$
 determined by (\ref{eq:15}) will be denote $\pi_0$
  and is called the {\it canonical projection} 
from the first group to the second.
In general (as shown below the example of the type of local groups), 
there is no section of $\bB_0(G)$ over $B_0(G)$ 
is also a subgroup of $\bB_0 (G)$. 
But it is very useful to know that we can at least define
 sections over the subgroups and
subsets of $B_0(G)$ have been introduced in $n^0~6$ and $7$. 

Let $\Phi\in L^2(G)$. For any automorphism $\alpha$ of $G$ 
we set
\[
\bd_0(\alpha)\Phi(x)=\abs{\alpha}^{\frac{1}{2}}\Phi(x\alpha).
\]
\begin{expl}
$\bd_0(\alpha)$ is unitary.
\int \abs{\alpha}\abs{\Phi(x\alpha)}^2 = \int \abs{\Phi(x)}^2
\end{expl}
For any seconde degree characher $f$ of $G$, we set
\[
\bt_0(f)\Phi(x) = \Phi(x)f(x).
\]
For any isomorphism $\gamma$ from $G^*$ to $G$, we set
\[
\bd'_0(\gamma)\Phi(x) = |\gamma|^{-\frac{1}{2}} \Phi^*(-x\gamma^{*-1}),
\]
\begin{expl}
Fourier transform is unitary.
\end{expl}
where, $\Phi^*$ denotes the Fourier transform of $\Phi$.
We can easily verify that $\bd_0$, $\bt_0$, $\bd'_0$ lifting of  
$d_0, t_0, d'_0$ defined in $n^o 6$ to  $\bB_0(G)$, 
i.e. $d_0 = \pi_0 \circ \bd_0$, $t_0=\pi_0\circ \bt_0$, 
$d'_0 = \pi_0 \circ \bd'_0$. 
\begin{expl}

\end{expl}
Moreover, $\bd_0$ and $\bt_0$ are monomrophisms
from $B_0(G)$ to the group of the automorphisms of $G$, and the group
$X_2(G)$ respectively; and when $\alpha$, $f$, $\gamma$ are like above,
we have:
\[
\bd_0(\alpha)^{-1}\bt_0(f)\bd_0(\alpha) = \bt_0(f^\alpha),\quad
\bd'_0(\gamma \alpha) = \bd'_0(\gamma)\bd_0(\alpha),\quad
\bd'_0(\alpha^{*-1}\gamma) = \bd_0(\alpha)\bd'_0(\gamma).
\]

The propersiton~\ref{p:1} of $n^o~7$ 
allows us raise $B_0 (G)$ for all the element of the set
$\Omega_0(G)$ defined in this propersition.
 Indeed, according to this, any $s\in \Omega_0(G)$
 is has a unique form as (\ref{eq:8}); 
 $s$ is given by (\ref{eq:8}), we set
\[
\br_0(s)=\bt_0(f_1)\bd'_0(\gamma)\bt_0(f_2).
\]
An easy calculation shows this formula;  
by writing, as usual, $S = (\sigma ,f)$,  
$\sigma = \begin{pmatrix}\alpha &\beta \\ \gamma & \delta\end{pmatrix}$,
 we obtains
\[
\br_0(s)\Phi(x) = 
\abs{\gamma}^{\frac{1}{2}}\int \Phi(x\alpha + x^*\gamma) f(x,x^*) dx^*.
\]
The conditions of this formula valid are obviously same as 
those of the formula of  definition transformation of Fourier, 
who was useful has to clarify $\bd'_0$; 
for example, 
it is valid almost everywhere if $\Phi\in L^2(G)\cap L^1(G)$;
it is valid for all x if $\Phi\in S(G)$,
the two members then defining a same function of $S(G)$.


\end{document}

